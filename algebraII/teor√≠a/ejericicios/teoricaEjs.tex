\documentclass[12pt]{article}

\usepackage[margin=1in]{geometry}
\usepackage{enumerate}
\usepackage{amsmath}
\usepackage{amssymb}
\usepackage{amsfonts}
\usepackage{mathtools}
\usepackage{amsthm}


\begin{document}
Ej 1.3)
Sean $e$ y $e'$ dos elementos neutros de un grupo $G$. Luego sea $x \in G$

Sabemos $xe = ex = x = e'x = xe'$ por lo que $ex = e'x$ 

Sabiendo que existe inverso tenemos $exx^{-1} = e'xx^{-1}$

Y suponiendo que $xx^{-1} = e$ luego $ee = e'e$

iFnalmente considerando que $e, e' \in G$ y que un elemento de $g \in G$ operado con un elemento neutro da el mismo elemento $g$ tenemos $e = e'$

Ej 1.8) Sale por inducción primero probas $a^{n + 1} = a^{n}a^{1}$ y despues salen los otros dos


Ej 1.9)

Veamos que $L_{g}$ es biyectiva.

Inyectividad: Sean $x , x' \in G$ tal que $gx = f(x) = f(x') = gx'$

Luego como $g \in G$ existe elemento inverso y elemento neutro

Luego $g^{-1}gx = g^{-1}gx' \Rightarrow ex = ex' $ entonces $x = x'$

Suryectividad: Sea $r \in G$ ahora queremos ver si $r = gx$ con $ g \in G$ tiene solución.

Por ejericio 1.4) sabemos que tiene y que es única además $g^{-1}r = x$ y como $g^{-1} \in G$ y $g \in G$ y la operación de $G$ es cerrada luego $x \in G$ 

Luego $r$ tiene pre imagen $\forall r \in G$ 

Ambos procesos son se usan de forma análoga para probar la otra biyección

Ej 1.23) 
\begin{itemize}
  \item Sabemos que $e \in Z(G)$ por que $ex = x = xe$ luego $ex = xe \quad \forall x \in G$
  \item Sea $r \in Z(G)$ sabemos que $rh = hr \quad \forall h \in G$ 

    Luego como $r \in G$ tiene inverso, usémoslo para multiplicar ambos lados varias veces con el inverso $r^{-1}$

    $r^{-1}rh = r^{-1}hr \Rightarrow r^{-1}rhr^{-1} = r^{-1}hrr^{-1} \Rightarrow  hr^{-1} = r^{-1}h \quad \forall h \in G$
  
    Finalmente $r^{-1} \in Z(G)$

  \item Sea $r \in Z(G) \quad r_{1} \in Z(G)$ entonces tenemos $rh =hr \quad$ y $\quad r_{1}h = h r_{1} \quad \forall h \in G$ 

    Luego multiplicando convenientemente sabiendo que existe inverso 

    $h = r^{-1}hr$ y tambié $r_{1}hr_{1}^{-1} = h$ por lo tanto $r^{-1}hr = r_{1}hr_{1}^{-1} \quad \forall h \in G$

    Devuelta multiplicando $hr = rr_{1}hr_{1}^{-1} \Rightarrow hrr_{1} = rr_{1}h \quad \forall h \in G$

    Luego $rr_{1} \in Z(G)$

\end{itemize}

Ej 1.24) 
\begin{itemize}
  \item Sabemos que $e \in C_{G}(g) \quad \forall g \in G$ por que $ge = eg$
  \item  Sea $r \in C_{G}(g)$ entonces $gr = rg $ 

    Luego como $r \in G$ tiene inverso 

    Por lo tanto $g = rgr^{-1} \Rightarrow r^{-1}g =gr^{-1} \Rightarrow gr^{-1} = r^{-1}g$

    Entonces $r^{-1} \in C_{G}(g)$

  \item Sean $r,r_{1} \in C_{G}(g)$ luego $gr =rg$ y tambien $gr_{1} = r_{1}g$ ahora multiplicando estratégicamente por inversos llegamos a $r^{-1}gr = g$ y por otro lado $g = r_{1}gr_{1}^{-1}$

    Entonces tenemos $r^{-1}gr = r_{1}gr_{1}^{-1} \Rightarrow gr = r r_{1}gr_{1}^{-1} \Rightarrow grr_{1} = rr_{1}g $

    Luego $rr_{1} \in C_{G}(g)$
\end{itemize}


\end{document}
