\documentclass[12pt]{article}

\usepackage[margin=1in]{geometry}
\usepackage{enumerate}
\usepackage{amsmath}
\usepackage{amssymb}
\usepackage{amsfonts}
\usepackage{mathtools}
\usepackage{amsthm}


\begin{document}
Ej 1.3)
Sean $e$ y $e'$ dos elementos neutros de un grupo $G$. Luego sea $x \in G$

Sabemos $xe = ex = x = e'x = xe'$ por lo que $ex = e'x$ 

Sabiendo que existe inverso tenemos $exx^{-1} = e'xx^{-1}$

Y suponiendo que $xx^{-1} = e$ luego $ee = e'e$

Finalmente considerando que $e, e' \in G$ y que un elemento de $g \in G$ operado con un elemento neutro da el mismo elemento $g$ tenemos $e = e'$

Ej 1.9)

Veamos que $L_{g}$ es biyectiva.

Inyectividad: Sean $x , x' \in G$ tal que $gx = f(x) = f(x') = gx'$

Luego como $g \in G$ existe elemento inverso y elemento neutro

Luego $g^{-1}gx = g^{-1}gx \Rightarrow ex = ex' $ entonces $x = x'$

Suryectividad: 


\end{document}
