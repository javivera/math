\documentclass[11pt]{report}

\usepackage[margin=1in]{geometry}
\usepackage{enumerate}
\usepackage{amsmath}
\usepackage{amssymb}
\usepackage{mathtools}
\usepackage{amsfonts}
\usepackage{amsthm}
\usepackage{graphicx}
\usepackage{fancyhdr}
\pagestyle{fancy}

\newcommand{\n}{\aleph_{0}}
\newcommand{\F}{\mathhbb{F}}
\newcommand{\Q}{\mathbb{Q}}
\newcommand{\C}{\mathbb{C}}
\newcommand{\R}{\mathbb{R}}
\newcommand{\K}{\mathbb{K}}
\newcommand{\E}{\mathbb{E}}
\newcommand{\I}{\mathbb{I}}
\newcommand{\Z}{\mathbb{Z}}
\newcommand{\N}{\mathbb{N}}
\newcommand{\Ra}{\Rightarrow}
\newcommand{\ra}{\rightarrow}
\newcommand{\ol}{\overline}
\newcommand{\norm}[1]{\left\lVert#1\right\rVert}
\newcommand{\open}{\mathrm{o}}


\theoremstyle{definition}
\newtheorem{definition}{Definición}[section]
\newtheorem*{remark}{Observación}
\newtheorem{theorem}{Teorema}
\newtheorem{lemm}{Lema}
\newtheorem{corollary}{Corolario}[theorem]
\newtheorem{lemma}[theorem]{Lema}
\newtheorem{prop}{Proposición}
\newtheorem{ej}{Ejercicio}


\fancyhead[R]{Espacios Métricos}
\fancyhead[L]{Alumno Javier Vera}
\fancyhead[C]{Cálculo Avanzado}

\DeclarePairedDelimiter\Floor\lfloor\rfloor
\DeclarePairedDelimiter\Ceil\lceil\rceil

\begin{document}
	\begin{ej}
		
	\end{ej}

	\begin{ej}
		
	\end{ej}

	\begin{ej}
		
	\end{ej}

	\begin{ej}
		
	\end{ej}

	\begin{ej}
		
	\end{ej}

	\begin{ej}a
		\begin{enumerate}
			\item Mostrar que el intervalo $(0,1] \subseteq \R$ no es compacto.
				\begin{proof}
					Tenemos la sucesión de Cauchy $a_n = \frac{1}{n}$ que no converge , por lo tanto no es completo				
				\end{proof}
				
				
			\item Sea $S = (a,b) \cap \Q$ con $a,b \in \R \setminus \Q$. Probar que $S$ es un subconjunto cerrado y acotado pero no compacto de $(\Q,d)$, donde $d$ es la métrica usual de $\R$
				\begin{proof}
					Acotado por $a$ y $b$ es evidente, que es cerrado es facil ver que es igual a su clausura
					
					No es completo por que tenemos la sucesion $a_n = a + \frac{1}{n}$ que es de Cauchy , pero no converge
				\end{proof}
		\end{enumerate}
	\end{ej}
	
	\begin{ej}
		Sea $E = \{e^{(n)} \in \ell^{\infty} \  / \ n \in \N\}$, donde cada sucesión $e^{(n)} = (e_k^{(n)})_k$ esta definida por 

		$e_k^{(n)} = 0$ si $k \neq n$ y es 1 si $k = n$

		Probar que $E$ es discreto, cerrado y acotado

		\begin{proof}
			Es facil ver que cualquier subconjunto de $E$ es abierto.

			Sea $A \subset E$ , sea $x \in A$ si tomamos $B(x,\frac{1}{2}) $ sabemos que $B(x,\frac{1}{2}) = \{x\} \subseteq A$

			Por lo tanto $A$ es abierto. Para ver que cualquier sub es cerrado , tomamos $F \subset E$ y miramos su complemento $F^c$, que como es sub de $E$ tiene que ser abierto por lo recién demostrado. Entonces $F$ es cerrado

			Por lo tanto todo subconjunto de $E$ es abierto y cerrado, mostrando que $E$ es discreto.

			Cerrado: Tomemos una sucesión $a_k$ de $E$ convergente, como $E$ es discreto las únicas sucesiones convergentes son las constantes y una sucesión que está contenida en $E$ y que eventualmente es constante tiene que se constantemente algo en $E$.

			Si no la sucesión tendría elementos fuera de $E$ lo cual es absurdo por que es una sucesión de elementos de $E$

			Acotado: Tomamos cualquier $e^n \in E$ y vemos $B(e^n,2)$.

			Ahora dado cualquier elemento $e^j \in E$ sabemos que $d(e^j_k,e^n_k) = \sup_{k \in \N}|e^j_k - e^n_k| \leq 1$

			Entonces $e^j \in B(e^n,2)$ y esto vale para cualquier $e^j\in E$ por lo tanto $E \subseteq B(e^n,2)$. 

			Entonces $E$ es acotado. Pero no es totalmente acotado.

			Por que si tomamos $\epsilon = \frac{1}{2}$ y queremos cubrir $E$ tomando bolas de radio $\epsilon$ , vamos a necesitar una bola para cada elemento de $E$. 

			Pero $E$ tiene numerables elementos, entonces no podemos cubrir con finitas bolas de radio $\epsilon$

			Entonces $E$ no es compacto.
		\end{proof}
	\end{ej}
	
	\begin{ej}
		Sea $c_0 = \{(x_n)_n\subset \R \ / \ \lim_{n\ra\infty} x_n = 0\}$. Se define en $c_0$ la métrica
		$$ d(x,y) = \sup\{|x_n - y_n| \ / \ n \in \N\}$$

		Demostrar que la bola cerrada $\ol B (x,1) = \{y \in c_0 \ / \ d(x,y) \leq 1\}$ no es compacta.

		\begin{proof}
			Sabemos que $c_0$ es completo , entonces la bola cerrada es completa. Lo que tiene que fallar es totalmente acotada
		\end{proof}
	\end{ej}

	\begin{ej}
		Sea $X$ un espacio métrico y sea $(x_n)_n \subset X$ tal que $\lim_{n\ra\infty } a_n = a \in X$. Probar que el conjunto $K = \{a_n \ / \ n \in \N\} \cup \{a\} \subset X$ es compacto
		\begin{proof}
 			Dado un cubrimiento de abiertos de $K$. Sabemos que en alguno de esos abiertos $A$ está $a$. 

			Ahora como $A$ es abierto tenemos $B(a,r) \subset A$. Y dado que $a_n$ converge a $a$. 

			$\exists n_0$ tal que $d(a_n,a) < r \quad \forall n \geq n_0$. Entonces $\{a_n : n \geq n_0\} \subseteq B(a,r) \subseteq A$

			Ahora sea $x_n$ con $n < n_0$ sabemos que hay algún abierto del cubrimiento para cada uno (si nó no sería cubrimiento)

			Por ser finitos , tenemos finitos abiertos del cubrimiento. Y si a esos le agregamos $A$, tenemos un subcubrimiento finito de $K$. 

			Esto muestra que $K$ es compacto
				
		\end{proof}
	\end{ej}
	
	\begin{ej}
		Probar que todo espacio métrico compacto es separable
		\begin{proof}
			Dado un cubrimiento del espacio, tengo sub cubrimiento finito , por ser compacto,  entonces tengo subcubrimiento contable. Por ende es separable	
		\end{proof}
	\end{ej}

	\begin{ej} Sea $(X,d)$ un espacio métrico. Probar que:
		\begin{enumerate}
			\item Si $(X,d)$ es compacto, todo subconjunto cerrado de $X$ es compacto.
				\begin{proof}
					Sabemos que todo cerrado dentro de un completo es completo. Y todo conjunto dentro de un TTA es TTA

					Entonces cualquier subconjunto cerrado es completo y TTA por lo tanto compacto.
				\end{proof}
			\item Todas unión finita y toda intersección (finita o infinita) de subconjuntos compactos de $X$ es compacta
				\begin{proof}
					Unión finita de cerrados es cerrada , por lo tanto la unión es un cerrado dentro de un compacto y entonces completo.

					Por otro lado subconjunto de un TTA es TTA. Entonces esta unión finita por ser subconjunto de un TTA es TTA.

					Finalmente es compacto

					Lo mismo pasa con intersección de cerrados , es cerrado , por lo tanto completo, y sigue siendo un sub de un TTA entonces es TTA, por lo tanto compacto
				\end{proof}
			\item Un subconjunto $F \subset X$ es cerrado si y sólo si $F \cap K$ es cerrado para todo compacto $K \subset X$
				\begin{proof}
					$\Ra )$ Si $F$ es cerrado y $K$	compacto, entonces intersecarlos es intersecar cerrados, por lo tanto es cerrado

					$\Leftarrow )$ Sea $(x_n)_n \subseteq F$ convergente a $x$. 

					Por ejercicio pasado sabemos que $A = \{x_n : n \in \N\} \cup \{x\}$ es compacto

					$F \cap A$ es cerrado, por lo tanto $(x_n)_n \subset F \cap A$ es una sucesión convergente en un cerrado

					Por lo tanto $x \in F \cap A$ por lo tanto $x \in F$.

					Entonces $F$ es cerrado
				\end{proof}
		\end{enumerate}
	\end{ej}
	
	\begin{ej}
		Sean $(Y,d)$ e $(Y,d')$ espacios métricos. Se consiera $(X\times Y,d_{\infty})$, donde
		$$ d_{\infty}((x_1,y_2),(x_2,y_2))= max\{d(x_1,x_2),d'(y_1,y_2)\}$$

		Probar que $(X\times Y,d_{\infty})$ es compacto si y sólo si $(X,d)$ e $(Y,d')$ son compactos.
		\begin{proof}
			$\Ra $) Sea $(x_n)_n \subset X$  e $(y_n)_n \subset Y$ entonces $(x_n,y_n)_n \subset X \times Y$

			Por lo tanto existe un sub $(x_{n_k},y_{n_k})$ convergente a $(x,y)$

			entonces $d_{\infty }((x_{n_k},y_{n_k}),(x,y)) < \epsilon \quad \forall n \geq n_0$

			Por lo tanto $d(x_{n_k},x) < \epsilon \quad \forall n \geq n_0$ y lo mismo pasa con $y_{n_k}$

			Entonces dado $x_n$ encontramos un sub convergente, por lo tanto $X$ es compacto

			Y lo mismo pasa con $Y$

			$\Leftarrow ) $ Sea $(x_n,y_n)_n \subset X \times Y$. 

			Entonces $(x_n)_n \subset X$ e $(y_n)_n \subset Y$

			Por ser $X$ compacto, existe un sub $x_{n_k}$ convergent a digamos $x \in X$ 

			Ahora si vemos $y_{n_k} \subset Y$ y tenemos en cuenta que $Y$ compacto

			Existe $y_{n_{k_j}}$ convergente a digamos $y \in Y$

			Ademas como $x_{n_k}$ converge a $x$ entonces $x_{n_{k_j}}$ también

			Por lo tanto tenemo que $d_{\infty}((x_{n_{k_j}},y_{n_{k_j}}),(x,y)) = max\{d(x_{n_{k_j},x}),d'(y_{n_{k_j}},y)\} <  \epsilon \quad \forall n \geq n_0$

			Por lo tanto encontramos una sub convergente entonces $X \times Y$ es compacto
		\end{proof}
	\end{ej}
	
	\begin{ej}
		Sea $X$ un espacio métrico compacto y sea $f : X \ra \R$ una función contínua tal que $f(x) > 0$ para todo $x \in X$. Probar que existe $\epsilon >0$ tal que $f(x) \geq \epsilon$ para todo $x \in X$
		\begin{proof}
			Supongamos que no es cierto entonces $\forall \epsilon > 0 $ existe un $x \in X$ tal que $f(x) < \epsilon$

			Ahora si tomamos $\epsilon = \frac{1}{n}$ vamos a tener una sucesión $(x_n)_n \subset X$ tal que $f(x_n) < \frac{1}{n}$

			Ahora sabemos que existe un sub $x_{n_k}$ que converge a $x\in X$ entonces $f(x_{n_k})$ converge a $f(x) \in \R$

			Pero además $0\leq f(x_{n_k}) < \frac{1}{n_k}$ por lo tanto $0 \leq \lim_{k \ra \infty} f(x_{n_k}) \leq \lim_{k\ra \infty}$ entonces $0 \leq f(x) \leq 0$.

			Finalmente $f(x) = 0$ lo que es absurdo
		\end{proof}
		
	\end{ej}

	\begin{ej}
		Sea $(X,d)$ un espacio métrico.
		\begin{enumerate}
			\item Sean $F \subset X$ un cerrado y $x \in X\setminus F$. Probar que no es cierto en general que exista un punto $y \in F$ tal que $d(x,y) = d(x,F)$. Es decir la distancia entre un punto y un cerrado puede no realizarse 
				\begin{proof}
					Tomemos como espacio $X = \{0\} \cup (1,2)$

					Ahora el $(1,2)$ es cerrado en $X$. Toda sucesión convergente de $(1,2)$ converge dentro.
					
					A lo sumo uno podría pensar que converge a $1$ o $2$, pero esos no están en el espacio métrico, entonces la sucesiones que "tienden" a $1$ o $2$ no convergen.

					Sin embargo la distancia entre $0$ y $(1,2)$ no se realiza

				\end{proof}
			\item Sean $K\subset X$ un compacto y $x \in X \setminus K$. Probar que existe $y \in K$ tal que $d(x,K) = d(x,y)$. Es decir, la distancia entre un punto y un compacto siempre se realiza.
				\begin{proof}
					Sabemos que $d(x,K)$ es un infirmo , entonces tenemos una sucesión $d(x,k_n)$ que converge al ínfimo, a $d(x,K)$.

					Además por ser $K$ compacto existe una $k_{n_j}$ convergente a $k \in K$

					Entonces tenemos $d(x,k_{n_k})$ una sub de $d(x,k_n)$ por ende converge a lo mismo.

					Como $k_{n_j}$ converge a $k$ entonces $d(x,k_{n_j})$ converge a $d(x,k)$. 

					Esto vale por que $d(x,k)$ es contínua 	

					Por lo tanto $d(x,k_{n_j})$ converge a $d(x,k)$ y también converge a $d(x,K)$

					Por lo tanto $d(x,K) = d(x,k)$ con $k \in K$. Entonces ese ínfimo es un mínimo.

					La distancia se realiza
					
				\end{proof}
			\item Probar que si $X$ tiene la propiedad de que toda bola cerrada es compacta (por ejemplo)
				\begin{proof}
					Usando el iii) sería directo
				\end{proof}
				
			\item Sean $F,K \subset X$ dos subconjuntos disjuntos de $X$ tales que $F$ es cerrado y $K$ es compacto.

				Probar que la distancia $d(F,K)$ entre $F$ y $K$ es positiva, pero puede no realizarse.
				\begin{proof}
					En el ii) hay un ejemplo de un punto (compacto) y un cerrado , tal que su distancia no se realiza y es positiva
				\end{proof}
				
			\item Sean $K_1,K_2 \subset X$ dos subconjuntos compactos de $X$ tales que $K_1 \cap K_2 = \emptyset$. Probar que existen $x_1 \in K_1$ y $x_2 \in K_2$ tales que $d(K_1,K_2) = d(x_1,x_2)$. Es decir, la distancia entre dos compactos siempre se realiza

				\begin{proof}
					En principio $d(K_1,K_2)$ es un ínfimo entonces tenemos una sucesión $d(x_n,y_n)$ convergente a $d(K_1,K_2)$

					Ahora como $K_1,K_2$ son compactos entonces $K_1 \times K_2$ es compacto también

					Por lo tanto $(x_n,y_n)_n \subset K_1\times K_2 $ tiene sub sucesión $(x_{n_k},y_{n_k})$ convergente a $(x,y) \in K_1 \times K_2$

					Dado que $d$ es contínua $d(x_{n_k},y_{n_k})$ converge a $d(x,y)$, 

					Pero además $d(x_{n_k},y_{n_k})$ converge a $d(K_1,K_2)$ por ser sub converge de $d(x_n,y_n)$

					Por lo tanto  $d(K_1,K_2) = d(x,y)$ con $x\in K_1,y \in K_2$.

					Entonces la distancia se realiza
				\end{proof}
		\end{enumerate}
	\end{ej}

	\begin{ej}
		Sea $(X,d)$ un espacio métrico completo. Se define
		$$ \mathcal{K} (X) = \{K\subset X \ / \ K \text{ es compacto y no vacío}\}$$

		\begin{enumerate}
			\item Sea $d'(A,B) = \sup_{a\in A}\{d(a,B)\}$. Verificar que, en general, d' $no$ es una métrica en $\mathcal{K}(X)$
				\begin{proof}
					Sea $A = [0,1]$ y $B = [1,100]$

					$d(A,B) = \sup_{a\in A}\{d(a,B)\} = 0$ pero $d(B,A) = 99$
				\end{proof}
			\item Se define $d: \mathcal{K} \times \mathcal{K} (X) \ra \R$ como $d(A,B) = \max\{ d'(A,B),d'(B,A))\}$. Proba que para todo $\epsilon >0$ vale
				$$ d(A,B) < \epsilon \quad \iff \quad A\subset B(B,\epsilon) \quad y\quad  B\subset B(A,\epsilon),$$

					donde $B(C,\epsilon) = \{x \in X \ / \ d(x,C) < \epsilon\}$

					\begin{proof}
						$d(A,B) < \epsilon \iff d'(A,B) < \epsilon \ \  y \ \ d'(B,A) < \epsilon$

							$\iff d(a,B) < \epsilon \quad \forall a\in A \ \ y \ \ d(b,A)<\epsilon \quad \forall b \in B $

								$\iff a \in B(B,\epsilon) \quad \forall a \in A \ \ y \ \  b \in B(A,\epsilon) \quad \forall b\in B$ 

									$\iff A \subset B(B,\epsilon) \ \ y \ \ B \subset B(A,\epsilon)$
					\end{proof}
				\item Probar que $d$ es una métrica en $\mathcal{K} (X)$

					\begin{enumerate}
						\item $d(A,A) = 0 $ es trivial
						\item $d(A,B) = \max\{d'(A,B),d'(B,A)\} = \max\{d'(B,A),d'(A,B)\} = d(B,A)$
						\item Sin pérdida de generalidades supongamos $d(A,C) = d'(A,C) = \sup_{a\in A}\{d(a,C)\}$
							$$d(a,c) < d(a,b) + d(b,c) \leq \iff \inf_{c \in C}\{d(a,c)\} \leq d(a,b) + d(b,c) \iff d(a,C) \leq d(a,b) + d(b,c)$$

								$$ d(a,C) \leq \inf_{b \in B}\{d(a,b)\} + d(b,c)\iff d(a,C) \leq d(a,B) + d(b,c)$$

									$$ d(a,C ) \leq d(a,B) + \inf_{c\in C}\{d(b,c)\} \iff d(a,C) \leq d(a,B) + d(b,C) $$
										Y haciendo lo mismo pero con supremos nos queda:

										$$ d(A,C) \leq d(A,B) + d(B,C)$$
							Luego encadenando supremos llegamos a que $\sup_{a\in A} d(a,b) + d$ 

							Si hubiese sido $d'(C,A)$ arrancabamos con $d(c,a)$ y concluíamos lo mismo
					\end{enumerate}
		\end{enumerate}
	\end{ej}
	
	\begin{ej}
		Dado un cubrimiento por abiertos $(U_i)_{i \in I}$ de un espacio métrico $(X,d)$ un número $\epsilon >0$ se llama numero de Lebesgue de $(U_i)_{i \in I}$ si para todo $x \in X$ existe $j \in I$ tal que $B(x,\epsilon) \subset U_j$. Probar que todo cubrimiento por abiertos de un espacio métrico compacto tiene un número de Lebesgue.
		\begin{proof}
			Como $\bigcup_{i \in I} U_i$ es un cubrimiento por abiertos.

			Entonces para cada $x\in X$ tenemos un $U_i^x \subseteq \bigcup U_i$ tal que $x \in U_i^x $

			Como $U_i^x$ es abierto tenemos $r(x) >0$ tal que $B(x,2r(x)) \subset U_i^x$

			Ahora la unión de todas esas bolas es cubrimiento por abierto, y por ser $X$ compacto tiene subcubrimiento finito

			Por lo tanto tenemos un $\epsilon = \min \{r(x) : x\in X\}$.

			$\epsilon > 0 $ por que es el mínimo de cosas mayores que cero.

			Dado $y \in X$ tenemos que existe algún $x \in X$ tal que $y \in B(x,r(x)) $ por ser cubrimiento

			Ahora sabemos que $B(y,\delta) \subset B(y,r(x)) \subseteq B(x,2r(x)) \subseteq U_i^x$

			Veamos la inclución del medio. Sea $z \in B(y,r(x))$. 

			$$ d(z,x) \leq d(z,y)  + d(y,x) \leq r(x) + r(x) = 2r(x) \Ra z \in B(x,2r(x))$$

			Finalmente dado encontramos $\delta >0$ tal que dado cualquier $y \in X$ sucede $B(y,\delta) \subseteq U_i$ para algún abierto del cubrimiento

		\end{proof}
	\end{ej}
	
	\begin{ej}
		Sea $(X,d)$ un espacio métrico. Se dice que una familia $(F_i)_{i\in I}$ de subconjuntos de $X$ tiene la propiedad de intersección finita (P.I.F) si cualquier subfamilia finita de $(F_i)_{i \in I}$ tiene intersección no vacía.
		Probar que los siguientes enunciados son equivalentes:
		\begin{enumerate}
			\item $X$ es compacto.
				\begin{proof}
					asd
				\end{proof}
			\item Toda familia $(F_i)_{i \in I}$ de subconjuntos cerrados de $X$ con la P.I.F tiene intersección no vacía.
			\item Todo subconjunto infinito de $X$ tiene un punto de acumulación en $X$.
			\item Toda sucesión en $X$ tiene una subsucesión convergente.
			\item $X$ es completo y totalmente acotado.
		\end{enumerate}
	\end{ej}
	
	\begin{ej}
		Sean $(X,d)$ e $(Y,d')$ espacios métricos y $f: X \ra Y$ contínua y biyectiva. Probar que si $(X,d)$ es compacto, entonces $f$ es un homemorfismo
		\begin{proof}
			Podemos mirar $f^{-1}$ por la biyectividad nos asegura que es una función bien definida. 

 			Veamos que es contínua. Sea $(f(x_n))_n \subset Y$  convergente a $f(x_0)$

			Ahora miramos $f^{-1}(f(x_n)) = (x_n)_n \subseteq X$. 

			Dada cualquier sub $x_{n_k}$ por compacidad tenemos una subsub $x_{n_{k_j}}$ convergente a un $x_1$ 

			Veamos que $x_1 = x_0$

			Supongamos que son diferentes. Entonces $(f(x_{n_{k_j}}))_j$ converge a $f(x_1)$ que es diferente a $f(x_0)$ por inyectividad de $f$.

			Pero esto es absurdo por que $(f(x_{n_{k_j}}))_k$ es un subsucesión de $(f(x_n))_n$ que convergía a $f(x_0)$

			Entonces $x_1 = x_0$. Lo que muestra que $x_{n_{k_j}}$ converge a $x_0$

			Y esto lo podemos hacer con cualquier subsucesión de $x_n$. Por lo tanto $x_n$ converge a $x_0$

			Entonces $f^{-1}$ es contínua. Por lo tanto $f$ es homeomorfismo

		\end{proof}
	\end{ej}
	
	\begin{ej}
		Sea $(X,d)$ un espacio métrico compacto. Probar que para cada espacio métrico $(Y,d')$, la proyección $\pi :X \times Y \ra Y$ definida por $\pi (x,y) = y$ es cerrada.
		\begin{proof}
			Sea $F \subseteq X \times Y$ un cerrado. Queremos ver que $\pi (F)$ es cerrado

			Tomemos $(y_n)_n \subseteq \pi(F)$ convergente a $y$, queremos ver que $y \in \pi (F)$

			Para cada $y_n$ tengo un $x_n \in X$ tal que $\pi (x_n,y_n) = y_n$ y $(x_n,y_n) \in F$

			Ahora como $(x_n)_n \subseteq X$ que es compacto, existe $(x_{n_k})_k$ que converge a un $x \in X$

			Y como $y_{n_k}$ es sub de $y_n$ sigue convergiendo a $y$

			Por lo tanto $(x_{n_k},y_{n_k}) = (x_{n_k},\pi(x_{n_k},y_{n_k}))$ converge a $(x,y)$

			Pero esta sub sub está contenida en $F$ que es cerrado. Entonces $(x,y) \in F$

			Por lo tanto $\pi (x,y) = y \in \pi (F)$

		\end{proof}
		
	\end{ej}
	
	\begin{ej}
		Sean $(X,d)$ e $(Y,d)$ espacios métricos y sea $f : X \ra Y$ una función. Probar que si $Y$ es compacto y el gráfico de $f$ es cerrado en $(X\times Y, d_{\infty})$, entonces $f$ es contínua. Comparar con el ejercicio 15 de la práctica 3.
		\begin{proof}
			Tomemos $x_n$ convergente a $x$.  Luego mirmoes cualquier sub $f(x_n)_n$

			Como $(f(x_{n_k}))_k$ está en un compacto , tiene un sub sub $f(x_{n_{k_j}})$ convergentea $y$

			Además $(x_{n_{k_j}})_j$ sigue convergiendo a $x$ por lo tanto $(x_{n_{k_j}},f(x_{n_{k_j}})) \subseteq Gr(f)$ coverge a $(x,y)$

			Como $Gr(f)$ es cerrado $(x,y) \in Gr(f)$ por lo tanto $y = f(x)$

			Mostrando que $f(x_{n_{k_j}})$ converge a $f(x)$

			Ahora esto lo podemos hacer con cualquier sub de $f(x_n)$ y siempre va a tener las sub sub van a converge $(x,f(x))$

			Por lo tanto $f(x_n)$ converge a $f(x)$

			Mostrando que $f$ es contínua.

		\end{proof}
	\end{ej}
	
	\begin{ej}.
		\begin{enumerate}
			\item Sea $f : \R_{\geq a} \ra \R$ una función que es uniformemente contínua en $[a,b]$ y también en $[b,+\infty )$. Probar que $f$ es uniformemente contínua en $\R_{\geq a}$
				\begin{proof}
					Dado $\frac{\epsilon}{2} >0$ tenemos que existe $\delta_1$ tal que $d(x,y) < \delta_1 \Ra d(f(x),f(y))< \epsilon \quad \forall x,y \in [a,b]$

					Y tenemos un $\delta_2$ que sirve para $x,y \in [b,+\infty]$

					Ahora si tomamos $\delta = \min\{\delta_1,\delta_2\}$ sabemos que nos sirve para los dos

					Veamos que pasa cuando tenemos un $x \in [a,b]$ y un $y \in [b,+\infty]$

					$$d(x,y) < d(x,b) + d(b,y) < \delta$$ 

					Luego tanto $d(x,b)$ como $d(b,y)$ seguro son menores que sus deltas
					
					Entonces tenemos por un lado $d(f(x),f(b)) < \frac{\epsilon}{2}$ y por otro $d(f(b),f(y)) < \frac{\epsilon}{2}$

					Entonces $$ d(f(x),f(y)) < d(f(x),f(b)) + d(f(b),f(y)) \leq \frac{\epsilon}{2} + \frac{\epsilon}{2} = \epsilon$$
	
					Mostrando que $f$ es uniformemente contínua para cualquier par $x \in [a,b], \ y \in [b,+\infty)$  

					Concluyendo que $f$ es uniformemente contínua en $[a,+\infty) = \R_{\geq a}$
				\end{proof}
				
							
				
			\item Deducir que $\sqrt{x}$ es uniformemente contínua en $\R_{\geq 0}$
					
				\begin{proof}
					Dado $\epsilon$ tomamos $\delta = \epsilon^2$

					$$ |\sqrt{x} - \sqrt{y}|^2 < |\sqrt{x} -\sqrt{y}|.|\sqrt{x} + \sqrt{y}| =  |\sqrt{x}^2 - \sqrt{y}^2| = |x -  y| < \delta = \epsilon^2$$

					Como la raíz cuadrada es una función estrictamente creciente 

					$$|\sqrt{x} - \sqrt{y}| < \epsilon^2 \iff |\sqrt{x} - \sqrt{y}| < \epsilon$$

					Entonces:

					$$ |x+y | < \delta \Ra |\sqrt{x} + \sqrt{y}| < \epsilon$$
				\end{proof}

			\item Sea $f : \R \ra \R$ contínua y tal que $\lim_{x \ra - \infty} f(x) = \lim_{x \ra + \infty} f(x) = 0$. Probar que $f$ es uniformemente contínua en $\R$
				\begin{proof}
					Dado $\epsilon$ sabemos que existe $\alpha_1$ tal que $$d(f(x),0) < \frac{\epsilon}{2} \quad \forall x \in (-\infty,\alpha] $$

					Entonces dados $x,y \in (-\infty,\alpha] $ tenemos $d(f(x),f(y)) < d(f(x),0) + d(0,f(y)) = \epsilon$

					Por lo tanto tomando cualquier $\delta$ tenemos: 
					$$d(x,y) < \delta \Ra d(f(x),f(y)) < \epsilon \quad \forall x,y \in (-\infty,\alpha] $$

					Por lo tanto $f$ es uniformemente contínua en $(-\infty,\alpha]$

					Lo mismo pasa con un $\beta$ y con $x,y \in [\beta,+\infty)$ y $d(f(x),f(y))$

					Entonces $f$ es uniformemente contínua en $[\beta,+\infty)$

					Además $f$ es unfiormemente contínua en $[\alpha,\beta]$ por ser un compacto y $f$ contínua 

					Ahora usando $[\beta,+\infty)$ y $[\alpha,\beta]$ caémos en las hipótesis de la primera parte

					Por lo tanto $f$ es contínua en $[\alpha,+\infty)$
				\end{proof}
				
				
		\end{enumerate}
	\end{ej}
	
	\begin{ej}
		Sea $f : \R \ra \R$ una función contínua y abierta.
		\begin{enumerate}
			\item Probar que $f$ no tiene extremos locales; es decir, no existen $x_0 \in \R$ y $\epsilon > 0$ tales que $f(x_0) \leq f(x)$ (resp. $f(x_0) \geq f(x)$) para todo $x \in (x_0 - \epsilon,x_0 + \epsilon)$

			\item Comprobar que existen $a,b \in \R \cup \{-\infty,+\infty\}$ tales que $f(\R) = (a,b)$
			\item Mostrar que $f: \R \ra (a,b)$ es un homeomorfismo y que ella y su inversa son funciones monótonas
		\end{enumerate}
	\end{ej}
	
	
	
		
\end{document}
