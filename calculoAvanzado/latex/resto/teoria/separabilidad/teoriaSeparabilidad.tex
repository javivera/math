\documentclass[12pt]{article}

\usepackage[margin=1in]{geometry}
\usepackage{enumerate}
\usepackage{amsmath}
\usepackage{amssymb}
\usepackage{mathtools}
\usepackage{amsfonts}
\usepackage{amsthm}
\usepackage{graphicx}
\usepackage{fancyhdr}
\pagestyle{fancy}


\newcommand{\Z}{\mathbb{Z}}
\newcommand{\n}{\aleph_{0}}
\newcommand{\F}{\mathbb{F}}
\newcommand{\Q}{\mathbb{Q}}
\newcommand{\C}{\mathbb{C}}
\newcommand{\R}{\mathbb{R}}
\newcommand{\K}{\mathbb{K}}
\newcommand{\E}{\mathbb{E}}
\newcommand{\I}{\mathbb{I}}
\newcommand{\N}{\mathbb{N}}
\newcommand{\Ra}{\Rightarrow}
\newcommand{\ra}{\rightarrow}
\newcommand{\ol}{\overline}
\newcommand{\norm}[1]{\left\lVert#1\right\rVert}
\newcommand{\open}{\mathrm{o}}



\theoremstyle{definition}
\newtheorem{definition}{Definición}[section]
\newtheorem*{remark}{Observación}
\newtheorem{theorem}{Teorema}
\newtheorem{lemm}{Lema}
\newtheorem{corollary}{Corolario}[theorem]
\newtheorem{lemma}[theorem]{Lema}
\newtheorem{prop}{Proposición}
\newtheorem{ex}{Ejemplo}
\newtheorem{ej}{Ejercicio}


\fancyhead[R]{Separabilidad}
\fancyhead[L]{Javier Vera}
\fancyhead[C]{Cálculo Avanzado}
\begin{document}

Separabilidad 

\newpage
\begin{definition}
  Sea $(\E,d)$ un espacio métrico. Un subconjunto $D \subseteq \E $ se dice denso (en $\E$) si $\ol D = \E$ 
\end{definition}
\begin{ex}
  $\ol \Q = \R$ Entonces $\Q$ es denso en $\R$
  $\ol (0,1) = [0,1]$ entonces $(0,1)$ es denso en $[0,1]$
\end{ex}
\begin{remark}
  $D$ denso en $\E$ y tenemos  $f: D \ra \E '$ continua no necesariamente existe $f' : \E \ra \E '$ continua

  Lo que significa que tener una funcion continua en un denso de $B$ no necesariamente podemos extenderla a $B$ y seguir teniendo continuidad.

  Contraejemplo $f:(0,1) \ra \R$ dada por $f(x) = \frac{1}{x}$

  Pero si en cambio tenemos una función uniformemente continua SI podemos
\end{remark}
\begin{definition}
  Una función $f: \E \ra \E '$ se llama homemorfismo si es biyectiva, continua y su inversa es continua  

  Dos espacios $(\E,d)$ y $(\E ' , d')$ se dicen homemorfos si existe un homeomorfismo   $f: \E \ra \E '$ homemorfismo
\end{definition}
\begin{remark}
  Si $\E$ y $\E '$ son espacios homemorfos, entonces hay una correspondencia entre los abiertos de $\E$ y los de $\E '$ entonces topológicamente son lo mismo

  Sea $A$ abierto de $\E$ , $f(A) = (f^{-1})^{-1}(A)$ que es abierto en $\E '$ esto vale por que $f^{-1} $ es continua, por que $f$ es homemorfismo y preimagen de abierto es abierto

  Y por otro lado si $f(A)$ abierto entonces $A = f^{-1}(f(A))$ es abierto de $E$

  Entonces $$A \text{ abierto de }\E  \iff f(A) \text{ es abierto de }\E' $$ 
\end{remark}
\begin{remark}
  Dada $f$ biyectiva, puede suceder que $f$ esa continua pero su inversa no lo sea

  $id: (\R,\delta) \ra (\R, | \cdot |)$ es continua pero su inversa $id^{-1}: (\R, |\cdot|) \ra (\R ,\delta)$ no es continua

Es mas si la identidad entre dos espacios es un homemorfismo entonces sus métricas son equivalentes
\end{remark}
\begin{definition}
  Si $f: \E \ra \E '$ es biyectiva y $d(x,y) = d'(f(x),f(y))$, diremos que $f$ es una isomemtría. 
\end{definition}
\begin{remark}
  Si $f : \E \ra \E '$ una isomemtría, entonces tanto $f$ como $f^{-1}$ son uniformemente continuas. En particular $f$ es homemorfismo
  \begin{proof}
    Como $f$ isomorfismo $d(x,y) = d'(f(x),f(y)) \quad x,y \in \E$

    $w,z \in E ' \quad $ $d'(w,z) = d'(f(f^{-1}(w)),f(f^{-1}(z))) = d(f^{-1}(w), f^{-1}(z))$ 
  
Entonces dado $\epsilon$ tomamos $\delta = \epsilon$ y nos sirve para todo el dominio
  \end{proof}
\end{remark}
\begin{definition}[Una definicion equivalente de densidad]

  $$ D \text{ denso en } \E \iff \quad \forall x \in \E, \quad \forall r>0 ,\quad  B(x,r) \cap D \neq \emptyset$$ 
\end{definition}
\begin{remark}
  $\E$ es denso en $\E$

  Sea $(\E,\delta)$, $\ol A = A \quad \forall A \subseteq \E$ entonces $D$ denso en $\E \iff D = \E$ 
\end{remark}
\begin{remark}[Ejemplos mas interesantes]
  $$\ell^1 = \{(a_n)_{n \in \N} / \sum_{n=1}^{\infty}|a_n| < \infty\} \quad d_1(a,b) = \sum_{n = 1}^{\infty}|a_n - b_n|$$

  $$\ell^{\infty} = \{(a_n)_{n \in \N} / \sup_{n \in \N}|a_n| < \infty\} \quad d_{\infty}(a,b) = \sup_{n \in \N} |a_n - b_n|$$

  Sea $D = \{(a_n)_n / \exists m \text{ tal que } a_n = 0 \quad \forall n \geq m\}$ entonces $D$ es denso en $\ell^1$

Sea $a \in \ell^1 , r>0$ entonces $\sum_{n=1}^{\infty}|a_n| \leq \infty \Ra \exists n_0 \in \N  / \sum_{n \geq n_0}^{\infty} |a_n| \leq r$

Sea $b = (a_1,a_2, \dots , a_{n_0},0,0, \dots)$ entonces $b \in D$. Luego $d_1(a,b) = \sum_{n=1}^{\infty} |a_n - b_n|$ 

Cada uno de estos módulos es 0 o $a_n$ entonces $\sum_{n=1}^{\infty} |a_n - b_n|=   \sum_{n\geq n_0}^{\infty} |a_n| \leq r$

Finalmente $d_1(a,b) \leq r$

Entonces para cada $a \in \ell^1$ y para cada $\epsilon > 0$ existe $b \in D$ tal que $d(a,b) \leq \epsilon$

O lo equivalente dado un $a \in A$ $\forall \epsilon >0 \quad \exists b\in D$ tal que  $ b \in B(a,\epsilon) \cap D$ 

Por lo que $B(a,\epsilon) \cap D \neq \emptyset$ entonces $D$ es denso en $\ell^1$

Ejercicio para el lectór probar que no es denso en $\ell^{\infty}$
\end{remark}
\begin{definition}
 Un espacio métrico $\E$ se llama separable si contiene un subconjunto $D$ numerable y denso 
\end{definition}
\begin{remark}
  $\R$ y $\R^n$ son separables

  $(\E,\delta)$ es el único denso es $\E$ entonces 

  $$\E  \text{ separable} \iff \E \text{ es a lo sumo numerable}$$
\end{remark}
\begin{ex}
  $\ell^1$ es separable, el $D$ que usamos arriba no nos sirve , por que NO es numerable a pesar de ser denso

  Pero sea $A_{\Q}=\{(a_n)_n \subseteq \Q :\exists m\in \N / a_n = 0 \quad \forall n \geq m\}$

  Probar que es denso en $\ell^1$ y que es numerable
\end{ex}
\begin{definition}
  Sea $(\E,d)$ un espacio métrico. Una familia $\mathcal{B}$ de abiertos se dice una base de abiertos de $\E$ si todo abierto de $\E$ se escribe como unión de conjuntos de la familia $\mathcal{B}$ 
\end{definition}
\begin{remark}
  La familia $$ \mathcal{B}_0 = \{B(x,r) : x \in \E, r>0\}$$

  es un base de abiertos de $\E$
  \begin{proof}
    $G$ abierto de $\E$, dado $x \in G$, $\exists r_x >0 $ tal que $x \in B(x,r_x) \subseteq G$ 

    Ahora afirmo $G \subseteq \bigcup_{x \in G}B(x,r_x) \subseteq G$

    Entonces $G = \bigcup_{x \in G}B(x,r_x)$

    En particular los intervalos abiertos forman una base de abiertos de $\R$
  \end{proof}
\end{remark}
\begin{theorem}
  Sea $(\E,d)$ un espacio métrico. Una familia (de abiertos) $\mathcal{B}$ es una base de abiertos de $\E$ si y sólo si para todo abierto $U \subseteq \E$ y todo $x \in U,$ existe $A \in \mathcal{B}$ tal que $x \in A \subseteq U$
  \begin{proof}
  $\Ra )$ $U$ abierto, $x \in U$, tenemos que $U = \bigcup_{i \in I} A_i$ con $A_i \in \mathcal{B}$

  $x \in U$ entonces $\exists i_0 $ tal que $x \in A_{i_0}\subseteq U$ y además $A_{i_0} \in \mathcal B$

$\Leftarrow )$ $G$ abierto de $\E$, $\forall x \in G$, $\exists A_x \in \mathcal{B}$ tal que $x \in A_x \subseteq G$

Entonces $G = \bigcup_{x\in G}A_x$. Entonces dado un abierto lo escribimos como unión de abiertos de la base
  \end{proof}
\end{theorem}

\begin{ex}
  Probar que $\mathcal{B}$ es una base de abiertos si y sólo si para todo $x \in \E$ y todo entorno $V$ de $x$ existe $A \in \mathcal{B}$ tal que $x \in A \subseteq V.$
  \begin{proof}
  $\Ra )$ Sea $\mathcal{B}$ una base de abiertos y $V$ un entorno de $x$

    Como $V$ entorno existe $U$ abierto tal que $x \in U \subseteq V$

    Dado que $U$ abierto $x \in U = \bigcup_{i \in I} A_i$ con $A_i \in \mathcal B$

    Pero entonces $x \in A_i \subseteq V$ para algún $i \in I$

  $\Leftarrow )$ Sea $x \in \E$ sabemos que existe un $A \in \mathcal{B}$ base tal que $x \in A$
  
  Pero entonces $\E = \bigcup A_i$ con $A_i \in B$
  \end{proof}
\end{ex}
\begin{remark}
  Si $\E = \R^n$ con la métrica euclídea, entonces la familia
$$ \mathcal{B}_1 = \{B(x,q): x \in \Q^n, q \in \Q, q >0\}$$

es una base de abiertos, y es numerable

en particular los intervalos de extremos racionales forman una base de abiertos de $\R$
\end{remark}
\begin{remark}
  $R^n$ es separable , tiene base numerable de abiertos, cumple Lindelof 
\end{remark}
\begin{theorem}
  Sea $(\E,d)$ un espacio métrico. Las siguientes afirmaciones son equivalentes:
  \begin{enumerate}
    \item $(\E,d)$ es separable.
    \item $\E$ tiene una base numerable de conjuntos abiertos
    \item $\E$ tiene la propiedad de Lindelof: todo cubrimiento por abiertos contiene un subcubrimiento (a lo sumo) numerable
  \end{enumerate}
  \begin{proof}
$i) \Ra ii)$ Sea $D$ denso numerable y $\mathcal{B} = \{B(a,q): a \in D, q \in \Q , q>0\}$

$\mathcal{B} \sim D \times \Q \sim \N \times \N$ entonces $\mathcal{B}$ es numerable

Veamos que $\mathcal{B}$ es base: $G \subseteq \E$ abierto, $x \in G$ existe $r>0$ tal que $B(x,r) \subseteq G$ 

Como $D$ es denso $B(x,\frac{r}{2}) \cap D \neq \emptyset$ entonces $\exists a \in B(x,\frac{r}{2}) \cap D$ Luego $d(a,x) < \frac{r}{2}$. 

Sea $q \in \Q$ tal que $d(x,a) < q < \frac{r}{2}$, Ahora tenemos $d(x,a) < q$ entonces $x \in B(a,q)$

Ahora sea $y \in B(a,q)$ entonces $d(y,x) \leq d(y,a) + d(a,x) < q + \frac{r}{2} < r$

Entonces $y \in B(x,r)$ entonces $B(a,q) \subseteq B(x,r) \subseteq G$

Juntando todo tenemos que para cualquier $x \in G$ encontramos $B(a,q) \in \mathcal{B}$ tal que $x \in B(a,q) \subseteq G$ con $B(a,q) \in \mathcal{B}$ y esto vale para cualquier $G$ abierto.

Entonces $\mathcal{B}$ es base y es numerable

$ii) \Ra iii)$ Sea $\mathcal{C}$ cubrimiento por abiertos de $S \subseteq \E$ y sea $\mathcal{B}$ base numerable de $\E$

Definamos $\mathcal{B}_{\mathcal{C}} = \{A \in \mathcal{B} : \exists G \in \mathcal{C} \text{ tal que } A \subseteq G\}$

$\mathcal{B}_{\mathcal{C}}$ es a lo sumo numerable por ser un subconjunto de $\mathcal{B}$

Dado $A \in \mathcal{B}_{\mathcal{C}}$ existe algún $G \in \mathcal{C}$ (llamémoslo $G_A$) tal que $A \subseteq G_{A}$

Y definamos $\mathcal{C}_0 = \{G_A : A \in \mathcal{B}_{\mathcal{C}}\}$ este es a lo sumo numerable por que su cardinal es menor o igual que el de $\mathcal{B}_{\mathcal{C}}$ (podría ser que un mismo $G_A$ contenga mas de un $A$)

Ahora $$\bigcup_{A \in \mathcal{B}_{\mathcal{C}}}A \subseteq \bigcup_{G \in \mathcal{C}_0} G \subseteq \bigcup_{G \in \mathcal{C}} G  $$

Veamos que $$\bigcup_{G \in \mathcal{C}} G \subseteq  \bigcup_{A \in \mathcal{B}_{\mathcal{C}} } G$$

Sea $x \in \bigcup_{G \in \mathcal{C}} G$ entonces $ x \in G$ para algún $G \in \mathcal{C}$ entonces como $\mathcal{B}$ es base sabemos existe $A \in \mathcal{B}$ tal que $x \in A \subseteq G$ ($A \in \mathcal{B}_{\mathcal{C}}$ pues $A \subseteq G$)

Demostrado esto y usando las inclusiones de arriba tenemos 
$$ \bigcup_{G \in \mathcal{C}_0} G = \bigcup_{G \in \mathcal{C}} G$$

Entonces $\bigcup_{G \in \mathcal{C}_0} G$ es subcubrimiento y además sabemos que es numerable por que $\mathcal{C}_0$ es numerable

$iii) \Ra i)$ Dado $n \in \N$ tenemos $\E \subseteq \bigcup_{x \in \E} B(x, \frac{1}{n})$ (esto en realidad es una igualdad pero la inclusión deja más claro que son cubrimientos). Por Lindeloff existe subcubrimiento a lo sumo numerable 

Ahora las bolas son de radio fijo , entonces podemos quedarnos con numerables $x \in \E$

Entonces para cada $n \in \N$ $\exists (x_k^n)_{k \in \N}$ tal que $\E \subseteq \bigcup_{k \in \N} B(x_k^n,\frac{1}{n})$(Una vez fijado el $n \in \N$ nos armamos la sucesion de $x \in \E$ que nos dado el cubrimiento con las bolas de esos centros y radio $\frac{1}{n}$ esta notación representa todo eso con cualquier $n \in \N$)

$D = \{x_k^n : n \in \N, k \in \N \}$ Esto es a lo sumo numerable 

$ \N \times \N \sim D $ con $f(x,k) = x_k^n$ que es sobreyectiva

Densidad, sea $x \in \E$, dado $r>0$ tomamos $n_0$ tal que $\frac{1}{n_0} < r$ entoncs $\E \subseteq \bigcup_{k \in \N}B(x_k^{n_0},\frac{1}{n_0})$

Como $x \in \E$ entonces $\exists k_0$ tal que $x \in B(x_{k_0}^{n_0},\frac{1}{n_0})$ entonces $d(x,x_{k_0}^{n_0}) < \frac{1}{n_0} < r$

$x_{k_0}^{n_0} \in B(x,r) $ y además $x_{k_0}^{n_0} \in D$ por lo tanto $B(x,r) \cap D \neq \emptyset$

Encontramos $D$ es denso y numerable de $\E$ por lo tanto $\E$ es separable

\end{proof}
\end{theorem}

\begin{definition}
  Sea $(\E,d)$ un espacio métrico y $x \subseteq \E$ un subespacio métrico. Es decir, consideramos en $X$ la métrica $d_X$ inducida: dados $x,y \in X$
  $$ d_X (x,y) = d(x,y)$$

  Entonces, para $x \in X$ u $r >0$, la bola en $X$ o relativa a $X$ de centro $x$ y radio $r$ es 

  $$ B_X(x,r) = \{y \in X : d_X(x,y) < r\} = B(x,r) \cap X$$
\end{definition}

\begin{definition}
  Sea $X$ subespacio métrico de $\E$. Decimos que $A \subseteq X$ es abierto relativo de $X$ o abierto en $X$ si para todo $x \in A$ existe $r>0$ tal que $B_X(x,r )\subseteq A$
\end{definition}
\newpage
\begin{prop}
  Sea $X$ subespacio métrico de $\E$ y $A \subseteq X$. Son equivalentes:
  \begin{enumerate}
    \item $A$ es abierto relativo de $X$
    \item Existe $U\subseteq \E$ abierto (en $\E$) tal que $A = U \cap X$
  \end{enumerate}
  \begin{proof}
$i) \Ra ii)$ Sea $x \in A$ entonces $x \in U$ entonces $\exists r >0$ tal que $B(x,r) \subseteq U$

Entonces $B(x,r) \cap X \subseteq U \cap X = A$ por lo tanto $B_X(x,r) \subseteq A$

$i) \Ra ii)$ para cada $x \in A$ existe $r_x >0$ tal que $B_X(x,r_x) \subseteq A$

Entonces $A = \bigcup_{x \in A} B_X(x,r_x)$ pero entonces tomemos $U =\bigcup_{x \in A} B(x,r_x)$ que es abierto por ser unión de abiertos. 

Y ahora tenemos $A = U \cap X$ con $U$ abierto
  \end{proof}
\end{prop}
\begin{definition}
  Sea $X$ subespacio métrico de $\E$. Decimos que $B \subseteq X$ es cerrado relativo de $X$ o cerrado en $X$ si $X \setminus B$ es abierto relativo de $X$ 
\end{definition}
\begin{prop}
  Sea $X$ subespacio métrico de $\E$ y $B \subseteq X$. Son equivalentes:
  \begin{enumerate}
    \item $B$ es cerrado relativo de $X$
    \item Existe $F\subseteq \E$ cerrado (en $\E$) tal que $B = F \cap X$
  \end{enumerate}
  \begin{proof}
    
  \end{proof}
\end{prop}

\begin{ej}
  Para pensar: dar una definición de clausura relativa a $X$ a partir de bolas y ver cómo se relaciona con la definición de cerrado relativo.
  \begin{proof}
    
  \end{proof}
\end{ej}

\newpage
Completitud
\newpage

\fancyhead[R]{Completitud}
\fancyhead[L]{Javier Vera}
 \fancyhead[C]{Cálculo Avanzado}

\begin{definition}
  Una sucesión $(x_n)_n$ se dice de Cauchy si para todo $\epsilon > 0$ existe $n_0 \in \N$ (que depende de $\epsilon$) tal que si $n,m \geq n_0$ entonces $d(x_n,x_m) \leq \epsilon$ 
\end{definition}

\begin{theorem}
  Sea $(\E,d)$ un e.m y $(x_n)_n \subseteq \E$
  \begin{enumerate}
    \item Si $(x_n)_n$ es de Cauchy, entonces el conjunto $\{x_n : n \in \N \}$ es acotado.
      \item Si $(x_n)_n$ es de Cauchy y contiene alguna subsucesión convergente entonces $x_n$ es convergente
      \item Si $(x_n)_n$ es convergente, entonces es de Cauchy
  \end{enumerate}
  \begin{proof}
Ya probamos la $1)$ y la $3)$
  $2)$ Sea $(x_{n_k})_k$ subsucesión convergente a $ x \in \E$.

  Sea $\epsilon >0 \quad \exists k_1 / d(x_{n_k},x) < \frac{\epsilon}{2} \quad \forall k \geq k_1$

  $x_n$ es de Cauchy, entonces $\exists n_1 / d(x_n,x_m) < \frac{\epsilon}{2} \quad \forall n,m \geq n_0$

  Ahora tomemoes $n \geq n_0$ y sea $k \geq k_1 / n_{k} \geq n_0$

  Con esto tenemos $d(x_n,x) \leq d(x_n,x_{n_{k}}) + d(x_{n_{k}},x) < \epsilon$
  \end{proof}
\end{theorem}
\begin{definition}
  Un espacio métrico $(\E,d)$ se dice completo si toda sucesión de Cauchy es convergente a un punto $x \in \E$ 
\end{definition}
\begin{ex}
  $\R$ es completo y también $\R^n$ se vió en taller

  Idea: Si $x_n$ es una sucesión de Cauchy, entonces por el teorema de arriba es acotada. Ahora sabemos que una sucesión acotada tiene una subsucesión convergente, entonces usando nuevamente el teorema , $x_n$ tiene que converger. Luego $\R$ es completo por que toda sucesión de Cauchy converge en $\R$ 
\end{ex}
\begin{ej}
  Sea $\E$ un conjunto no vacío con la métrica discreta, es ¿$(\E,\delta) $ completo ?
\end{ej}

\begin{ex}
  El espacio $C([0,1])$ es completo con $d_{\infty}(x,y) = \max_{t \in [0,1]} |x(t)-y(t)|$
  \begin{proof}
    Tomemos una sucesión de Cauchy $(x_n)_n \subseteq C([0,1])$ y $x_n: [0,1] \ra \R$.

    Ahora dado $\epsilon >0$, $\exists n_0 \in \N /\max_{t \in [0,1]} |x_n(t) - x_m(t)| < \epsilon \quad \forall n,m \geq n_0 $

    Ahora si fijo $t$ tenemos que $(x_n(t))_n \subseteq \R$ es una sucesión de Cauchy de $\R$ entonces como $\R$ es completo $\lim_{n \ra \infty} x_n(t) = x(t)$

    Ahora sabemos dado que $x_n$ es de Cauchy , usando la definición podemos afirmar que si el máximo para cualquier $t \in [0,1]$ es menor que $\epsilon$ entonces

    $|x_n(t) - x_m(t)| < \frac{\epsilon}{2} \quad \forall n,m \geq n_0 \quad \forall t \in [0,1]$

    Ahora si hacemos que $m \ra \infty$ llegamos a $|x_n(t) - x(t)| \leq \frac{\epsilon}{2} < \epsilon \quad \forall n \geq n_0 \quad \forall t \in [0,1]$

    Entonces $\max_{t \in [0,1]} |x_n(t) - x(t)| < \epsilon \quad \forall n \geq n_0$

    Luego $x_n$ converge uniformemente a $x$ en $[0,1]$

    Sabemos que entonces $x$ es continua entonces $x \in C([0,1])$

    Pero además $d_{\infty} (x_n,x) \ra 0$ entonces $x_n$ converge
  \end{proof}
\end{ex}
\begin{ex}
  El espacio $C([0,2])$ no es completo con

  $$ d_1(x,y) = \int_{0}^2 |x(t) - y(t)| dt$$
  \begin{proof}
    Miremos 
$$
x_n(t) = \left\{
        \begin{array}{ll}
            t^n & \quad 0 \leq t < 1 \\
            1 & \quad 1 \leq t \leq 2
        \end{array}
    \right.
$$  

Ahora $x_n$ converge a $x$ dada por 
$$
x(t) = \left\{
        \begin{array}{ll}
            0 & \quad 0 \leq t < 1 \\
            1 & \quad 1 \leq t \leq 2
        \end{array}
    \right.
$$
Pero esta función NO es continua, entonces no está en el espacio , pero entonces $x_n$ no convergia , por que no se puede converge a algo que no está en el espacio
\end{proof}
\end{ex}
\begin{definition}
  Dado $A \subseteq \E$, con $A \neq \emptyset$, se define su diámetro

  $$ \delta(A) = \sup \{d(x,y) : x \in A , y \in A\}$$
\end{definition}
\begin{remark}
  El diámetro no necesariamente es el doble del radio , por ejemplo sea $(\E,\delta)$ la bola $B(0,\frac{1}{2}) = \{0\}$

  Entonces $\delta(B(0,\frac{1}{2})) = 0$
\end{remark}
\begin{ej}
  $\delta(A) = \delta (\ol A)$
  \begin{proof}
    Una inclusión es fácil , la vuelta un poco mas dificíl
  \end{proof}
\end{ej}
\begin{theorem}[Principio de encaje de Cantor ]
  Sea $(\E,d)$ un e.m completo y una sucesión decreciente de conjuntos no vacíos $A_1 \supset A_2 \supset A_3 \supset \dots ,$ tales que $\delta(A_j) \ra 0$ cuando $j \ra \infty$, entonces existe un único $x \in \bigcap_n^{\infty} \ol{A_n}$
\end{theorem}
\begin{proof}
  $A_n \neq \emptyset \forall n \in\N$ entonces $\exists a_n \in A_n$

  Además $\delta (A_n) \ra 0$ entonces dado $\epsilon >0$ $\exists n_0/ \quad \delta(A_n) < \epsilon \quad \forall n \geq n_0$

  Ahora sean $n,m \geq n_0$ entonces $a_n \in A_n \subseteq A_{n_0}$ y $a_m \in A_m \subseteq A_{n_0}$

  Por lo tanto $d(a_n,a_m ) < \epsilon \quad \forall n,m \geq n_0$. Entonces $(a_n)_n$ es de Cauchy.

  Como  $\E$ es completo $a_n$ es convergente a un $x \in \E$. Mostremos que $x \in \bigcap \ol{A_n}$

  Sea $k \in \N$. Si $n \geq k$ sucede $a_n \in A_n \subseteq A_k$ entonces $(a_n)_{n \geq k} \subseteq A_k$

Ahora esto es un subsucesión de $(a_n)_n$ por lo tanto converge a $x$ también luego tengo una sucesión contenida $A_k$ que converge a $x$ por lo tanto $x \in A_k$ y este $k \in \N$ era arbitrario entonces puedo hacerlo $\forall k \in \N$

Entonces $x \in \bigcap_k^{\infty} \ol{A_k}$

Supongamos que tenemos $x, x' \in \bigcap_n^{\infty} \ol{A_n}$ entonces para cada $n\in\N \quad x,x' \in \ol{A_n}$ entonces $d(x,x') \leq \delta(\ol{A_n}) = \delta(A_n) \ra 0$ entonces $d(x,x') \ra 0$ luego $x = x'$

Entonces $x $ es el único en la intersección
\end{proof}
\begin{remark}
  No vale si $\E$ no completo, por ejemplo $\E = (0,1)$ y $A_n = (0, \frac{1}{n})$ $\ol{A_n} = (0,\frac{1}{n}]$ 

  Entonces $A_1 \supseteq A_2 \supseteq \dots $ y también $\delta (A_j) \ra 0$ sin embargo $\bigcap_{m\in \N} \ol{A_n} = \emptyset$

  Pensar $\E = (C([0,1]),d_1)$ no es completo.

  Sea $A_n = \{x \in C[0,1] / x(0) = 1 \quad \int_0^1|x(t)|dt \leq \frac{1}{n}\}$ 

  Pensar si este ejemplo sirve. Observación tiene que ser no vacía la intersección y además tener un único elemento
\end{remark}
\begin{prop}
  Sean $\E ,\E '$ espacios métricos, $\E '$ completo, y $D \subseteq \E$ un subconjunto denso. Si $f : D \ra \E '$ es uniformemente continua, entonces existe una única $F : \E \ra \E '$ uniformemente continua tal que $F|_d = f$
\end{prop}

\begin{theorem}
  Sea $(\E,d)$ un e.m completo y $X \subseteq \E$ un subespacio. Entonces, $X$ es completo si y sólo si $X$ es cerrado (como subconjunto de $\E$)
  \begin{proof}
  $\Ra ) $ Sea $x \in \ol X \subseteq \E$ entonces $\exists (x_n)_n \subseteq X / \quad x_n \ra x$ esta sucesión converge en $\E$ pero no sabemos si converge en $X$

Ahora como $x_n$ converge en $\E$ entonces es de Cauchy en $\E$ . Ahora como $d_X(x_n,x_m) = d_{\E}(x_n,x_m)$

Ahora dado que la distancia en $X$ es la inducida por la distancia en $\E$ entonces esta sucesión también es de Cauchy en $X$, pero como $X$ es completo entonces esta sucesión converge en $X$.

Entonces $\exists y \in X / \quad x_n \ra y$ y esto vale para ambas distancias $d_x,d_{\E}$

Pero entonces $y = x $ luego $x \in X$. Luego dada una sucesión convergente en $X$ entonces converge en $X$ por lo tanto $X$ es cerrado

$\Leftarrow )$ Sea $(x_n)_n \subseteq X$ de Cauchy. Como $d_X(x_n,x_m) = d_{\E}(x_n,x_m)$ entonces $(x_n)_n$ es de Cauchy en $\E$ como $\E$ es completo existe $a \in \E/ \quad x_n \ra a$

Ahora $(x_n)_n \subseteq X$ y $x_n \ra a$ entonces $a \in \ol X = X$ por ser $X$ cerrado

Entonces $x_n \ra a$ con $d_{\E}$ y sabemos $(x_n)_n \subseteq X$ y también $a \in X$ 

Entonces $x_n \ra a$ con $d_X$ (esto sucede por que la distancia en $X$ es la inducida por $\E$) por lo tanto $(x_n)_n$ es convergente en $X$

Esto lo podemos hacer para cualquier sucesión de Cauchy en $X$ por lo tanto $X$ es completo
  \end{proof}
\end{theorem}
\begin{remark}
  $d: \E \ra \E '$ continua, no necesariamente manda sucesiones de Cauchy en sucesiones de Cauchy, pero si es uniformemente continua si

  Métricas equivalentes no necesariamente respetan completitud, algo es que es completo en una métrica puede no ser completo en otra métrica equivalente.
  
  Intuición si dos métridas $d,d'$ son equivalentes entonces $id: (E,d) \ra (E,d')$ es homeomorfismo, ahora como continuidad no necesriamente respeta sucesiones de Cauchy , entonces puede que esto nos mande sucesiones de Cauchy en sucesiones que no son de Cauchy

  Ahora si en cambio son fuertemente equivalentes (uniformemente equivalentes) entonces si respetan completitud. Vale por que la identidad esta vez es uniformemente equivalente por lo tanto manda Cauchy en Cauchy

  Es más, veremos mas adelante que si tenemos un homemorfismo uniforme entre dos espacios métricos cualquiera entonces uno va a ser completo si y sólo si el otro es completo 
\end{remark}
\begin{remark}
  Si $\E = \Q$ su completación podría ser $\R$

  Si $\E = (0,1)$ su completación sería $[0,1]$ (aunque $[-1,2]$ o $\R$ también sean completos que lo contienen)

  Si $\E \subseteq M$ con $M$ completo, ¿la completación de $\E$ sería $M$?. No necesariamente, mejor tomar $\ol{\E}$
\end{remark}
\begin{definition} 
  Sean $(\E,d), (\E,d')$ espacios métricos. Una función $T : \E \ra \E '$ se dice inmersión isométrica o función isométrica si $d'(T(x),T(x) = d(x,y) \quad \forall x,y \in \E$ (La diferencia con una isometríá es que esta es inyectiva)
\end{definition}
\begin{definition}
  Sea $(\E,d)$ un espacio métrico. Decimos que el espacio métricos $(\hat{E},\rho)$ es una completación de $\E$ si existe una inmersión isométrica $T: \E \ra \hat{\E}$ tal que $T(\E)$ es denso en $\hat{\E}$ 
\end{definition}
\begin{remark}
  Cuando uno tiene una inmersión isométrica podemos decir que el espacio en imagén es casi lo mismo que el espacio en el dominio

Si quisieramos ser mas rigurosos en realidad la completación es el par $((\hat{E},\rho), T)$
\end{remark}
\begin{theorem}
  Todo espacio métrico de $\E$ tiene una completación (puede ser completado). La completación es única salvo isometrías.
\end{theorem}
\begin{proof}
  Para ver que la completación es única debemos ver que dadas dos completaciones $((\hat{E},d_1),T)$ y $( (\hat{\E_2}, d_2),T_2 )$ 

  Tenemos que ver que existe una isometría biyectiva $F :\hat{\E} \ra \hat{\E_2}$ lo que nos diría que $\hat{\E}$ y $\hat{\E_2}$ son escencialmente el mismo espacio

  Pero además nos asegura que $F \circ T = T_2$

  Sea $M = \{(x_n)_n \subseteq \E /\quad  x_n \text{ es de Cauchy}\}$

  Definimos una relación de equivalencia dada por $x_n \sim y_n \iff \lim_{n \ra \infty}d(x_n,y_n) = 0$

  De esa forma definimos:

  $\hat{E} = \frac{M}{\sim} = \{\text{Clases de equivalencia de M según } \sim\} = \{[(x_n)_n] : x_n \text{ es de Cauchy}\}$

Observación: $[(x_n)_n]$ se refiere a la clase de equivalencia

Sea $z \in \hat{E}, z = [(x_n)_n ]$ para algún $(x_n)_n \subseteq \E$ de Cauchy

Y lo mismo $w \in \hat{\E}, w = [(y_n)_n ]$ para algún $(y_n)_n \subseteq \E$ de Cauchy

Y definimos la distancia $\rho (z,w) = \lim_{n \ra \infty}(x_n,y_n)$ este límite existe lo vimos en la práctica 

Faltaría ver que $\rho$ está bien definida, para esto necesitamos ver que si 

$z = [(x_n) ] = [x_n']$ y $w = [(y_n) ] = [(y_n')]$ entonces $\lim d(x_n,y_n) = \lim d(x_n',y_n')$ es trivial y queda como ejercicio para el lectór

Ahora definimos $T : \E \ra \hat{\E}$ dada por $T(x) = [(x,x,x, \dots) ] = [(x_n)_n ]$ con $x_n = x \quad \forall n \in \N$

Esta es una inmersión isométrica, para verlo basta primero probar que existe $T' : \E \ra \frac{M'}{\sim}$ dada por $T'(x)= [(x,x,x,\dots) ]$ y esta relación es la misma que antes, pero en en el conjunto $M' = \{(x_n)_n \subseteq \E : x_n \text{ es convergente}\}$ y ver que esta $T'$ es una isometria. 

Teniendo esto y sabiendo que $M' \subseteq M$ (las sucesiones convergentes son de Cauchy siempre) entonces $\frac{M'}{\sim} \subseteq \hat{E}$ podemos ver que $T$ es lo mismo que $T'$pero extendiendo el codominio por lo tanto es suryectiva, pero sigue siendo inyectiva

Teniendo $T: \E \ra \hat{E}$ inmersión isométrica falta ver que $T(\E)$ es denso en $\hat{E}$ ($\ol{T(E)} = \hat{E}$)

Entonces tomemos $z \in \hat{E}, z = [(x_n) ]$ tal que $x_n$ es de Cauchy, entonces $z = [(x_1,x_2,x_3, \dots ]$

  Para cada m, tenemos que $T(x_m) = [(x_m,x_m,x_m,\dots) ]$ estos $x_m$ son términos de $z$ 

  Sabemos $(x_n)_n$ es de Cauchy, dado $\epsilon > 0$ existe $n_0$ tal que $d(x_n,x_m) < \epsilon \quad \forall n,m \geq n_0$

  Entonces si $m \geq n_0$ tenemos $\rho(T(x_m),x_n) = \rho ((x_m,x_m,x_m, \dots),x_n) = \lim_{n \ra \infty} d(x_m,x_n) $ 
  
  Además sabemos que a partir de un $n_0$ $d(x_m,x_n) \leq \epsilon$ para cualquier $n \geq n_0$

  Pero entonces $\rho(T(x_m),[x_n]) = \lim_{n \ra \infty} d(x_m,x_n) < \epsilon$

  Por lo tanto $T(x_m) \ra [x_n]$ por esto es importante que $\rho$ de lo mismo con cualquier representante de la clase de equivalencia $[x_n]$ (hecho ya demostrado)

  Entonces para cualquier $[x_n ] \in \hat{E}$ encontramos una sucesión $(T(x_m))_m \subseteq T(\E)$ tal que $T(x_m) \ra \hat{E}$ por lo tanto $\hat{E} \subseteq \ol{T(\E)}$ y la otra inclusión siempre se da

  Finalmente $\ol{T(\E)} = \hat{E}$ probando que $T(\E)$ es denso en $\hat{E}$
\end{proof}
\end{document}
