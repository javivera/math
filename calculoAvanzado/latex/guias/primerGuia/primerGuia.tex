\documentclass[12pt]{article}

\usepackage[margin=1in]{geometry}
\usepackage{enumerate}
\usepackage{amsmath}
\usepackage{amssymb}
\usepackage{mathtools}
\usepackage{amsfonts}
\usepackage{amsthm}
\usepackage{graphicx}
\usepackage{fancyhdr}
\pagestyle{fancy}

\newcommand{\n}{\aleph_{0}}
\newcommand{\F}{\mathhbb{F}}
\newcommand{\Q}{\mathbb{Q}}
\newcommand{\C}{\mathbb{C}}
\newcommand{\R}{\mathbb{R}}
\newcommand{\K}{\mathbb{K}}
\newcommand{\E}{\mathbb{E}}
\newcommand{\I}{\mathbb{I}}
\newcommand{\N}{\mathbb{N}}
\newcommand{\Ra}{\Rightarrow}
\newcommand{\ra}{\rightarrow}
\newcommand{\ol}{\overline}
\newcommand{\norm}[1]{\left\lVert#1\right\rVert}

\theoremstyle{definition}
\newtheorem{definition}{Definición}[section]
\newtheorem*{remark}{Observación}
\newtheorem{theorem}{Teorema}
\newtheorem{lemm}{Lema}
\newtheorem{corollary}{Corolario}[theorem]
\newtheorem{lemma}[theorem]{Lema}
\newtheorem{prop}{Proposición}



\fancyhead[R]{Espacios Normados}
\fancyhead[L]{Alumno Javier Vera}
\fancyhead[C]{Cálculo Avanzado}
\begin{document}

\noindent 1) Probar las siguientes igualdades
\begin{enumerate}[i.]
  \item $$B \setminus \bigcup_{i \in \I} A_{i} = \bigcap_{i \in \I} (B \setminus A_{i})$$
    \begin{proof}
    
    $\subseteq )$ Sabemos $x \in B $ y $x \notin \bigcup_{i \in \I} A_{i}$ 

    Luego $x \in B$ y $x \notin \bigcup A_{i} \quad \forall i \in \I $

    Entonces $x \in B \setminus A_{i} \quad \forall i \in \I$

    $\Ra x \in \bigcap B \setminus A_{i}$

    $\supseteq )$ Sabemos $x \in B \setminus A_{i} \quad \forall i \in \I$

    Luego para cada $i \in \I$ sabemos $x \in B$ y $x \notin A_{i}$

    $\Ra x \in B \setminus \bigcup A_{i}$
    
    \end{proof}
  \item $$B \setminus \bigcap_{i \in \I} A_{i} = \bigcup_{i \in \I} (B \setminus A_{i})$$
    \begin{proof}

    	$\subseteq )$ Sabemos $x \in B $ y $x \notin \bigcap A_{i}$ 

    	Luego existe algún $i \in \I$ tal que $x \notin A_{i}$ (quizas para todos los $i \in I$ sucede que $x \notin A_{i}$ pero con uno alcanza)

    	Entonces existe algún $i \in \I$ tal que $x \in B$ y $x \notin A_{i} \Ra B \setminus A_{i} $

    	$\Ra x \in \bigcup (B \setminus A_{i})$

    	$\supseteq )$ Tenemos $x \in B \setminus A_{i}$ para algún $i \in \I$

    	Luego $x \in B$ y $x \notin A_{i}$ para algún $i \in \I$

    	Entonces $x \in B$ y $x \notin \bigcap A_{i} \quad \forall i \in \I$

    	$\Ra x \in B \setminus \bigcap A_{i}$
    
    \end{proof}
  \item $$\bigcup_{i \in \I} (A_{i} \cap B) = B \cap (\bigcup_{i \in \I} A_{i})$$
    \begin{proof}
    $\subseteq )$ Tenemos $x \in A_{i} \cap B$ para algún $i \in \I$  
    
    Luego $x \in B$ y $x \in A_{i}$ para algún $i \in \I \Ra x \in \bigcup A_{i}$

    Entonces $x \in B$ y $x \in \bigcup A_{i}$

    $\Ra x \in B \cap (\bigcup A_{i})$
    \end{proof}
\end{enumerate}

\newpage
\noindent
3) Sea $f : X \ra Y$ una función, $A, B $ subconjuntos de $X$
\begin{enumerate}[i.]
  \item $f(A \cup B) = f(A) \cup f(B)$
    \begin{proof}
    $\subseteq )$ Sea $y \in f(A \cup B)$ entonces $\exists x \in A \cup B /  f(x) = y$ 

    Luego $x \in A$ y $ x \in B$ 

    Entonces  $ y = f(x) \in f(A)$ y por otro lado $y = f(x) \in B$

    Finalmente $y = f(x) \in f(A) \cup f(B)$

    $\supseteq )$ Sea $y \in f(A) \cup f(B)$ luego $y \in f(A)$ e $y \in f(B)$

    Entonces $\exists x \in A  $ tal que $ f(x) = y $ luego $ x \in A \cup B$

    Luego $y = f(x) \in f(A \cup B)$
    
    \end{proof}
  \item $f(A \cap B) \subseteq f(A) \cap f(B)$
    \begin{proof}
    
    Sea $ y \in f(A \cap B)$ luego $\exists x \in A \cap B$ tal que $f(x) = y$

    Luego $x \in A $ y $x \in B$ luego $y = f(x) \in f(A)$ e $y = f(x) \in f(B)$

    Finalmente $y = f(x) \in f(A) \cap f(B)$ 
    
    \end{proof}
  \item Sea $A_{i \in \N} $ una familia de infinitos conjuntos, entonces 
    \begin{enumerate}
      \item $f(\bigcup A_{i}) = \bigcup f(A_{i})$ 
	\begin{proof} $\subseteq )$ Sea $y \in f(\bigcup A_{i})$ luego $\exists x \in \bigcup A_{i}$ tal que $f(x) = y$

	Entonces $\exists A_{j}$ tal que $x \in A_{j}$ por lo que $y = f(x) \in f(A_{j}) \subseteq \bigcup f(A_{i})$

        $\supseteq )$ Sea $y \in \bigcup f(A_{i})$ luego $\exists j \in \N $ tal que $y \in f(A_{j})$

	Luego $\exists x \in A_{j} $ tal que $y = f(x)$ luego $x \in \bigcup A_{i}$

	Finalmente $y = f(x) \in f (\bigcup A_{i})$

        \end{proof}

      \item $f(\bigcap A_{i}) \subseteq \bigcap f(A_{i})$
	\begin{proof}
	
	Sea $y \in f(\bigcap A_{i})$ luego $\exists x \in \bigcap A_{i}$

	Entonces $x \in A_{i} \quad \forall i \in \N$

	Luego $y = f(x) \in f(A_{i}) \quad \forall i \in \N$ 

	Finalmente $y \in \bigcap f(A_{i})$

        \end{proof}

      \item La última inclusión puede ser estricta. 
	\begin{proof}

	Sea $f(x) = 3 \quad \forall x \in X$ y $A = {1}, B={2}$

	Luego $3 = f(A) \cap f(B) = 3 = \{3\} $ que es distinto a $f(A \cap B) = f(\{\emptyset\}) = \emptyset$

        \end{proof}

     \end{enumerate}
\end{enumerate}

4) Sean $f: X \ra Y$ una función, $A \subseteq X$ y $B,B_{1},B_{2} \subseteq Y.$ Luego vale:

\begin{enumerate}[i.]
  \item $A \subseteq f^{-1} (f(A))$
    \begin{proof}
      
      Sea $x \in A$ luego $f(x) \in f(A)$ por lo tanto, como $ x\in f^{-1}(f(x)) \subseteq f^{-1}(f(A))$  
    
    Entonces $x \in f^{-1}(f(A)) $ 
  \end{proof}
  \item $f(f^{-1}(B)) \subseteq B$
    \begin{proof}
      Sea $y \in f(f^{-1}(B))$ entonces $\exists x \in f^{-1}(B) / f(x) = y$

      Pero entonces $f(x) \in B \Ra y \in B$
    \end{proof}
     
  \item $f^{-1}(Y \setminus B) = X \setminus f^{-1}(B)$
    \begin{proof}
      $\subseteq )$ Sea $x \in f^{-1}(Y \setminus B)$ luego $f(x) \in Y \setminus B$

      Entonces $f(x) \notin B$ entonces $x = f^{-1}(f(x)) \notin f^{-1}(B) $

      Por otro lado $f(x) \in Y$ entonces $x \in f^{-1}(Y)$

      Juntando todo $x \in f^{-1}(Y) \setminus f^{-1}(B)$

      O lo que és lo mismo $X \setminus f^{-1}(B)$

      $\supseteq )$ Sea $x \in X \setminus f^{-1}(B)$ 

      Entonces $x \in X$ entonces $f(x) \in f(X)=Y $

      Tambien $x \notin f^{-1}(B)$ por lo que $f(x) \notin B$

      Luego $f(x) \in Y \setminus B$

      Finalmente $x = f^{-1}(f(x)) \in f^{-1}(Y \setminus B)$
     \end{proof}

   \item $f^{-1}(B_{1} \cup B_{2}) = f^{-1}(B_{1}) \cup f^{-1}(B_{2})$
      \begin{proof}
        $\subseteq )$ Sea $ x \in f^{-1}(B_{1} \cup B_{2})$ luego $f(x) \in B_{1} \cup B_{2}$

	Luego $f(x) \in B_{1}$ por lo que $x \in f^{-1}(B_{1})$

	Finalmente $x \in f^{-1}(B_{1}) \cup f^{-1}(B_{2})$

        $\supseteq )$ Sea $x \in f^{-1}(B_{1}) \cup f^{1}(B_{2})$

	Luego $x \in f^{-1}(B_{1})$ entonces $f(x) \in B_{1}$

	Por tanto $f(x) \in B_{1} \cup B_{2}$

	Finalmente $x \in f^{-1}(B_{1} \cup B_{2})$
      \end{proof}

    \item $f^{-1}(B_{1} \cap B_{2}) = f^{-1}(B_{1}) \cap f^{-1}(B_{2})$
      \begin{proof}
        $\subseteq )$ Sea $x \in f^{-1}(B_{1} \cap B_{2})$ entonces $f(x) \in B_{1} \cap B_{2}$

	Por lo que $f(x) \in B_{1}$ esto implica $x \in f^{-1}(B_{1})$

	Tambien $f(x) \in B_{2}$ que implica $f(x) \in f^{-1}(B_{2})$

	Finalmente $f(x) \in f^{-1}(B_{1}) \cap f^{-1}(B_{2})$

        $\supseteq )$ Sea $x \in f^{-1}(B_{1}) \cap f^{-1}(B_{2})$

        Luego $x \in f^{-1}(B_{1})$ por lo que $f(x) \in B_{1}$ y con el mismo argumento $f(x) \in B_{2}$

	Entonces tenemos $f(x) \in B_{1} \cap B_{2} $

        Finalmente $x \in f^{-1}(B_{1} \cap B_{2})$
      \end{proof}
 \end{enumerate}

4)b) Sean $f: X \ra Y$ una función y $B_{i}$ Una familia infinita de subconjuntos de $Y$ vale:

\begin{enumerate}[i.]
  \item $f^{-1}(\bigcup B_{i}) = \bigcup f^{-1}(B_{i})$
    \begin{proof}
      $\subseteq )$ Sea $x \in f^{-1}(\bigcup B_{i})$ entonces $f(x) \in \bigcup B_{i}$

      Luego $f(x) \in B_{j}$ para algún $B_{j}$

      Por ende $x \in f^{-1}(B_{j}) \subseteq \bigcup f^{-1}(B_{i})$

      Finalmente $x \in \bigcup f^{-1}(B_{i})$

      $\supseteq )$ Por hipótesis sabemos $\exists j \in \N$ tal que $x \in f^{-1}(B_{j})$

      Por lo que $f(x) \in B_{j}$ y entonces $f(x) \in \bigcup B_{j}$

      Luego $x \in f^{-1} (\bigcup B_{j})$
    \end{proof}
  \item $f^{-1}(\bigcap B_{i}) = \bigcap f^{-1}(B_{i})$

    $\subseteq )$ Sea $x \in f^{-1}(\bigcap B_{i})$ luego $f(x) \in \bigcap B_{i}$

    Entonces $f(x) \in B_{i} \quad \forall i \in \N$ luego $x \in f^{-1}(B_{i}) \quad \forall i \in \N$

    Finalmente $x \in \bigcap f^{-1}(B_{i})$

    La otra inclusión sale de la misma forma que todos los ejercicios arriba , queda como ejercicio para alguién con muchas ganas

\end{enumerate}

5) Sea $f: X \ra Y$ una función. Probar que $f(f^{-1}(B)) = B$ para cada $B \subseteq Y$ si y sólo si $f$ es suryectiva

\begin{proof}
 $\Leftarrow)$ Por ejercicio anteriór sabemos que $f(f^{-1}(B)) \subseteq B $ probemos la otra inclusión.

  Sea $y \in B$ luego $y \in Y$ como $f$ suryectiva $\exists x \in X$ tal que $f(x) = y$ equivalentemente $x = f^{-1}(y)$

  Luego $y = f(x) = f(f^{-1}(y)) \in f(f^{-1}(B)) $

  Entonces $y \in f(f{-1}(B)) \quad \forall y \in B $ y por ende $B \subseteq f(f^{-1}(B))$ 

  Finalmente $B = f(f^{-1}(B)) $ para cualquier $B \subseteq Y$

  $\Ra )$ Tenemos la igualdad para cada $B \subseteq Y$ en particular vale para $Y$ 

  Luego $f(f^{-1}(Y)) = Y$ por lo tanto $f$ es suryectiva 

  Si no fuera suryectiva tiene que existir algún $y \in Y$ tal que $f^{-1}(y) = \emptyset$ 

  Por lo que $f^{-1}(y) \notin f^{-1}(Y)$ entonces $y \notin f(f^{-1}(Y))$

  Finalmente $Y \neq f(f^{-1}(Y))$
\end{proof}

\noindent
6) Sea $f: X \ra Y$ una función. Luego las siguientes afirmaciones son equivalentes:
\begin{enumerate}
  \item $f$ es inyectiva
  \item $f(A \cap B) = f(A) \cap f(B)$ para todo $A,B \subseteq X$
  \item $f^{-1}(f(A)) = A$ para todo $A \subseteq X$
  \item $f(A) \cap f(B) = \emptyset$ para todo par de subconjuntos $A,B $ tales que $A \cap B = \emptyset$
  \item $f(A \setminus B) = f(A) \setminus f(B)$ para todo $B \subseteq A \subseteq X$
    \begin{proof}
  $1) \Ra 2)$ Sabemos que $f(A \cap B) \subseteq f(A) \cap f(B)$ probemos la otra inclusión
  
  Sea $y \in f(A) \cap f(B)$ luego $y \in f(A)$ y $y \in f(B)$ 

  Por esto sabemos que $\exists x \in A$ tal que $f(x) = y$ y tambien que $\exists x' \in B$ tal que $f(x') = y$

  Luego $f(x) = y = f(x')$ y como $f$ es inyectiva tenemos que $x = x'$

  Luego ambos $x,x' $ (que son el mismo) estan en $A$ y ambos están en $B$

  Resumiendo $x \in A \cap B$ y por ende $y = f(x) \in f(A \cap B)$

  Luego $f(A \cap B) = f(A) \cap f(B)$

$2 \Ra 3)$ Por ej anteriór sabemos que $A \cap B \subseteq f^{-1}(f(A \cap B))$. Probemos la otra inclusión 

Sea $x \in f^{-1}(f(A \cap B))$ entonces $f(x) \in f(A \cap B)$

Y como $f(A \cap B) = f(A) \cap f(B)$ tenemos $f(x) \in f(A) \cap f(B)$  

Entonces $f(x) \in f(A)$ luego $x \in A$ por otro lado $f(x) \in f(B)$ luego $x \in B$

Finalmente $x \in A \cap B$

$3 \Ra 4 )$ Supongamos que $f(A) \cap f(B) \neq \emptyset$ luego $\exists y \in f(A) \cap f(B)$

Luego tenemos $y \in f(A)$ entonces $\exists x \in A$ tal que $f(x) = y$

Y también $y \in f(B)$ entonces $\exists x' \in B$ tal que $f(x') = y$

Entonces tenemos $x' = f^{-1}(f(x')) = f^{-1}(y) = f^{-1}(f(x)) = x$

Luego $x = x'$ por lo que $x \in A \cap B$ lo que es absurdo

Provino de suponer $f(A) \cap f(B) \neq \emptyset$

Entonces $f(A) \cap f(B) = \emptyset$

$4 \Ra 5 )$ $\supseteq )$ Sea $y \in f(A) \setminus f(B)$ 

Luego $y \in f(A)$ entonces $\exists x \in A$ tal que $f(x) = y$

Por otro lado $y \notin f(B)$ entonces $\nexists x \in B$ tal que $f(x) = y$

Luego $x \in A \setminus B$ por lo que $f(x) \in f(A \setminus B)$

$\subseteq )$ Sea $y \in f(A \setminus B)$ luego $\exists x \in A \setminus B$ tal que $f(x) = y$

Luego $x \in A$ por ende $f(x) \in A$

Y tambien $x \notin B$ por ende $f(x) \notin B$

Finalmente $y = f(x) \in f(A) \setminus f(B)$

\end{proof}
\end{enumerate}



\end{document}
