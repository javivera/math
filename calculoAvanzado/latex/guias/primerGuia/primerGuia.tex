\documentclass[12pt]{article}

\usepackage[margin=1in]{geometry}
\usepackage{enumerate}
\usepackage{amsmath}
\usepackage{amssymb}
\usepackage{mathtools}
\usepackage{amsfonts}
\usepackage{amsthm}
\usepackage{graphicx}
\usepackage{fancyhdr}
\pagestyle{fancy}

\newcommand{\n}{\aleph_{0}}
\newcommand{\F}{\mathhbb{F}}
\newcommand{\Q}{\mathbb{Q}}
\newcommand{\C}{\mathbb{C}}
\newcommand{\R}{\mathbb{R}}
\newcommand{\K}{\mathbb{K}}
\newcommand{\E}{\mathbb{E}}
\newcommand{\I}{\mathbb{I}}
\newcommand{\N}{\mathbb{N}}
\newcommand{\Ra}{\Rightarrow}
\newcommand{\ra}{\rightarrow}
\newcommand{\ol}{\overline}
\newcommand{\norm}[1]{\left\lVert#1\right\rVert}

\theoremstyle{definition}
\newtheorem{definition}{Definición}[section]
\newtheorem*{remark}{Observación}
\newtheorem{theorem}{Teorema}
\newtheorem{lemm}{Lema}
\newtheorem{corollary}{Corolario}[theorem]
\newtheorem{lemma}[theorem]{Lema}
\newtheorem{prop}{Proposición}



\fancyhead[R]{Espacios Normados}
\fancyhead[L]{Alumno Javier Vera}
\fancyhead[C]{Cálculo Avanzado}
\begin{document}

\noindent 1)
\begin{enumerate}[i.]
  \item $$B \setminus \bigcup_{i \in \I} A_{i} = \bigcap_{i \in \I} (B \setminus A_{i})$$
    \begin{proof}
    $\subseteq )$ Sabemos $x \in B $ y $x \notin \bigcup_{i \in \I} A_{i}$ 

    Luego $x \in B$ y $x \notin \bigcup A_{i} \quad \forall i \in \I $

    Entonces $x \in B \setminus A_{i} \quad \forall i \in \I$

    $\Ra x \in \bigcap B \setminus A_{i}$
    
$\supseteq )$ Sabemos $x \in B \setminus A_{i} \quad \forall i \in \I$

Luego para cada $i \in \I$ sabemos $x \in B$ y $x \notin A_{i}$

$\Ra x \in B \setminus \bigcup A_{i}$


  \end{proof}
\item $$B \setminus \bigcap_{i \in \I} A_{i} = \bigcup_{i \in \I} (B \setminus A_{i})$$

  \begin{proof}
  $\subseteq )$ Sabemos $x \in B $ y $x \notin \bigcap A_{i}$ 

    Luego existe algún $i \in \I$ tal que $x \notin A_{i}$ (quizas para todos los $i \in I$ sucede que $x \notin A_{i}$ pero con uno alcanza)

    Entonces existe algún $i \in \I$ tal que $x \in B$ y $x \notin A_{i} \Ra B \setminus A_{i} $

  $\Ra x \in \bigcup (B \setminus A_{i})$

$\supseteq )$ Tenemos $x \in B \setminus A_{i}$ para algún $i \in \I$

Luego $x \in B$ y $x \notin A_{i}$ para algún $i \in \I$

Entonces $x \in B$ y $x \notin \bigcap A_{i} \quad \forall i \in \I$

$\Ra x \in B \setminus \bigcap A_{i}$


   \end{proof}
\item $$\bigcup_{i \in \I} (A_{i} \cap B) = B \cap (\bigcup_{i \in \I} A_{i})$$

  \begin{proof}
  $\subseteq )$ Tenemos $x \in A_{i} \cap B$ para algún $i \in \I$  
  
  Luego $x \in B$ y $x \in A_{i}$ para algún $i \in \I \Ra x \in \bigcup A_{i}$

  Entonces $x \in B$ y $x \in \bigcup A_{i}$

  $\Ra x \in B \cap (\bigcup A_{i})$
\end{proof}
 
\end{enumerate}
3) Sea $f : X \ra Y$ una función, $A, B $ subconjuntos de $X$
\begin{enumerate}[i.]
  \item $f(A \cup B) = f(A) \cup f(B)$
    \begin{proof}
      $\subseteq )$ Sea $y \in f(A \cup B)$ entonces $\exists x \in A \cup B /  f(x) = y$ 

      Luego $x \in A$ y $ x \in B$ 

     Entonces  $ y = f(x) \in f(A)$ y por otro lado $y = f(x) \in B$

     Finalmente $y = f(x) \in f(A) \cup f(B)$

   $\supseteq )$ Sea $y \in f(A) \cup f(B)$ luego $y \in f(A)$ e $y \in f(B)$

   Entonces $\exists x \in A  $ tal que $ f(x) = y $ luego $ x \in A \cup B$

   Luego $y = f(x) \in f(A \cup B)$
    \end{proof}

  \item $f(A \cap B) \subseteq f(A) \cap f(B)$

    \begin{proof}
      Sea $ y \in f(A \cap B)$ luego $\exists x \in A \cap B$ tal que $f(x) = y$

Luego	$x \in A $ y $x \in B$ luego $y = f(x) \in f(A)$ e $y = f(x) \in f(B)$

Finalmente $y = f(x) \in f(A) \cap f(B)$ 
    \end{proof}

  \item Sea $f(x) = 3 \quad \forall x \in X$ y $A = {1}, B={2}$

    Luego $3 = f(A) \cap f(B) = 3 = \{3\} $ que es distinto a $f(A \cap B) = f(\{\emptyset\}) = \emptyset$

\end{enumerate}

\end{document}
