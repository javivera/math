\documentclass[12pt]{article}

\usepackage[margin=1in]{geometry}
\usepackage{enumerate}
\usepackage{amsmath}
\usepackage{amssymb}
\usepackage{mathtools}
\usepackage{amsfonts}
\usepackage{amsthm}
\usepackage{graphicx}
\usepackage{fancyhdr}
\pagestyle{fancy}

\newcommand{\n}{\aleph_{0}}
\newcommand{\F}{\mathhbb{F}}
\newcommand{\Q}{\mathbb{Q}}
\newcommand{\C}{\mathbb{C}}
\newcommand{\R}{\mathbb{R}}
\newcommand{\K}{\mathbb{K}}
\newcommand{\E}{\mathbb{E}}
\newcommand{\I}{\mathbb{I}}
\newcommand{\Z}{\mathbb{Z}}
\newcommand{\N}{\mathbb{N}}
\newcommand{\Ra}{\Rightarrow}
\newcommand{\ra}{\rightarrow}
\newcommand{\ol}{\overline}
\newcommand{\norm}[1]{\left\lVert#1\right\rVert}
\newcommand{\open}{\mathrm{o}}


\theoremstyle{definition}
\newtheorem{definition}{Definición}[section]
\newtheorem*{remark}{Observación}
\newtheorem{theorem}{Teorema}
\newtheorem{lemm}{Lema}
\newtheorem{corollary}{Corolario}[theorem]
\newtheorem{lemma}[theorem]{Lema}
\newtheorem{prop}{Proposición}
\newtheorem{ej}{Ejercicio}


\fancyhead[R]{Espacios Métricos}
\fancyhead[L]{Alumno Javier Vera}
\fancyhead[C]{Cálculo Avanzado}
\begin{document}
\begin{ej}
  Sean $(X,d)$ e $(Y,d')$ espacios métricos y sea $f: X \ra Y$. Probar que:
  \begin{enumerate}[i.]
    \item $f$ es continua en $x_0 \in X$ si y sólo si para toda sucesión $(x_n)_{n \in \N} \subseteq X$ tal que $x_n \ra x_0$, la sucesión $(f(x_n))_{n \in \N} \subseteq Y$ converge $f(x_0)$
    \item Son equivalentes:
      \begin{enumerate}
	\item $f$ es continua
	\item para todo $G \subseteq Y$ abierto, $f^{-1}(G)$ es abierto en $X$
	\item para todo $F \subseteq Y$ cerrado, $f^{-1}(F)$ es cerrado en $X$
      \end{enumerate}
  \end{enumerate}
  \begin{proof}
$i) \Ra)$ $\epsilon >0$ por continuidad sabemos $\exists \delta >0$ tal que $f(B(x_0,\delta)) \subseteq B(f(x_0),\epsilon)$

  Luego por convergencia sabemos $\exists n_0 \in \N$ tal que $d(x_n,x_0) \leq \delta \quad \forall n \geq n_0$ 

  Entonces $f(x_n) \in f(B(x_0,\delta)) \subseteq B(f(x_0),\epsilon) \quad \forall n \geq n_0$ por lo que $d(f(x_n),f(x_0)) \leq \epsilon$ 

  Ahora esto lo podemos hacer con cualquier $\epsilon$

  Finalmente $\forall \epsilon >0 \quad \exists n_0 \in \N$ tal que $d(f(x_n),f(x_0))\leq \epsilon \quad \forall n \geq n_0$

$\Leftarrow )$ Supongamos que $f$ no es continua en $x_0$ 

Entonces $\exists \epsilon > 0$ tal que para cada $\delta > 0$ $\exists y_{\delta}$ y $d(y_{\delta},x_0) \leq \delta$ pero $ d(f(y_{\delta}),f(x_0)) > \epsilon $

Si miro $\delta = \frac{1}{n}$ tengo sucesión $x_n = y_{\delta}$ tal que $x_n \ra x_0$. Pero $d(f(x_n),f(x_0)) > \epsilon \quad \forall n \in \N$

Entonces $x_n \ra x_0$ pero $f(x_n) \not\ra f(x_0)$ lo que es absurdo

$ii)$ Está todo hecho en el apunte teórico
  \end{proof}
\end{ej}

\begin{ej}
  Decidir cuales de las siguientes funciones son continuas:
  \begin{enumerate}
    \item $f:(\R^2,d) \ra (\R, |\cdot|)$ dada por $f(x,y) = x^2 + y^2$, donde $d$ representa la métrica euclídea
    \item $id_{\R^2}:(\R^2,\delta) \ra (\R^2,d_{\infty})$, la función identidad donde $\delta$ es la métrica discreta
    \item $id_{\R^2}: (\R^2,d_{\infty}) \ra (\R^2,\delta)$
    \item $i:(\E,d) \ra (X,d)$ la inclusión, donde $\E \subseteq X$
  \end{enumerate}
  \begin{proof}
  $1)$ Sea $(x_n,y_n) \ra (x,y)$ entonces $x_n \ra x$ e $y_n \ra y$ luego $x_n^2 \ra x^2$ e $y_n^2 \ra y^2$ 


  Luego $f(x_n,y_n) = x_n^2 + y_n^2 \ra x^2 + y ^2 = f(x,y)$ entonces $f$ es continua

$2)$ Sea $(x_n,y_n) \ra (x,y)$, por ser discreta la métrica $\exists n_0 / (x_n,y_n) = (x,y) \quad \forall n \geq n_0$

Entonces $f(x_n,y_n) = (x,y) \quad \forall n \geq n_0$ por lo tanto $f(x_n,y_n) \ra (x,y) = f(x,y)$

Observación esto hubiera funcionado con cualquier métrica en el espacio de llegada, por que las sucesiones que son eventualmente constantes siempre son convergentes

$3)$ Sea $(x_n,y_n) = (\frac{1}{n},\frac{1}{n})$ sabemos que $(x_n,y_n) \ra (0,0)$.

Sin embargo $f(x_n,y_n) = (x_n,y_n)$ que con la métrica discreta NO converge dado que $(x_n,y_n) \neq (x,y) \quad \forall n \in \N$ entonces $d((x_n,y_n),(x,y)) = 1 \quad \forall n \in \N$

$4)$ Sea $x_n \ra x $ es trivial ver que $f(x_n) = x_n \ra x = f(x)$
  \end{proof}
\end{ej}
\newpage
\begin{ej}
  Sean $f,g,h:[0,1] \ra \R$ definidas por:

 $f(x) = 0 $ si $ x\notin \Q$ $f(x) = 1$ si $x \in \Q$ 

 $g(x) = xf(x)$
 
 $h(x) = 0$ si $x \notin \Q$, $h(x) = \frac{1}{n}$ si $x = \frac{m}{n}$ con $(m:n ) = 1$, $h(x) = 1$ si $x=0$

 Probar que:
 \begin{enumerate}
   \item $f$ es discontinua en todo punto
   \item $g$ sólo es continua en $x =0$
   \item $h$ es continua en $[0,1] \setminus \Q$
 \end{enumerate}
 \begin{proof}
 $1)$ Dado $x \in \Q$ Sea $(x_n)_{n \in \N} \subseteq \I$ tal que $x_n \ra x$. Entonces $f(x_n) = 0 \quad \forall n \in \N$

 Por lo tanto $f(x_n) \ra 0 \neq 1 = f(x) $ entonces $f$ no es continua en $x \in \Q$

 Usando el mismo argumento vemos que no es continua en $\I$ tampoco, por lo tanto no es continua en ningún lado

 $2)$  Con sucesiones es facil ver que es continua en el 0. 

 Sea $x \neq 0$ con $x \in \Q$ entonces tomamos $(x_n)_n \subseteq \I $ tal que $x_n \ra x$

 $g(x_n) = x_nf(x_n)= x_n.0 = 0 \quad \forall n \in \N $ entonces  $g(x_n) \ra 0 \neq x = x.1 = xf(x) = g(x)$

 Con un argumento análogo vemos que tampoco es continua en $x\neq 0 \quad x\in \I$

 $3)$ $h$ restringida a $[0,1] \setminus \Q$ es una función constantemente 0. Asi que es trivialmente continua en todo punto
 \end{proof}
 \end{ej}
 
 \begin{ej}
   Probar que un espacio métrico de $X$ es discreto si y sólo si toda función de $X$ en un espacio métrico arbitrario es continua
   \begin{proof}
   $\Ra ) $ Sea $X$ discreto sabemos que toda sucesión $(x_n)_n \subseteq X$ convergente es eventualmente constante, por ende $f(x_n)$ es eventualmente constante y por ende convergente sin importar la métrica del espacio de llegada de $f$

 $\Leftarrow )$ Sea $id: X \ra (\E,\delta)$ continua. Sea $a_n \ra a$ entonces $f(a_n) \ra f(a)$ pero como la imagen tiene la métrica discreta $\exists n \in \N \quad f(a_n) = f(a) \quad \forall n \geq n_0$. Ahora nuestra función es la identidad, esto nos dice que $a_n = a \quad \forall n \geq n_0$. O en otras palabras que todas sucesión convergente es a partir de algún momento constante, y esto sucede SOLO en espacios métricos discretos, por ende $X$ es discreto
   \end{proof}
 \end{ej}

 \begin{ej}
   (Métricas topológicamente equivalentes)
     \begin{enumerate}
     \item Supongamos que existen constantes $c_1,c_2 \in \R_>0$ tales que
       $$ d_1(x,y) \leq c_1 d_2(x,y) \leq c_2d_1(x,y)$$

       para todo $x,y \in X$. Probar que $d_1$ y $d_2$ son topológicamente equivalentes.
     \item Probar que dos métricas $d_1$ y $d_2$ son topológicamente equivalentes si y sólo si la función identidad $id_X : (X,d_1) \ra (X,d_2)$ es homeomorfismo.
     \item Probar que en $\R^n$ todas las métricas $d_p$ con $1 \leq p \leq \infty$ son topológicamente equivalentes.
     \item Consideremos en $\R$ la métrica 
       $$ d'(x,y)= \biggl |\frac{x}{1 + |x|} - \frac{y}{1 + |y|} \biggl | $$

       Probar que es topológicamente equivalente a la métrica usual $d(x,y)= |x-y|$
   \end{enumerate}
   \begin{proof}
   $i)$ Sea $y \in B_2(x,r)$ entonces $d_2(x,y) < r$ entonces $c_1 d_2(x,y) < c_1  r$ 

   Pero entonces por hipótesis $d_1(x,y) \leq c_1d_2(x,y) < c_1r$ 

   Por lo tanto $y \in B_1(x,r')$ con $r' = c_1r$. Luego $B_2(x,r) \subseteq B_1(x,r')$

   La otra inclusión es análoga

 $ii)$ $\Ra )$ Sabemos que $X$ tiene los mismos abiertos con ambas métricas.

 Sea $A \subseteq (X,d_1)$ abierto entonces $id^{-1}(A) = A$ que es abierto en $(X,d_2)$ entonces $id^{-1}(A)$ es abierto

 Y esto pasa tambien con la función inversa $id^{-1}$. Por ende ambas son continuas.


 $\Leftarrow ) $ Sean $id$ continua entonces sea $A$ abierto de $(X,d_2)$ sabemos que $id^{-1}(A) = A$ tiene que ser abierto de $(X,d_1)$ y lo mismo con la inversa, por ende cualquier abierto con una métrica lo es con la otra, entonces son topológicamente equivalentes 

 $iii)$ Veamos que $d_p$ es equivalente a $d_{\infty}$ 

 Sabemos que $d_p(x,y)^p = \sum_{i =1 }^n |x_i - y_i|^p \leq n \sup | x_i - y_i | ^p = n d_{\infty}(x,y)^p$

 Luego $d_p(x,y) \leq \sqrt nd_{\infty}(x,y)$

 Por otro lado $|x_i - y_i|^p \leq \sum_{i=0}^n |x_i - y_i|^p$ entonces $\sup |x_i - y_i|^p \leq \sum_{i=0}^n |x_i -y_i|^p$

 Luego $d_{\infty}(x,y)^p \leq d_p(x,y)^p $ por lo tanto $d_{\infty}(x,y) \leq d_p(x,y)$

 Juntando todo $d_{\infty}(x,y) \leq d_p(x,y) \leq \sqrt n d_{\infty}(x,y)$

 Por lo tanto $d_p$ es equivalente $d_{\infty}$ y esta última es equivalente a $d_1$
   \end{proof}
 \end{ej}

 \begin{ej}
   Consierando $\R^n$ con la métrica euclídea, probar que:
   \begin{enumerate}[i.]
     \item $A = \{(x,y) \in \R^2 : x^2 +ysen(e^x -1) = -2\}$ es cerrado
     \item $B = \{(x,y,z) \in \R^3 : -1 \leq x^3 -3y^4 + z -2 \leq 3\}$ es cerrado
     \item $C = \{(x_1,x_2,x_3,x_4,x_5) \in \R^5 : 3 < x_1 -x_2\}$ es abierto
   \end{enumerate}
   
   Mencione otras dos métricas para las cuales siguen valiendo estas afirmaciones
   \begin{proof}
   $i)$ Sea $f(x,y) =  x^2 +ysen(e^x -1) $ tenemos que $f$ es continua por ser suma y mulitplicación de continuas

   Entonces como $\{-1\}$ es cerrado con la métrica euclídea por lo tanto $f^{-1}(\{-1\}) = B$ tiene que ser cerrado

 $ii)$ Lo mismo con preimagen del $[-1,3 ]$

 $iii)$ $f((x_1,x_2,x_3,x_4,x_5)) = x_1 - x_2$ es claramente continua

 por lo tanto como $(3,+\infty)$ es abierto, entonces  $f^{-1}((3,+\infty)) = C$ es abierto
   \end{proof}
 \end{ej}

 \begin{ej}
   Consideremos las funciones $E,I: C([0,1]) \ra \R$ definidas por:
   $$ E(f) = f(0) \text{ e } I(f) = \int_{0}^{1}f(x)dx$$
   \begin{enumerate}
     \item Demostrar que si utilizamos en $C([0,1])$ la distancia $d_{\infty}$ ambas resultan continuas.
     \item Demostrar que si en cambio utilizamos en $C([0,1])$ la distancia $d_1$, $I$ es una función continua pero $E$ no lo és
     \item Analizar si es posible que una función $F: C([0,1]) \ra \R$ sea continua para la distancia $d_1$ pero no para $d_{\infty}$
   \end{enumerate}
   \begin{proof}
   $1)$ Sea $\epsilon > 0$ entonces $|E(f),E(g)| \leq |f(0) - g(0)| \leq \sup_{t \in [0,1]}{|f(t) - g(t)|} = d_{\infty}(f,g)$. Tomemos $\delta = \epsilon$ luego $d_{\infty}(f,g) < \delta$ implica $d(E(f),E(g)) = |E(f) - E(g)| < \delta = \epsilon$

   $|I(f) - I(g)|  = |\int f(x)dx - \int g(x)dx| \leq \int |f(x) - g(x)|dx $ $\leq \int \sup{|f(t) - g(t)|dx}$

   Además $\int \sup{|f(t) - g(t)|dx} = \int d_{\infty}(f,g) = d_{\infty}(f,g)$

   Si tomamos $d_{\infty}(f,g) < \delta = \epsilon$ tenemos $d(I(f),I(g)) < \int d_{\infty}(f,g) = \int \epsilon = \epsilon.(1-0) = \epsilon$

 \noindent $2)$ De la misma forma que antes es facil ver que si $d_1(f,g) < \epsilon$ entonces $|I(f) - I(g)| < \epsilon$

 Ahora para $E$ sea $h: \R \ra \R$ dada por $f_n(t) = (1-t)^n$ tenemos que $d_1(f_n,0) = \int_{0}^1 (1-t)^n$

 $\int (1-t)^n = \biggl |_{0}^{1} \frac{(t -\frac{t^2}{2})^{n+1}}{n+1} = \frac{\frac{1}{2}^{n+1}}{n+1} = \frac{1}{(n+1)2^{n+1}} \ra 0$ por ende $f_n$ converge a $0$ (la función $0$)

 Pero $E(f_n) = 1 \quad \forall n \in \N$ por lo tanto $E(f_n) \ra 1 \neq 0 = E(0)$

 \noindent $3)$ No es posible. 

 Sea $F$ continua con $d_1$ entonces dado $\epsilon >0$ existe $\delta >0$ tal que $F(B_1(f,\delta)) \subseteq B_{\R}(F(f),\epsilon)$. 

 Ahora tomemos la bola pero en distancia infinito. Sea $ g \in B_{\infty}(f,\delta)$

 Luego $d_1(f,g) = \int |f(x)-g(x)|dx \leq \sup_{x \in [0,1]} |f(x) - g(x)| = d_{\infty}(f,g) < \delta$

 Por lo tanto $g \in B_1(f,\delta)$ entonces $B_{\infty}(f,\delta) \subseteq B_1(f,\delta)$ 

 Luego $F(B_{\infty}(f,\delta)) \subseteq F(B_1(f,\delta)) \subseteq B_{\R}(F(f),\epsilon)$

 Entonces $F$ es continua con $d_{\infty}$
   \end{proof}
 \end{ej}
  
 \begin{ej}
   Sean $X,Y$ espacios métricos y sea $f: X \ra Y$ una función continua. Probar que el gráfico de $f$, definido por 
   $$G(f) = \{(x,f(x)) \in X \times Y : x \in X \}$$

   es cerrado en $X \times Y$ ¿Es cierta la afirmación recíproca?

   \begin{proof}
     Sea $(p_n)_n \subseteq G(f)$ tal que $p_n \ra p$ queremos ver que $p \in G(f)$


     Sabemos que $d_G((x_1,x_2),(y_1,y_2)) = d_X(x_1,y_1) + d_Y(x_2,y_2)$

     
     $p_n = (x_n,f(x_n)) \ra p = (x,y)$. 

     Por lo tanto $d_X(x_n,x) \leq d_X(x_n,x) + d_Y(f(x_n),y) = d_G(p_n , p)$

     También $d_Y(f(x_n),y) \leq d_Y(f(x_n),y) + d_X(x_n,x) = d_G(p_n,p)$

     Usando limite en ambas desigualdades tenemos $d_X(x_n,x) \ra 0$ y $d_Y(f(x_n),y) \ra 0$

     Pero entonces como $f$ es continua y $x_n \ra x$ tenemos que $f(x_n) \ra f(x)$ por lo tanto $f(x) = y$

     Finalmente $p=(x,f(x))$ por lo tanto $p \in G(f)$. Entonces toda sucesión convergente de $G(f)$ converge en $G(f)$

     Por lo tanto $G(f)$ es cerrado

     La recíproca no vale , por ejempl con $f(x) = \frac{1}{x}$ y $f(0) = 0$, esta tiene como grafico a $\R \times \R$ que es cerrado por ser producto de cerrados , pero sin embargo no es continua
   \end{proof}
 \end{ej}
 \newpage
 \begin{ej}
   Sea $f:(X,d) \ra (Y,d')$ una función. Analizar la validez de las siguientes afirmaciones:
   \begin{enumerate}[i.]
     \item Si $X = \bigcup_{i \in I} U_i$, con cada $U_i$ abierto y $f|_{U_i}$ continua para todo $i \in I$

       entonces $f:X \ra Y$ continua

       \item Si $X = \bigcup_{i \in I} F_i$, con cada $F_i$ cerrado y $f|_{F_i}$ continua para todo $i \in I$

       entonces $f:X \ra Y$ continua

     \item Si $X = \bigcup_{i = 1}^m F_i$, con cada $F_i$ cerrado y $f|_{F_i}$ continua para cada $i = 1, \dots , m$

       entonces $f:X \ra Y$ continua

     \item Si $X = \bigcup_{i = 1}^m X_i$,  y $f|_{X_i}$ continua para cada $i = 1, \dots , m$ 

       entonces $f:X \ra Y$ continua

   \end{enumerate}
\begin{proof}
$i)$  Sea $V$ abierto de $Y$ tenemos que:
$$ f^{-1}(V) = f^{-1}(V) \cap X = f^{-1}(V) \cap \bigcup_{i \in I} U_i = \bigcup_{i \in I} f^{-1}(V) \cap U_i = \bigcup_{i\in I}f|_{U_i}^{-1}(V)$$

Por continuidad de la función restringida, $f|_{U_i}^{-1}(V)$ es abierto de $U_i$ para cada $i \in I$

Pero $U_i$ es abierto de $X$ para cada $i \in I$ entonces estos abiertos lo son de $X$ también , luego unión de estos abiertos es abierto de $X$ 

Por lo tanto $f^{-1}(V)$ es abierto de $X$ entonces $f$ es continua

$ii)$ Sea $f$ cualquier discontinua, ahora dado $X = \bigcup_{x \in X} \{x\}$ es un cubrimiento de cerrados

Y además $f|_{\{x\}}: \{x\} \ra Y$ es continua para cualquier $x \in X$ esto es facíl de ver dado que $\{x\}$ es abierto y cerrado en $\{x\}$ y $\{\emptyset \}$ también es abierto y cerrado en $\{x\}$ y estos dos son las únicas posibles preimagenes dadas por $f|_{\{x\}}$ de $Y$. Por lo tanto para cualquier abierto de $Y$ su preimágen será abierta y lo mismo para cualquier cerrado de $Y$

Pero entonces $f$ cumple las hipótesis , pero no es continua. Entonces la afirmación es falsa

$iii)$ Siguiendo la idea del $i)$. Sea $V$ cerrado

$$ f^{-1}(V) = f^{-1}(V) \cap X = f^{-1}(V) \cap \bigcup_{i = 1}^m  F_i = \bigcup_{i = 1}^m f^{-1}(V) \cap F_i = \bigcup_{i =1}^m f|_{F_i}^{-1}(V)$$

Ahora cada uno de estos $f|_{F_i}^{-1} (V)$ es cerrado de $F_i$ respectivamente , pero $F_i$ es cerrado de $X$ por lo tanto cada uno era entonces cerrado de $X$ luego tengo $f^{-1}(V) = \bigcup_{i =1}^m f|_{F_i}^{-1}(V) $ únion finita de cerrados de $X$ por lo tanto es cerrado. Entonces $f$ es continua

$iv)$ Si tomamos el siguiente cubrimiento $X = (-\infty,0] \cup (0,+\infty)$  

Sea $f: X \ra Y$ dada por:
$$
f(x) = \left\{
        \begin{array}{ll}
            0 & \quad x \leq 0 \\
            1 & \quad x > 0
        \end{array}
    \right.
$$

Esta es discontinua, pero si miro $A$ abierto que no contiene al $0$ entonces $f|_{(-\infty , 0]}^{-1}(A) = \{\emptyset\}$ que es abierto. Si miro $A$ abierto que contiene al cero $f|_{(-\infty , 0]}^{-1}(A) = (-\infty,0]$ que es abierto de $(-\infty,0]$ por ende $f|_{(-\infty,0]}$ es continua. Y algo similar se puede ver para el otro intervalo.

Una mas facil es usar $f(x) = 0 \quad \forall x \in X$.

\end{proof}
 \end{ej}
 \begin{ej}
   Sea $(X,d)$ un espacio métrico y sea $f: X \ra \R.$ Probar que $f$ es continua si y sólo si para todo $\alpha \in \R$, los conjuntos $A = \{x \in X : f(x) < \alpha \}$ y $B = \{x \in X : f(x) > \alpha\}$ son abiertos
   \begin{proof}
   $\Ra )$ Sea $f$ continua, luego preimagen de abierto es abierto, entonces $A = f^{-1}((- \infty,\alpha))$ es abierto, lo mismo pasa con $B$

 $\Leftarrow )$ Cualquier abierto $A$ en $\R$ lo podemos escribir como $A = (f(x)-\epsilon,f(x)+\epsilon)$. Luego tomando $\alpha = f(x) + \epsilon$ y otro $\alpha = f(x) - \epsilon$

 $\{y \in X : f(y) < f(x) + \epsilon\} \cap \{ y \in X : f(x) - \epsilon <f(y)\}$ por hipótesis es intersección de abiertos entonces es abierto , pero esta intersección es claramente la preimagen de $A$. Entonces para cualquier abierto $A$ su preimagen por $f$ es abierta, por lo tanto $f$ es continua
   \end{proof}
 \end{ej}
 \begin{ej}
   Sea $(X,d)$ un espácio métrico y sea $A$ un subconjunto de $X$. Probar que la función $d_A: X \ra \R$ definida por $d_A(x) = d(x,A) = \inf_{a \in A} d(x,a)$ es (uniformemente) continua
   \begin{proof}

     Sabemos por práctica pasada $|f(x) - f(y)| = |d_A(x) - d_A(y)| \leq d(x,y)$

    Dado $\epsilon > 0 $ puedo tomar $\delta = \epsilon $ y eso me implica $f(B(x,\delta)) \subseteq B(f(x),\epsilon) \quad \forall x \in X$

    Por lo tanto $d_A(x)$ es uniformemente continua
   \end{proof}
 \end{ej}
 \begin{ej}
   Teorema de Urysohn. Sea $(X,d)$ un espacio métrico y sean $A,B$ cerrados disjuntos de $X$
   \begin{enumerate}
     \item Probar que existe una función $f: X \ra \R$ continua tal que: 
       $$ f|_A \equiv 0  \quad \quad f|_{B} \equiv 1 \quad \quad y \quad \quad 0 \leq f(x) \leq 1 \quad \forall x \in X$$
       Sugerencia: Considerar la función $f(x) = \frac{d_A(x)}{d_A(x) + d_B(x)}$
     \item Deducir que existen abiertos $U,V \subseteq X$ disjuntos tales que $A \subseteq U$ y $B \subseteq V$
   \end{enumerate}
   \begin{proof}
   $1)$    La sugerencia sirve , es continua por ser division y suma de continuas y el denominador nunca es $0$ por que $A$ y $B$ son disjuntos y además son cerrados entonces la única forma de que $d_A(x) = 0 = d_B(x)$ es que $x \in A \cap B $ 

 $2)$ Sea $U = f^{-1}((-\infty,\frac{1}{2}))$ y $V=f^{-1}((\frac{1}{2},+\infty))$ y $A \subseteq f^{-1}(0) \subseteq f^{-1}( (-\infty,\frac{1}{2})) = U$

 Lo mismo pasa con $V$
   \end{proof}
 \end{ej}
 \begin{ej}
   Consideremos en $\Z$ y $\Q$ la métrica inducida por la usual de $\R$. Sea $f: \Z \ra \Q$ una función.
   \begin{enumerate}
     \item Probar que $f$ es continua. ¿ Sigue valiendo si $f$ toma valores irracionales?
     \item Suponiendo que $f$ es biyectiva ¿puede ser un homemorfismo?
   \end{enumerate}
 \end{ej}
 \begin{proof}
 $i)$ En $\Z$ todo subconjunto es abierto y cerrado a la vez, considerando esto la preimagen de cualquier conjunto será un conjunto de $\Z$ por lo tanto preimagen de cualquier abierto será abierto (y cerrado)

   Esto vale para cualquer espacio de llegada , dado que $\Z$ sigue teniendo las mismas características 

 $ii)$ La inversa de $f$ es $f^{-1}: \Q \ra \Z$ tiene como espacio de llegada un espacio discreto , por lo tanto todo en este espacio es abierto y cerrado , pero entonces necesito que toda preimagen sea abierta y cerrada para tener continuidad

 Y esto sucede si el espacio de salida es discreto tambien, o si la preimagen de cualquier cosa es TODO el espacio de salida, o lo que es lo mismo si la función es constante. Ahora si fuera constante no sería una biyección, y $\Q$ no es discreto con la métrica usual , asi que no es posible tener un homemorfismo entre estos dos espacios con estas métricas
 \end{proof}
 \begin{ej}
   Sea $(X,d)$ un espacio métrico, y sea $\Delta : X \ra X \times X$ la aplicación diagonal definida por $\Delta (x) = (x,x)$. Probar que 
   \begin{enumerate}
     \item $\Delta$ es un homemorfismo entre $X$ y $\{(x,x) : x \in X \} \subseteq X \times X$
     \item $\Delta (x) $ es cerrado en $X \times X $
   \end{enumerate}
   \begin{proof}
   $i)$ Si usamos la distancia orgánica en $X \times X$ dada por $d((x,x),(y,y)) = d(x,y) + d(x,y)$ Luego dado $\epsilon >0$

   Tomamos $\delta = \frac{\epsilon}{2}$ entonces $d((x,x),(y,y)) = 2d(x,y) < 2\delta = \epsilon$. Entonces la aplicación es uniformemente continua haciendo algo muy similar vemos que la inversa también es uniformemente continua (solo hay que pasar dividiendo el 2 y tomar $\delta = 2\epsilon$)

 $ii)$ Sea $(x_n,x_n) \ra (x,x)$ tal que $(x_n,x_n)_n \subseteq X \times X$ Supongamos que $(x,x) \notin X \times X$

 Pero entonces tengo $(x_n)_n \subseteq X$ con $x_n \ra x$ y $x \notin X$ lo que es absurdo, porque $X$ es todo el espacio métrico , si algo converge tiene que converger en $X$

 Otra forma de verlo es que $\Delta (X)$ es el grafico de la función $id : X \ra X$ y los gráficos son cerrados
   \end{proof}
 \end{ej}
 \begin{ej}
   Sean $(X,d)$ e $(Y,d')$ espacio métrico. Una aplicación $f: X \ra Y$ se dice $abierta$ si $f(A)$ es abierto para todo abierto $A \subseteq X$ y se dice $cerrada$ si $f(F)$ es cerrado para todo cerrado $F \subseteq X$
   \begin{enumerate}[i.]
     \item Suponiendo que $f$ es biyectiva, probar que $f$ es abierta (cerrada) si y sólo si $f^{-1}$ es continua.
     \item Dar un ejemplo de una función de $\R$ en $\R$ continua que no sea abierta
     \item Dar un ejemplo de una función de $\R$ en $\R$ continua que no sea cerrada
     \item Mostrar con un ejemplo que una función puede ser biyectiva, abierta y cerrada , pero no continua
   \end{enumerate}
   \begin{proof}
 $i)$ $\Ra )$ Llamemos $g = f^{-1}$ con $g: Y \ra X$ por comodida. Supongamos que $g$ no es continua, entonces existe un abierto en $A \subseteq X$ tal que $g^{-1}(A) \subseteq Y$ no es abierto. 

     Pero $g^{-1} = (f^{-1})^{-1} = f$ entonces $g^{-1}(A) = f(A) $ por lo tanto tenemos un abierto $A \subseteq X$ tal que $f(A)$ no es abierto , esto es absurdo

   $\Leftarrow )$ Como $g$ es continua, para todo $A \subseteq X$ abierto $g^{-1}(A) = f(A) \subseteq Y$ abierto

   Esto nos dice que $A \subseteq X$ abierto entonces $f(A)$ es abierto

   Acá se usó biyectividad para asegurar que existia $f^{-1}$

   Observación interesante, toda función de $\R$ a $\R$ biyectiva, con la métrica usual tiene inversa continua, por ende toda función biyectiva de $\R$ a $\R$ es abierta (y cerrada)

 $ii)$ $f: (\R,d_1) \ra (\R,d_1)$ tal que $f(x) = 1$ (o cualquier constante) luego $f(\R) = \{1\}$ , entonces $\R$ es abierto de $\R$ pero $f(\R) = \{1\}$ no es abierto de $\R$. Por ende $f$ no es abierta (cerrada)

 $iii)$ $f(t) = e^t$ es continua , $\R$ es cerrado pero $f(\R) = (0,+\infty)$ no es cerrado

 $iv)$ Sea $id: (\R,d_1) \ra (\R,\delta)$ donde $\delta$ es la distancia discreta, esta función NO continua, es biyectiva, su inversa es continua, (el dominio de su inversa es discreto) entonces por $i)$ es abierta (cerrada). 
   \end{proof}
 \end{ej}
 \begin{ej}
   Sean $(X,d)$ e $(Y,d')$ espacios métricos y sea $f: X \ra Y$ una función.
   \begin{enumerate}
     \item Probar que $f$ es continua si y sólo si $f(\ol{E}) \subseteq \ol{f(E)}$ para todo subconjunto $E \subseteq X$

       Mostrar con un ejemplo que la inclusión puede ser estricta.
     \item Probar que $f$ es continua y cerrada si y sólo si $f(\ol{E}) = \ol{f(E)}$ para todo subconjunto $E \subseteq X$
   \end{enumerate}
   \begin{proof}
 $1) \Ra )$ Sea $y \in f(\ol E)$ entonces $\exists x \in \ol E$ tal que $f(x) = y$

     Como $x \in \ol E$ existe $(x_n)_n \subseteq E$ con $x_n \neq x$ tal que $x_n \ra x$

     Entonces $(f(x_n))_n \subseteq f(E)$ con $f(x_n) \neq f(x)$ 

     Por continuidad de $f$ tenemos $f(x_n) \ra f(x) = y$. Por lo tanto $y \in \ol{f(E)}$

     Si tomamos $id:(\R,\delta) \ra (\R,d_1)$ sabemos que es continua y $id(\ol{\Q}) = id(\Q) = \Q$ que esta contenido estrictamente en $\ol{id(\Q)} = \ol{\Q} = \R$

   $\Leftarrow )$ 

 $2)$ $\Ra )$ Sabemos por el $1)$ que $f(\ol E) \subseteq \ol{f(E)}$ por otro lado sabemos $f(E) \subseteq f(\ol E)$

 Pero $f(E)$ es cerrado y como la clausura de $f(E)$ es el menor cerrado que contiene a $f(E)$ entonces $\
 \ol{f(E)} \subseteq f(\ol E)$. Luego $f(\ol E) = \ol{f(E)}$

 Observación $f(E)$ seguro está contenido en $f(\ol E)$ para que esto no se cumpla necesitamos que $f(E) = f(\ol E)$ que esto es lo que usamos en el contra ejemplo de arriba. $id(\Q) = id(\ol \Q)$
   \end{proof}
 \end{ej}
 \begin{ej}
   Un subconjunto $D$ de un espacio métrico $X$ se dice denso si $\ol D = X$
   \begin{enumerate}
     \item Sean $(X,d)$ e $(Y,d')$ espacios métricos y sea $D \subseteq X$ denso. Sean $f,g: X \ra Y$ funciones continuas. Probar que si $f|_D = g|_D$, entonces $f =g$
     \item Sea $f: \R \ra \R$ una función continua tal que $f(x+y) = f(x) + f(y)$ para todo $x,y \in \Q$. Probar que existe $\alpha \in \R$ tal que $f(x) = \alpha x$ para todo $x \in \R$
   \end{enumerate}
   \begin{proof}
 $1)$  Sea $x \in X$ como $D$ es denso en $X$ entonces existe $(x_n)_n \subseteq D$ tal que $x_n \ra x$

 Por continuidad sabemos que $f(x_n) \ra f(x)$ y $g(x_n) \ra g(x)$. Luego por hipótesis sabemos que $f|_D = g|_D$ entonces $f(x_n) = g(x_n)$, pero entonces $f(x_n)$ y $g(x_n)$ tienden a lo mismo por lo tanto $f(x) = g(x)$

 $2)$ Proponemos $\alpha = f(1)$ ahora mostremos que $f(n) = \alpha n$

 Por inducción $(n = 1) $ vale por que $f(1) = f(1).1 = \alpha . 1$
 
 $(n \Ra n+1$) Por hipótesis del ejercicio $f(n +1 ) =f(n) + f(1) $

 Por hipótesis inductiva $f(n) + f(1)= \alpha n + \alpha = \alpha (n+1)$ entonces $f(n+1) = \alpha (n+1)$

 Esto vale para todo $n \in \N$
 
 Ahora tomemos $n > m$ entonces $f(m) = f(n + (m -n)) = f(n) + f(m-n)$ 

 Pero entonces $f(m -n) = f(m) - f(n) = \alpha m - \alpha n = \alpha (m - n)$ 

 Ahora como a cualquier $z \in \Z$ tal que $z \neq 0$ lo podemos escribir como una resta de $m,n \in \N$ tal que $m \neq n$ tenemos que vale para cualquier $z \in \Z$ 

 $f(1) = f(1 + 0) = f(1) + f(0)$ entonces $0 = f(0)$

 Ahora fijemos $z \in \Z$ y mostremos que $f(\frac{z}{2^m}) = \alpha \frac{z}{2^m} \quad \forall m \in \N_0$

 Por inducción, el caso $n=0$ sabemos que se cumple por que $f(\frac{z}{2^0}) = f(z) = \alpha z$

 $(n \Ra n+1)$ Tenemos $f(\frac{z}{2^{n}}) = f(\frac{z}{2^{n+1}} + \frac{z}{2^{n+1}}) = f(\frac{z}{2^{n+1}}) + f(\frac{z}{2^{n+1}}) =2f(\frac{z}{2^{n+1}}) $

 Entonces $f(\frac{z}{2^{n+1}}) = f (\frac{z}{2^n}).\frac{1}{2} = \alpha \frac {z}{2^n}.\frac{1}{2} = \alpha \frac{z}{2^{n+1}}$

 Luego consideramos $D = \{\frac{z}{2^n}: z \in \Z , n \in \N \}$ y $g : \R \ra \R$ dada por $g(x) = \alpha x$

 Sabemos que estas funciones coninciden en $D$ que es denso en $\R$ 

 Entonces $f(x) = g(x) \quad \forall x \in \R$
   \end{proof}
 \end{ej}
 \begin{ej}
   Sean $(X,d)$ e $(Y,d') $ espacios métricos. Consideramos en $X \times Y$ la métrica $d_{\infty}$
   \begin{enumerate}
     \item Probar que las proyecciones $\pi_1 : X \times Y \ra X$ y $\pi_2 : X \times Y \ra Y$ son continuas y abiertas.

       Mostrar con un ejemplo que pueden no ser cerradas

     \item Sea $(E,\delta)$ un espacio métrico y sea $f: E \ra X \times Y $ una aplicación. Probar que $f$ es continua si y sólo si $f_1 = \pi_1 \circ f$ y $f_2 = \pi_2 \circ f$ lo son
   \end{enumerate}
   \begin{proof}
Probemos la uniformidad continua:

    $d_X(\pi_1((x_1,y_1)),\pi_2( (x_2,y_2))) = d_X(x_1,x_2) \leq \sup\{d_X(x_1,x_2), d_Y(y_1,y_2)\} = d_{\infty}( (x_1,y_1),(x_2,y_2))$ entonces $\pi_1$ es uniformemente continua.

     Un razonamiento similar nos lleva a probar que $\pi_2$ es uniformemente continua también

     Veamos que $f$ es abierta

     Sea $U \subseteq X\times Y$ abierto queremos ver que $\pi_1(U)$ es abierto. Ahora tomemos un $(x,y) \in U$ 

     Sabemos que existe $\epsilon >0 $ tal que $B_{\epsilon}(x,y) \subseteq U$ entonces $\pi_1 (B_{\epsilon}(x,y)) \subseteq \pi_1(U)$

     Sea $x' \in B_{\epsilon}(x) \subseteq X$ entonces $d_{\infty}((x',y),(x,y)) = \sup\{d(x',x), d(y,y)\} = d(x',x)< \epsilon$ 

 Pero entonces $(x',y) \in B_{\epsilon}(x,y)$ por lo tanto $x' = \pi_1 (x',y) \in \pi_1(B_{\epsilon}(x,y)) \subseteq \pi_1 (U)$

 Por lo tanto $\forall x' \in B_{\epsilon}(x)$ se da que $x' \in \pi_1(U)$

 Entonces $B_{\epsilon}(x) \subseteq \pi_1(U)$ y esto vale para cualquier $(x,y) \in U$ entonces vale para cualquier  $ x \in \pi_1(U) $ luego $\pi_1(U)$ es abierto

 Finalmente dado $U$ abierto tenemos que $\pi_1(U)$ es abierto

 $Contraejemplo:$
 Tomemos $F = \{(x,y) \in \R \times \R : xy = 1\}$ es cerrado , el único punto que podría ser de acumulación pero que no esté en $F$ es el $(0,0)$ pero es facil ver que no puede existir una sucesión en $F$ que converga al $(0,0)$ pero $\pi_1 (F)$ no es cerrado por que es $\R \setminus \{0\}$

 $2) \Ra )$  supongamos $f_1$ no es continua , pero como $\pi_1$ es continua , y $f_1$ es composición de esta con $f$ entonces para que $f_1$ no sea continua $f$ no tiene que ser continua, absurdo, de la misma forma vemos que $f_2$ es continua.

$\Leftarrow ) $ Tenemos que $\pi_1 \circ f$ y $\pi_2 \circ f$ son continuas 

Entonces dado $\epsilon > 0$ existe $\delta_1$ tal que $f_1(B_{\delta_1}(x,y)) \subseteq B_{\frac{\epsilon}{2}}(f_1(x,y))$

Y existe $\delta_2 >0$ tal que $f_2(B_{\delta_2}(x,y)) \subseteq B_{\frac{\epsilon}{2}}(f_2(x,y))$

Equivalentemente $\pi_1(f(B_{\delta_1}(x,y))) \subseteq B_{\frac{\epsilon}{2}}(\pi_1(f(x,y)))$ también $\pi_2(f(B_{\delta_2}(x,y))) \subseteq B_{\frac{\epsilon}{2}}(\pi_2(f(x,y)))$

Tomando $\delta = \min\{\delta_1,\delta_2\}$ tendriamos ambas inclusiones entonces 

Dado $(x',y') \in B_\delta(x,y)$ entonces $\pi_i(f(x',y')) \in \pi_i(f(B_{\delta}(x,y)))  \subseteq B_{\frac{\epsilon}{2}}(\pi_i(f(x,y))) $

Luego tenemos $\pi_1(f(x',y')) \in B_{\frac{\epsilon}{2}}(\pi_1(f(x,y))) $ también $\pi_2(f(x',y')) \in B_{\frac{\epsilon}{2}}(\pi_2(f(x,y)))$

  Luego tenemos $d(\pi_1(f(x',y')),\pi_1(f(x,y))) < \frac{\epsilon}{2}$ también $d(\pi_2(f(x',y')),\pi_2(f(x,y))) < \frac{\epsilon}{2}$

  Juntando esto último dado $(x',y') \in B_{\delta}(x,y)$ entonces $f(x',y') \in f(B_{\delta}(x,y))$

Luego $d_{\infty}(f(x,y),f(x',y')) =\sup \{d_X(f_1(x',y'),f_1(x,y)), d_Y(f_2(x',y'),f_2(x,y))\} $

Y esto último es menor o igual que si sumaramos $d_X(f_1(x',y'),f_1(x,y)) +  d_Y(f_2(x',y'),f_2(x,y))$

Que es igual a $ d_X(\pi_1(f(x',y')),\pi_1(f(x,y))) + d_Y(\pi_2(f(x',y')),\pi_2(f(x,y))) < \frac{\epsilon}{2} + \frac{\epsilon}{2} = \epsilon$

  Entonces para cualquier $(x',y') \in B_{\delta}(x,y)$ sucede que $f(x',y') \in B_{\epsilon}(f(x,y))$

Finalmente $f(B_{\delta}(x,y)) \subseteq B_{\epsilon}(f(x,y))$ por lo tanto $f$ es continua

O si nó , $\forall (x',y')$ tal que $d((x',y'),(x,y)) \leq \delta$ tenemos $d(f(x',y'),f(x,y))\leq \epsilon$
   \end{proof}
 \end{ej}
 \begin{ej}
 Sea $(X,d)$ un espacio métrico y sea $f: X \ra \R$ una función. Se dice que $f$ es $semicontinua$ $inferiormente$ (respectivamente $superiormente$) en $x_0 \in X$ si para todo $\epsilon > 0$ existe $\delta >0$ tal que 
 $$ d(x,x_0) < \delta \Ra f(x_0) - \epsilon < f(x) \quad \text{(respectivamente. } f(x_0) + \epsilon > f(x) ). $$

 \begin{enumerate}[i.]
   \item $f$ es continua en $x_0$ si y sólo si $f$ es semicontinua inferiormente y superiormente en $x_0$
   \item $f$ es semicontinua inferiormente si y sólo si $f^{-1}(\alpha,+\infty)$ es abierto para todo $\alpha \in \R$
   \item $f$ es semicontinua superiormente si y sólo si $f^{-1}(-\infty,\alpha)$ es abierto para todo $\alpha \in \R$
   \item si $A \subseteq X$ y $X_A: X \ra \R$ es su función característica, entonces $X_A$ es semicontinua inferiormente (resp. superiormente) si y sólo si $A$ es abierto (resp. cerrado)
 \end{enumerate}
$i)$ $f$ es continua en $x_0 \iff$  ( $\forall \epsilon$ existe $\delta >0$ tal que $\forall x \in X$ que cumple $ d(x,x_0) < \delta $ $\Longrightarrow$ $f(x_0) - \epsilon < f(x)$ y $f(x_0) + \epsilon > f(x)$ que es lo mismo a $|f(x_0) - f(x)| < \epsilon$ o lo mismo $d(f(x),f(x_0)) < \epsilon$)

Observación para la vuelta tenemos dos deltas y usamos el mínimo de ambos. 

\noindent $ii)$ $\Ra )$ Sea $x \in f^{-1}(\alpha,+\infty)$ entonces $f(x) > \alpha$ por lo tanto $f(x) - \alpha > 0$

Dado $\epsilon = f(x) - \alpha$ y dado que $f$ es semicontinua, sabemos que existe $\delta >0$ tal que si $y \in B_{\delta}(x)$ entonces $f(x) - \epsilon < f(y)$ pero entonces $f(y) > \alpha$ por lo tanto $y \in f^{-1}(\alpha , +\infty)$ 

Entonces para cualquier $x \in f^{-1}(\alpha,+\infty)$ tenemos que $B_{\delta}(x) \subseteq f^{-1}(\alpha,+\infty)$ por lo tanto este úlitmo es abierto

$\Leftarrow )$ Sea $\alpha = f(x) - \epsilon $ Sabemos que $f^{-1}(f(x) - \epsilon,+\infty)$ es abierto entonces existe $\delta > 0$ tal que $B_{\delta}(x) \subseteq  f^{-1}(f(x) - \epsilon,+\infty)$

Tomemos $y \in X$ tal que $d(x,y) < \delta$ tenemos que $y \in B_{\delta}(x)$ por lo tanto $f(y )\in f(B_{\delta}(x))$

Por lo tanto $f(y) \in f(f^{-1}(f(x) - \epsilon,+\infty)) = (f(x) - \epsilon ,+\infty)$

Entonces para todo $\epsilon >0 $ existe un $\delta >0$ tal que si $d(y,x) < \delta$ entonces $f(x) - \epsilon < f(y)$

Y esto vale para cualquier $x \in X$ que tomemos, luego $f$ es semicontinua inferiormente 

$iv)$ $\Ra )$ Sea $f = X_A$ ahora 

$$
f^{-1}(\alpha,+\infty) = \left\{
        \begin{array}{ll}
            \R & \quad \alpha < 0 \\
            A & \quad 0 \leq \alpha < 1 \\
	  \emptyset & \quad 1 \leq \alpha
        \end{array}
    \right.
$$

Ahora para que $f$ sea semicontinua inferiormente necesitamos que $f^{-1}(\alpha,+\infty)$ sea abierto para todo $\alpha \in \R$. Sabemos que $\R$ y $\emptyset$ son abiertos entonces 

$f$ es semicontinua inferiormente $\iff$ $A$ es abierto 

Para semicontinua superiormente usamos 

$$
f^{-1}(-\infty,\alpha) = \left\{
        \begin{array}{ll}
            \emptyset & \quad \alpha \leq 0 \\
            \R \setminus A & \quad 0 < \alpha \leq 1 \\
	 \R& \quad 1 < \alpha
        \end{array}
    \right.
$$

Tiene que ser abierto. Entonces $\R \setminus A$ tiene que ser abierto y es abierto $\iff$ $A$ es cerrado 
 \end{ej}
 \begin{ej}
   Sean $(X,d)$ e $(Y,d')$ espacios métricos y sea $f: X \ra Y$ una función que satisface:
   $$d'(f(x_1),f(x_2)) \leq c d(x_1,x_2) $$

   \noindent para todo $x_1,x_2 \in X$, donde $c \geq 0$. Probar que $f$ es uniformemente continua.
 \end{ej}
 \begin{proof}
   Sea $\epsilon >0$ puedo tomar $\delta = \frac{\epsilon}{c}$

 \noindent  Entonces $\forall x_1,x_2 \in X $ tq $d(x_1,x_2) < \delta$ tenemos $d'(f(x_1),f(x_2)) \leq cd(x_1,x_2) < c \frac{\epsilon}{c} = \epsilon$
 \end{proof}
 \begin{ej}a
   \begin{enumerate}
     \item Sean $(X,d)$ e $(Y,d')$ espacios métricos, $A \subseteq X$ y $f: X \ra Y$ una función. Probar que si existe $\alpha > 0, \quad (x_n)_n, (y_n)_n \subseteq A$ sucesiones y $n_0 \in \N$ tales que 
       \begin{enumerate}
	 \item $d(x_n,y_n) \ra 0$ para $n \ra \infty$
	 \item $d'(f(x_n),f(y_n)) \geq \alpha$ para todo $n \geq n_0$

	   entonces $f$ no es uniformemente continua en $A$
       \end{enumerate}
     \item Verificar que la función $f(x) = x^2$ no es uniformemente continua en $\R$ ¿Y en $\R_{\leq -\pi}$?
     \item Verificar que la función $f(x) = sen(\frac{1}{x})$ no es uniformemente continua en $(0,1)$
   \end{enumerate}
   \begin{proof}
   $1)$ La negación de continuidad uniforme es que existe algún $\epsilon$ tal que $\forall \delta > 0 $ existe $x,y \in X$ que cumple que $d(x,y) < \delta$ pero $d(f(x),f(y)) > \epsilon$

   Por hipótesis dado $\delta > 0 $ sabemos que existe $n_{1_{\delta}} \in \N$ tal que $d(x_n,y_n) < \delta \quad \forall n \geq n_{1_{\delta}}$

   Por la otra hipótesis tenemos que existe $n_{2_{\delta}} \in \N$ tal que $d(f(x_n),f(y_n)) \geq \alpha \quad \forall n \geq n_{2_{\delta}}$

   Entonces si tomamos $n_{0_{\delta}} = \max \{n_{1_{\delta}},n_{2_{\delta}}\}$

   Tenemos que $d(x_n,y_n) < \delta$ pero $d(f(x_n),f(y_n)) \geq \alpha \quad \forall n \geq n_{0_{\delta}}$ 

   Y para cualquier $\delta > 0 $ podemos encontrar dicho $n_{0_{\delta}}$

   Entonces si tomamos $\epsilon < \alpha$ sucede que $\forall \delta $ existen $(x_n)_n,(y_n)_n \in X$

   Que cumplen que $d(x_n,y_n) < \delta$ pero $d(f(x_n),f(y_n)) \geq \alpha > \epsilon \quad \forall n \geq n_{0_{\delta}}$

   En particular para cada $\delta$ tenemos $d(x_{n_{0_{\delta}}},y_{n_{0_{\delta}}}) < \delta$ pero  $d(f(x_{n_{0_{\delta}}}),f(y_{n_{0_{\delta}}})) \geq \alpha > \epsilon$  

   Por lo tanto $f$ no es uniformemente continua

 \noindent $2)$ Sea $x_n = n +  \frac{1}{n}$ $y_n = n$ tenemos que $d(x_n,y_n) = |n + \frac{1}{n} - n| = \frac{1}{n} \ra 0$

 Sin embargo $d(f(x_n),f(y_n)) = d(n^2 + 2 +\frac{1}{n} - n^2) = d(2 + \frac{1}{n}) = |2 + \frac{1}{n}| \ra 2$

 Entonces $x^2$ no es uniformemente continua en $\R$
   \end{proof}
 \end{ej}
 
 \begin{ej}
 Sea $f: (X,d) \ra (Y,d')$ una función uniformemente continua y sea $(x_n)_n$ una sucesión de Cauchy en $X$. Probar que $(f(x_n))_n$ es una sucesión de Cauchy en $Y$
 \begin{proof}
   Como $f$ unif continua 

   Dado $\epsilon >0$  $\exists \delta >0$ tal que todo $x,y \in X$ que cumple $d(x,y) \leq \delta $ implica $d(f(x),f(y)) < \epsilon$

   Sea $a_n$ una sucesión de Cauchy , tomando el $\delta$ de la uniformidad 

   Tenemos que $\exists n_0$ tal que $d(x_n,x_m) < \delta \quad \forall n,m \geq n_0$

   Pero entonces $d(f(x_n),f(x_m)) < \epsilon \quad \forall n,m \geq n_0$

   Y esto lo podemos hacer con cualquier $\epsilon > 0$, por que cualquier $\epsilon > 0$ nos va a dar un $\delta$ por uniformidad y este $\delta$ nos va a dar un $n_0$ por Cauchy. Luego este $n_0$ nos va a asegurar la cercania en la imagen

   Por lo tanto $f(x_n)$ es de Cauchy.

 \end{proof}
 \end{ej}
 \begin{ej} Resolver
   \begin{enumerate}[i.]
     \item Dar un ejemplo de una función $f: \R \ra \R$ acotada y continua pero no uniformemente continua
     \item Dar un ejemplo de una función $f: \R \ra \R$ no acotada y uniformemente continua
   \end{enumerate}
   \begin{proof}
     a
   \end{proof}
 \end{ej}
 \begin{ej}
   Sea $f:(X,d) \ra (Y,d')$ una función uniformente continua, y sean $A,B \subseteq X$ conjuntos no vacíos tales que $d(A,B) = 0$. Probar que $d'(f(A),f(B)) = 0$ 
 \end{ej}
 \begin{proof}
   Tengo $d(A,B) = \inf \{d(a,b):a \in A , b \in B \} = 0$ entonces existe una sucesión de distancias que tienden a cero. $d(x_n,y_n) \ra 0$ 

   Ahora supongamos que $d'(f(A),f(B)) \neq 0$ entonces $ d'(f(A),f(B)) = \alpha > 0$ (No puede ser menor que 0 por que son distancias , toda son mayores o iguales que 0).

 Entonces para cualquier par $x_n,y_n \in X$ $d(f(x_n),f(y_n))) \geq \alpha$ si para algun par la distancia fuera menor , entonces el conjunto de distancias tendría una distancia menor que $\alpha$ por lo tanto el ínfimo del conjunto sería menor que $\alpha $ luego $d'(f(A),f(B)) < \alpha$

 Pero entonces seguro $d(f(x_n),f(y_n)) \geq \alpha \quad \forall n \in \N$ equivalentemente $d(f(x_n),f(y_n)) \not\ra 0$ y esto nos dice que $f$ no es uniformemente continua. Lo cual es absurdo, provino de suponer que $d'(f(A),f(B)) \neq 0$
\end{proof}
\end{document}
