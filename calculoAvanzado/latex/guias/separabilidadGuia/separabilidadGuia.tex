\documentclass[12pt]{article}

\usepackage[margin=1in]{geometry}
\usepackage{enumerate}
\usepackage{amsmath}
\usepackage{amssymb}
\usepackage{mathtools}
\usepackage{amsfonts}
\usepackage{amsthm}
\usepackage{graphicx}
\usepackage{fancyhdr}
\pagestyle{fancy}

\newcommand{\n}{\aleph_{0}}
\newcommand{\F}{\mathhbb{F}}
\newcommand{\Q}{\mathbb{Q}}
\newcommand{\C}{\mathbb{C}}
\newcommand{\R}{\mathbb{R}}
\newcommand{\K}{\mathbb{K}}
\newcommand{\E}{\mathbb{E}}
\newcommand{\I}{\mathbb{I}}
\newcommand{\Z}{\mathbb{Z}}
\newcommand{\N}{\mathbb{N}}
\newcommand{\Ra}{\Rightarrow}
\newcommand{\ra}{\rightarrow}
\newcommand{\ol}{\overline}
\newcommand{\norm}[1]{\left\lVert#1\right\rVert}
\newcommand{\open}{\mathrm{o}}


\theoremstyle{definition}
\newtheorem{definition}{Definición}[section]
\newtheorem*{remark}{Observación}
\newtheorem{theorem}{Teorema}
\newtheorem{lemm}{Lema}
\newtheorem{corollary}{Corolario}[theorem]
\newtheorem{lemma}[theorem]{Lema}
\newtheorem{prop}{Proposición}
\newtheorem{ej}{Ejercicio}


\fancyhead[R]{Separabilidad}
\fancyhead[L]{Alumno Javier Vera}
\fancyhead[C]{Cálculo Avanzado}

\begin{document}
\begin{ej}
  Probar que $\R^n$ (con la distancia euclídea) es separable.
  \begin{proof}
    Ya probamos antes que $\Q^n$ es denso en $\R^n$ y sabemos que es numerable. Entonces $\R^n$ es separable

  \end{proof}
\end{ej}
\begin{ej}
Sea $\R^{(\N)} = \{(a_n)_n \subseteq \R : \exists n_0 \text{ tal que } a_n = 0 \quad \forall n \geq n_0\}$. Se considera la aplicación $d_{\infty} : \R^{(\N)} \times \R^{(\N)} \ra \R$ definida por $d_{\infty}((a_n)_n,(b_n)_n) = \sup_{n \in \N}|a_n - b_n| $. Probar que $(\R^{(\N)},d_{\infty})$ es un espacio métrico separable

\begin{proof}
Propongamos $\Q^{(\N)}$ como subconjunto, este es numerable (práctica i), veamos que es denso en $\R^{(\N)}$

Tomemos $a_n \in \R^{(\N)}$ ahora dado $\epsilon >0$ sabemos, por densidad de $\Q$ en $\R$ que para cada $n < n_0 \in \N$ existe un $b_n \in \Q$ tal que $d_{\R}(a_n,b_n) < \epsilon$.

Luego para cada $n \geq n_0$ decimos $b_n = 0$. Luego esta sucesión esta en $\Q^{(\N)}$ y además $\sup_{n \in \N}|a_n - b_n| < \epsilon$ por que lo es para cada $n \in \N$ (No es un $\leq$ por que solo hay finitas restas diferentes de cero dado que ambas sucesiones son 0 despues de $n_0$ y esas finitas restas son todas menores que epsilon entonces en realidad aquí el supremos es un máximo y es estrictamente menor que $\epsilon$)
\end{proof}
\end{ej}
\begin{ej}
  Sea $(X,d)$ un espacio métrico. Se dice que una familia $\mathcal{A} = (U_j)_{j \in J}$ de abiertos de $X$ es base de abiertos de $X$ si todo abierto de $X$ se puede escribir como unión de miembros de $\mathcal{A}$. Probar que $\mathcal{A}$ es una base de abiertos de $X$ si y sólo si verifica la siguiente condición : `Para todo abierto $G$ de $X$ y para todo $x \in G$ existe $j \in J$ tal que $x \in U_j \subseteq G$ '
  \begin{proof}
  $\Ra )$ sea $G$ abierto entonces $G = \bigcup_{j \in J} U_j$ luego sea $x \in G$ tenemos que $x \in \bigcup U_j$

  Pero entonces existe un $j_0 \in J$ tal que $x \in U_{j_0} \subseteq \bigcup U_j \subseteq G$a

$\Leftarrow )$ Sea $G$ un abierto para cada $x \in G$ existe un $j \in J$ tal que $x \in U_j \subseteq G$

Pero entonces puedo escribir a $G$ como la unión de $U_j \subseteq \mathcal{A}$ y esto vale para cualquier abierto , por lo tanto $\mathcal{A}$ es base de $X$

  \end{proof}
\end{ej}
\begin{ej}
  Sea $(X,d)$ un espacio métrico que verifica que cada cubrimiento abierto de $X$ tiene un subcubrimiento numerable. Probar que $X$ es separable
  \begin{proof}
    Sea $\bigcup_{x \in X} B(x,\frac{1}{n}) = X$ son cubrimiento por abiertos de $X$ para cada $n \in \N$

    Ahora para cada uno de ellos tenemos un subcubrimiento numerable $\bigcup_{(x_n)_n \subseteq X} B(x_n\frac{1}{n})$

    Ahora si unimos los $x_n$ de cada cubrimiento tenemos un conjunto numerable (por ser union numerable de numerables). Llamemosló $A$

    Veamos que es denso. Sea $x \in X$ sabemos que para cada $n \in \N$ tenemos un cubrimiento de $X$ por lo tanto tenemos una $B(x_n,\frac{1}{n})$ tal que $x$ esta en ella por lo tanto $x_n \in B(x,\frac{1}{n})$ y esto vale para cada $n \in \N$ 

    Por lo tanto para cada $x \in X$ sucede que para cada radio existe un $x_n \in A$ tal que $x_n \in B(x,r)$ por lo tanto $A$ es denso en $X$
  \end{proof}
\end{ej}
\begin{ej}
  Probar que todo subespacio de un espacio métrico separable es separable
  \begin{proof}
    Sea $A$ un espacio separable entonces existe $D \subseteq A$ denso y numerable
  \end{proof}
\end{ej}
\begin{ej}
Parte a)
  Sea $(X,d)$ un espacio métrico separable. Probar que toda familia de subconjuntos de $X$ no vacíos, abiertos y disjuntos dos a dos es a lo sumo numerable. Deducir que el conjunto de puntos aislados de $X$ es a lo sumo numerable
  \begin{proof}
    Sea $D$ denso numerable de $X$ y sea $\mathcal{U}$ el conjunto de abiertos disjuntos no vacíos.

    Sabemos que $\forall U \in \mathcal{U}$ $U \cap D \neq \emptyset$ por densidad de $D$ (Si no tendría un abierto $U$ tal que $D \cap U = \emptyset$ y como $U$ abierto si tomo $x \in U$ existe un radio $r>0$ tal que $B(x,r) \subseteq U$ y por lo tanto $B(x,r) \cap D = \emptyset$ lo que es absurdo)

    Ahora puedo hacer una función $\phi: \mathcal{U} \ra D$ tal que $\phi(U) \in U \cap D \quad \forall U \in \mathcal{U}$

    Veamos que $\phi$ es inyectiva. Supongamos $U,V \in \mathcal{U}$ tal que $\phi(U) = \phi(V)$ 

  Tenemos $\phi(U) \in U \cap V$ y $\phi (V) \in V \cap D$ entonces $\phi(U) = \phi(V) \in (V \cap U ) \cap D$

	  Y sabemos que $V \cap U \neq \emptyset$ por que si no $(V \cap U ) \cap D = \emptyset$ pero entonces $\phi$ no estaría bien definida. Ahora como $V \cap U \neq \emptyset$ y ambos están en $\mathcal{U}$ entonces $V = U$. Si no tendríamos dos abiertos de $\mathcal{U}$ que no son disjuntos

  Entonces $\phi$ es inyectiva por lo tanto $\# \mathcal{U} \leq \# D = \n$ entonces $\mathcal{U}$ es a lo sumo numerable

Parte b) Sea $x$ un punto aislado podemos encontrar un $r >0$ tal que $B(x,r) \cap X = \{x\}$ y esto lo podemos hacer para cualquier $x \in X$ entonces podemos definir a $\mathcal{U}$ como el conjunto de todas estas bolas y estas son disjuntas de dos a dos son abiertas y no vacías , por ende a lo sumo puede haber numerables de ellas, y como para cada una hay un $x $ aislado , entonces tenemos a lo sumo numerables $x$ puntos aislados

Otra forma. Sea $A$ el conjunto de los puntos aislados de $X$, si $x \in A$ entonces existe $\epsilon > 0$ tal que $B(x,\epsilon) \cap X = \{x\}$ entonces $\{x\}$ es un conjunto abierto en $X$ 

Por lo tanto $\mathcal{U} = \{\{x\}: x \in A\}$ es un conjunto de abiertos disjuntos , y ahora usamos la parte a
  \end{proof}
\end{ej}
\begin{ej}
  Sean $(X,d)$ e $(Y,d')$ espacios métricos. Probar que $(X \times Y,d_{\infty})$ es separable si y sólo si $(X,d)$ e $(Y,d')$ son separables
  \begin{proof}
  $\Leftarrow )$ Sean $D_1$ denso numerable de $X$ y $D_2$ denso numerable de $Y$

  Son ambos numerables entonces existen $f: D_1 \ra \N$ y $g: D_2 \ra \N$ sobreyectivas

  Y entonces tenemos $h: D_1 \times D_2 \ra \N \times \N$ dada por $h(x,y) = (f(x),g(y))$

  Entonces $D_1 \times D_2 \sim \N \times \N \sim \N$ por lo tanto $D_1 \times D_2$ es numerable

  Sea $(x,y) \in X \times Y$ , $D_1$ es denso en $X$ entonces dado $r>0$ $B(x,r) \cap D_1 \neq \emptyset$

  Por lo tanto existe algun $d_1 \in D_1 $ tal que $d(d_1,x) < r$

  También para el mismo $r > 0$ $B(y,r) \cap D_2 \neq \emptyset$ entonces existe $d_2 \in D_2$ tal que $d'(d_2,y) < r$

  Entonces $d_{\infty}((x,y),(d_1,d_2)) < \sup \{d(x,d_1),d'(y,d_2)\} = r$

  Y esto vale para cualquier $r>0$ entonces $D_1 \times D_2$ es denso en $X \times Y$

  Luego $X \times Y$ es separable
  \end{proof}
\end{ej}
\begin{ej}
  ¿Es el espacio $(\ell_{\infty},d_{\infty})$ separable ? 
  \begin{proof}
    Tomemos en cuenta el subespacio métrico de $\ell_{\infty}$ dado por $(\ell_{(\N)}, d_{\infty})$

    Con $\ell_{(\N)} = \{(a_n)_n \subseteq \N : a_n \text{ es acotada }\}$

    Consideremos el conjunto $A = \{f: \N \ra \{1,2\} \text{ tal que } f \text{ es inyectiva}\}$

    Sabemos que $\# A = 2^{\n} = \mathfrak{c}$

 Cada una de esas funciones nos da una sucesión de naturales claramente acotadas por $2$

 Entonces tenemos $\mathfrak{c}$ sucesiones acotadas , ahora si tomamos $r = \frac{1}{2}$ para cada una de esas sucesiones tenemos $\mathfrak{c}$ abiertos y todos estos son disjuntos en $\ell_{(\N)}$ dado que si no lo fueran existirian dos sucesiones $(x_n)_n \neq (y_n)_n$ en $ \ell_{(\N)}$ tales que $\sup_{n \in \N}|x_n - y_n| = d_{\infty}(x_n,y_n) < \frac{1}{2}$  y esto es imposible salvo que $x_n  = y_n$ entonces tenemos no numerables abiertos disjuntos en $\ell_{(\N)}$. Por lo tanto $\ell_{(\N)}$ no puede ser separable pero entonces $\ell_{\infty}$ tiene un subespacio métrico que no es separable, por lo tanto no es separable
  \end{proof}
\end{ej}
\begin{ej}
  Sean $X,Y$ espacios métricos. Sea $f: X \ra Y$ una función continua y suryectiva. Probar que si $X$ es separable, entonces $Y$ es separable
  \begin{proof}
    $D$ denso y numerable en $X$ entonces $f(D)$ es numerable, veamos que es denso en $Y$

    Sea $y \in Y$ entonces existe un $x \in X$ tal que $f(x) = y$ 

  Ahora como $D$ es denso en $X$ existe $(x_n)_n \subseteq D$ tal que $x_n \ra x$ 

  Como $f$ es continua tenemos que $f(x_n) \ra f(x) = y$ y además $(f(x_n))_n \subseteq f(D)$

  Por lo tanto $f(D)$ es denso en $Y$, por lo tanto $Y$ es separable

  \end{proof}
\end{ej}
\begin{ej}
  Sea $(X,d)$ un espacio métrico y sea $(x_n)_n \subseteq X$. Probar que:
  \begin{enumerate}
    \item $\lim_{n \ra \infty} x_n = x$ si y sólo si para toda sucesión $(x_{n_k})_k, \lim_{k \ra \infty} x_{n_k}= x$
    \item Si existe $x \in X$ para el cual toda sucesión $(x_{n_k})_k$ de $(x_n)_n$ tiene una subsucesión $(x_{n_{k_j}})_j$ tal que $\lim_{j \ra \infty}x_{n_{k_j}} = x$, entonces $(x_n)_n$ converge y $\lim x_n = x$. ¿ Vale la recíproca?
    \item Si $(x_n)_n$ es convergente, entonces $(x_n)_n$ es de Cauchy , ¿Vale la recíproca?
     \item Si $(x_n)_n$ es convergente, entonces es acotada
     \item Si $(x_n)_n$ es de Cauchy y tiene una subsucesión $(x_{n_k})_k$ tal que $\lim_{k \ra \infty} x_{n_k} = x \in X$, entonces $(x_n)_n$ converge y $\lim x_n = x$
  \end{enumerate}
  \begin{proof}
$1) \Ra)$ Sabemos que para todo $\epsilon > 0$ existe $n_0$ tal que $d(x_n,x) < \epsilon \quad \forall n \geq n_0$

Entonces tomemos cualquier subsucesión $x_{n_k}$ afirmamos que vale $d(x_{n_k},x) < \epsilon \quad \forall n_k \geq n_0$ 

Sabemos que dado que $x_{n_k}$ existe algún $ x_n$ tal que $x_n = x_{n_k}$ y tiene que cumplir $n \geq n_k$ 

Entonces como $n \geq n_k \geq n_0$ sucede que $x_{n_k} = x_n \in B(x,\epsilon)$ y esto vale para cualquier $x_{n_k}$ con $n_k \geq n_0$

$\Leftarrow )$ Supongo que $\lim x_n \neq x$ sabemos que $x_n$ es subsucesión de si mismo , entonces tenemos una subsucesión de que $x_n$ tal que $\lim x_{n_k} \neq x$ lo que es absurdo

$2)$ 

$3)$ Sea $x_n$ convergente entonces dado $\epsilon > 0$ $x_n \in B(x,\epsilon) \quad \forall n \geq n_0$, 

Luego $x_n , x_m \in B(x,\epsilon) \quad \forall n,m \geq n_0$ 

Entonces $d(x_n,x_m) < diam(B(x,\epsilon)) = 2\epsilon \quad \forall n,m \geq n_0$ por lo tanto $x_n$ es de Cauchy.

La recíproca no vale si estamos en un espacio métrico NO completo. Por ejemplo en $\Q$ tenemos la sucesión $a_n = \sqrt{2} + \frac{1}{n}$ esta sucesión NO converge en $\Q$ pero es de Cauchy, dado que para $\epsilon > 0$ existe un $n_0$ tal que $a_n \in B(\sqrt{2},\epsilon) \quad \forall n \geq n_0$. Entonces usando un razonamiento igual al de la ida, tenemos que $a_n$ es de Cauchy

$4)$ Como $x_n$ convergente $x_n$ es de Cauchy 

Entonces dado $\epsilon > 0 $ tenemos que $d(x_n,x_m) < \epsilon \quad \forall n,m \geq n_0$ 

Por lo tanto si fijamos $x_m$ con algún $m \geq n_0$ tenemos $d(x_n,x_m) < \epsilon \quad \forall n \geq n_0$ 

Luego $x_n < x_m + \epsilon = S_1 \quad \forall n \geq n_0$ 

Sabemos que hay finitos $x_n$ tal que $n < n_0$ luego sea $S_2 = \max \{x_n : n < n_0\}$ 

Finalmente tomamos $S_0 = \max \{S_1,S_2\}$ sucede que $x_n < S_0 \quad \forall n \in \N$ 

$5)$ Una forma interesante de hacerlo. Sea $(X,d)$ un espacio métrico $(a_n)_n \subseteq X$ de Cauchy, y sea $a_{n_k}$ un sub convergente a $L \in X$. Ahora si completamos a $X$ tenemos que $a_n$ converge digamos a $P$, pero como no puede converge a algo diferente a una subsucesion entonces $P = L$ por lo tanto $P \in X$ entonces $a_n$ convergía en $X$ desde un principio

Sea $\epsilon > 0$ como $x_n$ es de Cauchy tenemos que existe $n_0$ tal que $d(x_n,x_m) < \frac{\epsilon}{2} \quad \forall n \geq n_1$

Mismo epsilon existe $n_2$ tal que $d(x_{n_k},x) < \frac{\epsilon}{2} \quad \forall n_k \geq n_2$

Por lo tanto tomemos $n_0 = \max \{n_1,n_1\}$ y fijemos un $n_k \geq n_0$

luego tenemos que $$d(x_n,x) \leq d(x_n,x_{n_k}) + d(x_{n_k},x) \leq \frac{\epsilon}{2} + \frac{\epsilon}{2} = \epsilon \quad \forall n \geq n_0$$

Por lo tanto $x_n \ra x$
  \end{proof}
\end{ej}
\begin{ej}
  Probar que si toda bola cerrada de un espacio métrico $X$ es un subespacio completo de $X$, entonces $X$ es completo
  \begin{proof}
    Tomemos $(x_n)_n \subseteq X$ de Cauchy. Sabemos entonces que está contenida en una bola $B_1(x,r)$ y está contenida en la misma bola cerrada , siempre pensándola como bola de $X$. Ahora como esta bola cerrada tiene que ser completa, entonces $x_n \ra x \in \ol{B_1}(x,r)$ por lo tanto $x \in X$ por lo tanto toda sucesión de Cauchy converge en $X$ entonces $X$ es completo  
  \end{proof}
\end{ej}
\begin{ej}
  Sea $(X,d)$ un espacio métrico
  \begin{enumerate}
    \item Probar que todo subespacio completo de $(X,d)$ es un subconjunto cerrado de $X$
    \item Probar que si $X$ es completo, entonces todo subconjunto $F \subseteq X$ cerrado, es un subespacio completo de $X$
  \end{enumerate}
  \begin{proof}
  $1)$ Sea $(Y,d)$ un subespacio métrico de $(X,d)$ entonces toda sucesión de Cauchy de $Y$ converge en $Y$, pero además toda sucesión convergente de $Y$ es de Cauchy , por lo tanto toda sucesión convergente de $Y$ converge en $Y$.

  Ahora si miro a $Y$ como subconjunto de $X$ sigue pasando lo mismo por lo tanto $Y$ es cerrado

$2)$ Sea $F \subseteq  X$ cerrado , ahora sea $(x_n)_n \subseteq F \subseteq X$ de Cauchy entonces converge a un $x \in X$ por que $X$ es completo

Ahora ese $x$ tiene que estár en $F$ si no tendríamos una sucesion $(x_n)_n \subseteq F$ cerrado tal que $x_n$ no converge en $F$ lo cual es absurdo.

Por lo tanto toda sucesión $(x_n)_n \in F$ de Cauchy converge enotonces $F$ es completo
  \end{proof}
\end{ej}
\begin{ej}
  (Teorema de Cantor). Probar que un espacio métrico $(X,d)$ es completo si y sólo si toda familia $(F_n)_n$ de un subconjunto de $X$ cerrados, no vacíos tales que $F_{n + 1} \subseteq F_n$ para todo $n \in \N$ y $diam(F_n) \ra 0$ tiene un único punto en la intersección
  \begin{proof}
  $\Ra )$ Sea $F_n$ una familia que cumple las hipótesis , para cada $n \in \N$ me quedo con un $(x_n)_n \subseteq X$ ahora tengo una sucesión que es de Cauchy, por que para cada $\epsilon >0$ existe un $diam(F_n) \leq \epsilon$ entonces todo $x_n,x_m \in F_{n}$ cumplen que $d(x_n,x_m) < diam (F_{n}) \leq \epsilon $.

  Dado que $X$ es completo $x_n$ converge, pero además $(x_n)_n \subseteq \bigcap_{n \in I} F_n$ y esta por ser intersección de cerrados , es cerrado , por lo tanto $x_n \ra x \in \bigcap F_n$

  Supongo que hay otro $y \in \bigcap F_n$ ahora como $y \neq x$ existe un $r>0$ tal que $y \notin B(x,r)$

  Ahora si tomo un $F_n$ tal que $diam(F_n) < \frac{r}{2}$ se que $x \in F_n$ entonces para cualquier $z \in F_n$ tengo que $d(x,z) < 2 \frac{r}{2} = r$ por lo tanto $F_n \subseteq B(x,r)$ pero entonces $y \notin F_n$ por que si no $y \in B(x,r)$. Entonces existe un $F_n$ tal que $y \notin F_n$ lo que es absurdo por que dijimos que $y \in \bigcap F_n$ el absurdo vino de suponer que existia otro $y \in \bigcap F_n$ por lo tanto no existe otro 
  
$\Leftarrow )$ Sea $(x_n)_n \subseteq X$ una sucesión de Cauchy $G_n = \{x_n : n \geq n_0\}$ y sea $F_n = \ol{G_n}$

Como $G_n \supseteq G_{n +1}$ entonces $F_n \supseteq F_{n+1}$

Además como $x_n$ es de Cauchy , dado un $\frac{\epsilon}{2}$ sabemos que existe $N \in \N$ tal que 
$$d(x_m,x_n) < \frac{1}{\epsilon} \quad \forall n,m \geq N$$

Entonces $G_n \in B(x_N,\frac{\epsilon}{2})$ si $ n \geq N$ pero entonces $F_n \subseteq \ol{B}(x_N,\frac{\epsilon}{2})$ 

Entonces $diam(F_n) \leq \epsilon$ y esto lo puedo hacer para cualquier $\epsilon$ en particular para $\epsilon$ cada vez mas chicos. De esta forma estamos en la hipótesis , por lo tanto existe $x \in \bigcap F_n$. 

Luego sabemos que dado $\epsilon > 0$ existe $N \in \N$ tal que $diam(F_N) < \epsilon$

Ahora sabemos también que $\{x\} \cup G_N \subseteq F_N$ entonces para todo $n \geq N$ $d(x,x_n) < \epsilon$

Por lo tanto $x_n \ra x$ entonces cualquier sucesíon de Cauchy converge, por lo tanto $X$ es completo
  \end{proof}
\end{ej} 
\begin{ej}
  Sean $(X,d)$ e $(Y,d')$ espacios métricos. Probar que $(X \times Y ,d_{\infty})$ es completo si y solo si $(X,d)$ e $(Y,d')$ son completos.
  \begin{proof}
$\Ra )$ Sean $(x_n) \subseteq X$ e $(y_n)_n \subseteq Y$ sucesiones de Cauchy entonces 

Veamos que $(x_n,y_n)_n \subseteq X \times Y$ es de Cauchy

Sabemos que dado $\epsilon$ existen $n_1, n_2 \in \N$ tal que $d(x_n,x_m) \leq \frac{\epsilon}{2} \quad \forall n,m \geq n_1$

También $d(y_n,y_m) \leq \frac{\epsilon}{2} \quad \forall n,m \geq n_2$. Tomemos $n_0 = \max\{n_1,n_2\}$

Ahora $d_{\infty}( (x_n,y_n),(x_m,y_m)) = \sup \{d(x_n,x_m),d(y_n,y_m)\} \leq \frac{\epsilon}{2} < \epsilon \quad \forall n,m \geq n_0$

Entonces $(x_n,y_n)_n \subseteq X \times Y$ es de Cauchy, como $X \times Y$ es completo converge a un $(x,y)$

Dado $\epsilon > 0 \quad $  $d(x_n,x) < \sup\{d(x_n,x),d(y_n,y)\} = d_{\infty}( (x_n,x),(y_n,y)) < \epsilon \quad \forall n \geq n_0$

Por lo tanto $x_n \ra x \in X$, lo mismo pasa con $y_n \ra y \in Y$

Finalmente cualquier sucesión de Cauchy de $X$ converge y cualquier sucesión de Cauchy de $Y$ también , por lo tanto ambos son completos

$\Leftarrow )$ Sea $(x_n,y_n)_n \subseteq X \times Y$ de Cauchy. 

Sea $\epsilon > 0$ $\exists n_0/ \quad$ $d_{\infty}( (x_n,y_n),(x_m,y_n)) = \sup\{d(x_n,x_m),d(y_n,y_m)\} \leq \epsilon \quad \forall n,m \geq n_0$

Entonces $d(x_n,x_m) \leq \epsilon \quad \forall n,m \geq n_0$ si nó el supremo de antes sería mas grande que $\epsilon$ 

Y también $d(y_n,y_m) \leq \epsilon \quad \forall n,m \geq n_0$

Por lo tanto $x_n$ e $y_n$ son de Cauchy, por hipótesis convergen a $x$ e $y$ respectivamente

Pero entonces dado $\epsilon$ existe $n_1$ tal que $d(x_n,x) < \epsilon$ y un $n_2$ tal que $d(y_n,y) < \epsilon$

Si tomamos $n_0 = \max\{n_1,n_2\}$ tenemos que
$$d_{\infty}( (x_n,y_n),(x,y)) = \sup\{d(x_n,x),d(y_n,y)\} < \epsilon \quad \forall n \geq n_0$$

Finalmente para cualquier $(x_n,y_n)_n \subseteq X \times Y$ de Cauchy entonces $(x_n,y_n)_n$ converge, entonces $X \times Y$ es Completo
  \end{proof} 
\end{ej}
\begin{ej}
  \begin{enumerate}[i.] Resolver
    \item Sea $X$ un espacio métrico y sea $B(X) = \{f: X \ra \R : f \text{ es acotada }\}$. Probar que $(B(X),d_{\infty})$ es un espacio métrico completo, donde $d_{\infty} = \sup_{x \in X} |f(x) - g(x)|$ 
    \item Sean $a,b \in \R$ tal que $a<b$. Probar que $(C[a,b],d_{\infty})$ es un espacio métrico completo, donde $d_{\infty}(f,g) = \sup_{x \in [a,b]} |f(x) - g(x)|$ 
    \item Probar que $c_0 = \{(a_n)_n \subseteq \R \ | \ a_n \ra 0\}$ es un espacio métrico completo con la distancia $d_{\infty}( (a_n)_n,(b_n)_n) = \sup_{x \in \N}|a_n -b_n| $
  \end{enumerate}
  \begin{proof}
  $i)$ Sea $(f_n)_n \subseteq B(X)$ una sucesión de Cauchy. 
 
  Dado $\epsilon >0 $ existe $n_0 \in \N$ tal que $d(f_n,f_m) =\sup_{x \in X} |f_n(x) - f_m(x)| \leq \epsilon \quad \forall n,m \geq n_0$ 

	  Pero entonces si fijo $x$ tengo $|f_n(x) - f_m(x)| < \epsilon \quad \forall n,m \geq n_0$ 

	  Por lo tanto $(f_n(x))_n \subseteq \R$ es de Cauchy, como $\R$ es completo $f_n(x)$ converge digamos a un $f(x)$

	  Veamos primero que $f(x)$ es acotada.

	  Sabemos por un lado que $f_n$ es acotada $\forall x \in X$ por que $f_n \in B(X)$

	  Por ser de Cauchy tenemos que $d_{\infty}(f_N,f_n)=\sup_{x \in X}|f_N(x) - f_n(x)| < 1 \quad \forall n \geq N$ 

	Entonces $|f_n(x) - f_N(x)| < 1 \quad \forall x \in X \quad \forall n \geq N $ luego $-1 +f_N(x) < f_n(x) < 1 +f_N(x)$

	  Ahora si hacemops $n \ra \infty$ tenemos $-1 + f_N(x) \leq f(x) \leq 1 + f_N(x)$

	  Este $N$ está fijo , lo mismo el $x$ por lo tanto $f_N(x) = M$ que es una constante

	  Luego $f(x)$ está acotado y esto además sabemos que vale para cualquier $x$  

	  Finalmente $f$ esta acotada

	  Veamos que $f_n \ra f$ Como $f_n$ de Cauchy $d_{\infty}(f_n,f_m) \leq \frac{\epsilon}{2} \quad \forall n,m \geq n_0$

	  Entonces $|f_n(x) - f_m(x)| < \frac{\epsilon}{2} \quad \forall n,m \geq n_0$ para cualquier $x \in X$.

	  Si hacemos tender $m \ra \infty$ tenemos que $|f_n(x) - f(x)| \leq \frac{\epsilon}{2} \quad \forall n \geq n_0 \quad \forall x \in X$

	  Luego $\sup_{x \in X} |f_n(x) - f(x)| < \epsilon \quad \forall n\geq n_0$ 

	  Entonces dado $\epsilon > 0$ existe $n_0 \in \N$ tal que $d_{\infty}(f_n,f) \leq \epsilon \quad \forall n \geq n_0$

	  Luego $f_n \ra f$ entonces $B(X)$ es completo

	  $ii)$ Este espacio es un subespacio de $B(X)$ por que toda función en este espacio es acotada entonces basta con ver que es cerrado

	  Sea $(f_n)_n \subseteq C[a,b]$ convergente a $f$ veamos que $f$ es una función continua en $[a,b]$

	  Por un lado dado $\epsilon > 0 $ tenemos $d_{\infty}(f_n,f) = \sup_{x \in X} |f_n(x),f(x)| < \frac{\epsilon}{3} \quad \forall n \geq n_0 $

	  Por otro lado $f_n$ es continua por ejemplo en $x$ para cualquier $n \in \N$ 

	  Entonces $\exists \delta > 0 $ tal que $d(x,y) < \delta$ implica $d(f(x),f(y)) < \frac{\epsilon}{3}$

	  Luego $d(f(x),f(y)) \leq d(f(x),f_n(x)) + d(f_n(x),f_n(y)) + d(f_n(y),f(y)) $

	  $\leq d_{\infty}(f_n,f) + d(f_n(x),f_n(y)) + d_{\infty}(f_n,f) < \frac{\epsilon}{3}  + \frac{\epsilon}{3} + \frac{\epsilon}{3} = \epsilon$

	  Entonces dado $\epsilon$ exsite $\delta$ que cumple $\forall y$ tal que $d(x,y) < \delta$ implica $d(f(x),f(y)) < \epsilon \quad$

	  


   $ iii)$ Sabemos $\ell_{\infty}$ es completo, y $c_0 \subseteq \ell_{\infty}$ por lo tanto basta ver que es cerrado para saber que es completo.

  Sea $(a_k^n)_n \subseteq c_0$ convergente a $a_k$ veamos que $(a_k)_k \subseteq c_0$ 

  Dado $\epsilon$ existe $n_0$ tal que $d(a_k,a_k^n) < \frac{\epsilon}{2} \quad \forall n \geq n_0$

  Ahora fijamos un $n \geq n_0$. Existe $k_0$ tal que $d(a_k^n,0) < \frac{\epsilon}{2} \quad \forall k \geq k_0$

  Esto quiere decir que el $\epsilon$ nos da algúnos $a_k^n$ para elegir e intercalar

  $d(a_k,0) \leq d(a_k,a_k^n) + d(a_k^n,0) < \frac{\epsilon}{2} + \frac{\epsilon}{2} = \epsilon \quad \forall k \geq k_0$

  Por lo tanto $a_k$ converge a 0 y entonces $(a_k)_k \subseteq c_0$. Por lo tanto $c_0$ es cerrado y entonces completo
  \end{proof}
\end{ej}
\begin{ej}
	Sea $(X,d)$ un espacio métrico y sea $D \subseteq X$ un subconjunto denso con la propiedad que toda sucesión de Cauchy $(a_n)_n \subseteq D$ converge a $X$. Probar que $X$ es completo 
	\begin{proof}
		Sea $(x_n)_n \subseteq X$ de Cauchy. 

		$D$ denso luego $\forall n \in \N$ existe un $d_n \in D$ tal que $d(x_n,d_n) < \frac{1}{n}$, ahora tenemos $(d_n)_n \subseteq D$ que cumple $d(d_n,x_n) \ra 0$

		Dado $\epsilon > 0$ existe $n_0 \in \N$ tal que $\frac{1}{n_0} < \frac{\epsilon}{3}$ también tal que $d(x_m,x_m) < \frac{\epsilon}{3} \quad \forall n,m \geq n_0$

		$d(d_m,d_n) \leq d(d_m,x_m) + d(x_m, x_n) + d(x_n, d_n) \leq \frac{1}{m} + \frac{\epsilon}{3} + \frac{1}{n} \leq \frac{1}{n_0} + \frac{\epsilon}{3} + \frac{1}{n_0} \leq \epsilon \quad \forall n,m \geq n_0$

		Entonces $d_n$ es de Cauchy y está contenida en $D$ por lo tanto converge a un  $x \in X$

		Ahora dado $\epsilon > 0$ como $d(d_n,x_n) \ra 0$ tenemos un $n_1$ tal que $d(d_n,x_n) < \frac{\epsilon}{2} \quad \forall n \geq n_1$ 

		Además por convergencia de $d_2$tenemos $n_2 \in \N$ tal que $d(d_n,x) < \frac{\epsilon}{2} \quad \forall n \geq n_2$

		Si tomamos $n_0 = \max\{n_1,n_2\}$ tenemos $d(x_n,x) \leq d(x_n,d_n) + d(d_n,x) \leq \epsilon \quad \forall n \geq n_0$

		Entonces $x_n \ra x \in X$ por lo tanto $X$ es completo

	\end{proof}
\end{ej}
\begin{ej}
Consideremos $\R$ en la métrica

	$$ d'(x,y) = \biggl |\frac{x}{1 + |x|} - \frac{y}{1 + |y|}\biggl |$$

	
\noindent (práctica 3, ej 5) d' es topológicamente equivalente a la métrica usual $d(x,y) = |x - y|$.
	
	Probar que $(\R,d')$ no es completo
	\begin{proof}
usando la sucesión $n$ y despues despejas y se cancelan cosas
	\end{proof}
\end{ej}
\begin{ej}
  Sea $f: (X,d) \ra (Y,d')$ un homemorfismo uniforme. Probar que $(X,d)$ es completo si y solo si $(Y,d')$ es completo.
  En particular si un espacio métrico $X$ es completo para una métrica lo es para cualquier otra métrica uniformemente equivalente
  \begin{proof}
  $\Ra )$
    Sea $(a_n)_n \subseteq Y$ de Cauchy , por ser $f^{-1}$ uniformemente continua entonces preserva Cauchy , luego $f^{-1}(a_n)$ es de Cauchy

  Como $X$ es completo entonces $b_n = f^{-1}(a_n)$ converge digamos a un $b = f^{-1}(a)$

  Ahora como $f$ continua y $b_n \ra b$ entonces $f(b_n) \ra f(b)$ o lo que es lo mismo $a_n \ra a$

  Entonces $a_n$ converge , por lo que $Y$ es completo

$\Leftarrow )$ Sea $(x_n)_n \subseteq X$ de Cauchy, entonces $f(x_n)$ es de Cauchy , por lo tanto $f(x_n) \ra f(x)$

Y por continuidad de la inverda $x_n \ra x$. Entonces $x_n$ converge, $X$ es completo
  \end{proof}
\end{ej}
\newpage
\begin{ej}
  Sean $X$ e $Y$ espacios métricos, $Y$ completo. Sea $D \subseteq X$ denso y sea $f: D \ra Y$ una función uniformemente continua. Probar que $f$ tiene una única extensión continua a todo $X$, es decir, existe una única función $F: X \ra Y$ continua tal que $F|_D = f $ 

  (Más aún, $F$ es uniformemente continua)	  
  \begin{proof}
    Sea $x \in X$ por ser $D$ denso tenemos que $(d_n)_n \in D$ tal que $d_n \ra x$ y esta converge, por lo tanto es de Cauchy.

    Como $f$ es uniforme continua $f(d_n)$ también es de Cauchy en $Y$, entonces converge a $d$

    Ahora podemos definir $F(x) = \lim f(d_n)$ para cualquier $d_n \ra x$

    Veamos que está bien definida. Supongamos que tenemos dos sucesiones que convergen a dicho $x \in X$ $x_n$ e $y_n$

    Entonces $d(x_n,y_n) \ra 0$ como $f$ uniforme entonces $d(f(x_n),f(y_n)) \ra 0$

	  Por lo tanto da igual que sucesión convergente a $x$ usemos la función $F$ queda unívocamente definida
  \end{proof}
\end{ej}
\end{document}
