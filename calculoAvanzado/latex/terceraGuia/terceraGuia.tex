\documentclass[11pt]{report}

\usepackage[margin=1in]{geometry}
\usepackage{enumerate}
\usepackage{amsmath}
\usepackage{amssymb}
\usepackage{mathtools}
\usepackage{amsfonts}
\usepackage{amsthm}
\usepackage{graphicx}
\usepackage{fancyhdr}
\pagestyle{fancy}

\newcommand{\n}{\aleph_{0}}
\newcommand{\F}{\mathhbb{F}}
\newcommand{\Q}{\mathbb{Q}}
\newcommand{\C}{\mathbb{C}}
\newcommand{\R}{\mathbb{R}}
\newcommand{\K}{\mathbb{K}}
\newcommand{\E}{\mathbb{E}}
\newcommand{\I}{\mathbb{I}}
\newcommand{\Z}{\mathbb{Z}}
\newcommand{\N}{\mathbb{N}}
\newcommand{\Ra}{\Rightarrow}
\newcommand{\ra}{\rightarrow}
\newcommand{\ol}{\overline}
\newcommand{\norm}[1]{\left\lVert#1\right\rVert}
\newcommand{\open}{\mathrm{o}}


\theoremstyle{definition}
\newtheorem{definition}{Definición}[section]
\newtheorem*{remark}{Observación}
\newtheorem{theorem}{Teorema}
\newtheorem{lemm}{Lema}
\newtheorem{corollary}{Corolario}[theorem]
\newtheorem{lemma}[theorem]{Lema}
\newtheorem{prop}{Proposición}
\newtheorem{ej}{Ejercicio}


\fancyhead[R]{Espacios Métricos}
\fancyhead[L]{Alumno Javier Vera}
\fancyhead[C]{Cálculo Avanzado}

\DeclarePairedDelimiter\Floor\lfloor\rfloor
\DeclarePairedDelimiter\Ceil\lceil\rceil

\begin{document}

\begin{ej}
	Sea $(X,d)$ un espacio métrico y sea $(x_n)_n \subseteq X$. Probar:
	\begin{enumerate}
		\item $\lim x_n = x$ si y sólo si para toda subsucesión $(x_{n_k})_k, \quad \lim_{k \ra \infty} x_{n_k} = x$
			\begin{proof}
				$\Ra )$	Como $x_n$ converge dado $\epsilon >0$ existe $n_0 \in \N$ tal que $d(x_n,x) < \epsilon \quad \forall n \geq n_0$

				Por ser $x_{n_k}$ subsucesión tenemos que $n_{k_0} \geq n_0 $ entonces $d(x_{n_k},x) < \epsilon \quad \forall n_k \geq n_{k_0}$

				Y esto lo podemos hacer con cualquier epsilon por lo tanto $x_{n_k}$ converge a $x$

				$\Leftarrow )$ Si vale para toda subsucesión en particular vale para $x_n$ que es una subsucesión, entonces $x_n \ra x$
			\end{proof}

		\item Si existe $x \in X$ para el cual toda subsucesión $(x_{n_k})_k$ de $(x_n)_n$ tiene un subsucesión $(x_{n_{k_j}})_j$ tal que $\lim_{j} x_{n_{k_j}} = x$, entonces $(x_n)_n$ converge y $\lim_n x_n = x$
			
		\begin{proof}
			Supongamos que $x_n$ no converge a $x$.

			Entonces $\exists \epsilon >0$ tal que para cada $N \in \N$ $\exists n_0 > N$ con $d(x_{n_0},x) \geq \epsilon$

			Entonces para $N_1 \in \N$ nos quedamos con un $n_1 \geq N_1$ tal que $d(x_{n_1},x) \geq \epsilon$

			Y ahora tomamos $N_2 > n_1$ tiene que existir un $n_2 > N_2 > n_1$ tal que $d(x_{n_2},x) \geq \epsilon$

			Si nos vamos quedando con todos los $x_{n_k}$ y teniendo en cuenta como los tomamos nos aseguramos que $x_{n_k}$ es una subsucesión de $x_n$

			Pero entonces existe $x_{n_{k_j}}$ convergente a $x$. Y esto es absurdo , por que TODOS los términos de $x_{n_k}$ cumplían $d(x_{n_k},x) \geq \epsilon$ por ende como $x_{n_{k_j}}$ es sub tiene que cumplir lo mismo entonces $d(x_{n_{k_j}},x) \geq \epsilon \quad \forall j \in \N$
		\end{proof}

		\item Si $(x_n)_n$ es convergente entonces es de Cauchy, ¿Vale la recíproca?

		\begin{proof}
			$\Ra )$ Sea $x_n$ convergente a $x$ 

			Tomemos $\epsilon > 0$ sabemos que existe $n_0 \in \N$ tal que $d(x_n,x) < \frac{\epsilon}{2} \quad \forall n \geq n_0$

			Ahora si miramos $d(x_n,x_j) \leq d(x_n,x) + d(x,x_j) \leq \frac{\epsilon}{2} + \frac{\epsilon}{2} = \epsilon \quad \forall n,j \geq n_0$

			La vuelta no vale cuando no es completo. Por ejemplo los racionales, tenemos la sucesion $x_n = \sqrt{2} + \frac{1}{n}$ es facil ver que es de Cauchy, sin embargo no converge en $\Q$ por que $\sqrt{2}$ no está en $\Q$
		\end{proof}
		
	\item Si $(x_n)_n$ es de Cauchy, entonces es acotada.

		\begin{proof}
		Dado $\epsilon > 0$ sabemos que existe $n_0 \in \N$ tal que $d(x_n,x_m) \leq \epsilon \quad \forall n,m \geq n_0$

		Entonces si fijo $x_n$ con $n \geq n_0$ tenemos que $d(x_n,x_m) \leq \epsilon \quad \forall m \geq n_0$

		Equivalentemente $x_m \in B(x_n,\epsilon) \quad \forall m \geq n_0$ entonces todos los términos mayores que $n_0$ están acotados. Llamemosle a su cota $M$

		Y los anteriores son finitos, entonces cada uno es menór que un $M_i$

		Entonces puedo tomar una cota para todos $C = \max_{i}\{M_i,M\}$
		\end{proof}
	\item Si $(x_n)_n$ es de Cauchy y tiene un subsucesión $(x_{n_k})_k$ convergente a $x \in X$, entonces $(x_n)_n$ converge a $x$
		\begin{proof}
			Dado un $\epsilon >0$ tenemos un $n_0 \in \N$ tal que $d(x_n,x_m) \leq \frac{\epsilon}{2} \quad \forall n,m \geq n_1$

			Y además tenemos un $n_2 \in \N$ tal que $d(x_{n_k},x) \leq \frac{\epsilon}{2} \quad \forall n_k \geq n_2$

			Si $n_0 = \max\{n_1,n_2\}$ luego $d(x_n ,x )\leq d(x_n,x_{n_k}) + d(x_{n_k},x) = \frac{\epsilon}{2} + \frac{\epsilon}{2} = \epsilon \quad \forall n_k,n \geq n_0$
		\end{proof}
	\end{enumerate}
\end{ej}

\begin{ej}
	Probar que si toda bola cerrada de un espacio métrico $X$ es un subespacio completo de $X$ entonces $X$ es completo.

	\begin{proof}
		Tomemos $(x_n)_n \subseteq X$ de Cauchy. Sabemos entonces que está acotada , por ende a partir de un momento podemos meterla en una bola en particular cerrada y para los elementos que nos quedan miramos de todos ellos el que nos genere una distancia mas grande a la bola , ahora sumamos ese radio al de la bola y tenemos una nueva bola cerrada que tiene todos los elementos de la sucesión.

		Como es una bola cerrada es una espacio métrico completo , por ende $x_n$ que estaría contenida en dicho nuevo espacio y ademas sería de cauchy por que el nuevo espacio tiene la misma métrica, tiene que ser convergente a un $x$ en ese espacio , pero ese espacio es un subespacio de $X$ por ende $x_n$ converge a algo en $X$. Y esto lo podemos hacer con cualquier sucesión de Cauchy.

		Por ende $X$ es completo
	\end{proof}
\end{ej}
	\begin{ej}
		Sean $A$ y $B$ subespacios de un espacio métrico. Probar que si $A$ y $B$ son completos, entonces $A \cup B$ y $A \cap B$ son completos.
		\begin{proof}
			Sea $(x_n)_n \subseteq A\cap B$ de Cauchy entonces $(x_n)_n \subseteq A$ por lo tanto $x_n$ converge en $A$

			Entonces $x_n$ converge en $A$.

			Por otro lado $(x_n)_n \subseteq B$ entonces $x_n$ converge en $B$

			Por unicidad de límite $x_n$ converge en $A \cap B$, entonce $A \cap B$ es completo

			Sea $(x_n)_n \subseteq A\cup B$ de Cauchy. Entonces $(x_n)_n \subseteq A$ entonces $x_n$ converge en $A$

			Por lo tanto $x_n$ converge en $A \cup B$, entonces $A \cup B$ es completo
		\end{proof}
	\end{ej}
	
	\begin{ej}
		Sea $(X,d)$ un espacio métrico.
		\begin{enumerate}
			\item Probar que todo subespacio completo de $(X,d)$ es un subconjunto cerrado de $X$
				\begin{proof}
					Sea $(Y,d)$ subespacio métrico completo de $(X,n)$

					Ahora si tomamos $(x_n)_n \subseteq Y$ convergente, sabemos que entonces es de Cauchy, pero entonces converge en $Y$ dado que es completo. Por lo tanto $Y$ es cerrado y está contenido en $X$, es un cerrado de $X$. (En realidad se puede ver que es cerrado en cualquier lado , por ser completo. Por que completitud es una propiedad intrínseca)
				\end{proof}

			\item Probar que si $X$ es completo, entonces todo subconjunto $F\subseteq X$ cerrado, es un subespacio completo de $X$
				\begin{proof}
					Sea $(x_n)_n \subseteq F$ una sucesión de Cauchy, esta misma es entonces una sucesión de Cauchy de $X$, entonces converge.

					Ahora si miramos a $F$ como subconjunto de $X$ sabemos que es cerrado

					Entonces $x_n$ es una sucesión de $F$ que converge, como $F$ es cerrado , tiene que converger allí

					Entonces $x_n$ converge en $F$

					Esto vale para cualquier sucesion de cauchy de $F$. Por lo tanto $F$ es completo

				\end{proof}
		\end{enumerate}
	\end{ej}
	
	\begin{ej}
		Sean $(X,d)$ e $(Y,d')$ espacios métricos. Consideramos en $X \times Y$ la métrica $d_{\infty}$ definida por

		$$ d_{\infty}((x_1,y_1),(x_2,y_2)) = \max\{d(x_1,x_2),d'(y_1,y_2)\}$$

		Probar que $(X\times Y, d_{\infty})$ es completo si y sólo si $(X,d)$ e $(Y,d')$ son compleots.
		\begin{proof}
			$\Ra )$ Sean $(x_n)_n$ e $(y_n)_n$ sucesiones de Cauchy de $X$ e $Y$ respectivamente.

			Entonces dado $\epsilon >0$ existe $n_1 \in \N$ tal que $d(x_n,x_m) \leq \epsilon \quad \forall n,m \geq n_1$ 

			También existe $n_2 \in \N $ tal que $d'(y_n,y_m) \leq \epsilon \quad \forall n,m \geq n_2$

			Tenemos entonces una sucesión $(x_n,y_n)_n \subseteq X \times Y$, veamos que es de Cauchy

			$d_{\infty}((x_n,y_n),(x_m,y_m)) = \max \{d(x_n,x_m),d(y_n,y_m)\}$

			Podemos suponer que el máximo es cualquier de las dos sin pérdida de generalidades.

			Entonces $d_{\infty}\{((x_n,y_n),(x_m,y_m)) = d(x_n,x_m) \leq \epsilon \quad \forall n,m \geq n_1$. 

			Esto lo puedo hacer para cualquier $\epsilon$. Por lo tanto $(x_n,y_n)$ es de Cauchy, entonces converge

			Supongamos que converge a $(x,y)$ 

			Entonces $ \max \{d(x_n,x),d'(y_n,y)\} = d_{\infty}((x_n,y_n),(x,y)) \ra 0$

			Pero el máximo es mas grande que las dos distancias, por lo tanto es mas grande que cualquier de ellas 

			Entonces $ 0 < d(x_n,x) \leq d_{\infty}((x_n,y_n)(x,y)) \ra 0$. Entonces $d(x_n,x) \ra 0$ 

			Y lo mismo pasa con $(y_n,y)$. Mostrando que ambas convergen. 

			Entonces $X$ es de Cauchy e $Y$ es de Cauchy

			$\Leftarrow$) Sea $(x_n,y_n)_n \subseteq X\times Y$ una sucesión de Cauchy.

			Dado $\epsilon > 0$ existe $n_0 \in \N$ tal que $d_{\infty}((x_n,y_n),(x_m,y_m)) \leq \epsilon \quad \forall n,m\geq n_0$ 
			
			Entonces por como está  definida la distancia infinito. Tenemos que $d(x_n,x_m) \leq \epsilon \quad \forall n,m \geq n_0$

			Y lo mismo con $y_n$. Esto es por que ambas distancias son mas pequeñas que la distancia infinito.

			Pero entonces ambas sucesiones son de Cauchy en sus respectivos espacios completos, por ende convergen

			Entonces usando el $\epsilon$ que teníamos sabemos que existe $n_1 \in \N$ tal que $d(x_n,x) \leq \epsilon \quad \forall n\geq n_1$ 

			Y también existe $n_2 \in \N$ tal que $d(y_n,y) \leq \epsilon \quad \forall n\geq n_2$.

			Ahora si tomamos $n_0 = \max\{n_1,n_2\}$ tenemos que $d_{\infty}((x_n,y_n),(x,y)) \leq \epsilon \quad \forall n \geq n_0$

			Entonces $(x_n,y_n)$ converge, y lo podemos hacer con cualquier sucesión de Cauchy.

			Entonces $X\times Y$ es completo
		\end{proof}
	\end{ej}

	\begin{ej}
		\begin{enumerate}
			\item Sea $X$ un espacio métrico y sea $B(X) = \{f: X\ra \R / f $ es acotada $\}$. 

				Probar que $(B(X),d_{\infty})$ es un espacio métrico completo, donde $d_{\infty}(f,g) = \sup_{x\in X}|f(x)-g(x)|$

				\begin{proof}
					Sea $f_n$ de Cauchy entonces dado $\epsilon >0$ tenemos $d(f_n,f_m) \leq \epsilon \quad \forall n,m\geq n_0$	

					Por lo tanto $\sup_{x\in X}|f_n(x) - f_m(x)| < \epsilon \quad \forall n,m \geq n_0$

					En particular si fijo $x \in X$ tenemos que $|f_n(x) - f_m(x)| < \epsilon \quad \forall n,m \geq n_0$

					Esto nos dice que $(f_n(x))_n \subseteq \R$ es de Cauchy, como $\R$ es completo entonces converge

					Con lo cual $(f_n(x))_n$ converge a $f(x)$. Y esto vale con cualquier $x \in X$ 

					Vamos a proponer esta $f$ como límite de $f_n$

					Por convergencia de $f_n(x)$ dado un $\epsilon >0$ existe $n_0 \in \N$ tal que $|f_n(x) - f(x)| < \epsilon \quad \forall n \geq n_0$

					Si fijamos un $x \in X$ tenemos que $|f_n(x)-f_m(x)| < \epsilon \quad \forall n,m\geq n_0$

					Si hacemos limite de $n$ tendiendo a infinito $|f(x) - f_m(x)| \leq \epsilon \quad \forall m \geq n_0$

					Y esto vale para cada $x \in X$ entonces $\sup_{x\in X}|f(x)-f_m(x)| = d_{\infty}(f,f_m) \leq \epsilon$

					Por lo tanto $f_n$ converge a $f$ si es que $f$ está en el conjunto.

					Tomando el mismo $\epsilon$ y una $f_n$ cualquier con $n \geq n_0$ tenemos que 
					$$|f_n(x) - f(x)| \leq \epsilon \quad \forall x \in X \iff f_n(x) - \epsilon \leq f(x) \leq \epsilon + f_n(x)$$

					Y sabemos que $f_n$ es acotada entonces $M_1< f_n(x) < M_2 \quad \forall x \in X$

					Entonces $M_1 - \epsilon < f(x) < M_2 + \epsilon $ acá $\epsilon$ está fijo y es un número real.

					Por lo tanto $f$ está acotada
				\end{proof}
				
				
			\item Sean $a,b\in \R, a < b$. Probar que $(C[a,b],d_{\infty})$ es un espacio métrico completo, donde $d_{\infty}(f,g) = \sup_{x\in [a,b]}|f(x)-g(x)|$

				\begin{proof}
					Dado que $[a,b]$ es compacto y continua manda compactos en compactos, estas funciones son todas acotadas 

					Por que su imagen es tta entonces es acotada. Entonces basta ver que el conjunto es cerrado.

					Tomo una sucesión $(f_n)_n \subseteq (\mathcal{C}[a,b],d_{\infty})$ tal que $f_n$ converge a $f$

				Entonces dado $\epsilon > 0 $ existe $n_0\in \N$ tal que $d_{\infty}(f_n,f) < \frac{\epsilon}{3} \quad \forall n \geq n_0$.

				Esto quiere decir que $\sup_{x\in X} |f_n(x) - f(x) | < \frac{\epsilon}{3} \quad \forall n \geq n_0$

				Entonces $|f_n(x)-f(x)| < \frac{\epsilon}{3} \quad \forall x \in X$ si tomo cualquier $n \geq n_0$ fijo

				Además $f_n$ con ese $n \in \N$ fijo es continua entonces en un compacto, entonces es uniformemente contínua 
			$$\exists \delta >0 \ / \ d(x,y) < \delta \Ra d(f_n(x),f_n(y)) < \frac{\epsilon}{3}$$

				Ahora usando ese delta tenemos que $d(x,y) < \delta$ implica 
				$$d(f(x),f(y)) < d(f(x),f_n(x)) + d(f_n(x),f_n(y)) + d(f_n(y),f(y)) \leq \frac{\epsilon}{3} + \frac{\epsilon}{3} + \frac{\epsilon}{3}= \epsilon  $$

				Por lo tanto $f$ es continua.

				Entonces está en $\mathcal{C}[a,b]$ mostrando que la sucesión converge en el conjunto. 

			 	Finalmente $\mathcal{C}[a,b]$ es cerrado, por lo tanto es compoeto

				\begin{remark} Sin usar compacto, tenemos la demo de convergencia uniforme implica puntual
					
				        Por convergencia uniforme para cualquier $x \in X$ sucede $d(f(x),f_n(x)) < \frac{\epsilon}{3} \quad \forall n\geq n_0$ 

					Además $f_{n_0}$ con $n_0 \in \N$ fijo es contínua en todo punto , en particular lo es en $x_0$

					Por lo tanto dado $\epsilon$ existe un $\delta$ tal que $d(x,x_0) < \delta \Ra d(f_{n_0}(x),f_{n_0}(x_0))< \frac{\epsilon}{3} = \epsilon$.

					Entonces $f$ es contínua en $x_0$ y esto lo podemos hacer para cualquier $x_0 \in X$


				Ahora usando aquél delta tenemos $d(x,x_0) < \delta$ implica:
				$$ d(f(x),f(x_0)) < d(f(x),f_{n_0}(x)) + d(f_{n_0}(x),f_{n_0}(x_0)) + d(f_{n_0}(x_0),f(x_0)) $$
				\end{remark}
				\end{proof}
				
				
			\item Probar que $C_0 := \{(a_n)_n \subseteq \R / a_n \ra 0 \}$ se un espacio métrico completo con la distancia $d_{\infty}((a_n),(b_n)) = \sup_{n\in\N} |a_n-b_n|$

				Estas son funciónes que van de $\N \ra \R$ y además por converge a 0, están acotadas, dado que tienen finitos términos distintos que 0

				Probemos entonces que es cerrado y sería cerrado en un completo, por lo tanto completo

				Sea $a^k $ convergente a $a$ entonces $\sup_{n \in \N}|a^{k}_n - a_n|=  d_{\infty}(a^k_n,a_n) < \frac{\epsilon}{2} \quad \forall k\geq k_0$
				
				Entonces para cada $n \in \N$ tenemos $d(a_n,a^{k_0}_n) < \frac{\epsilon}{2}$
			
				Además dado ese $k_0$ sabemos que $d(a^{k_0}_n , 0 ) < \frac{\epsilon}{2} \quad \forall n \geq n_0$

				Finalmente dado $\epsilon$ tenemos:

			$$ d(a_n,0) < d(a_n,a^{k_0}_n) + d(a^{k_0}_n , 0) < \epsilon \quad \forall n\geq n_0$$

			Entonces $a_n$ converge a 0 por lo tanto $(a_n)_n \subseteq C_0$. Mostrando que $C_0$ es cerrado
		\end{enumerate}
	\end{ej}


	\begin{ej}
		Sea $(X,d)$ un espacio métrico y sea $\mathcal{D} \subseteq X$ un subconjunto denso con la propiedad que toda sucesión de Cauchy $(a_n)_n \subseteq \mathcal{D}$ converge en $X$. Probar que $X$ es completo.
		\begin{proof}
			Sea $(x_n)_n \subseteq X$ un sucesión de Cauchy.

			Como $\mathcal{D}$ es denso, para cada $x_n$ existe un $d_n \in D$ tal que $d(x_n,d_n) \leq \frac{1}{n}$

			Veamos que $d_n$ es de Cauchy.

			Ahora dado un $\epsilon > 0 $ tenemos sabemos que existe  $n_1 \in \N$ tal que $\frac{2}{n_1} + \epsilon ' \leq \epsilon$

			Dado dicho $\epsilon ' > 0 $ sabemos por Cauchy existe un $n_2\in \N$ tal que $d(x_n,x_m )\leq \epsilon ' \quad \forall n,m \geq n_1$

			Ahora si tomamos $n_0 = \max\{n_1,n_2\}$ Tenemos que:
			
			$$d(d_n,d_m) < d(d_n,x_n) + d(x_n,x_m) + d(x_m,d_m) = \frac{1}{n} + \epsilon ' + \frac{1}{m} \leq \frac{2}{n_1} + \epsilon '=\epsilon \quad \forall n,m \geq n_0$$

			Entonces $d_n$ es de Cauchy , como está contenida en $D$ converge a un $d$

			Ahora propongo mi $d$ como candidato a límite de $a_n$.

		Dado $\epsilon >0$ sabemos que existe un $n_1 \in \N$ tal que $\frac{1}{n_1} + \frac{\epsilon}{2} < \epsilon$

			dado que $d_n$ converge a $d$, tenemos que existe $n_2$ tal que $d(d_n,d) \leq \frac{\epsilon}{2} \quad \forall n \geq n_2$

			Si tomamos nuevamente $n_0 = \max\{n_1,n_2\}$

			$d(a_n,d) \leq d(a_n,d_n) + d(d_n,d) \leq \frac{1}{n} +\frac{\epsilon}{2} \leq \frac{1}{n_1}+\frac{\epsilon}{2}  = \epsilon \quad \forall n \geq n_0 $ 

			Entonces $a_n$ converge a $d$, como $d\in D \subseteq X$ entonces $d \in X$.

			Finalmente $a_n$ converge en $X$. Entonces $X$ es completo
		\end{proof}
	\end{ej}

	\begin{ej}
	$Teorema$ $de$ $cantor$

	Probar que un espacio métrico $(X,d)$ es completo si y sólo si toda familia $(F_n)_n$ de suconjuntos de $X$ cerrados, no vacíos tales que $F_{n+1} \subset F_n$ para todo $n \in \N$ y $diam(F_n) \ra 0$ tiene un único punto en la intersección

		\begin{proof}
			$\Ra$) Tenemos una familia de subconjuntos que cumple las propiedades dadas.

			Entonces podemos armar una sucesión $x_n$ donde cada elemento es algún elemento de $F_n$

			Veamos que es de Cauchy. Dado $\epsilon >0$ y considerando que $diam(F_n) \ra 0$. Sabemos que existe algún cerrado $F_{n_0}$ tal que $diam(F_{n_0}) < \epsilon$.

			Ahora dado cualquier par de elementos $x_n, x_m$ tales que $n,m \geq n_0$ sabemos que pertenecen a algún cerrado que está contenido en $F_{n_0}$. Por lo tanto $d(x_n,x_m) \leq diam(F_{n_0}) < \epsilon$.

			Entonces $(x_n)_n$ es de Cauchy. Ahora por ser $X$ completo, $x_n$ converge digamos a $x$

			Ahora, para dado un $F_{n_0}$ sabemos $x_n \in F_{n_0} \quad \forall n \geq n_0$ entonces $F_{n_0}$ contiene una subsucesión de $x_n$, por ser subsucesión converge a $x$ también y por ser cerrado $x\in F_{n_0}$ 

			Esto vale para cualquier $F_{n}$ por lo tanto $x$ está en la intersección de todos ellos

			Ahora supongamos que existe $x'\neq x$ en la intersección. 

			Obviamente $x_n$ no puede converger a él. Por unicidad de convergencia. 

			Por lo tanto existe $\epsilon > 0$ tal que $\forall n_0\in \N$ existe $n \geq n_0$ tal que $d(x_n,x') > \epsilon$

			Ahora , sabemos que existe $n_0 \in \N$ tal que $diam(F_{n_0}) < \epsilon$ y además  $(x_n)_n \subseteq F_{n_0} \quad \forall n \geq n_0$.

			Pero entonces $x' \notin F_{n_0}$ por que si no $d(x_n,x) < \epsilon \quad \forall n\geq n_0$

			Por lo tanto $x' \notin \bigcap_{i \in I} F_i$ lo que es absurdo, provino de suponer que existia un $x'\neq x$ en la intersección, entonces no existe otro elemento diferente de $x$ en la intersección

			Otra forma mas facil es notar que como $x,x'$ están en la intersección $d(x,x') \ra 0$

			Entonces $x = x'$

			$\Leftarrow$ ) Sea $(x_n)_n \subseteq X$ una sucesion de Cauchy supongamosla no constante, si lo fuera ya sabemos que converge.

			Entonces nos armamos $F_j = \{(x_n)_n \subseteq X : n \geq j \}$ y luego los clausuramos. 

			Está claro que $F_{j+1} \subseteq F_j$. Y además por cauchy sabemos que $diam(F_n) \ra 0$. 

			Esto vale por que para cada $\epsilon = \frac{1}{j}$ existe un $n_0 \in \N$ tal que $d(x_n,x_m) \leq \epsilon \quad \forall n,m \geq n_0$. 

			Pero entonces en algún momento $j = n_0$ y esto lo podemos hacer para cualquier epsilon , va a existir algún $j = n_0(\epsilon)$.


			Entonces estamos dentro de las hipótesis, por lo tanto podemos afirmar que existe un único punto $x$ tal que $x \in \bigcap_{j \in J} F_j$. 
		
			Proponemos ese $x$ como punto al que converge $x_n$
	
			Dado un $\epsilon$ tenemos algún $j$ tal que $diam(F_j) < \epsilon$.

			Entonces $x_n \in F_j \quad \forall n \geq j$ y además $x\in F_j$ por pertenecer a la intersección de todos

			Por lo tanto $d(x_n,x) < \epsilon \quad \forall n\geq j$
		\end{proof}	
	\end{ej}
	\begin{ej}
		Sea$(X,d)$ un espacio métrico completo sin puntos aislados. Probar que $X$ tiene cardinal mayor o igual que $\mathfrak{c}$. Deducir que si además $X$ es separable, entonces $\# X = \mathfrak{c}$ (Para esto último, puede servir un ejercicio de la práctica anteriór)

		\begin{proof}
			Tomo dos puntos $x_1,x_2 \in X$ diferentes, se que no son aislados, 

			Ahora tomo dos bolas cerradas. $r = \frac{d(x_1,x_2)}{2},  \ol B(x_1,r), \ol B(x_2,r)$ estas son disjuntas.

			Ahora miramos $\ol B(x_1,r)$, tomamos dos puntos $x_{1,2} \neq x_{1,1}$ que estén en la bola. 

			Dichos punto existen, por que si nó $x_1$ , sería aislado

			Tomando $r' = \min\{\frac{d(x_{1,1},x_{1,2})}{2},r - d(x_{1,2},x_1), r-d(x_{1,1},x_1)\}$

			Tenemos dos bolas $\ol B(x_{1,1},\frac{r'}{2})$ y $\ol B(x_{1,2},\frac{r'}{2})$ ambas contenidas en $\ol B(x_1,r)$ y disjuntas

			Si seguimos haciendo esto infinitamente siempre con alguna bola, vamos a tener una sucesión de cerrados encajados de diametro tendiendo a 0, y como el espacio es completo , esto implica que existe un $x \in X$ en la intersección de dichas bolas.

			Ahora si prestamos atención en cada paso podemos elejir entre dos bolas, por lo tanto las elecciones que tomamos nos dan una sucesión de 1s y 2s. 

			Entonces para cada una de estas sucesiones de 1s y 0s distintas, tenemos un $x \in X$

			Por lo tanto $\# X \geq \# \{1,2\}^{\N} = 2^{\n} = \mathfrak{c}$

			Si además $X$ es separable, tenemos un denso numerable $D \subseteq X$

			Entonces como es denso para cada $x \in X$ tenemos una sucesión de $D$ que converge a él

			Por lo tanto $\# X \leq \# D^{\N} = \# \N^{\N} = \mathfrak{c}$
		\end{proof}
		
	\end{ej}

	\begin{ej}
		Sean $(X,d)$ e $(Y,d')$ espacios métricos y sea $f: X \ra Y$. Probar que:
		\begin{enumerate}
			\item $f$ es continua en $x_0 \in X$ si y sólo si para toda sucesión $(x_n)_n \subset X$ tal que $x_n \ra x_0$. La sucesión $(f(x_n))_n \subset Y$ converge a $f(x_0)$

			\begin{proof}
				La ida es trivial.

				$\Leftarrow ) $ Supongamos que $f$ no es continua en $x_0$ entonces dado $\epsilon >0 $ sabemos que $\forall \delta >0 $ si $d(x,x_0) < \delta \Ra d(f(x),f(x_0)) > \epsilon$.

				Ahora si tomamos $\delta = \frac{1}{n}$ y para cada delta me quedo con algún $x_n$

				Por lo tanto tengo una sucesión $x_n$ que converge a $x$

				Sin embargo $d(f(x_n),f(x)) \geq \epsilon $ lo que es absurdo, provino de suponer que $f$ no era continua.

				Entonces $f$ es continua.
			\end{proof}
			\item 	Son equivalentes:
				\begin{enumerate}[(a)]
					\item $f$ es continua
					\item Para todo $G \subseteq Y$ abierto, $f^{-1}(G)$ es abierto en $X$
					\item Para todo $F \subseteq Y$ cerrado, $f^{-1}(F)$ es cerrado en $X$
				\end{enumerate}
				\begin{proof}
					$(a) \Ra (b) $ Sea $x_0 \in f^{-1}(G)$ qvq existe un entorno abierto $V$ de $x_0$

					$f(x) \in G$ que es abierto, entonces existe $\epsilon>0$ tal que $B(f(x_0),\epsilon) \subseteq G$

					Entonces por continuidad existe un $\delta > 0 $ tal que $f(B(x_0,\delta)) \subseteq B(f(x_0),\epsilon)\subseteq G$

					Entonces $f^{-1}(f(B(x_0),\delta))\subseteq f^{-1}(G)$, por lo tanto $B(x_0,\delta) \subseteq f^{-1}(G)$

					$(a) \Ra (c)$ Sea $(x_n)_n \in f^{-1}(F)$ convegente a $x$ entonces $(f(x_n))_n \subseteq F $.

					Por continuidad $f(x_n) $ converge a $f(x)$ y como $F$ es cerrado entonces $f(x) \in F$

					Por lo tanto $x = f^{-1}(f(x)) \in f^{-1}(F)$. Entonces probamos que toda sucesión de $f^{-1}(F)$ que converge, converge en $f^{-1}(F)$.

					Por lo tanto $f^{-1}(F) $ es cerrado

					$(b) \Ra (a)$ Tomemos $x_0 \in X$ veamos que $f$ es continua en $x_0$

					Como $x_0 \in X$ existe $y \in Y$ tal que $f(x_0) = y$. 

					Ahora dado $\epsilon >0$ tenemos que $B(y,\epsilon)$ es un abierto de $Y$

					Por lo tanto $f^{-1}(B(y,\epsilon))$ es abierto de $X$ y además $f^{-1}(y) = x_0$ pertenece a ese conjunto trivialmente

					Entonces por ser abierto existe $\delta >0$ tal que $B(x_0,\delta) \subseteq f^{-1}(B(y,\epsilon)) = f^{-1}(B(f(x_0),\epsilon))$

					Entonces aplicando $f$ de ambos lados tenemos que $f(B(x_0),\delta)\subseteq B(f(x_0),\epsilon)$.

					Por lo tanto $f$ es continua en $x_0$

					También podríamos haber usado que si $x\in B(x_0,\delta)$ entonces $d(x,x_0) < \delta$

					Y además $x \in f^{-1}(B(f(x_0),\epsilon))$, por lo tanto $f(x) \in B(f(x_0), \epsilon)$

					Entonces $d(f(x),f(x_0)) < \epsilon$, por lo tanto dado un $\epsilon$ encontramos un $\delta$ que cumple lo necesario

					$(c) \Ra (b)$ Probemos que dad cualquier funcione $f$ y cualquier conjunto $A$ entonces

					$$ f^{-1} (X\setminus A) = X \setminus f^{-1}(A)$$

					$\subseteq ) $ Sea $x \in f^{-1}(X\setminus A)$ entonces $f(x) \in X \setminus A$ por lo tanto $f(x) \notin A$

					Entonces $x \notin f^{-1}(A)$. Finalmente $x \in X \setminus f^{-1}(A)$

					$\supseteq )$ Sea $x \in X \setminus f^{-1}(A)$ entonces $x \notin f^{-1} (A) $.

					Luego $f(x) \notin A$ entonces $f(x) \in X \setminus A$, finalmente $x \in f^{-1}(X \setminus A)$

					Tomemos un abierto $G \subseteq Y$ por lo que acabamos de probar $X \setminus f^{-1}(G) = f^{-1}(X \setminus G)$.

					Como $G$ es abierto $X \setminus G$ es cerrado, entonces por hipótesis $f^{-1}(X \setminus G)$ es cerrado.

					Entonces $X \setminus f^{-1}(G)$ es cerrado , por lo tanto $f^{-1}(G)$ es abierto
				\end{proof}
				
				
		\end{enumerate}
	\end{ej}
	
	\begin{ej}
		Decidir cuáles de las siguientes funciones son continuas:
			\begin{enumerate}
				\item $f: (\R^2, d) \ra (\R, |.|), f(x,y) = x^2 + y^2$. Donde $d$ es la métrica euclídea

				\item $id_{\R^2}: (\R^2,\delta) \ra (\R^2,d_{\infty})$. La función identidad, donde $\delta$ representa la métrica discreta.
					\begin{proof}
						Sabemos que en un conjunto con la métrica discreta, todo subconjunto es abierto y cerrado, por lo tanto si agarro cualquier abierto en la imagen , su preimagen es un abierto, entonces es continua
					\end{proof}
				 		
				\item $id_{\R^2}: (\R^2,d_{\infty}) \ra (\R^2,\delta)$. La función identidad, donde $\delta$ representa la métrica discreta.

					\begin{proof}
						Sea $(x_n,y_n) = (\frac{1}{n},\frac{1}{n})$ esta sucesión converge a 0 con $d_{\infty}$

						Dado $\epsilon >0$ existe $n_0 \in \N$ tal que $\frac{1}{n_0} < \epsilon$

						Entonces $d((x_n,y_n),0) = max\{|x_n - 0 |,|y_n - 0|\} = max\{|\frac{1}{n}|,|\frac{1}{n}|\} = \frac{1}{n} \leq \frac{1}{n_0} < \epsilon \quad \forall n\geq n_0$

						Sin embargo $f((x_n,y_n))$ no converge dado que no es constante a partir de ningún momento, y en un espacio discreto las unicas sucesiones convergentes son las que son constantes a partir de algún momento
					\end{proof}

				\item $i:(E,d) \ra (X,d)$, la inclusión, donde $E \subset X$

					\begin{proof}
						Dado un abierto  $V \subseteq X$ sabemos que $V \cap E$ es un abierto relativo a $E$
					\end{proof}
					
				\item $f:(\mathcal{C}[0,1],d_{\infty}) \ra (\R,|\cdot|), f(\phi) = \phi (0)$

					Veamos que es continua en $\phi_0$. Dado $\epsilon >0$ tenemos si pedimos $\delta = \epsilon$ alcanza

					Veamosló: $d(\phi,\phi_1) = \sup_{x\in [0,1] }|\phi(x)-\phi_1 (x)| < \delta$

					Pero entonces en partcular evaluar en el cero también va a ser menór.

					Entonces $|f(\phi) - f(\phi_1)| = |\phi(0) - \phi_1(0)| < \delta = \epsilon $

					Entonces dado $\epsilon$ tomamos $\delta = \epsilon$ y viemos que $d_{\infty}(\phi,\phi_1) < \delta \Ra d_{|\cdot|}(f(\phi),f(\phi_1)) < \epsilon$

					Por lo tanto $f$ es continua en $\phi_0$ y esto lo podemos hacer para cualquier función del dominio.

					Por ende $f$ es continua en todo el dominio					
					
			\end{enumerate}
	\end{ej}

	\begin{ej}
		Sean $f,g,h : [0,1] \ra \R$

		\[
 		 f(n) =
  			\begin{cases}
                                   0 & \text{if $x\notin \Q$} \\
                                   1 & \text{if $x\in \Q$} \\
  			\end{cases}
		\]

		$$g(x) = x.f(x)$$

		\[
 		 h(n) =
  			\begin{cases}
                                   0 & \text{if $x\notin \Q$} \\
				   \frac{1}{n} & \text{if $x =\frac{m}{n} \ , \ (m:n)=1$} \\
				   1 & \text{if $x =0$} \\
  			\end{cases}
		\]


		Probar que 

		\begin{enumerate}
			\item $f$ es discontinua en todo punto.
				\begin{proof}
					Sea $x_0 \in \Q$, ahora sea $\epsilon = \frac{1}{2}$ entonces 

					Entonces para cualquier $\delta >0$ que tomemos sabemos que $B(x_0,\delta)$ contiene algún irracional, por densidad.

					Entonces $0 \in f(B(x_0,\delta)) \subseteq B(f(x_0),\epsilon) = B(1,\frac{1}{2})$, lo que es absurdo.

					Entonces $f$ no es continua en $x_0$ entonces no es continua en ningún racional 

					Con el mismo argumento tendríamos que  $1 \in B(0,\frac{1}{2})$

				\end{proof}
			\item $g$ sólo es continua en $x=0$
				\begin{proof}
					Dado $\epsilon >0 $, si tomamos $\delta = \epsilon$ sucede:

					$ g(B(0,\delta)) \subseteq [0,\delta]  \cap \Q \subseteq B(0,\epsilon) = B(f(0),\epsilon) $

					Por lo tanto es continua en $0$

					Supongamos $f$ es continua en un $x_0 \in \Q$ con $x_n \neq 0$. 

					Ahora si agarramos una sucesión de irracionales $x_n$ que converga a $x_0$ (podemos hacerlo por densidad de irracionales en $[0,1]$)

					Tenemos que $g(x_n) = 0 $ por lo tanto $f(x_n)$ converge a $0$ pero $g(x) = x \neq 0$

					Entonces $g(x_n)$ no converge a $g(x)$, por lo tanto $g$ no es continua en $\Q \setminus \{0\}$

					Hacemos lo mismo con un $x_0 \in \I$ , hay una sucesión de racional convergiendo a él, cuando le aplicamos $g$ nos queda constantemente 1, por lo tanto converge a 1 , pero $g(x_0) = 0$

					Entonces $g(x_n)$ no converge a $g(x_0)$, por lo tanto $g$ no es continua en $\I \setminus \{0\}$

					Entonce $g $ es continua únicamente en el 0

				\end{proof}
				
			\item $h$ solo es continua en $[0,1] \setminus \Q$
				
				Veamos que es contínua en los irracionales. Sea $x_0 \in \I$

				Tomemos $(x_n)_n$ convergente a $x_0$. Si a partír de algún momento fuese irracional , es evidente que $f(x_n)$ es constantemente 0 entonces $f(x_n)$ converge a $f(x_0) = 0$

				Entonces miremos las sucesiones que no tienen tal momento , entonces $(x_n)_n$ siempre tiene racionales.

				Ahora es lo mismo mirar la subsucesión $x_{n_k} $ solo de racionales, esta tiene que converger también a $x_0$

				Supongamos que el exponente de cada uno de estos elementos tiene una cota superiór $m>0$

				Bueno en algún momento tenemos algun $\frac{k}{m} < x_0 < \frac{k+1}{m}$ y como $m$ es el mas grande que puedo tomar entonces solo podría cambiar $k$ para acercarme, pero eso no me va a servír 

				Entonces la sucesión no podría converger, por lo tanto dicho $m$ no existe, y el denominador tiene que agrandarse infinitamente para que la sucesión pueda converger a $x_0$

				Entonces ahora si miramos nuevamente $x_n$ tenemos que $f(x_n) = 0$ para los irracionales y $f(x_n) = \frac{1}{n}$ para los racionales por lo tanto converge a $0$ que es $f(x_0)$

				Entonces es contínua en $x_0$ y esto vale para cualquier irracional

				Tenemos que es contínua en irracionales. 

				Supongamos que es contínua en un $x_0 \in \Q$ tal que $x_0 \in [0,1]$

				Ahora tenemos una sucesión $(x_n)_n$ de irracionales que converge a $x_0$

				Como $f$ es contínua en racionales  $f(x_n)$ converge a $f(x_0)$. 

				Pero $f(x_n)$ es constantemente 0 y $f(x_0)$  es diferente de 0, por que $x_0 \notin \I$

		\end{enumerate}



		\end{ej}
	\begin{ej}
		
		Probar que un espacio métrico $X$ es discreto si y sólo si toda función de $X$ en un espacio métrico arbitrario es continua.

		\begin{proof}
			$\Ra )$ Si es discreto sabemos que todo sub de $X$ es abierto y cerrado , por lo tanto si tomo un abierto en la imagen, y miro su preimagen , por ser sub de $X$ es también abierta, entonces $f$ es continua.

			$\Leftarrow )$ Como puedo tomar cualquier espacio métrico tomo $(X,\delta)$ con $\delta$ la métrica discreta. 

			Ahora como puedo tomar cualquier función tomo la identidad

			$id: (X,d) \ra (X,\delta)$ por hipótesis tiene que ser contínua.

			Entonces si agarro un abierto en $(X,\delta)$ tiene que ser abierto en $(X,d)$

			Ahora cualquier conjunto $A$ que agarre en $(X,d)$.

			$A = f(A) \subseteq (X,\delta)$ y por lo tanto es abierto y cerrado en $(X,\delta)$ 


			Entonces tenemos que $A$ es abierto y cerrado en $(X,\delta)$

			Ahora como $f$ es contínua entonces también es abierto y cerrado en $(X,d)$

			Entonces $(X,d)$ es discreto



		\end{proof}
	\end{ej}
	
	\begin{ej}
		Considerando en cada $\R^n$ la métrica euclídea, probar que:
		\begin{enumerate}
			\item $A= \{(x,y)\in\R^2 \ / \ x^2+ysen(e^x -1) = -2\}$ es cerrado
				\begin{proof}
					La función $x^2 + ysen(e^x -1)$ es una composición de continuas, por lo tanto es continua.

					$A = f^{-1}(\{-2\})$ que es un cerrado de $\R$  , por lo tanto $A$ es preimagen de cerrado y es cerrado

				\end{proof}
			\item $A = \{(x,y,z) \in \R^3 \ / \ -1 \leq x^3 - 3y^4 + z -2 \leq 3\}$ es cerrado.
				\begin{proof}
					Lo mismo $A$ es preimagen del cerrado $[-1,3]$ por una función continua, entonces es cerrado
				\end{proof}
				
			\item $A = \{(x_1,x_2,x_3,x_4,x_5) \in \R^5 \ / \ 3 < x_1 - x_2\}$ es abierto
				\begin{proof}
					$A = f^{-1}((3,+\infty))$ con $f((x_1,x_2,x_3,x_4,x_5)) = x_1 -x_2$

					Veamos que $f$ es continua, tomamos $(x_n)_n \subseteq \R^5$ convergente.

					Sabemos que entonces coordenada a coordenada tiene que converger, por que el espacio usa la métrica euclídea 
					$d(x_n,x) \ra 0$ entonces $\sqrt{(x^1_n - x^1)^2+ \cdots + (x^5_n - x^5)^2} \ra 0$

					Por lo tanto como son todas sumas, cada sumando tiene que tender a 0

					Pero entonces 

					$$d(f(x_n),f(x)) = d(x^1_n - x^2_n , x^1 -x^2) = |x^1_n -a x^1 -x_n^2 + x^1| \leq |x^1_n -x^1| + |-x_n^2 + x^2| \ra 0$$

					Por lo tanto $f(x_n) \ra f(x)$

					Ahora como $(-3,+\infty)$ es abierto, entonces $A$ es preimagen de abierto , por lo tanto abierto

				\end{proof}
				\begin{remark} Estas tres afirmaciones siguen valiendo con la métrica discreta y con $d_1$, 

				Con la distreta sabemos que todo es contínuo. Veamos con $d_1$

				Ya sabemos que las funciones en cuestión son continuas con $d_2$ (métrica euclídea)

				Entonces dado un $\epsilon$ tenemos un $\delta$ tal que $d_2(x , x_0) < \delta \Ra d(f(x_n),f(x)) < \epsilon $  

				Si usamos el mismo delta con $d_1$ tendríamos

				$d_2(x_n,x) < d_1(x_n,x) < \delta \Ra d(f(x_n),f(x)) \leq \epsilon$

				Probando que $f$ sigue siendo continua con $d_1$

				Comentario: Vale para cualquier métrica de las equivalentes en $\R$ , por que las métricas equivalentes conservan abiertos y cerrados
				
				\end{remark}
		\end{enumerate}
	\end{ej}
	
	\begin{ej}
		Sean $X, Y$ espacios métricos y sea $f: X \ra Y$ una función continua. Probar que el gráfico de $f$ definido por 

		$$ G(f) = \{(x,f(x)) \in X \times Y : x\in X\}$$

		es cerrado en $X \times Y$ ¿ Es cierta la afirmación recíproca?

		\begin{proof}
			Tomemos una sucesión $(x_n,f(x_n)) \in G(f)$ convergente a $(a,b)$ queremos ver que entonces $(a,b) \in G(f)$

			Pero sabemos que $d((x_n,f(x_n)),(x,f(x)) = d_X(x_n,x) + d_Y(f(x_n),f(x))$. Dado que $f$ es continua ambas tienden a 0

			Entonces $(x_n,f(x_n))$ converge a $(x,f(x))$ por lo tanto $(a,b) = (x,f(x)) \in G(f)$

			Y esto lo podemos hacer con cualqueir suceisón convergente.

			Mostrando que $G(f)$ es cerrado
			
			No es cierta la recíproca, tomemos la función $id : (X,d) \ra (X,\delta)$ con $X$ no discreto , entonces, por un argumento similar al usado antes $id$ no es continua.

			Pero $G(f)$ es cerrado, veámoslo:

			Sea $(x_n,f(x_n))_n \subseteq G(f)$ un sucesión convergente a $(x,f(x))$

			Esto quiere decir que dado $\epsilon > 0$ existe $n_0 \in \N$ tal que 
			$$d(x_n,x) + d(f(x_n),f(x)) =  d((x_n,f(x_n)),(x,f(x))) < \epsilon \quad \forall n \geq n_0$$

			Por lo tanto $d(f(x_n),f(x)) < \epsilon \quad \forall n \geq n_0$, entonces $f(x_n)$ converge a $f(x)$

			Como la imagen es discreta esto significa que la sucesión es constante a partír de algún momento

			Pero entonces $(x_n,f(x_n))_n$ es constante a partír de algún momento y es constantemente $(x,f(x))$

			Como todos sus términos estan en $G(f)$ resulta que $(x,f(x)) \in G(f)$

			Mostrando que el gráfico es cerrado.

			Entonces mostramos una función que tiene gráfico cerrado pero que no es contínua, por lo tanto no es cierta la recíproca


		\end{proof}
		
		
	\end{ej}
	
	\begin{ej}
		
	\end{ej}

	\begin{ej}
		Sea $(X,d)$ un espacio métrico y sea $f: X \ra \R$. Probar que $f$ es contínua si y sólo si para todo $\alpha \in \R$, los conjuntos $\{x \in X: f(x) < \alpha\}$ y $\{x\in X : f(x) > \alpha\}$ son abiertos
		\begin{proof}
			$\Ra ) $ Sabemos que esos conjuntos son preimagen $(- \infty , \alpha)$ y $(\alpha,+\infty)$ respectivamente, entonces dado que $f$ es contínua tienen que ser abiertos

			$\Leftarrow )$ Sabemos que cualquier abierto de $\R$ se puede escribir como $A = (f(x) - \epsilon,f(x) + \epsilon)$ para algún $x \in X$

			Ahora sabemos que la preimagen de A es $\{x \in X : f(y) < f(x) + \epsilon \} \cap \{x \in X : f(y) >  f(x) - \epsilon\}$

			Que por hipótesis es una intersección de dos abiertos , por lo tanto es abierto 

			Entonces preimagen de cualquier abierto es abierto, por lo tanto $f$ es contínua
		\end{proof}
		
		
	\end{ej}
	
	\begin{ej}
		Sea $(X,d)$ un espacio métrico y sea $A$ un subconjunto de $X$. Probar que la función $d_A: X \ra \R$ definida por $d_A(x) = d(x,A) = \inf_{a\in A}d(x,a)$ es uniformemente contínua

		\begin{proof}
			Esto sale facil si usamos el ejercicio de la guía pasada que prueba $|d_A(x) - d_A(y)| \leq  d(x,y)$

			Dado $\epsilon >0 $ si tomo $\delta = \epsilon$ sabemos que 

			$$d(x,y) < \delta = \epsilon \Ra d(d_A(x),d_A(y)) = |d_A(x)-d_A(y)| < \epsilon \quad \forall x,y \in X$$

			Entonces $d_A$ es uniformemente contínua
		\end{proof}
		
		
	\end{ej}
	
	

	\begin{ej}
		$Teorema$ $de$ $Urysohn$	

		Sea $(X,d)$ un espacio métrico y sean $A,B$ cerrados disjuntos de $X$
		\begin{enumerate}
			\item Probar que existe una función $f: X \ra \R$ continua tal que:
				$$ f|_A \equiv 0 \ \ , \ \ f|_B \equiv 1 \ \ y \ \ 0 \leq f(x) \leq 1 \quad \forall x \in X$$

				Sugerencia: Considerar la función $f(x) = \frac{d_A(x)}{d_A(x) + d_B(x)}$
				\begin{proof}
					La función de la sugerencia sirve perfecto, está evidentemente entre 0 y 1 

					Cuando tomamos $x \in A$ nos da 0 dado que $d_A(x) = 0$

					Y cuando tomamos $x \in B$ entonces nos da $\frac{d_A(x)}{d_A(x)} = 1$ 
				\end{proof}
			\item Deducir que existen abiertos $U,V \subseteq X$ disjuntos tales que $A \subset U$ y $B \subset V$

				si tomamos $ U = f^{-1}(-\infty,\frac{1}{2})$ y $V = f^{-1}(\frac{1}{2},+\infty)$

				Sabemos que $0 \in U $ por lo tanto $A \subseteq f^{-1}(0)\subset U$ , lo mismo con $B \subset V$

				Y ambos son abiertos , por ser preimagen de abiertos de una contínua
				
		\end{enumerate}
	\end{ej}
	
	\begin{ej}
		Sea $f: \Z \ra \Q$ una función.
		\begin{enumerate}
			\item Probar que $f$ es contínua ¿Sigue valiendo si $f$ toma valores irracionales?
				\begin{proof}
					Es contínua por que $\Z$ es un espacio métrico discreto, entonces toda función que salga de él será contínua, no es así si cambiamos $\Z$ por $\Q$.  

					Por ejemplo la función que es 1 en todos los enteros y 0 en todos los racionales no enteros, no es contínua. 	

					Tengo una sucesion de racionales no enteros que se acerca a un entero , pero su imagen es constantemente 0 que es diferente que 1 que sería la imagen del límite de esa sucesión
				\end{proof}
			\item Suponiendo que $f$ es biyectiva ¿puede ser un homeomorfismo?

				No no puede, se prueba viendo $f^{-1}$ para que una sucesión en la imagen converga entonces tiene que ser constante a partír de un momento y esto no necesariamente vale en el dominio

		\end{enumerate}
	\end{ej}
	
	\begin{ej}
		Sea $(X,d)$ un espacio métrico, y sea $\Delta : X \ra X \times X$ la aplicación diagonal definida por $\Delta (x) = (x,x)$. Probar que:
		\begin{enumerate}
			\item $\Delta$ es un homeomorfismo entres $X$ y $\{(x,x):x\in X\}$
				\begin{proof}
					Veamos que $\Delta$ es contínua

					Tomemos $(x_n)_n \subseteq X$ cualquiera convergente a $x$ 

					Tenemos que dado $\epsilon >0$ existe $n_0 \in \N$ tal que $d(x_n,x) \leq \frac{\epsilon}{2}\quad \forall n \geq n_0$

					$d(\Delta (x_n),\Delta (x)) = d((x_n,x_n),(x,x)) = d(x_n,x) + d(x_n,x) = \frac{\epsilon}{2} + \frac{\epsilon}{2} = \epsilon \quad \forall n \geq n_0$  

					Entonces $\Delta$ es contínua

					Veamos que $\Delta^{-1}$ es contínua:

					Sea $(x_n,x_n)_n \subseteq \{(x,x):x\in X\}$ convergente a $(x,x)$

					Dado $\epsilon >0$ existe $n_0 \in \N$ tal que $d(x_n,x) + d(x_n,x) = d((x_n,x_n),(x,x)) < \epsilon \quad \forall n \geq n_0$ 

					Entonces $d(x_n,x) < \epsilon \quad \forall n \geq n_0$

					$d(\Delta^{-1}((x_n,x_n)),\Delta^{-1}((x,x))) = d(x_n,x) < \epsilon \quad \forall n \geq n_0$

					Que ambas son biyectivas es trivial
				\end{proof}
				
			\item $\Delta(X)$ es cerrado en $X \times X$	
				\begin{proof}
					Sea $(x_n,x_n)_n \subseteq \Delta (X)$ convergente a $(x,x)$

					Pero $(x,x) \in \Delta (X) \quad \forall x \in X$

					Entonces la sucesión converge dentro del conjunto por lo tanto es cerrado

					Además se puede ver a $\Delta (X)$ como el gráfico de la función $id : X \ra X$

					Y los gráficos son cerrados
				\end{proof}
		\end{enumerate}
	\end{ej}
	
	\begin{ej}
		Sean $(X,d)$ e $(Y,d')$ espacios métricos. Una aplicación $f: X \ra Y$ se dice $abierta$ si $f(A)$ es abierto para todo abierto $A \subset X$ y se dice $cerrada$ si $f(F)$ es cerrado para todo cerrado $f \subset X$
		\begin{enumerate}
			\item Probar que si $f$ es biyectiva entonces, $f$ es abierta (cerrada) si y sólo si $f^{-1}$ es contínua
				\begin{proof}
					$\Ra )$ Veamos $g = f^{-1}$ con $g : Y \ra X$. Supongamos que $g$ no es contínua, entonces existe un abierto $A \subseteq X$ tal que $g^{-1}(A) \subseteq Y$ no es abierto.
					Pero $g^{-1} = (f^{-1})^{-1} = f$ entonces $g^{-1}(A) = f(A)$ por lo tanto tenemos un abierto $A$ tal que $f(A)$ no es abierto, lo que es absurdo. Entonces para todo abierto su preimagen por $g$ es abierto , por lo tanto $g$ es contínua

					$\Leftarrow )$ Usémos la misma $g$, como es contínua, para todo $A \subseteq X$ abierto $g^{-1}(A) = f(A)\subseteq Y$ abierto.

					Esto nos dice que $A \subseteq X$ abierto entonces $f(A)$ es abierto

					No veo bien donde se usó biyectividad per se, en la vuelta te dicen que tenes una $f^{-1}$ contínua, asi que medio que ya está asumido que existe la inversa de $f$ y está bien definida, quizas para la parte uno.

					\begin{remark}
						Toda función de $\R$ en $\R$ biyectiva, con la métrica usual (o equivalentes) tiene inversa contínua, por ende toda función biyectiva de $\R$ a $\R$ biyectiva es abierta (y cerrada)
					\end{remark}
				\end{proof}
			\item Dar un ejemplo de una función de $\R$ en $\R$ contínua que no sea abierta. 
				\begin{proof}
					La función $f(x) = 2$ , no es abierta, por que cualquier abierto te lo manda al $\{2\}$ que es un cerrado de $\R$
				\end{proof}

			\item Dar un ejemplo de una función de $\R$ en $\R$ contínua que no sea cerrada.
				\begin{proof}
					$f(x) = \frac{1}{1+|x|}$ y además $f(0) - \frac{1}{2}$

					$f(\R) = (0,1)$ pero $\R$ es cerrado , por ser el conjunto universo, entonces mandé un cerrado a un abierto 
				\end{proof}
		
			\item Mostar con un ejemplo que una función puede ser biyectiva, abierta y cerrada pero no contínua.
				\begin{proof}
					$id: (\R,d) \ra (\R,\delta)$, con $d$ la métrica euclídea y $\delta$ la métrica discreta.

					Esta es cláramente biyectiva, también abierta y cerrada por que TODO en la imagen es abierto y cerrado (imagen discreta)

					Además la inversa es contínua, por que su dominio es discreto.

					Sin embargo es obviamente no contínua (se puede ver con sucesiones tomas una sucesions que converge en el dominio, pero no converge en la imagen , por que las únicas que convergen son constantes a partír de algún momento)



				\end{proof}
		\end{enumerate}
	\end{ej}

	\begin{ej}
		Sean $(X,d)$ e $(Y,d')$ espacios métricos y sea $f: X \ra Y$ una función.
		\begin{enumerate}
			\item Probar que $f$ es continua si y sólo si $f(\ol E) \subset \ol{f(E)}$ para todo subconjunto $E$ de $X$
				Mostrar con un ejemplo que la inclusión puede ser estricta.
				\begin{proof}
					$\Ra )$ Sea $f$ contínua , dado un $E$ que temos ver que $f(\ol E)$ está en la clausura de $f(E)$

					O lo mísmo , dado un $y \in f(\ol E)$ queremos ver que $\forall r >0 \quad B(y,r) \cap f(E) \neq \emptyset$

					Ahora como $y \in f(\ol E)$ entonces existe un $x \in \ol E$ tal que $f(x) = y$

					Entonces tenemos una sucesión $(x_n)_n \subseteq E$ tal que $x_n \ra x$

					Como $f$ es continua $f(x_n) \ra f(x) = y$ y $(f(x_n)_n \subseteq f(E)$

					Entonces tomando el $r>0$ sabemos que existe un $n_0 \in \N$ tal que $d(f(x_n),y) \leq r \quad \forall n \geq n_0$ 

					En particular existe algun $n_1 \in \N$ tal que $d(f(x_{n_1}),y) < r$ 

					Equivalentemente $f(x_{n_1}) \in B(y,r)$ con $f(x_{n_1}) \in f(E)$. Entonces $B(y,r) \cap f(E) \neq \emptyset$ 

					Y esto lo podemos hacer con cualquie $r>0$. Por lo tanto $y \in \ol{f(E)}$

					Finalmente $f(\ol E) \subseteq \ol{f(E)}$

					$\Leftarrow )$ Sea $x_n$ una sucesión convergente a $x$. Queremos ver que $f(x_n)$ converge a $f(x)$ 

					Ahora tomemos una sub sucesión $x_{n_k}$ cualquiera. Si esta pasa infinitas veces por $x$ entonces nos tomamos la sub sucesión $x_{n_{k_j}} = x$  y entonces $f(x_{n_{k_j}}) = f(x)$ por lo tanto converge.

					Entonces para un $f(x_{n_k})$ de este tipo encontramos un $f(x_{n_{k_j}})$ que converge a $f(x)$

					Si no pasa infinitas veces, entonces a partir de algún momento $n_{k_0}$ deja de pasar por $x$

					Entonces nos quedamos con el conjunto $E = \{ x_{n_{k}}: k \geq k_0 \}$
					
					Sabemos que $x \in \ol E$ justamente por que $x_{n_k}$ converge a $x$

					Y por hipótesis sabemos que $f(\ol E) \subset \ol{f(E)}$

					Por lo tanto $f(x) \in \ol{f(E)}$, entonces tenemos una sucesión de elementos de $f(E)$ que converge a $f(x)$

					Ahora esta sucesión la podemos armar como subsucesión de $f(x_{n_k})$, si esto no fuera cierto, entonces existiría $n_{k_1}$ tal que $d(f(x_{n_{k}}),f(x)) > \epsilon \quad \forall n_k \geq n_{k_1}$.

					Pero nuestro conjunto $E$ tiene solamentes elementos $f(x_{n_k})$ por lo tanto no podría tener una sucesión de $E$ que converja a $f(x)$, lo cual es absurdo, entonces puedo armarme una sub-subsucesión $f(x_{n_{k_j}})$ que converja a $f(x)$

					Y por último, si $x_n$ nunca era $x$ entonces podemos repetír el argumento recién usado.

					Entonces teniendo $f(x_n)$ y tomando cualquier subsucesion $f(x_{n_k})$ pudimos encontrar una subsubsucesión $f(x_{n_{k_j}})$ que converje a $f(x)$, por lo tanto $f(x_n)$ converje a $f(x)$.

					Mostrando que $f$ es continua

				\end{proof}
				
				
			\item Probar que $f$ es continua y cerrada si y sólo si $f(\ol E) = \ol{f(E)}$ para todo subconjunto $E$ de $X$
				\begin{proof}
				$\Ra$) Sea $E$ un conjunto , usando la hipótesis junto con el $i)$ tenemos $f(\ol E) \subseteq \ol{f(E)}$

				Ahora veamos la otra inclusión. Sabemos que $f(E) \subseteq f(\ol E)$ entonces $\ol{f(E)} \subseteq \ol{f(\ol E)}$

				Como $f$ es cerrada, $f(\ol E)$ es cerrado entonces $\ol{f(\ol E)} = f(\ol E)$

				Finalmente $\ol{f(E)} \subseteq f(\ol E)$ entonces $\ol{f(E)} = f(\ol E)$

				$\Leftarrow$) Por una de las inclusiones y usando el $i)$ tenemos la continuidad. Veamos que $f$ es cerrada

				Dado un $E$ cerrado , sabemos que $f(E) = f(\ol E)$ y por hipótesis tenemos que $f(\ol E) = \ol{f(E)}$

				Entonces tenemos que $f(E) = \ol{f(E)}$ por lo tanto es cerrado

				Finalmente para cualquier cerrado su imagen por $f$ es un cerrado, por lo tanto $f$ es cerrada
				\end{proof}
		\end{enumerate}
	\end{ej}

	\begin{ej}a

		\begin{enumerate}
			\item Sean $(X,d)$ e $(Y,d')$ espacios métricos y sea $D \subset X$ denso. Sean $f,g : X \ra Y$ funciones contínuas. Probar que si $f|_D = g|_D$ entonces $f = g$	

				\begin{proof}
					Supongo que existe un $x \in X\setminus D$ tal que $f(x) \neq g(x)$

					Ahora por ser denso tengo una sucesión $x_n $ que converge a $x$

					Y por continuidad $f(x_n)$ converge a $f(x)$ y $g(x_n)$ converge a $g(x)$

					Y por coincidir en el denso y esta sucesion estar en el denso $f(x_n) = g(x_n)$

					Pero entonces por unicidad de límite tienen que converger a lo mismo por lo tanto $g(x) = f(x)$

					Lo que es absurdo, por lo tanto no existe dicho $x$

					Entonces $f$ y $g$ coinciden fuera del denso y ya coincidian en el denso entonces coinciden en todos lados $(f =g)$
				\end{proof}
				
			\item Sea $f : \R \ra \R$ una función contínua tal que $f(x+y) = f(x) +f(y)$ para todo $x,y \in \Q$.
 
				Probar que existe $\alpha \in \R$ tal que $f(x)= \alpha x$ para todo $x \in \R$
		\end{enumerate}
	\end{ej}
	
	\begin{ej}
		Sean $(X,d)$ e $(Y,d')$ espacios métricos. Consideramos en $X \times Y$ la métrica $d_{\infty}$
		\begin{enumerate}
			\item Probar que las proyecciones $\pi_1 : X \times Y \ra X$ y $\pi_2 : X \times Y \ra Y$ son contínuas y abiertas. Mostrar con un ejemplo que pueden ser no cerradas
				\begin{proof}
					Probemos continuidad (uniforme) 
					$$d_X(\pi_1((x_1,y_1)),\pi_1((x_2,y_2)) = d_X(x_1,x_2) < d_X(x_1,x_2) + d_Y(y_1,y_2) = d_{\infty}((x_1,x_2),(y_1,y_2)) $$		

					Probemos que son abiertas:

					Sea $U \subseteq X \times Y$ abierto. Dado $(x_1,y_1) \in U$ tenemos un $r>0$ tal que $B((x_1,y_1),r) \subseteq U$

					Entonces $\pi_1(B((x_1,y_1),r)) \subseteq \pi_1(U)$. Ahora me gustaría ver que $B(x_1,r) \subseteq \pi_1(U)$. 

					Y esto mostraría que dado cualquier $(x_1,y_1) \in U$ que es lo mismo que decir dado cualquier $x_1 \in \pi_1(U)$ tengo una bola $B(x_1,r) \subseteq \pi_1(U)$. Mostrando que $\pi_1$ es abierta

					Entonces tomemos un $x_1' \in B(x_1,r)$ sabemos que $d_X(x_1,x_1') < r $

					Luego $d_{\infty}((x_1',y_1),(x_1,y_1)) = d(x_1', x_1) + d(y_1,y_1) = d(x_1',x_1) < r$

					Entonces $(x_1',y_1) \in B((x_1,y_1),r)$ por lo tanto $x_1'= \pi_1(x_1',y_1) \in \pi_1(B((x_1,y_1),r)) \subseteq \pi_1(U)$

					Como esto vale para cualquier $x_1'\in B(x_1,r)$ entonces $B(x_1,r) \subseteq \pi_1(U)$

					Mostrando lo que queríamos y por lo tanto que $\pi_1$ es abierta. 

					Un razonamiento análogo prueba que $\pi_2$ es abierta

					Ejemplo de no cerrada:

					Sea $F = \{(x,y) \in \R^2 : xy = 1\}$. Es cerrado , el único punto que podría ser de acumulación sin estra en el conjunto es el $(0,0)$ y es facil ver que ninguna sucesión de $F$ converge a él.

					Sin embargo $\pi_1(F) = \R \setminus \{0\}$ que no es cerrado
				\end{proof}
			\item Sea $(\Z,d '')$ un espacio métrico y sea $f : \Z \ra X \times Y$ una aplicación. Probar que $f$ es contínua si y sólo si $f_1 = \pi_1 \circ f$ y $f_2 = \pi_2 \circ f$ lo son
				
				\begin{proof}
					$\Ra )$ $f_1$ y $f_2$ eson composiciones de contínuas, por lo tanto contínuas.

					$\Leftarrow$) Sea $(x_n)_n \subseteq \Z$ convergente a $x$ 

					Entonces sabemos que $f_1(x_n)$ converge a $f_1(x)$ y lo mismo con $f_2(x_n)$, por que ambas con contínuas. Entonces para cada uno tenemos un $n_k$, tal que $d(f_1(x_n),f_1(x)) < \frac{\epsilon}{2} \quad \forall n \geq n_k$ y lo mismo con $f_2$

					Tomamos $n_0$ el máximo entre ámbos y ahora tenemos que dado ese $\epsilon$ ambas sucesiones están cerca del límite a partir de $n_0$

					Ahora teniendo en cuenta que $f(x) = (f_1(x),f_2(x)) $ tenemos $f(x_n) = (f_1(x_n),f_2(x_n))$, 

					$d_{\infty}((f_1(x_n),f_2(x_n)),(f_1(x),f_2(x))) = d_X(f_1(x_n),f_1(x)) + d_Y(f_2(x_n),f_2(x)) < \epsilon \quad \forall n\geq n_0$

					Luego tenemos que $f(x_n) = (f_1(x_n),f_2(x_n))$ converge a $(f_1(x),f_2(x)) =  f(x)$

					Por lo tanto $f$ es contínua

					Otra forma. Dado $\epsilon >0$ existe $\delta_1 > 0 $ tal que $d''(x',x) < \delta $ entonces $d(f_1(x'),f_1(x)) < \frac{\epsilon}{2}$

					
					Lo mismo sucede con $f_2$ y con un $\delta_2$ usando la $d'$. Ahora si tomamos $\delta = \min\{\delta_1,\delta_2\}$

					Tenemos que $d''(x',x) < \delta$ entonces $d_{\infty}(f(x'),f(x)) = d_{\infty}((f_1(x'),f_2(x')),(f_1(x),f_2(x)))$ 

					Esto es igual a $d(f_1(x'),f_1(x)) + d'(f_2(x'),f_2(x)) < \frac{\epsilon}{2} + \frac{\epsilon}{2} = \epsilon$
				\end{proof}
				
				
		\end{enumerate}
	\end{ej}
	
	\begin{ej}
		Sean $X,Y$ espacios métricos. Sea $f: X \ra Y$ una función contínua y suryectiva.
		\begin{enumerate}
			\item Probar que si $X$ es separable, entonces $Y$ es separable.
				\begin{proof}
					Tomemos un cubrimiento por abiertos. $Y = \bigcup_{i \in I} A_i$

					Ahora tenemos $X = f^{-1}(Y) = \bigcup_{i \in I} f^{-1} (A_i)$ esto podemos hacerlo por que $f$ es suryectiva

					Como $A_i$ son abiertos su preimagen es abierto entonces tenemos un cubrimiento por abiertos de $X$

					$X$ es separable, por lo tanto tiene un sub cubrimiento numerable

					Entonces $X = \bigcup_{j \in J} A_j $ con $J \subset I$. Finalmente $Y = f(X) = \bigcup_{j \in J} f(f^{-1}(A_j))$

					Por lo tanto $Y = \bigcup_{j \in J} A_j$. Mostrando que $Y$ tiene subcubrimiento numerable

					
				\end{proof}
			\item ¿ Es cierto que si $X$ es completo, entonces $Y$ es completo?
				\begin{proof}
					No es cierto $id:(\Q,\delta) \ra (\Q,d)$ con $\delta$ la métrica discreta y $d$ la métrica euclídea

					Las sucesiones de Cauchy en $\Q$ con le discreta son sucesiones constantes por lo tanto convergentes

					Sin embargo $\Q$ no es completo con la métrica discreta
				\end{proof}
		\end{enumerate}
	\end{ej}
	
	\begin{ej}
		Saa $(X,d)$ un espacio métrico y sea $f : X \ra \R$ una función. Se dice que $f$ es $semicontinua$ $inferiormente$ (resp. $superiormente$) en $x_0 \in X$ si para todo $\epsilon >0$ existe $\delta >0$ tal que 

		$$ d(x,x_0) < \delta \Longrightarrow f(x_0) < f(x) + \epsilon \quad (\text{ resp } f(x_0)+\epsilon > f(x))$$

		Probar que:

		\begin{enumerate}
			\item $f$ es contínua en $x_0$ si y sólo si $f$ es semicontínua inferiormente y superiormente en $x_0$
				\begin{proof}
					$\Ra )$ Como $f$ es contínua en todos lados en particular es en $x_0 \in X$ 

					Entonces dado $\epsilon > 0$ tenemos un $\delta$ tal que $d(x,x_0) < \delta \Ra d(f(x),f(x_0)) < \epsilon$ 

					Equivalentemente $|f(x)-f(x_0)| < \epsilon$ entonces $f(x_0) - \epsilon< f(x) < f(x_0) + \epsilon$

					Mostrando que $f$ es semicontínua inferior y superiormente en cualquier $x_0 \in X$

					$\Leftarrow$ Como tenemos ambas semicontinuidades dado $\epsilon$ tenemos:

					$d(x,x_0) < \delta_1 \Ra f(x_0) < f(x) + \epsilon$

					$d(x,x_0) < \delta_2 \Ra f(x) - \epsilon< f(x_0)$

					Entonces tomando $\delta = \min\{\delta_1,\delta_2\}$

					Tenemos $d(x,x_0) < \delta \Ra f(x) - \epsilon < f(x_0) < f(x) + \epsilon \iff |f(x_0) - f(x)| < \epsilon$

					Por lo tanto $f$ es contínua en $x_0$, para cuaquier $x_0 \in X$ 
				\end{proof}
				
			\item $f$ es semicontínua inferiormente si y sólo si $f^{-1}(\alpha,+\infty)$ es abierto para todo $\alpha \in \R$
				\begin{proof}
					$f$ semicontínua inferiormente entonces dado $\epsilon >0$ existe $\delta >0$ tal que 
					$$ d(x,x_0) < \delta \Ra f(x_0) < f(x) + \epsilon$$

					Sea $x_0 \in f^{-1}(\alpha,+\infty)$ entonces $f(x_0) > \alpha$ entonces $f(x_0) - \alpha > 0 $

					Entonces tomamos $\epsilon ' = f(x_0) - \alpha $ tendríamos:

					$$\exists \delta >0 \text{ tal que } d(x,x_0 ) < \delta \Ra f(x_0) < f(x) + \epsilon ' \iff f(x) > \alpha \iff x \in f^{-1}(\alpha , +\infty)$$

					Pero entonces $x \in B(x_0,\delta) \iff d(x,x_0) < \delta \iff x \in f^{-1}(\alpha,+\infty)$

					Por lo tanto $B(x_0,\delta) \subseteq f^{-1}(\alpha,+\infty)$. Y esto vale para cualquier $x_0 \in X$

					Por lo tanto $f^{-1}(\alpha,+\infty)$ es abierto

			
				\end{proof}
				
				
			\item $f$ es semicontínua superiormente si y sólo si $f^{-1}(-\infty,\alpha)$ es abierto para todo $\alpha \in \R$
				\begin{proof}
					Sale con un razonamiento análogo al anteriór
				\end{proof}
				
				
			\item Si $A\subset X$ y $\chi_A : X \ra \R$ es su función característica, entonces $\chi_A$ es semicontínua inferiormente (resp. superiormente) si y sólo si $A$  es abierto (resp. cerrado).

				Comparar con el ejercicio 17
				\begin{proof}
					\[
						\chi_A^{-1}((\alpha,+\infty)) =
  					\begin{cases}
                                   		\R & \text{if $\alpha \leq 0$} \\
                                   		A & \text{if $0 < \alpha \leq 1 $} \\
  						\emptyset & \text{if $\alpha > 1$}
  					\end{cases}
					\]
				\end{proof}
				Sabemos que $\R$ es abierto , $\emptyset$ es abierto , entonces esta preimagen es abierta si y sólo si $A$ es abierto

				Entonces $X_A$ es semicontínua inferiormente si y solo si $X_A^{-1}((\alpha,+\infty))$ es abierto para todo $\alpha \in \R$ si y solo si $A$ es abierto
		
				\[
					f^{-1}((-\infty,\alpha)) =
  \begin{cases}
                                   \emptyset & \text{if $\alpha < 0$} \\
                                   \R \setminus A & \text{if $0 \leq \alpha < 1$} \\
  				   \R & \text{if $1 \leq \alpha$}
  \end{cases}
\]

				Ahora tenemos algo similar $\R$ es abierto $\emptyset$ es abierto necesitamos que $\R \setminus A$ sea abierto y esto sucede si y sólo si $A$ es cerrado

				Es similar por que el 17 nos dice que si $f^{-1}((- \infty,\alpha)) = \{x\in X : f(x) < \alpha\}$

				y $f^{-1}((\alpha,+\infty)) = \{x\in X : f(x)> \alpha\}$ son abiertos , $f$ es contínua.

				Y este ej nos dice que si $f^{-1}((\alpha,+\infty))$ es abierto entonces $f $ es semicontínua inferiormente

				Y el otro nos dice que es semicontínua superiormente, y ambas juntas nos dice que $f$ es contínua
		\end{enumerate}
	\end{ej}
	
	\begin{ej}
		Sean $(X,d)$ e $(Y,d')$ espacios métricos y sea $f: X \ra Y$ una función que satisface:
		
		$$d'(f(x_1),f(x_2)) \leq c d(x_1,x_2) $$

		para todo $x_1,x_2\in X,$ donde $c \geq 0$. Probar que $f$ es uniformemente contínua.	

		\begin{proof}
			Si $c = 0$, entonces $d'(f(x_1),f(x_2)) = 0 \quad \forall x_1,x_2 \in X$

			Entonces dado un $\epsilon$ podemos tomar cualquier $\delta >0$.

			Si $c \neq 0$ dado $\epsilon > 0$ tomamos $\delta = \frac{\epsilon}{c} > 0$

			Entonces $d(x_2,x_2) < \delta \Ra d'(f(x_1),f(x_2)) < cd(x_1,x_2) < c \delta = \epsilon \quad \forall x_1,x_2 \in X$
		\end{proof}
		
		
	\end{ej}

	\begin{ej}a
		\begin{enumerate}
			\item Sean $(X,d)$ e $(Y,d')$ espacios métricos, $A \subset X$ y $f: X \ra Y$ una función. Probar que si existe $\alpha > 0, (x_n)_n, (y_n)_n \subset A$ sucesiones y $n_0 \in \N$ tales que:
				\begin{enumerate}
					\item $d(x_n,y_n) \ra 0$ para $n \ra +\infty$						
						
					\item $d'(f(x_n),f(y_n)) \geq \alpha \quad \forall n \geq n_0$

					entonces $f$ no es uniformemente contínua en $A$
				\end{enumerate}
				\begin{proof}
					Tomemos $\epsilon = \alpha$

					Como $d(x_n,y_n) \ra 0$ dado cualquier $\delta$ tenemos un $n_1 \in \N$ tal que $d(x_n,y_n) < \delta \quad \forall n \geq n_1$

					Tomamos $n_2 = \max\{n_1,n_0\}$ luego tomemos $x_{k},y_{k}$ con $k \geq n_2$

					Ahora por hipótesis sabemos que $d'(f(x_{k}),f(y_{k})) \geq \alpha$ (vale por que $k \geq n_2 \geq n_0$)

					Por lo tanto para cualquier $\delta$ nos podemos fabricar $x_1,x_2 \in X$

					Luego existe $\epsilon $ ($\alpha$) tal que $\forall \delta >0 $ tengo  $x,y \in X$ tal que $d(x,y)<\delta \Ra d(f(x),f(y)) \geq \alpha$

					Por lo tanto $f$ no es uniformemente contínua
				\end{proof}
				
				
			\item Verificar que la función $f(x)=x^2$ no es uniformente contínua en $\R$ ¿Y en $\R_{\leq -\pi}$?

				Sean $x_n = n + \frac{1}{n}$ e $y_n = n$ tenemos que $d(x_n,y_n) = |n + \frac{1}{n} -n| = \frac{1}{n}  \ra 0$

				Sin embargo $d(f(x_n),f(y_n)) = d(n^2 + 2 + \frac{1}{n^2} - n^2) |2 + \frac{1}{n^2}| \ra 2 $

				Entonces $d(x_n,y_n) \ra 0$ 

				Además si tomamos $\alpha = 1$ sabemos que existe $n_0$ tal que $d(f(x_n),f(y_n)) > \alpha \quad \forall n \geq n_0$
			\item Verificar que la función $f(x) = sen(\frac{1}{x})$ no es uniformemente contínua en $(0,1)$
		\end{enumerate}
	
	\end{ej}
	
	\begin{ej}a
		\begin{enumerate}
			\item Sea $f:(X,d) \ra (X,d')$ una función uniformemente contínua y sea $(x_n)_n$ una sucesión de Cauchy en $X$. Probar que $(f(x_n))_n$ es una sucesión de Cauchy en $Y$
				\begin{proof}
					Dado $\epsilon > 0$. Sabemos $\exists \delta> 0$ tal que 

					Para todo $x,y \in X$ que cumpla $d(x,y) < \delta \Ra d(f(x),f(y))< \epsilon $


					Sea $(x_n)_n \subseteq X$ de Cauchy entonces existe un $n_0 \in \N$ tal que $d(x_n,x_m) < \delta \quad \forall n,m\geq n_0$

					Entonces $d(f(x_n),f(x_m)) < \epsilon \quad \forall n,m \geq n_0$

					Por lo tanto $f(x_n)$ es de Cauchy

				\end{proof}
				
				
			\item Sea $f: (X,d) \ra (Y,d')$ un homeomorfismo uniforme. Probar que $(X,d)$ es completo si y sólo si $(Y,d')$ es completo

			       En particular, si un espacio métrico $X$ es completo para una métrica lo es para cualquier otra métrica uniformemente equivalente.
			       \begin{proof}
				       $\Ra$) Tenemos $(y_n)_n \subseteq Y$ de Cauchy. Como $f$ es homeo uniforme entonce $f^{-1}$ es uniformente contínua.

				Entonces $x_n = f^{-1}(y_n)$ es de Cauchy, por inciso i). Como $X$ es completo $x_n$ converge.

				Ahora por continuidad $f(x_n) = y_n$ converge también. Entonces $Y$ es completo.

				$\Leftarrow $) Con un razonamiento análogo vemos que $Y$ completo implica $X$ completo.

				\end{proof}
					
				Sabemos que tenemos la identidad $id: (X,d) \ra (X,d')$ esto es un homeomorfismo, veamos que es uniforme

				Como las métricas son uniformemente equivalentes $d(x,y) < \alpha d'(x,y)$

				Entonces por continuidad de $id^{-1}$ dado $\epsilon >0 $ tenemos que existe $\delta > 0$ tal que $d'(x,y) < \delta $ implica $d(x,y)< \epsilon$

				Además $ d(x,y) < \delta \iff \alpha d(x,y) < \alpha \delta $
		\end{enumerate}
		
	\end{ej}
	
	\begin{ej}a
		\begin{enumerate}
			\item Dar un ejemplo de una función $f: \R \ra \R$ acotada y contínua pero no uniformemente contínua.
				\begin{proof}
					

				\end{proof}
				
			\item Dar un ejemplo de una función $f : \R \ra \R$ no acotada y uniformemente contínua
				\begin{proof}
					
				\end{proof}
				
				
		\end{enumerate}
	\end{ej}
	
		

	\begin{ej}
		Sea $f:(X,d) \ra (Y,d')$ una función uniformemente continua y sean $A,B \subseteq X$ conjuntos no vacíos tales que $d(A,B) = 0$. Probar que $d'(f(A),f(B)) = 0$
		\begin{proof}
		  Sabemos que $d(A,B) = \inf\{(a,b):a \in A, b \in B\} = 0$, entonces tenemos una sucesión $d(a_n,b_n)$ que converge a 0

		  Ahora supongamos que $d'(f(A),f(B)) = \alpha > 0 $ entonces para $d(f(a),f(b)) \geq \alpha \quad \forall a \in A ,b \in B$

		  Por que sino tendría dos elementos $a'\in A, b'\in B$ tal que $d(f(a'),f(b')) < \alpha$ contradiciendo que $\alpha$ es un ínfimo.

		  Pero entonces en particular $d(f(a_n),f(b_n)) \geq \alpha \quad \forall n \in \N$

		  Pero entonces $f$ no debería ser uniformemnte contínua, lo que es absurdo.

		  Por lo tanto $d(f(A),f(B)) \leq 0$ como son distancias, son positivas. $d(f(A),f(B)) = 0$
		\end{proof}
		
		
	\end{ej}
	
	\begin{ej}
		Sean $X$ e $Y$ espacios métricos, $Y$ completo. Sea $D \subset X$ denso y $f:D \ra Y$ una función uniformente contínua. Probar que $f$ tiene una única extensión contínua a todo $X$, es decir, existe una única función $F: X \ra Y$ tal que $F|_D =f $ (Más aún, $F$ es uniformemente contínua)
		\begin{proof}
			Sea $x \in D$ entonces $F(x) = f(x)$

			Si en cambio $x \notin X \setminus D$, como $D$ es denso tenemos una sucesion $(d_n)_n \subseteq D$ que converge a $x$ 

			Entonces $F(x) = \lim_{n\ra +\infty} f(d_n)$ sabemos que este límite existe 

			Veamos que cualquier sucesion del denso que tomemos nos da lo mismo. Sean  $(x_n)_n,(y_n)_n \subseteq D$

			Tales que ambas convergen a $x \in X$. $d(x_n,y_n) < d(x_n,x) + d(x,y_n) < \epsilon \quad \forall n\geq n_0$

			Entonces $d(x_n,y_n) \ra 0$, como $f$ es uniformemente contínua 

			Dado $\epsilon>0 \quad \exists n_0 \in \N$ tal que $d(f(x_n),f(y_n)) < \epsilon \quad \forall n\geq n_0$ 

			(Salteé un par de pasos, dado $\epsilon$ primero tenes el $\delta$ y dado el $\delta$ tenes el $n_0$ por que $d(x_n,y_n) \ra 0$ y ahí llegas al $\epsilon$ en la imagen)

			Esto nos dice que $d(f(x_n),f(y_n)) \ra 0$ y además sabemos que tanto $f(x_n)$ como $f(y_n)$ convergen por continuidad

			Pero entonces $\lim f(x_n) = \lim f(y_n)$ por lo tanto da igual que sucesión tomemos y entonces $F(x)$ está bien definido, es único

			Veamos que $F$ es contínua, sabemos que es contínua para cualquier punto de $D$

			Entonces tomemos un $x_0 \in X$ tomemos una sucesion $x_n$ convergente a $x_0$

			La idea es que para cada término de esta $x_n$ tenemos una $(d_n)_n \subset D$ tal que $d_j^n \ra x_n$

			Entonces con esto probamos que tenemos una sucesión $(d_n)_n \subseteq D$ tal que $d(x_n,d_n) \ra 0$

			Entonces $f(d_n) \ra f(x_0)= F(x_0)$

			


		\end{proof}
		
		
	\end{ej}
	
	



\end{document}
