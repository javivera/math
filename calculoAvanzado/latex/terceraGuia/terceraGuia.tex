\documentclass[11pt]{report}

\usepackage[margin=1in]{geometry}
\usepackage{enumerate}
\usepackage{amsmath}
\usepackage{amssymb}
\usepackage{mathtools}
\usepackage{amsfonts}
\usepackage{amsthm}
\usepackage{graphicx}
\usepackage{fancyhdr}
\pagestyle{fancy}

\newcommand{\n}{\aleph_{0}}
\newcommand{\F}{\mathhbb{F}}
\newcommand{\Q}{\mathbb{Q}}
\newcommand{\C}{\mathbb{C}}
\newcommand{\R}{\mathbb{R}}
\newcommand{\K}{\mathbb{K}}
\newcommand{\E}{\mathbb{E}}
\newcommand{\I}{\mathbb{I}}
\newcommand{\Z}{\mathbb{Z}}
\newcommand{\N}{\mathbb{N}}
\newcommand{\Ra}{\Rightarrow}
\newcommand{\ra}{\rightarrow}
\newcommand{\ol}{\overline}
\newcommand{\norm}[1]{\left\lVert#1\right\rVert}
\newcommand{\open}{\mathrm{o}}


\theoremstyle{definition}
\newtheorem{definition}{Definición}[section]
\newtheorem*{remark}{Observación}
\newtheorem{theorem}{Teorema}
\newtheorem{lemm}{Lema}
\newtheorem{corollary}{Corolario}[theorem]
\newtheorem{lemma}[theorem]{Lema}
\newtheorem{prop}{Proposición}
\newtheorem{ej}{Ejercicio}


\fancyhead[R]{Espacios Métricos}
\fancyhead[L]{Alumno Javier Vera}
\fancyhead[C]{Cálculo Avanzado}

\DeclarePairedDelimiter\Floor\lfloor\rfloor
\DeclarePairedDelimiter\Ceil\lceil\rceil

\begin{document}

\begin{ej}
	Sea $(X,d)$ un espacio métrico y sea $(x_n)_n \subseteq X$. Probar:
	\begin{enumerate}
		\item $\lim x_n = x$ si y sólo si para toda subsucesión $(x_{n_k})_k, \quad \lim_{k \ra \infty} x_{n_k} = x$
			\begin{proof}
				$\Ra )$	Como $x_n$ converge dado $\epsilon >0$ existe $n_0 \in \N$ tal que $d(x_n,x) < \epsilon \quad \forall n \geq n_0$

				Por ser $x_{n_k}$ subsucesión tenemos que $n_{k_0} \geq n_0 $ entonces $d(x_{n_k},x) < \epsilon \quad \forall n_k \geq n_{k_0}$

				Y esto lo podemos hacer con cualquier epsilon por lo tanto $x_{n_k}$ converge a $x$

				$\Leftarrow )$ Si vale para toda subsucesión en particular vale para $x_n$ que es una subsucesión, entonces $x_n \ra x$
			\end{proof}

		\item Si existe $x \in X$ para el cual toda subsucesión $(x_{n_k})_k$ de $(x_n)_n$ tiene un subsucesión $(x_{n_{k_j}})_j$ tal que $\lim_{j} x_{n_{k_j}} = x$, entonces $(x_n)_n$ converge y $\lim_n x_n = x$
			
		\begin{proof}
			Supongamos que $x_n$ no converge a $x$.

			Entonces $\exists \epsilon >0$ tal que para cada $N \in \N$ $\exists n_0 > N$ con $d(x_{n_0},x) \geq \epsilon$

			Entonces para $N_1 \in \N$ nos quedamos con un $n_1 \geq N_1$ tal que $d(x_{n_1},x) \geq \epsilon$

			Y ahora tomamos $N_2 > n_1$ tiene que existir un $n_2 > N_2 > n_1$ tal que $d(x_{n_2},x) \geq \epsilon$

			Si nos vamos quedando con todos los $x_{n_k}$ y teniendo en cuenta como los tomamos nos aseguramos que $x_{n_k}$ es una subsucesión de $x_n$

			Pero entonces existe $x_{n_{k_j}}$ convergente a $x$. Y esto es absurdo , por que TODOS los términos de $x_{n_k}$ cumplían $d(x_{n_k},x) \geq \epsilon$ por ende como $x_{n_{k_j}}$ es sub tiene que cumplir lo mismo entonces $d(x_{n_{k_j}},x) \geq \epsilon \quad \forall j \in \N$
		\end{proof}

		\item Si $(x_n)_n$ es convergente entonces es de Cauchy, ¿Vale la recíproca?

		\begin{proof}
			$\Ra )$ Sea $x_n$ convergente a $x$ 

			Tomemos $\epsilon > 0$ sabemos que existe $n_0 \in \N$ tal que $d(x_n,x) < \frac{\epsilon}{2} \quad \forall n \geq n_0$

			Ahora si miramos $d(x_n,x_j) \leq d(x_n,x) + d(x,x_j) \leq \frac{\epsilon}{2} + \frac{\epsilon}{2} = \epsilon \quad \forall n,j \geq n_0$

			La vuelta no vale cuando no es completo. Por ejemplo los racionales, tenemos la sucesion $x_n = \sqrt{2} + \frac{1}{n}$ es facil ver que es de Cauchy, sin embargo no converge en $\Q$ por que $\sqrt{2}$ no está en $\Q$
		\end{proof}
		
	\item Si $(x_n)_n$ es de Cauchy, entonces es acotada.

		\begin{proof}
		Dado $\epsilon > 0$ sabemos que existe $n_0 \in \N$ tal que $d(x_n,x_m) \leq \epsilon \quad \forall n,m \geq n_0$

		Entonces si fijo $x_n$ con $n \geq n_0$ tenemos que $d(x_n,x_m) \leq \epsilon \quad \forall m \geq n_0$

		Equivalentemente $x_m \in B(x_n,\epsilon) \quad \forall m \geq n_0$ entonces todos los términos mayores que $n_0$ están acotados. Llamemosle a su cota $M$

		Y los anteriores son finitos, entonces cada uno es menór que un $M_i$

		Entonces puedo tomar una cota para todos $C = \max_{i}\{M_i,M\}$
		\end{proof}
	\item Si $(x_n)_n$ es de Cauchy y tiene un subsucesión $(x_{n_k})_k$ convergente a $x \in X$, entonces $(x_n)_n$ converge a $x$
		\begin{proof}
			Dado un $\epsilon >0$ tenemos un $n_0 \in \N$ tal que $d(x_n,x_m) \leq \frac{\epsilon}{2} \quad \forall n,m \geq n_1$

			Y además tenemos un $n_2 \in \N$ tal que $d(x_{n_k},x) \leq \frac{\epsilon}{2} \quad \forall n_k \geq n_2$

			Si $n_0 = \max\{n_1,n_2\}$ luego $d(x_n ,x )\leq d(x_n,x_{n_k}) + d(x_{n_k},x) = \frac{\epsilon}{2} + \frac{\epsilon}{2} = \epsilon \quad \forall n_k,n \geq n_0$
		\end{proof}
	\end{enumerate}
\end{ej}

\begin{ej}
	Probar que si toda bola cerrada de un espacio métrico $X$ es un subespacio completo de $X$ entonces $X$ es completo.

	\begin{proof}
		Tomemos $(x_n)_n \subseteq X$ de Cauchy. Sabemos entonces que está acotada , por ende a partir de un momento podemos meterla en una bola en particular cerrada y para los elementos que nos quedan miramos de todos ellos el que nos genere una distancia mas grande a la bola , ahora sumamos ese radio al de la bola y tenemos una nueva bola cerrada que tiene todos los elementos de la sucesión.

		Como es una bola cerrada es una espacio métrico completo , por ende $x_n$ que estaría contenida en dicho nuevo espacio y ademas sería de cauchy por que el nuevo espacio tiene la misma métrica, tiene que ser convergente a un $x$ en ese espacio , pero ese espacio es un subespacio de $X$ por ende $x_n$ converge a algo en $X$. Y esto lo podemos hacer con cualquier sucesión de Cauchy.

		Por ende $X$ es completo
	\end{proof}
\end{ej}
	\begin{ej}
		Sean $A$ y $B$ subespacios de un espacio métrico. Probar que si $A$ y $B$ son completos, entonces $A \cup B$ y $A \cap B$ son completos.
		\begin{proof}
			Sea $(x_n)_n \subseteq A\cap B$ de Cauchy entonces $(x_n)_n \subseteq A$ por lo tanto $x_n$ converge en $A$

			Entonces $x_n$ converge en $A$.

			Por otro lado $(x_n)_n \subseteq B$ entonces $x_n$ converge en $B$

			Por unicidad de límite $x_n$ converge en $A \cap B$, entonce $A \cap B$ es completo

			Sea $(x_n)_n \subseteq A\cup B$ de Cauchy. Entonces $(x_n)_n \subseteq A$ entonces $x_n$ converge en $A$

			Por lo tanto $x_n$ converge en $A \cup B$, entonces $A \cup B$ es completo
		\end{proof}
	\end{ej}
	
	\begin{ej}
		Sea $(X,d)$ un espacio métrico.
		\begin{enumerate}
			\item Probar que todo subespacio completo de $(X,d)$ es un subconjunto cerrado de $X$
				\begin{proof}
					Sea $(Y,d)$ subespacio métrico completo de $(X,n)$

					Ahora si tomamos $(x_n)_n \subseteq Y$ convergente, sabemos que entonces es de Cauchy, pero entonces converge en $Y$ dado que es completo. Por lo tanto $Y$ es cerrado y está contenido en $X$, es un cerrado de $X$. (En realidad se puede ver que es cerrado en cualquier lado , por ser completo. Por que completitud es una propiedad intrínseca)
				\end{proof}

			\item Probar que si $X$ es completo, entonces todo subconjunto $F\subseteq X$ cerrado, es un subespacio completo de $X$
				\begin{proof}
					Sea $(x_n)_n \subseteq F$ una sucesión de Cauchy, esta misma es entonces una sucesión de Cauchy de $X$, entonces converge.

					Ahora si miramos a $F$ como subconjunto de $X$ sabemos que es cerrado

					Entonces $x_n$ es una sucesión de $F$ que converge, como $F$ es cerrado , tiene que converger allí

					Entonces $x_n$ converge en $F$

					Esto vale para cualquier sucesion de cauchy de $F$. Por lo tanto $F$ es completo

				\end{proof}
		\end{enumerate}
	\end{ej}
	
	\begin{ej}
		Sean $(X,d)$ e $(Y,d')$ espacios métricos. Consideramos en $X \times Y$ la métrica $d_{\infty}$ definida por

		$$ d_{\infty}((x_1,y_1),(x_2,y_2)) = \max\{d(x_1,x_2),d'(y_1,y_2)\}$$

		Probar que $(X\times Y, d_{\infty})$ es completo si y sólo si $(X,d)$ e $(Y,d')$ son compleots.
		\begin{proof}
			$\Ra )$ Sean $(x_n)_n$ e $(y_n)_n$ sucesiones de Cauchy de $X$ e $Y$ respectivamente.

			Entonces dado $\epsilon >0$ existe $n_1 \in \N$ tal que $d(x_n,x_m) \leq \epsilon \quad \forall n,m \geq n_1$ 

			También existe $n_2 \in \N $ tal que $d'(y_n,y_m) \leq \epsilon \quad \forall n,m \geq n_2$

			Tenemos entonces una sucesión $(x_n,y_n)_n \subseteq X \times Y$, veamos que es de Cauchy

			$d_{\infty}((x_n,y_n),(x_m,y_m)) = \max \{d(x_n,x_m),d(y_n,y_m)\}$

			Podemos suponer que el máximo es cualquier de las dos sin pérdida de generalidades.

			Entonces $d_{\infty}\{((x_n,y_n),(x_m,y_m)) = d(x_n,x_m) \leq \epsilon \quad \forall n,m \geq n_1$. 

			Esto lo puedo hacer para cualquier $\epsilon$. Por lo tanto $(x_n,y_n)$ es de Cauchy, entonces converge

			Supongamos que converge a $(x,y)$ 

			Entonces $ \max \{d(x_n,x),d'(y_n,y)\} = d_{\infty}((x_n,y_n),(x,y)) \ra 0$

			Pero el máximo es mas grande que las dos distancias, por lo tanto es mas grande que cualquier de ellas 

			Entonces $ 0 < d(x_n,x) \leq d_{\infty}((x_n,y_n)(x,y)) \ra 0$. Entonces $d(x_n,x) \ra 0$ 

			Y lo mismo pasa con $(y_n,y)$. Mostrando que ambas convergen. 

			Entonces $X$ es de Cauchy e $Y$ es de Cauchy

			$\Leftarrow$) Sea $(x_n,y_n)_n \subseteq X\times Y$ una sucesión de Cauchy.

			Dado $\epsilon > 0$ existe $n_0 \in \N$ tal que $d_{\infty}((x_n,y_n),(x_m,y_m)) \leq \epsilon \quad \forall n,m\geq n_0$ 
			
			Entonces por como está  definida la distancia infinito. Tenemos que $d(x_n,x_m) \leq \epsilon \quad \forall n,m \geq n_0$

			Y lo mismo con $y_n$. Esto es por que ambas distancias son mas pequeñas que la distancia infinito.

			Pero entonces ambas sucesiones son de Cauchy en sus respectivos espacios completos, por ende convergen

			Entonces usando el $\epsilon$ que teníamos sabemos que existe $n_1 \in \N$ tal que $d(x_n,x) \leq \epsilon \quad \forall n\geq n_1$ 

			Y también existe $n_2 \in \N$ tal que $d(y_n,y) \leq \epsilon \quad \forall n\geq n_2$.

			Ahora si tomamos $n_0 = \max\{n_1,n_2\}$ tenemos que $d_{\infty}((x_n,y_n),(x,y)) \leq \epsilon \quad \forall n \geq n_0$

			Entonces $(x_n,y_n)$ converge, y lo podemos hacer con cualquier sucesión de Cauchy.

			Entonces $X\times Y$ es completo
		\end{proof}
	\end{ej}

	\begin{ej}
		\begin{enumerate}
			\item Sea $X$ un espacio métrico y sea $B(X) = \{f: X\ra \R / f $ es acotada $\}$. 

				Probar que $(B(X),d_{\infty})$ es un espacio métrico completo, donde $d_{\infty}(f,g) = \sup_{x\in X}|f(x)-g(x)|$

				\begin{proof}
					
				\end{proof}
				
				
			\item Sean $a,b\in \R, a < b$. Probar que $(C[a,b],d_{\infty})$ es un espacio métrico completo, donde $d_{\infty}(f,g) = \sup_{x\in [a,b]}|f(x)-g(x)|$
			\item Probar que $C_0 := \{(a_n)_n \subseteq \R / a_n \ra 0 \}$ se un espacio métrico completo con la distancia $d_{\infty}((a_n),(b_n)) = \sup_{n\in\N} |a_n-b_n|$
	
		\end{enumerate}
	\end{ej}


	\begin{ej}
		Sea $(X,d)$ un espacio métrico y sea $\mathcal{D} \subseteq X$ un subconjunto denso con la propiedad que toda sucesión de Cauchy $(a_n)_n \subseteq \mathcal{D}$ converge en $X$. Probar que $X$ es completo.
		\begin{proof}
			Sea $(x_n)_n \subseteq X$ un sucesión de Cauchy.

			Como $\mathcal{D}$ es denso, para cada $x_n$ existe un $d_n \in D$ tal que $d(x_n,d_n) \leq \frac{1}{n}$

			Veamos que $d_n$ es de Cauchy.

			Ahora dado un $\epsilon > 0 $ tenemos sabemos que existe  $n_1 \in \N$ tal que $\frac{2}{n_1} + \epsilon ' \leq \epsilon$

			Dado dicho $\epsilon ' > 0 $ sabemos por Cauchy existe un $n_2\in \N$ tal que $d(x_n,x_m )\leq \epsilon ' \quad \forall n,m \geq n_1$

			Ahora si tomamos $n_0 = \max\{n_1,n_2\}$ Tenemos que:
			
			$$d(d_n,d_m) < d(d_n,x_n) + d(x_n,x_m) + d(x_m,d_m) = \frac{1}{n} + \epsilon ' + \frac{1}{m} \leq \frac{2}{n_1} + \epsilon '=\epsilon \quad \forall n,m \geq n_0$$

			Entonces $d_n$ es de Cauchy , como está contenida en $D$ converge a un $d$

			Ahora propongo mi $d$ como candidato a límite de $a_n$.

		Dado $\epsilon >0$ sabemos que existe un $n_1 \in \N$ tal que $\frac{1}{n_1} + \frac{\epsilon}{2} < \epsilon$

			dado que $d_n$ converge a $d$, tenemos que existe $n_2$ tal que $d(d_n,d) \leq \frac{\epsilon}{2} \quad \forall n \geq n_2$

			Si tomamos nuevamente $n_0 = \max\{n_1,n_2\}$

			$d(a_n,d) \leq d(a_n,d_n) + d(d_n,d) \leq \frac{1}{n} +\frac{\epsilon}{2} \leq \frac{1}{n_1}+\frac{\epsilon}{2}  = \epsilon \quad \forall n \geq n_0 $ 

			Entonces $a_n$ converge a $d$, como $d\in D \subseteq X$ entonces $d \in X$.

			Finalmente $a_n$ converge en $X$. Entonces $X$ es completo
		\end{proof}
	\end{ej}

	\begin{ej}
	$Teorema$ $de$ $cantor$

	Probar que un espacio métrico $(X,d)$ es completo si y sólo si toda familia $(F_n)_n$ de suconjuntos de $X$ cerrados, no vacíos tales que $F_{n+1} \subset F_n$ para todo $n \in \N$ y $diam(F_n) \ra 0$ tiene un único punto en la intersección

		\begin{proof}
			$\Ra$) Tenemos una familia de subconjuntos que cumple las propiedades dadas.

			Entonces podemos armar una sucesión $x_n$ donde cada elemento es algún elemento de $F_n$

			Veamos que es de Cauchy. Dado $\epsilon >0$ y considerando que $diam(F_n) \ra 0$. Sabemos que existe algún cerrado $F_{n_0}$ tal que $diam(F_{n_0}) < \epsilon$.

			Ahora dado cualquier par de elementos $x_n, x_m$ tales que $n,m \geq n_0$ sabemos que pertenecen a algún cerrado que está contenido en $F_{n_0}$. Por lo tanto $d(x_n,x_m) \leq diam(F_{n_0}) < \epsilon$.

			Entonces $(x_n)_n$ es de Cauchy. Ahora por ser $X$ completo, $x_n$ converge digamos a $x$

			Además $(x_n)_n \subseteq \bigcap_{i \in I} F_i$ por como la construimos y esta intersección es una intersección de cerrados por ende cerrada.

			Por ende $(x_n)_n$ tiene que converger en la intersección. Entonces $x \in \bigcap_{i \in I}F_i$

			Ahora supongamos que existe $x'\neq x$ en la intersección. 

			Obviamente $x_n$ no puede converger a él. Por unicidad de convergencia. 

			Por lo tanto existe $\epsilon > 0$ tal que $\forall n_0\in \N$ existe $n \geq n_0$ tal que $d(x_n,x') > \epsilon$

			Ahora , sabemos que existe $n_0 \in \N$ tal que $diam(F_{n_0}) < \epsilon$ y además  $(x_n)_n \subseteq F_{n_0} \quad \forall n \geq n_0$.

			Pero entonces $x' \notin F_{n_0}$ por que si no $d(x_n,x) < \epsilon \quad \forall n\geq n_0$

			Por lo tanto $x' \notin \bigcap_{i \in I} F_i$ lo que es absurdo, provino de suponer que existia un $x'\neq x$ en la intersección, entonces no existe otro elemento diferente de $x$ en la intersección

			$\Leftarrow$ ) Sea $(x_n)_n \subseteq X$ una sucesion de Cauchy supongamosla no constante, si lo fuera ya sabemos que converge. Entonces si tomamos $\epsilon = \frac{1}{j}$ 

			Sabemos que para cada uno existe un $n_0 \in \N $ tal que $d(x_n,x_m) < \frac{1}{j} \quad \forall n \geq n_0$

			Entonces nos armamos $F_j = \{(x_n)_n \subseteq X : n \geq n_0 \}$ y luego los clausuro. Entonces $diam(\ol{F_j}) = \frac{1}{j}$

			Ahora a medida que voy agrandando el $j$ voy obteniendo conjuntos que están estrictamente incluidos en el anteriór, no es cierto que en cada paso suceda, pero si , para algún $j$ por que si no $\ol{F_j}$ sería a partir de algún $j_0$ siempre igual, por lo tanto el mismo $n_0$ nos serviría para todos los conjuntos, pero entonces la sucesión era constante. 

			Entonces me quedo con un grupo de $J$ tal que $\ol{F_{j+1}} \subset \ol{F_{j}}\quad \forall j \in J$, sabemos que su diametro tiende a 0, por como tomamos los $\epsilon$.

			Entonces estamos dentro de las hipótesis, por lo tanto podemos afirmar que existe un único punto $x$ tal que $x \in \bigcap_{j \in J} F_j$. 
		
			Proponemos ese $x$ como punto al que converge $x_n$
	
			Dado $\epsilon >0$, sabemos que existe un $j_0 \in J$ tal que $\frac{1}{j_0} < \epsilon$ 

			Además existe $n_0 \in \N$ tal que $d(x_n,x_m) \leq \frac{1}{j_0} < \epsilon \quad \forall n,m \geq n_0$

			Por lo tanto $x_n \in F_{j_0} \quad \forall n \geq n_0$ 

			Además por estar en la intersección de todos los $F_j$ sabemos que $x \in F_{j_0}$

			Por lo tanto como $diam(F_{j_0}) < \epsilon$ sucede que $d(x_n,x) < \epsilon \quad \forall n\geq n_0$.

			Y esto lo podemos hacer con cualquier $\epsilon >0$, entonces $x_n$ converge a $x$
		\end{proof}	
	\end{ej}
	\begin{ej}
		Sea$(X,d)$ un espacio métrico completo sin puntos aislados. Probar que $X$ tiene cardinal mayor o igual que $\mathfrak{c}$. Deducir que si además $X$ es separable, entonces $\# X = \mathfrak{c}$ (Para esto último, puede servir un ejercicio de la práctica anteriór)

		\begin{proof}
			
		\end{proof}
		
	\end{ej}

	\begin{ej}
		Sean $(X,d)$ e $(Y,d')$ espacios métricos y sea $f: X \ra Y$. Probar que:
		\begin{enumerate}
			\item $f$ es continua en $x_0 \in X$ si y sólo si para toda sucesión $(x_n)_n \subset X$ tal que $x_n \ra x_0$. La sucesión $(f(x_n))_n \subset Y$ converge a $f(x_0)$

			\begin{proof}
				La ida es trivial.

				$\Leftarrow ) $ Supongamos que $f$ no es continua en $x_0$ entonces dado $\epsilon >0 $ sabemos que $\forall \delta >0 $ si $d(x,x_0) < \delta \Ra d(f(x),f(x_0)) > \epsilon$.

				Ahora si tomamos $\delta = \frac{1}{n}$ y para cada delta me quedo con algún $x_n$

				Por lo tanto tengo una sucesión $x_n$ que converge a $x$

				Sin embargo $d(f(x_n),f(x)) \geq \epsilon $ lo que es absurdo, provino de suponer que $f$ no era continua.

				Entonces $f$ es continua.
			\end{proof}
			\item 	Son equivalentes:
				\begin{enumerate}[(a)]
					\item $f$ es continua
					\item Para todo $G \subseteq Y$ abierto, $f^{-1}(G)$ es abierto en $X$
					\item Para todo $F \subseteq Y$ cerrado, $f^{-1}(F)$ es cerrado en $X$
				\end{enumerate}
				\begin{proof}
					$(a) \Ra (b) $ Sea $x_0 \in f^{-1}(G)$ qvq existe un entorno abierto $V$ de $x_0$

					$f(x) \in G$ que es abierto, entonces existe $\epsilon>0$ tal que $B(f(x_0),\epsilon) \subseteq G$

					Entonces por continuidad existe un $\delta > 0 $ tal que $f(B(x_0,\delta)) \subseteq B(f(x_0),\epsilon)\subseteq G$

					Entonces $f^{-1}(f(B(x_0),\delta))\subseteq f^{-1}(G)$, por lo tanto $B(x_0,\delta) \subseteq f^{-1}(G)$

					$(a) \Ra (c)$ Sea $(x_n)_n \in f^{-1}(F)$ convegente a $x$ entonces $(f(x_n))_n \subseteq F $.

					Por continuidad $f(x_n) $ converge a $f(x)$ y como $F$ es cerrado entonces $f(x) \in F$

					Por lo tanto $x = f^{-1}(f(x)) \in f^{-1}(F)$. Entonces probamos que toda sucesión de $f^{-1}(F)$ que converge, converge en $f^{-1}(F)$.

					Por lo tanto $f^{-1}(F) $ es cerrado

					$(b) \Ra (a)$ Tomemos $x_0 \in X$ veamos que $f$ es continua en $x_0$

					Como $x_0 \in X$ existe $y \in Y$ tal que $f(x_0) = y$. 

					Ahora dado $\epsilon >0$ tenemos que $B(y,\epsilon)$ es un abierto de $Y$

					Por lo tanto $f^{-1}(B(y,\epsilon))$ es abierto de $X$ y además $f^{-1}(y) = x_0$ pertenece a ese conjunto trivialmente

					Entonces por ser abierto existe $\delta >0$ tal que $B(x_0,\delta) \subseteq f^{-1}(B(y,\epsilon)) = f^{-1}(B(f(x_0),\epsilon))$

					Entonces aplicando $f$ de ambos lados tenemos que $f(B(x_0),\delta)\subseteq B(f(x_0),\epsilon)$.

					Por lo tanto $f$ es continua en $x_0$

					También podríamos haber usado que si $x\in B(x_0,\delta)$ entonces $d(x,x_0) < \delta$

					Y además $x \in f^{-1}(B(f(x_0),\epsilon))$, por lo tanto $f(x) \in B(f(x_0), \epsilon)$

					Entonces $d(f(x),f(x_0)) < \epsilon$, por lo tanto dado un $\epsilon$ encontramos un $\delta$ que cumple lo necesario

					$(c) \Ra (b)$ Probemos que dad cualquier funcione $f$ y cualquier conjunto $A$ entonces

					$$ f^{-1} (X\setminus A) = X \setminus f^{-1}(A)$$

					$\subseteq ) $ Sea $x \in f^{-1}(X\setminus A)$ entonces $f(x) \in X \setminus A$ por lo tanto $f(x) \notin A$

					Entonces $x \notin f^{-1}(A)$. Finalmente $x \in X \setminus f^{-1}(A)$

					$\supseteq )$ Sea $x \in X \setminus f^{-1}(A)$ entonces $x \notin f^{-1} (A) $.

					Luego $f(x) \notin A$ entonces $f(x) \in X \setminus A$, finalmente $x \in f^{-1}(X \setminus A)$

					Tomemos un abierto $G \subseteq Y$ por lo que acabamos de probar $X \setminus f^{-1}(G) = f^{-1}(X \setminus G)$.

					Como $G$ es abierto $X \setminus G$ es cerrado, entonces por hipótesis $f^{-1}(X \setminus G)$ es cerrado.

					Entonces $X \setminus f^{-1}(G)$ es cerrado , por lo tanto $f^{-1}(G)$ es abierto
				\end{proof}
				
				
		\end{enumerate}
	\end{ej}
	
	\begin{ej}
		Decidir cuáles de las siguientes funciones son continuas:
			\begin{enumerate}
				\item $f: (\R^2, d) \ra (\R, |.|), f(x,y) = x^2 + y^2$. Donde $d$ es la métrica euclídea

				\item $id_{\R^2}: (\R^2,\delta) \ra (\R^2,d_{\infty})$. La función identidad, donde $\delta$ representa la métrica discreta.
					\begin{proof}
						Sabemos que en un conjunto con la métrica discreta, todo subconjunto es abierto y cerrado, por lo tanto si agarro cualquier abierto en la imagen , su preimagen es un abierto, entonces es continua
					\end{proof}
				 		
				\item $id_{\R^2}: (\R^2,d_{\infty}) \ra (\R^2,\delta)$. La función identidad, donde $\delta$ representa la métrica discreta.

					\begin{proof}
						Sea $(x_n,y_n) = (\frac{1}{n},\frac{1}{n})$ esta sucesión converge a 0 con $d_{\infty}$

						Dado $\epsilon >0$ existe $n_0 \in \N$ tal que $\frac{1}{n_0} < \epsilon$

						Entonces $d((x_n,y_n),0) = max\{|x_n - 0 |,|y_n - 0|\} = max\{|\frac{1}{n}|,|\frac{1}{n}|\} = \frac{1}{n} \leq \frac{1}{n_0} < \epsilon \quad \forall n\geq n_0$

						Sin embargo $f((x_n,y_n))$ no converge dado que no es constante a partir de ningún momento, y en un espacio discreto las unicas sucesiones convergentes son las que son constantes a partir de algún momento
					\end{proof}

				\item $i:(E,d) \ra (X,d)$, la inclusión, donde $E \subset X$

					\begin{proof}
						Dado un abierto  $V \subseteq X$ sabemos que $V \cap E$ es un abierto relativo a $E$
					\end{proof}
					
				\item $f:(\mathcal{C}[0,1],d_{\infty}) \ra (\R,|\cdot|), f(\phi) = \phi (0)$

					Veamos que es continua en $\phi_0$. Dado $\epsilon >0$ tenemos si pedimos $\delta = \epsilon$ alcanza

					Veamosló: $d(\phi,\phi_1) = \sup_{x\in [0,1] }|\phi(x)-\phi_1 (x)| < \delta$

					Pero entonces en partcular evaluar en el cero también va a ser menór.

					Entonces $|f(\phi) - f(\phi_1)| = |\phi(0) - \phi_1(0)| < \delta = \epsilon $

					Entonces dado $\epsilon$ tomamos $\delta = \epsilon$ y viemos que $d_{\infty}(\phi,\phi_1) < \delta \Ra d_{|\cdot|}(f(\phi),f(\phi_1)) < \epsilon$

					Por lo tanto $f$ es continua en $\phi_0$ y esto lo podemos hacer para cualquier función del dominio.

					Por ende $f$ es continua en todo el dominio					
					
			\end{enumerate}
	\end{ej}

	\begin{ej}
		Sean $f,g,h : [0,1] \ra \R$

		\[
 		 f(n) =
  			\begin{cases}
                                   0 & \text{if $x\notin \Q$} \\
                                   1 & \text{if $x\in \Q$} \\
  			\end{cases}
		\]

		$$g(x) = x.f(x)$$

		\[
 		 h(n) =
  			\begin{cases}
                                   0 & \text{if $x\notin \Q$} \\
				   \frac{1}{n} & \text{if $x =\frac{m}{n} \ , \ (m:n)=1$} \\
				   1 & \text{if $x =0$} \\
  			\end{cases}
		\]


		Probar que 

		\begin{enumerate}
			\item $f$ es discontinua en todo punto.
				\begin{proof}
					Sea $x_0 \in \Q$, ahora sea $\epsilon = \frac{1}{2}$ entonces 

					Entonces para cualquier $\delta >0$ que tomemos sabemos que $B(x_0,\delta)$ contiene algún irracional, por densidad.

					Entonces $0 \in f(B(x_0,\delta)) \subseteq B(f(x_0),\epsilon) = B(1,\frac{1}{2})$, lo que es absurdo.

					Entonces $f$ no es continua en $x_0$ entonces no es continua en ningún racional 

					Con el mismo argumento tendríamos que  $1 \in B(0,\frac{1}{2})$

				\end{proof}
			\item $g$ sólo es continua en $x=0$
				\begin{proof}
					Dado $\epsilon >0 $, si tomamos $\delta = \epsilon$ sucede:

					$ g(B(0,\delta)) \subseteq [0,\delta]  \cap \Q \subseteq B(0,\epsilon) = B(f(0),\epsilon) $

					Por lo tanto es continua en $0$

					Supongamos $f$ es continua en un $x_0 \in \Q$ con $x_n \neq 0$. 

					Ahora si agarramos una sucesión de irracionales $x_n$ que converga a $x_0$ (podemos hacerlo por densidad de irracionales en $[0,1]$)

					Tenemos que $g(x_n) = 0 $ por lo tanto $f(x_n)$ converge a $0$ pero $g(x) = x \neq 0$

					Entonces $g(x_n)$ no converge a $g(x)$, por lo tanto $g$ no es continua en $\Q \setminus \{0\}$

					Hacemos lo mismo con un $x_0 \in \I$ , hay una sucesión de racional convergiendo a él, cuando le aplicamos $g$ nos queda constantemente 1, por lo tanto converge a 1 , pero $g(x_0) = 0$

					Entonces $g(x_n)$ no converge a $g(x_0)$, por lo tanto $g$ no es continua en $\I \setminus \{0\}$

					Entonce $g $ es continua únicamente en el 0

				\end{proof}
				
			\item $h$ solo es continua en $[0,1] \setminus \Q$
				
				Sea $x_0 \in \I$. Ahora tomemos cualquier $(x_n)_n $ sucesión de racionale que converga a $x_0$

				Ahora $f(x_n)$
		\end{enumerate}



		\end{ej}
	\begin{ej}
		
	\end{ej}
		Probar que un espacio métrico $X$ es discreto si y sólo si toda función de $X$ en un espacio métrico arbitrario es continua.

		\begin{proof}
			
		\end{proof}
		
		
	
	
	\begin{ej}
		Sean $(X,d)$ e $(Y,d')$ espacios métricos y sea $f: X \ra Y$ una función.
		\begin{enumerate}
			\item Probar que $f$ es continua si y sólo si $f(\ol E) \subset \ol{f(E)}$ para todo subconjunto $E$ de $X$
				Mostrar con un ejemplo que la inclusión puede ser estricta.
				\begin{proof}
					$\Ra )$ Sea $f$ contínua , dado un $E$ que temos ver que $f(\ol E)$ está en la clausura de $f(E)$

					O lo mísmo , dado un $y \in f(\ol E)$ queremos ver que $\forall r >0 \quad B(y,r) \cap f(E) \neq \emptyset$

					Ahora como $y \in f(\ol E)$ entonces existe un $x \in \ol E$ tal que $f(x) = y$

					Entonces tenemos una sucesión $(x_n)_n \subseteq E$ tal que $x_n \ra x$

					Como $f$ es continua $f(x_n) \ra f(x) = y$ y $(f(x_n)_n \subseteq f(E)$

					Entonces tomando el $r>0$ sabemos que existe un $n_0 \in \N$ tal que $d(f(x_n),y) \leq r \quad \forall n \geq n_0$ 

					En particular existe algun $n_1 \in \N$ tal que $d(f(x_{n_1}),y) < r$ 

					Equivalentemente $f(x_{n_1}) \in B(y,r)$ con $f(x_{n_1}) \in f(E)$. Entonces $B(y,r) \cap f(E) \neq \emptyset$ 

					Y esto lo podemos hacer con cualquie $r>0$. Por lo tanto $y \in \ol{f(E)}$

					Finalmente $f(\ol E) \subseteq \ol{f(E)}$

					$\Leftarrow )$ Sea $x_n$ una sucesión convergente a $x$. Queremos ver que $f(x_n)$ converge a $f(x)$ 

					Ahora tomemos una sub sucesión $x_{n_k}$ cualquiera. Si esta pasa infinitas veces por $x$ entonces nos tomamos la sub sucesión $x_{n_{k_j}} = x$  y entonces $f(x_{n_{k_j}}) = f(x)$ por lo tanto converge.

					Entonces para un $f(x_{n_k})$ de este tipo encontramos un $f(x_{n_{k_j}})$ que converge a $f(x)$

					Si no pasa infinitas veces, entonces a partir de algún momento $n_{k_0}$ deja de pasar por $x$

					Entonces nos quedamos con el conjunto $E = \{ x_{n_{k}}: k \geq k_0 \}$
					
					Sabemos que $x \in \ol E$ justamente por que $x_{n_k}$ converge a $x$

					Y por hipótesis sabemos que $f(\ol E) \subset \ol{f(E)}$

					Por lo tanto $f(x) \in \ol{f(E)}$, entonces tenemos una sucesión de elementos de $f(E)$ que converge a $f(x)$

					Ahora esta sucesión la podemos armar como subsucesión de $f(x_{n_k})$, si esto no fuera cierto, entonces existiría $n_{k_1}$ tal que $d(f(x_{n_{k}}),f(x)) > \epsilon \quad \forall n_k \geq n_{k_1}$.

					Pero nuestro conjunto $E$ tiene solamentes elementos $f(x_{n_k})$ por lo tanto no podría tener una sucesión de $E$ que converja a $f(x)$, lo cual es absurdo, entonces puedo armarme una sub-subsucesión $f(x_{n_{k_j}})$ que converja a $f(x)$

					Y por último, si $x_n$ nunca era $x$ entonces podemos repetír el argumento recién usado.

					Entonces teniendo $f(x_n)$ y tomando cualquier subsucesion $f(x_{n_k})$ pudimos encontrar una subsubsucesión $f(x_{n_{k_j}})$ que converje a $f(x)$, por lo tanto $f(x_n)$ converje a $f(x)$.

					Mostrando que $f$ es continua

				\end{proof}
				
				
			\item Probar que $f$ es continua y cerrada si y sólo si $f(\ol E) = \ol{f(E)}$ para todo subconjunto $E$ de $X$
				\begin{proof}
				$\Ra$) Sea $E$ un conjunto , usando la hipótesis junto con el $i)$ tenemos $f(\ol E) \subseteq \ol{f(E)}$

				Ahora veamos la otra inclusión. Sabemos que $f(E) \subseteq f(\ol E)$ entonces $\ol{f(E)} \subseteq \ol{f(\ol E)}$

				Como $f$ es cerrada, $f(\ol E)$ es cerrado entonces $\ol{f(\ol E)} = f(\ol E)$

				Finalmente $\ol{f(E)} \subseteq f(\ol E)$ entonces $\ol{f(E)} = f(\ol E)$

				$\Leftarrow$) Por una de las inclusiones y usando el $i)$ tenemos la continuidad. Veamos que $f$ es cerrada

				Dado un $E$ cerrado , sabemos que $f(E) = f(\ol E)$ y por hipótesis tenemos que $f(\ol E) = \ol{f(E)}$

				Entonces tenemos que $f(E) = \ol{f(E)}$ por lo tanto es cerrado

				Finalmente para cualquier cerrado su imagen por $f$ es un cerrado, por lo tanto $f$ es cerrada
				\end{proof}
		\end{enumerate}
	\end{ej}
	
	


	\begin{ej}
		Sea $f:(X,d) \ra (Y,d')$ una función uniformemente continua y sean $A,B \subseteq X$ conjuntos no vacíos tales que $d(A,B) = 0$. Probar que $d'(f(A),f(B)) = 0$
	\end{ej}
	
	



\end{document}
