\documentclass[12pt]{article}

\usepackage[margin=1in]{geometry}
\usepackage{enumerate}
\usepackage{amsmath}
\usepackage{amssymb}
\usepackage{mathtools}
\usepackage{amsfonts}
\usepackage{amsthm}
\usepackage{graphicx}
\usepackage{fancyhdr}
\pagestyle{fancy}


\newcommand{\F}{\mathhbb{F}}
\newcommand{\C}{\mathbb{C}}
\newcommand{\R}{\mathbb{R}}
\newcommand{\K}{\mathbb{K}}
\newcommand{\E}{\mathbb{E}}
\newcommand{\N}{\mathbb{N}}
\newcommand{\Ra}{\Rightarrow}
\newcommand{\ra}{\rightarrow}
\newcommand{\ol}{\overline}
\newcommand{\norm}[1]{\left\lVert#1\right\rVert}
\newcommand{\n}{\aleph_{0}}


\theoremstyle{definition}
\newtheorem{definition}{Definición}[section]
\newtheorem*{remark}{Observación}
\newtheorem{theorem}{Teorema}
\newtheorem{lemm}{Lema}
\newtheorem{corollary}{Corolario}[theorem]
\newtheorem{lemma}[theorem]{Lema}
\newtheorem{ej}{Ejercicio}


\fancyhead[R]{Espacios Normados}
\fancyhead[L]{Alumno Javier Vera}
\fancyhead[C]{Calculo Avanzado}
\begin{document}
\begin{ej}
  
  Sea $ A = \{g: \N \ra \N : g \text{ es inyectiva}\}$ 

  $$\# A  = \mathfrak{c}$$

  \begin{proof}
    Vamos a ver que $\mathfrak{c} = 2^{\n} \leq \# A \leq \n^{\n} = \mathfrak{c} \Ra \# A = \mathfrak{c}$


    Primero tenemos que $A \subseteq \{f:\N \ra \N\}$

    Por ende $\#A \leq \# \{f: \N \ra \N\} = \#(\N^{\N}) = \n^{\n} \leq \mathfrak{c}^{\n} = (2^{\n})^{\n} = 2^{\n \n} = 2^{\n} = \mathfrak{c} $

    Por un lado sabemos que $(a^b)^c = a^{bc}$ con $a,b,c$ cardinales , eso se prueba en las guías. 

    Y por otro lado sabemos que $\n \n  = \# (\N \times \N) = \# \N = \n$

    Para la otra desigualdad quiero encontrar una $h$ funcion inyectiva 
  $$h: \{f:\N \ra \{2,3\}\} \ra \{g: \N \ra \N \text{ : $g$ inyectiva}\}$$

    Y esto me diría que $\mathfrak{c} = 2^{\n} = \# \{2,3\}^{\N} = \{f: \N \ra \{2,3\}\} \leq \# A$

    Afirmo que dicha función es $h(f)(x) = (f(x))^{x}$

    Primero voy a ver que $h$ está bien definida 

    Por como está definida (valga la rebundancia) $h$ es trivial ver que para toda $f:\N \ra \{2,3\} $ $h(f)$ es una función bien definida que va de naturales en naturales , basicamente podemos evaluar en cualquier $x \in \N$ a la función $h(f)$ y esto nos va a caer en $2^x$ o en $3^x$ que es un natural.  

    Tambien tenemos que ver que estas imagenes de $h$ son efectivamente funciones inyectivas 
    $$h(f)(x) = h(f)(x') \iff f(x)^x = f(x')^{x'}$$ 

    Sabemos que $f(z) = 2 $ o $ 3 \quad \forall z \in \N$ y que $x,x' \neq 0$

    Por unicidad de primos si $a,b$ primos y $x,x' > 0$ entonces $a^x = b^{x'} \iff a = b \text{ y } x = x'$

    $$ \text{Entonces } f(x)^x = f(x')^{x'} \iff f(x) = f(x') \text{ y } x = x'$$

    Juntando esto último tenemos 

    $$ h(f)(x) = h(f)(x') \iff x = x'$$

    Por ende $h(f)$ es una función inyectiva y además $h(f):\N \ra \N$ entonces $h$ está bien definida

    Resta ver que además es $h$ inyectiva, pero esto es directo usando su definición. 

    Sean $f_{1},f_{2} \in \{f: \N \ra \{2,3\}\}$

    $$ h(f_{1}) = f(f_{2}) \iff h(f_{1})(x) = h(f_{2})(x) \quad \forall x \in \N \iff (f_{1}(x))^x = (f_{2}(x))^x$$

    $$\text{Como } x \in \N \quad x \neq 0 \quad  (f_{1}(x))^x = (f_{2}(x))^x \iff f_{1}(x) = f_{2}(x) \quad \forall x \in \N \iff f_{1} = f_{2}$$

    Juntando todo 
    $$ h(f_{1}) = h(f_{2}) \iff f_{1} = f_{2}$$

    Entonces $h$ es inyectiva

    Juntando todo lo arriba expuesto , queda demostrado el ejercicio

  \end{proof}

\end{ej}



\end{document}
