\documentclass[12pt]{article}

\usepackage[margin=1in]{geometry}
\usepackage{enumerate}
\usepackage{amsmath}
\usepackage{amssymb}
\usepackage{mathtools}
\usepackage{amsfonts}
\usepackage{amsthm}
\usepackage{graphicx}
\usepackage{fancyhdr}
\pagestyle{fancy}

\newcommand{\n}{\aleph_{0}}
\newcommand{\F}{\mathhbb{F}}
\newcommand{\Q}{\mathbb{Q}}
\newcommand{\C}{\mathbb{C}}
\newcommand{\R}{\mathbb{R}}
\newcommand{\K}{\mathbb{K}}
\newcommand{\E}{\mathbb{E}}
\newcommand{\I}{\mathbb{I}}
\newcommand{\N}{\mathbb{N}}
\newcommand{\Ra}{\Rightarrow}
\newcommand{\ra}{\rightarrow}
\newcommand{\ol}{\overline}
\newcommand{\norm}[1]{\left\lVert#1\right\rVert}

\theoremstyle{definition}
\newtheorem{definition}{Definición}[section]
\newtheorem*{remark}{Observación}
\newtheorem{theorem}{Teorema}
\newtheorem{lemm}{Lema}
\newtheorem{corollary}{Corolario}[theorem]
\newtheorem{lemma}[theorem]{Lema}
\newtheorem{prop}{Proposición}



\fancyhead[R]{Espacios Normados}
\fancyhead[L]{Alumno Javier Vera}
\fancyhead[C]{Cálculo Avanzado}
\begin{document}
\thispagestyle{empty}

Palabras introductorias

\newpage
\thispagestyle{empty}

Cálculo Avanzado 

Universidad de Buenos Aires \\

Teoría

Cardinalidad \\

Javier Vera

\LaTeX 

\newpage
\thispagestyle{empty}
Cardinalidad es un tema que para el lectór en este momento de su vida puede parecer ajeno y anti intuitivo, pero en un análisis más profundo y con suerte habiendo entendido los conceptos mas adelante expuestos, él podrá confirmar que en realidad parecería ser una manera mas orgánica de definir a los 'numeros' y algunas de sus operaciones

\begin{definition}
  Sean $X$ e $Y$ dos conjuntos. Decimos que son coordinables (o equipotentes, o que tienen el mismo cardinal) si existe $f: X \ra Y$ biyectiva. [Notacion:$X \sim Y$
  \end{definition}
    \begin{remark}
      Esta relacion $\sim$ es de equivalencia, la demostración es trivial y queda como ejercicio de repaso para el lector
    \end{remark}
Ejemplos para arrancar:

\begin{center} $\N \sim \{$ numeros pares $\}$ \end{center}
\begin{center} $\Q \sim \N$ (se ve usando un argumento de diagonales)\end{center} 

\begin{remark}
  Definimos el cardinal de un conjunto $X$ como la clase de equivalencia de los conjuntos coordinables con $X$:

  $$ \# X = Card(X) = \{Y : X \sim Y\}$$

 Algunos cardinales importantes tiene su símbolo unico \\

 $\# (\N) = \n$
 
 $\#(\R) = \mathfrak{c}$

 $\# \{1,2,3 \dots ,n\} = n$

 $\# \{a,b\} = 2$ 

 Atención no confundir este último ejemplo y otros parecidos con `numeros' per se, son clases de equivalencia. Más adelante se verá que en algunos casos se comportan parecidos a los `numeros' que conocemos y que muchas veces directamente se comportan igual que ellos 
\end{remark}

\begin{remark}
Llamemos $\I_{n} = \{1,2,3 \dots ,n\}$ el intervalo inicial del conjunto $\N$ de los números naturales

Atención para poder llamar $n$ a $\# \I_{n}$ necesitamos que para $n \neq m$, $\I_{n}$ y $\I_{m}$ esten en distintas clases de equivalencia según $\sim$. Demostrémoslo..
\end{remark}
Pero antes un lema para facilitar el asunto 
\begin{lemma}
  
  Sea $A \subseteq \I_{n}$. Si existe $f:\I_{n} \ra A$ inyectiva, entonces $A = \I_{n}$
\end{lemma}


\begin{proof}
    Induccion

  $(n=1)$ para el lector

  $(n \ra n + 1) $. Sea $A \subseteq \I_{n+1}$, luego tomemos $f: \I_{n+1} \ra A$ inyectiva

  Supongamos $A \neq \I_{n+1}$ (algo tiene que estar en $\I_{n}$ y no en $A$):

    Caso I: $$n+1 \notin A \Ra A \subset \I_{n} \quad b = f(n+1)$$

    \begin{center}$f\restriction_{\I_{n}} : \I_{n} \ra A - {b} \subset A$ inyectiva, o lo que es lo mismo $f \restriction_{\I_{n}}:\I_{n} \ra A$ inyectiva \end{center}

    Por hipotesis inductiva $\I_{n} = A - {b} \subset A \subset \I_{n}$ absurdo

    Caso II: \begin{center} $n+1 \in A$. Sea $p \in \I_{n+1} \setminus A $ Luego $p \notin A$ \end{center}

    Sea $g: \I_{n+1} \ra \I_{n+1}$
    
   $$n+1 \mapsto p$$
    $$p \mapsto n+1$$
  $$x \mapsto x \quad x \neq p \quad x \neq n+1$$

  $g$ es biyectiva  y $g(A) \subset \I_{n}$

  Miremos $$g \circ f: \I_{n+1} \ra g(f(\I_{n+1})) = g(A)  $$

  $g \circ f$ es inyectiva por composición de inyectivas

  Pero ademas como $ p \notin A$ luego $n + 1 \notin g(A) $

  Luego tenemos una función que va desde $I_{n+1}$ hacia un conjunto que no tiene al $n + 1$ y está metido en $I_{n+1}$

  Entonces tenemos las hipótesis del caso I , por lo tanto esto es absurdo

  \end{proof}


\begin{theorem}
  Sean $n,m \in \N$ Entonces, $$\I_{n} \sim \I_{m} \iff n = m$$

  \begin{proof}
  $\Ra)$ Sabemos que $\I_{n} \subseteq \I_{m}$ o $\I_{n} \supseteq \I_{m}$ y sabemos que hay una inyección entre ellos por que $\I_{n} \sim \I_{m}$ luego por lema $\I_{n} = \I_{m} \Ra m = n$

$\quad \Leftarrow ) $ Por absurdo supongamos $n \neq m $ y sin perdida de generalidades $n < m \Ra \I_{n} \subset \I_{m}$

Entonces por hipótesis $\I_{n} \sim \I_{m}$ sabemos $\exists f: \I_{n} \ra \I_{m}$ biyectiva 

Pero entonces por lema $\I_{m} = \I_{n}$ absurdo 

$\Ra n = m$ 
  \end{proof}
\end{theorem}


  \begin{definition}
    Un conjunto $A$ es finito si existe $n \in \N $ tal que $A \sim \I_{n}$
  \end{definition}

  \begin{definition}
    Un conjunto $A$ es infinito si no es finito
  \end{definition}

  \begin{remark}
    Si uno puede definir conjuntos finitos sin usar los numeros naturales puede luego definir los numeros naturales a partir de los cardinales 
  \end{remark}

  \begin{definition}
    Un conjunto $A$ es numerable si $A \sim \N$. Equivalentemente si $\# A = \n$
    
  \end{definition}

\begin{remark}
  Decimos que $\# X \leq \# Y$ si existe $f:X \ra Y$ inyectiva
\end{remark}
\begin{remark}
  Decimos que $\# X < \# Y$ si $\# X \leq \# Y$ pero $X \nsim Y$
\end{remark}

\begin{remark}
  Dado un conjunto $X$ el conjunto de $partes$ de $X$ es $\mathcal{P}(X) = \{A : A \subset X \}$
\end{remark}

\begin{theorem}[Teorema de Cantor]
Sea $X$ un conjunto. Entonces $\# X < \# \mathcal{P}(X)$

\begin{proof}
  $f:X \ra \mathcal{P}(X)$ definida como $x \mapsto {x}$ esta es inyectiva luego $\# X \leq \# \mathcal{P}(X)$

Sea $g: X \ra \mathcal{P}(X)$ (si $ x \in X  \Ra g(x) \subseteq X$)

Ahora $x \in g(x)$ o $x \notin g(x)$. Definamos $B =\{x \in X : x \notin g(x)\}$

Supongamos que $B \in im(g)$ luego $\exists y / \quad B=g(y)$ 
\begin{itemize}
\item Ahora si $y \in B \Ra y \notin g(y)$ por como esta definido $B$ esto es absurdo

\item Si $y \notin B \Ra y \notin g(y)$ pero entonces $y \in B$ absurdo
\end{itemize}

Luego $\nexists y \in X / \quad g(y) = B \Ra B \notin im(g)$ 

Entonces $B \notin \mathcal{P}(X)$

$\Ra g$ no puede ser suryectiva

Observación Si $B = \emptyset$ luego si definimos $f(\emptyset) = \emptyset$ sabemos que $\emptyset \notin \emptyset$ dado que este no es nisiquiera un conjunto pero entonces $B=\{\emptyset \}$ pero entonces $B$ no era vacío, lo que es absurdo

Luego tenemos que $\nexists x / g(x) = B $ pero $B \subset P(X)$ luego $g$ no puede ser suryectiva
\end{proof}
\end{theorem}

\begin{remark}
  $X$ es numerable si y solo si $X$ se puede escribir como una sucesión de elementos distintos: $X = \{x_{n}\}_{n \in \N}$ con $x_{n} \neq x_{m}$ si $n \neq m$

  \begin{proof}
  $\Ra ) \quad X$ numerable, luego $ \exists f: \N \ra X$ biyectiva

  Definimos $x_{n} = f(n)$ entonces $X= \{x_{n}\}_{n \in \N}$
  
La vuelta es trivial
\end{proof}

\end{remark}

\begin{theorem}
  Sea $X$ infinito. Entonces existe $Y \subset X$ numerable

  \begin{proof}
    $X$ infinito $\Ra X$ no vacio $\Ra x_{1} \in X$

    $X \setminus {x_{1}} \neq \emptyset \Ra \exists x_{2} \in X \setminus {x_{1}}$ con $x_{2} \neq x_{1}$
  
Repitiendo esto inductivamente tenemos un conjunto numerable $Y$ subconjunto de $X$

  \end{proof}

\end{theorem}
\begin{theorem}
  Sea $X$ un conjunto. Entonces, $X$ es infinito si y solo si es coordinable con un subconjunto propio
  \begin{proof}
$\Ra )$  $X$ infinito luego por Teorema 4 

$X$ contiene algún $Y$ numerable $Y=\{y_{n}\}_{n \in \N}$

Sea $Y_{2} = \{y_{2},y_{4},y_{6} \dots \}$ luego $Y \sim Y_{2}$

$g: Y \ra Y_{2}$ dada por $g(y_{n}) = y_{2n}$

Luego $f: X \ra (X \setminus Y) \cup Y_{2}$ dada por  

\[
f(a,b) =
     \begin{cases}
       \text{$x$} &\quad\text{$x \notin Y$ }\\
       \text{$g(x)$} &\quad\text{$x \in Y$} \\
     \end{cases}
\]

es biyectiva

$\Leftarrow ) \quad$ Supongamos que $X$ es finito ahora sabemos por hipótesis $\exists Y \subset X$ tal que $X \sim Y$ 

Entonces $\exists f:X \ra Y$ biyectiva (en particular inyectiva) 

Pero por lema 3 tenemos $X = Y$ absurdo 

Observación: El lema 3 sirve para $A \subseteq \I_{n}$ osea para conjuntos finitos

Luego no existe dicha función, lo cual es absurdo tambien que provino de suponer $X$ finito
\end{proof}
\end{theorem}

\begin{theorem}
  Sea $X$ un conjunto. Las siguientes afirmaciones son equivalentes.
  \begin{itemize}
    \item $X$ es infinito
    \item Existe una función inyectiva de $\N$ a $X$ (o $X$ tiene un subconjunto numerable)
    \item $X$ es coordinable a un subconjunto propio
    \item $X$ es coordinable a $X \setminus {x_{0}}$ para cualquier $x_{0} \in X$
  \end{itemize}
\end{theorem}


\begin{theorem}[Teorema de Cantor-Schroeder-Bernstein]
  Si existen $f: X \ra Y, g:Y \ra X$ inyectivas, entonces existe $h:X \ra Y$ biyectiva
  \begin{proof}
    Definamos $\Phi: \mathcal{P}(X) \ra \mathcal{P}(X)$,
    $$\Phi (A) = X \setminus g(Y \setminus f(A)) $$

   Esta función es creciente, probemoslo.

   $$ A \subset B $$ 
   $$ f(A) \subset f(B)$$
   $$ Y \setminus f(A) \supset Y \setminus f(B)$$
   $$ g(Y \setminus f(A)) \supset g(Y \setminus f(B))$$
   $$ X \setminus g(Y \setminus f(A)) \subset X \setminus g(Y \setminus f(B)) $$ 
   $$ \Phi (A) \subset \Phi (B)$$
 
Luego $\Phi$ es creciente

Sea $\Omega = \{C \subset X : \Phi (C) \subset C \} $ se puede verificar que $\Omega$ no es vacío, $X \in \Omega$. Lugo tiene sentido definirse
$$ A = \bigcap_{C \in \Omega} C$$

Por como esta definido $A$ sabemos $A \in C \quad \forall C \in \Omega$ y ademas sabemos que $\Phi$ es creciente y que para $\forall C \in \Omega$ se da  $\Phi(C) \subset C$. Luego juntando todo tenemos

$$ \Phi (A) \subset \Phi (C) \subset C \quad \forall C \in \Omega$$

Luego $$ \Phi (A) \subset \bigcap_{C \in \Omega} C = A \Ra \Phi (A) \subset A$$

Usando $\Phi$ creciente $$ \Phi (\Phi (A)) \subset \Phi (A) \Ra \Phi (A) \in \Omega$$

Pero devuelta como 
\begin{center} $A \subset C \quad \forall C \in \Omega \quad $ y $\quad \Phi (A) \in \Omega \Ra A \subset \Phi (A) $ \end{center}
  
Finalmente $ A = \Phi (A) \Ra A = X \setminus (g(Y \setminus f(A)) $

  Ahora si definimos a partir de las $f$ y $g$ inyectivas que tenemos por hipótesis $X_{1} = A$, $Y_{1} = f(X_{1})$, $Y_{2} = Y \setminus Y_{1}$, $X \setminus X_{1} = X_{2}$ podemos ver que $$ X_{1} = X \setminus g (Y \setminus f(X_{1})) \iff g(Y \setminus f(X_{1})) = X \setminus X_{1} \iff g(Y_{2}) = X \setminus X_{1} = X_{2}$$ 
  
  Estas nuevas $f$, $g$ son inyectivas, por que vienen de la $f$ y $g$ que por hipótesis eran inyectivas y son suryectivas por como estan construidas (ejemplo $f(X_{1})$ es exactamente igual a $Y_{2}$ osea que todo elemento de $Y_{2}$ tiene preimagen , si no $f(X_{1}) \neq Y_{2}$

    Luego tenemos nuevas $f: X_{1} \ra Y_{1}$ y $g: Y_{2} \ra X_{2}$ biyectivas

    Con estas definimos $h : X \ra Y$ 

\[
h(x) =
     \begin{cases}
       \text{$f(x)$} &\quad\text{$x \in X_{1}$ }\\
       \text{$g(x)^{-1}$} &\quad\text{$x \in X_{2}$} \\
     \end{cases}
\]

que sabemos es biyectiva
  \end{proof}


\end{theorem}

\begin{corollary}
  El conjunto $\N \times \N$ es numerable

  \begin{proof}
    Por un lado $f: \N \ra \N \times \N$ dada por $f(n) = (n,1)$ es inyectiva

    Por otro lado $g: \N \times \N \ra \N $ dada por $f(n,m) = 2^n3^m $ tambien inyectiva

    Por Schroeder Bernstein tenemos que existe biyección, luego $\N \times \N \sim \N$
  \end{proof}
\end{corollary}

\begin{corollary}
  Para $n \in \N $ sea $X_{n}$ un conjunto numerable. Entonces, $X = \bigcup_{n} X_{n}$ es numerable (Unión numerable de numerables es numerable)

  \begin{proof}
    Cada $X_{n}$ es numerable $ \Ra \exists f_{n}: \N \ra X_{n}$ biyectiva

    Luego $g: \N \times \N \ra X = \bigcup_{n} X_{n}$ dada por $g(m,n) = f_{n}(m)$ suryectiva (no necesariamente inyectiva por que no necesariamente los $X_{n}$ son disjuntos)

    Pero con suryectividad sabemos que $X$ tiene que ser contable ($\# X \leq \# \N \times \N$) pero como es infinito luego $X$ es numerable 
  \end{proof}
  Notemos que de aqui es facil probar otro tipo de resultados por ejemplo que union de numerable de finitos (disjuntos) es numerable o que union numerable de finitos es contable 
\end{corollary}



\begin{corollary}
La relación $\leq$ entre cardinales es una relación de orden  
\end{corollary}

\begin{remark}
  Sea $X$ numerable, $Y \subset X$. Entonces, $Y$ es finito o numerable (o sea $Y$ es contable)

  \begin{proof}
    Supongamos $Y$ no es finito 

    Como $X$ es numerable. $X = \{x_{n} : n \in \N \}$ con $x_{n}$ distintos

    $n_{1} = min \{n \in \N / x_{n} \in Y \}$ 

    $n_{2} = min \{ n > n_{1} / x_{n} \in Y \}$

    $\dots$

    $ n_{k} = min \{ n > n_{k -1} / x_{n} \in Y \}$

    Sabemos que hay infinitos $n_{k}$ dado que $Y$ es infinito.

    Luego $Y = \{ (x_{n_{k}})_{k \in \N} \}$ probémoslo 
  
  $\supseteq ) $ es trivial

$\subseteq) $ Sea $y \in Y$ como $y \subset X$ sabemos $\exists m \in \N  / y = x_{m}$

Ahora como $n_{k}$ es una sucesión $k$ puede ser tan grande como uno quiera, por ende existen $n_{k} \leq m < n_{k + 1} = min \{n > n_{k} / x_{n} \in Y\}$ 

Luego $m = n_{k}$ si no llegariamos a un absurdo entonces $y = x_{m} = x_{m_{k}}$



  \end{proof}
\end{remark}

\begin{remark}
  Sea $X$ numerable,
  \begin{itemize}
    \item Si existe $f:X \ra Y $ inyectiva, entonces $Y$ es contable (a lo sumo numerable)
	\item Si existe $f:X \ra Y$ suryectiva, entonces $Y$ es contable
    \end{itemize}
\end{remark}

\begin{prop}
  Si $X$ es infinito, existe $Y \subseteq X$, $ Y$ numerable, tal que $X \sim X \setminus Y$

  \begin{proof}
  Como $X$ es infinito $\exists A \subseteq X$ numerable, $A = \{a_{n}: n \in \N\}$

  $Y = A_{1} = \{a_{2n -1} : n \in \N \} \quad A_{2} = \{a_{2n} : n \in \N \} \quad a_{n} \neq a_{m} \quad (n \neq m)$ 

  $f : A \ra A_{2} = A \setminus A_{1} = A \setminus Y$ dada por $f(a_{n}) = a_{2n}$ biyectiva 

  Luego sea $h(x): X \ra X \setminus Y$ 

\[
h(x) =
     \begin{cases}
       \text{$x$} &\quad\text{$x \in X \setminus A$ }\\
       \text{$f(x)$} &\quad\text{$x \in A$} \\
     \end{cases}
\]

  \end{proof}
\end{prop}

\begin{remark}
  Si $X$ es infinito no numerable ademas sirve cualquier $Y$ numerable. Esto va a ser obvio un par de demostraciones mas adelante. 
\end{remark}

\begin{lemma}
  Sea $X \sim Y$ y $X' \sim Y'$ además $X \cap X' = \emptyset = Y \cap Y'$

  $$ X \cup X' \sim Y \cup Y'$$

  \begin{proof}
    Tenemos $f: X \ra Y$ y $g: X' \ra Y'$ ambas biyectivas.

    Luego tenemos $h: X \cup X' \ra Y \cup Y'$ biyectiva dada por

\[
h(x) =
     \begin{cases}
       \text{$f(x)$} &\quad\text{$x \in X$ }\\
       \text{$g(x)$} &\quad\text{$x \in X'$} \\
     \end{cases}
\]



  \end{proof}
\end{lemma}


\begin{prop}
  Sea $X$ infinito y $X'$ numerable. Entonces $X \cup X' \sim X$ 
 
  Aclaración Tomo disjunta la unión por comodidad, pero no es necesario, por que se puede reescribir a $X'$ como $X''=X' \setminus X $

    Igualmente la gracia del ejercicio es cuando las cosas que estan en $X'$ pero no en $X$ son numerables, si no le  estarías agregando finitas cosas a $X$ y eso no es nada nuevo

  


  \begin{proof}
   Sea $Y \subseteq X$ numerable tal que $X \setminus Y \sim X$

  Sabemos $X' \cap X = \emptyset$ 

  $X \cup X' = [(X \setminus Y) \cup Y ] \cup X' = (X \setminus Y) \cup (Y \cup X') \sim X \setminus Y \cup Y = X$

  Esto vale por que $Y \cup X'$ es unión de numerables por ende vuelve a dar numerable y por ende es coordinable con $Y$ numerable. Y la relacion $\sim$ vale por el lema 8  
   
  \end{proof}

  \begin{remark}
    Se puede probar con unión no disjunta , pero no es trivial , los que esten en ambos $X, X'$ no cambian nada
  
    Aqui podemos ver un ejemplo $\# \R = \# \I \cup \Q = \# \I$

  \end{remark}
\end{prop}


\begin{remark}
  Sea $\{0,1\}^{\N} = \{(a_{n})_{n \in \N} : a_{n} \in \{0,1\}\}$ luego existe una biyección entre $\mathcal{P}(\N)$ y $\{0,1\}^{\N}$
  
  \begin{proof}
    Sea $f: \mathcal{P}(\N) \ra \{0,1\}^{\N}$ dada por $f(A) = (a_{n})_{n}$
  \[
    a_{n} =
     \begin{cases}
       \text{$1$} &\quad\text{$n \in A$ }\\
       \text{$0$} &\quad\text{$n \notin A$} \\
     \end{cases}
\]

\noindent
Probar que esta función es biyectiva queda como ejercicio para el lector

\noindent
Dada esta demostración notemos que $\# \{0,1\}^{\N} = \# \mathcal{P}(\N) > \n$
  \end{proof}
\end{remark}

\begin{prop}
  Sea $A_{0} = \{(a_{n})_{n \geq 1} \in \{0,1\}^{\N} :$ existe $m$ tal que $a_{n} = 0$ si $ n \geq m$ $\}$ es numerable

  Sea $B_{m} = \{(a_{n})_{n \geq 1} \in \{0,1\}^{\N} : a_{n} = 0 $ si $ n \geq m$ $\}$

Luego $$A_{0} = \bigcup_{m = 1}^{\infty} B_{m}$$ 

Sabemos que cada $B_{m}$ es finito luego union numerable de finitos es contable $A_{0}$ es contable, pero ademas sabemos que $A_{0}$ es infinito , por lo tanto es numerable

\end{prop}

\begin{prop}
 Sea $X = \{(a_{n})_{n \geq 1} : a_{n} \in \{0,1\}$ y $ a_{n} = 1 $ para infinitos valores de $ n$ $\}$

 Luego $X = \{0,1\}^{\N} \setminus A_{0}$ y tenemos $X \cup A_{0} = \{0,1\}^{\N}$ 

 Pero recordemos $X$ es infinito y $A_{0}$ es numerable luego $X \cup A_{0} \sim X$

 Luego $X \sim X \cup A_{0} \sim \{0,1\}^{\N} \Ra \# X = \# \{0,1\}^{\N} = \# \mathcal{P}(\N)$
\end{prop}

\begin{theorem}
  Finalmente podemos probar $\# \R = \# \mathcal{P}(\N) > \# \N$

  \begin{proof}
    Sabemos que $\# \mathcal{P}(\N) = \# X $

    Definimos $f: X \ra [0,1]$ como $$f(a) = \sum_{n = 1}^{\infty} \frac{a_{n}}{2^n}$$

que es biyectiva

Veamos que $\# [0,1] = \# \R$ 

Por un lado tenemos $\R \sim (-1,1)$ por medio de$f:\R \ra (-1,1)$ dada por $x \mapsto \frac{x}{1 + x^2}$ biyectiva

Por otro lado tenemos que $(-1,1) \sim (a,b)$ usando la recta que manada 

$-1 \mapsto a$ 

$1 \mapsto b$

Ahora sabemos que $(0,1) \sim \R$ agregarle numerables puntos a algo infinito no cambia su cardinal , agregarle el 0 y el 1 tampoco 

Luego $[0,1] \sim \R$

Juntando todo $\# \mathcal{P}(\N) = \# \R$
  \end{proof}
\end{theorem}


\begin{definition}
  Dados dos cardinales, $\alpha, \beta$ y $X$ e $Y$ disjuntos tales que $\alpha = \# X, \beta = \# Y$ 
Podemos definir las siguientes operaciones:

\begin{center}
\begin{tabular}{ c c  }
  Suma: & $\alpha + \beta = \# (X \cup Y)$  \\ 
  Producto: & $\alpha . \beta = \# (X \times Y)$  \\  
  Potencia: & $\alpha^{\beta} = \# \{F: Y \ra X\} = \#(X^Y)$     
\end{tabular}
\end{center}


Es importante que sean disjuntos para que todo esté bien definido

Veamos que la suma está correctamente definida:

Si tenemos $X \sim X'$ e $Y \sim Y'$ disjuntos $\Ra X \cup Y \sim X' \cup Y'$ 

Lo que acabamos de ver es que es lo mismo sumar $X$ a $Y$ que otros conjunto que sean coordinables (y por ende tengan el mismo cardinal) con alguno de ellos respectivamente

La multiplicación se ve de forma similar y el producto se ve aprovechando las funciones biyectivas que nos da la coordinabilidad y la $F$ que nos da el producto
\end{definition}

\begin{remark}
  Teniendo está aritmética de cardinales, podemos obtener ciertos resultados.

 \begin{enumerate}
      \item $\n + \n = \n$   
      \item $\mathfrak{c} + \mathfrak{c} = \mathfrak{c}$ 
      \item$ \mathfrak{c} + \n = \mathfrak{c}$
      \end{enumerate}
      
  \begin{proof}
    Aquí nos vamos a apoyar en el hecho de que podemos usar cualquier par de conjuntos que tengan el cardinal que necesitamos para probarlo para todo conjunto

  1) $\{$pares$\}$ $\cup$ $\{$impares$\}$

2) $[0,1) \cup [1,2) = [0,2)$ cada uno de estos es coordinable con $\R$

3) Lo mas directo es usando $\mathfrak{c} \leq \mathfrak{c} + \n \leq \mathfrak{c} + \mathfrak{c} = \mathfrak{c}$ esto usa Cantor-Bernstein por detras (tenemos dos funciones inyectivas entonces tenemos una biyectiva. Otra forma es notar que si a un conjunto infinito le agrego algo numerable no nos cambia el cardinal
    
  \end{proof}

\end{remark}

\begin{definition}
 Usando la definicion del producto y sabiendo $\# \{0,1\} = 2$ y $\# \N = \n$
  $\# \R = \mathfrak{c} = 2^{\N}$ 

  Teniendo esto en cuenta veamos que:

  \begin{enumerate}
    \item $\n.\n = \n$
    \item $\mathfrak{c}.\mathfrak{c} = \mathfrak{c}$
    \item $\mathfrak{c}.\n = \mathfrak{c}$
    \end{enumerate}

    \begin{proof}
    1) sabemos que $\n \times \n \sim \n$

  2) $\mathfrak{c}.\mathfrak{c} = \mathfrak{c}^2 = (2^{\n})^2 = 2^{\n.2} = 2^{\n + \n} = 2 ^{\n} = \mathfrak{c}$
  \end{proof} 
\end{definition}

\begin{remark}
  Dado $\mathfrak{c}.\mathfrak{c} = \mathfrak{c}$ tenemos resultados interesantes. 
  Primero $$\R \sim \R \times \R = \R^2$$

  De la misma manera pdoemos probar $$\R \sim \R^n \quad \forall n \in \N $$

Como cualquier intervalo (no vacío) tiene cardinal $\mathfrak{c},$ concluimos que para cualquier $\epsilon > 0$ y cualquier $n \in \N$, tenemos $$(0,\epsilon)\sim \R \sim \R^n$$
\end{remark}
\end{document}
