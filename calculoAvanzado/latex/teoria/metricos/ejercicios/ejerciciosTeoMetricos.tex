\documentclass[12pt]{article}

\usepackage[margin=1in]{geometry}
\usepackage{enumerate}
\usepackage{amsmath}
\usepackage{amssymb}
\usepackage{mathtools}
\usepackage{amsfonts}
\usepackage{amsthm}
\usepackage{graphicx}
\usepackage{fancyhdr}
\pagestyle{fancy}

\newcommand{\n}{\aleph_{0}}
\newcommand{\F}{\mathhbb{F}}
\newcommand{\Q}{\mathbb{Q}}
\newcommand{\C}{\mathbb{C}}
\newcommand{\R}{\mathbb{R}}
\newcommand{\K}{\mathbb{K}}
\newcommand{\E}{\mathbb{E}}
\newcommand{\I}{\mathbb{I}}
\newcommand{\N}{\mathbb{N}}
\newcommand{\Ra}{\Rightarrow}
\newcommand{\ra}{\rightarrow}
\newcommand{\ol}{\overline}
\newcommand{\norm}[1]{\left\lVert#1\right\rVert}

\theoremstyle{definition}
\newtheorem{definition}{Definición}[section]
\newtheorem*{remark}{Observación}
\newtheorem{theorem}{Teorema}
\newtheorem{lemm}{Lema}
\newtheorem{corollary}{Corolario}[theorem]
\newtheorem{lemma}[theorem]{Lema}
\newtheorem{prop}{Proposición}



\fancyhead[R]{Espacios Normados}
\fancyhead[L]{Alumno Javier Vera}
 \fancyhead[C]{Cálculo Avanzado}
\begin{document}
Ejercicio 1

Sea $x \in C[0,1]$,$\quad A = \{x \in C[0,1 ] / x(\frac{1}{2}) > 0\}$ es abierto

\begin{proof}
  Afirmo que para cualquier $x \in A$, tomamos $r = \frac{d_{\R}(x(\frac{1}{2}),0)}{2}$ entonces la bola $B(r,x) \subseteq A$ 

  Sea $y \in B(r,x)$ entonces $d_{\infty}(y(t),x(t)) < r$

  $$|y(1/2) - x(1/2) |  < \sup_{t \in [0,1]} |y(t) - x(t)| <  \frac{d_{\R}(x(\frac{1}{2}),0)}{2} = \frac{|x(\frac{1}{2})|}{2}$$

 Ahora si expandimos los módulos tenemos

 $$ - \left |\frac{x(1/2)}{2} \right | < y(1/2) - x(1/2) < \left |\frac{x(1/2)}{2} \right |$$

Considerando que $x(1/2) > 0$

$$\frac{1}{2}x(1/2) <  y(1/2) < \frac{3}{2} x(1/2)$$

Finalmente $$0 < \frac{1}{2} x(1/2) < y(1/2)$$

Entonces $y \in A \quad \forall y \in B(r,x)$

Por lo tanto $B(r,x) \subseteq A \quad \forall x \in C[0,1]$ 

Entonces todo punto de $A$ es interiór , otra forma de decir que $A$ es abierto
\end{proof}


\end{document}
