\documentclass[12pt]{article}

\usepackage[margin=1in]{geometry}
\usepackage{enumerate}
\usepackage{amsmath}
\usepackage{amssymb}
\usepackage{mathtools}
\usepackage{amsfonts}
\usepackage{amsthm}
\usepackage{graphicx}
\usepackage{fancyhdr}
\pagestyle{fancy}


\newcommand{\F}{\mathhbb{F}}
\newcommand{\C}{\mathbb{C}}
\newcommand{\R}{\mathbb{R}}
\newcommand{\K}{\mathbb{K}}
\newcommand{\E}{\mathbb{E}}
\newcommand{\N}{\mathbb{N}}
\newcommand{\Ra}{\Rightarrow}
\newcommand{\ra}{\rightarrow}
\newcommand{\ol}{\overline}
\newcommand{\norm}[1]{\left\lVert#1\right\rVert}

\theoremstyle{definition}
\newtheorem{definition}{Definición}[section]
\newtheorem*{remark}{Observación}
\newtheorem{theorem}{Teorema}
\newtheorem{lemm}{Lema}
\newtheorem{corollary}{Corolario}[theorem]
\newtheorem{lemma}[theorem]{Lema}

\fancyhead[R]{Espacios Normados}
\fancyhead[L]{Alumno Javier Vera}
\fancyhead[C]{Calculo Avanzado}
\begin{document}
\LaTeX \\

Calculo Avanzado

Universidad de Buenos Aires\\

Teoría 

Espacios Normados\\

Javier Vera


\newpage

Los espacios normados son casos particulares de los espacios vectoriales, en otras palabras, son espacios vectoriales con una norma. Dicho esto es evidente que deben tener una estructura algebraica de espacio vectorial y además una norma.
Ahora, ¿Qué es un norma?

\begin{definition}
	Sea $\E$ un espacio vectorial (Sobre $\R$ o $\mathbb{C}$). Una función $\norm{\cdot}:\E \ra [0,+\infty]$ es una norma si verifica las siguientes propiedades:
	 \begin{enumerate}[(1)]
		\item $\norm{x+y} \leq \norm{x} + \norm{y}$
		
		\item $\norm{\lambda \cdot x} = |\lambda| \cdot \norm{x}$

		\item $\norm{x} = 0$ si y solo si $x=0$

	\end{enumerate}
\end{definition}
\begin{definition}
  Un espacio vectorial con una norma se llama espacio vectorial normado o espacio normado
\end{definition}
\begin{remark}
	Si $\E$ es un espacio normado entonces es un espacio métrico con la distancia $d(x,y) = \norm {x-y}$
\end{remark}


\begin{remark}
Todo espacio normado es métrico , pero no todo epacio métrico es normado. 

	\begin{proof}Sea $(R^n,\delta)$ con $\delta$ la distancia discreta. Supongamos que la distancia discreta nos induce una norma, luego Sea $x \neq 0$ $ d(x,0) = \norm{x-0} \neq \lambda \norm{x-0} = \norm {\lambda x - 0 } = d(\lambda x,0)$ luego $d(x,0) \neq d(\lambda x,0)$ cosa que es absurda dada la distancia discreta
	\end{proof}
	Ejemplos de espacios normados

	$(R^n,\norm {\cdot}_{2})$
	
	$(R^n,\norm {\cdot}_{\infty})$ $\norm{x}_{\infty} = max_{1 \leq i \leq m} |x_{i}|$

\end{remark}
\begin{definition}
	Un espacio normado que es completo con la distancia $d(x,y) = \norm{x-y}$ se llama un espacio de Banch

	Ejemplos de Banach

	$\ell^{\infty} = \{(a_{n})_{n \in \N} \subset \R : a_{n} \leq \infty \quad \forall n \in \N \} $

	$\ell^1$

      \end{definition}
      \begin{remark}
	Si $\E$ es un espacio normado, la suma es continua en $\E \times \E$ y el producto es continua en $\R \times \E$ o en $\C \times \E$. 

	La demo es ejercicio de la practica, pero se ve con sucesiones facilmente, aprovechando que sabemos como se comportan las sucesiones en $R$  
      \end{remark}
      \begin{remark}
	En los espacios normados pasan algunas cosas que vemos en $R^n$ pero no en métricos generales
\begin{itemize}
  \item $\ol B(x,r) = \ol{B(x,r)}\quad $ y $\quad dim(B(x,r)) = 2r$
  \item $B(x,r) = x + B(0,r) = \{x+y : y\in B(0,r)\} = x +rB(0,1) = \{x+rz : z \in B(0,1)\}$
  \item $ x \neq 0 \Ra \norm{\frac{x}{\norm{x}}}= \frac{1}{\norm{x}} \norm{x} = 1  $
  \item $\frac{x}{2 \norm{x}} \in B(0,1) \quad \forall x \neq 0$
\end{itemize}
      \end{remark}
      \begin{remark}
	Atención. Cerrado y acotado no implica compacto en espacios normados como en $R^n$ , salvo que el espacio normado sea finito
      \end{remark}

      \begin{definition}
	Dos normas en un espacio vectorial son equivalentes si definen distancias equivalentes.

	Ejemplos no equivalentes: $C[0,1]$ con $d_{1}$ y $d_{\infty} $ no son equivalentes

	Equivalentes: $R^n$ con cualquier distancia (más abajo se demostrará)
      
      \end{definition}

      \begin{definition}
	Dos normas $\norm{\cdot}_{1}$ y $\norm{\cdot}_{2}$ si y solo si existens $c, \ol c > 0$ tales que 
	$$ c \norm{x}_{2} \leq \norm{x}_{1} \leq \ol c \norm{x}_{2} \quad \forall x \in \E $$
      
      \end{definition}
      \begin{remark}
	A diferencia de las distancias equivalentes que cumples solo una desigualdad norma cumple dos desigualdades
	Luego dos normas equivalentes dan métricas fuertemente equivalentes (a veces llamado uniformemente equivalentes)

\begin{proof}
$\Ra)$ Teniendo distancias fuertemente equivalentes sabemos que $c d_{2} \leq d_{1} \leq \ol c d_{2}$ el resto es trivial

$\quad \Leftarrow) $ Sabemos que $\norm{\cdot}_{1}$ y $\norm{\cdot}_{2}$ son equivalentes entonces dan los mismos abiertos. 

$ B(x,r)_{1} =$ bola con la distancia inducida por la norma 1

	$B(x,r)_{2}=$ bola con la distancia inducida por la norma 2

	Como $B(0,1)_{1}$ es abierta para $\norm{\cdot}_{2} $$\Ra \exists r > 0/ \quad B(0,r)_{2} \subset B(0,1)_{1}$

	Ahora tomemos $x \in \E x \neq 0$ 

	Luego $$ \norm{\frac{r}{2} \frac{x}{\norm{x}_{2}}}_{2} = \frac{r}{2} < r$$

	$$\frac{r}{2} \frac{x}{\norm{x}_{2}} \in B(0,r) \subset B(0,1)_{1} \Ra \norm{\frac{r}{2} \frac{x}{\norm{x}_{2}}}_{1} < 1 $$

	Entonces $$ \norm{x}_{1} \leq \frac{2}{r} \norm{x}_{2}$$

	La otra desigualdad es casi igual por tanto queda como ejericio para el lector
\end{proof}

\end{remark}
	
\begin{definition}
	Sean $(X,d_{X})$ e $(Y,d_{Y})$ dos espacios metrico, un mapa (función) $f:X \ra Y$ es llamado isometría si para cualquier $a,b \in X$ se tiene $$ d_{Y}(f(a),f(b)) = d_{X}(a,b)$$
\end{definition}

\begin{remark}
	Dos espacios métricos $X$ e $Y$ son llamados isométricos si existe alguna isometría entre ellos

\end{remark}

\begin{theorem}
	Si $\E$ es un espacio normado de dimension $n \in \N$ entonces es isométrico a $(R^{n},\norm{\cdot})$ y ademas la isometría que existe entre ellos es biyectiva 
\begin{proof}
  Como $\E$ es de dimension finita existe un isomorfismo $G:\E \ra \R^n $ dado por cambiar de cordenadas. Por comodidad tomemos su inversa $T: \R^n \ra \E$ tambien un isomorfismo. 

  Teniendo en cuenta la norma de $\E $ definimos 
  $$ \norm{x}_{2} = \norm{T(x)}_{\E}$$

  Veamos que es norma, sabemos que cae en $R^n$

1) $ \norm{0}_{2} = \norm{T(0)}_{\E} = \norm {0}_{\E} = 0$

2)$\norm{\lambda x}_{2} = \norm{T(\lambda x)}_{\E}=\norm{ \lambda T(x)}_{\E} = \lambda \norm{T(x)}_{\E} = \lambda \norm{x}_{2} $

La última propiedad sale trivialmente con las mismas ideas ya usadas, queda como ejercicio para el lector

Luego tenemos una isometría dada por el isomorfismo $T:(\R^n,\norm{\cdot}_{1}) \ra (\E, \norm{\cdot}_{\E})$ luego es inversible, pero la inversa de esta isometría es tambien una isometría. Veamoslo
	Sea $T$ una isometría entre espacios normados con inversa entonces $T^{-1}$ es tambien una isometría  
	$$ d_{1}(T^{-1}(x),T^{-1}(y)) = \norm {T^{-1}(x) - T^{-1}(y)}_{1} = \norm{T(T^{-1}(x) - T^{-1}(y))}_{\E}$$ 
	$$ = \norm{T(T^{-1}(x)) - T(T^{-1}(x))}_{\E} = \norm {x - y}_{\E} = d_{\E}(x,y) $$

	Luego

	$$ d_{\E}(x,y) = d_{1}(T^{-1}(x),T^{-1}(y))$$ 

	Por lo que $T^{-1}$ es una isometría

\end{proof}
\end{theorem}

\begin{corollary}
  Dos espacios normados con un isomorfismo isométrico $$T:(E,\norm{\cdot}_{E}) \ra (F \norm{\cdot}_{F})$$son ademas uniformemente homemorfos

	\begin{proof}
	  Por tener isomorfismo isométrico y considerando las distancias inducidas sabemos 
	$$  d_{F}(T(x),T(y)) = d_{E}(x,y)$$

	Luego sea $\epsilon > 0 $ tomemos $\epsilon = \delta$ luego
	$$ d_{E}(x,y) < \delta \Ra d_{F}(T(x),T(y)) = d_{E}(x,y) < \delta = \epsilon \quad \forall x,y \in \E$$

	$T$ es uniformemente continua, análogo para $T^{-1}$
	\end{proof}
\end{corollary}
\begin{theorem}
  En $R^{n}$ todas las normas son equivalentes

\begin{proof}
  Veamos que cualquier norma $\norm{\cdot}_{n}$ en $R^n$ es equivalente a $\norm{\cdot}_{1}$ Sabiendo$ \norm{x}_{1} = |x_{1}|+|x_{2}|+|x_{3}|+\dots +|x_{n}| $ y tomando $e_{n}$ el vector canonico n-esimo. 
 
 \noindent Sea $x \in \R^n \quad x=(x_{1},\dots,x_{n}) \Ra x = x_{1}e_{1}+x_{2}e_{2}+\dots+x_{n}e_{n}$

  \noindent Luego $$ \norm{x}_{n} \leq \norm{x_{1}e_{1}}_{n} + \norm{x_{2}e_{2}}_{n} + \dots + \norm{x_{n}e_{n}}_{n} = |x_{1}| \norm{e_{1}}_{n}+ \dots + |x_{n}| \norm{e_{n}}_{n}$$

  Sea $$M =\max_{1 \leq j \leq n}{\norm{e_{j}}_{n}} $$
Sabemos existe por que $n < \infty$
  
 \noindent Luego $$  |x_{1}| \norm{e_{1}}_{n}+ \dots + |x_{n}| \norm{e_{n}}_{n} \leq M(|x_{1}| + |x_{2}| +\dots + |x_{n}|) = M \norm{x}_{1}$$
   Finalmente $$ \norm{x}_{n} \leq M \norm{x}_{1}$$
  Ahora sea $g:(R^n, \norm{\cdot}_{1}) \ra \R \quad$ con $\quad g(x) = \norm{x}_{n}$

  $$ d_{\R}(g(x),g(y)) = | g(x) - g(y) | = |\norm{x}_{n} - \norm{y}_{n}| \leq \norm{x-y}_{n} \leq \ol c \norm{x-y}_{1} = \ol c d_{1}(x,y)$$
por desigualdad triangular y por lo que probamos arriba

Luego $g$ es continua por lo tanto manda compactos en compactos o lo que es lo mismo alcanza max y min en la imagen de un compacto

Sea $S = \{x \in \R^{n}: \norm{x}_{1}=1\}$ que es cerrado y acotado ,como está en $R^n$ es compacto

Sea $$m = \min_{x \in S}{g(x)} = \min_{x \in S}{\norm{x}_{n}}$$

Veamos $m \neq 0$. Supongamos $m = 0 \quad \exists x_{0} \in S / \quad \norm{x_{0}}_{n} = 0$ como $\norm{\cdot}_{n}$ es una norma $x_{0} = 0 \Ra \norm{0}_{1} = 0 \Ra x_{0} \notin S$ Absurdo

Sabiendo $x \neq 0$
$$\frac{x}{\norm{x}_{1}} \in S \Ra \norm{\frac{x}{\norm{x}_{1}}}_{n} \geq m \Ra \frac{1}{\norm{x}_{1}} \norm{x}_{n} \geq m \Ra \norm{x}_{n} \geq m\norm{x}_{1}$$
\end{proof}


\end{theorem}
\begin{corollary}
  Todo espacio normado $E$ finito es uniformemente homemorfo a $R^{n}$

  \begin{proof}
    Por ser $E$ finito sabemos que existe $T$ un isomorfismo isometrico con $R^{n} $ 

    $\Ra T $ es un homemorfismo uniforme 

    Luego como todas las normas de $R^n$ son equivalentes tenemos que la identidad entre $R^n$ con alguna norma y $R^n$ con cualquier otra norma es un homeomorfismo uniforme componiendo todo tenemos un isomorfismo uniforme con $R^n$ con cualquier norma

  $\Ra$ $E$ es uniformemente homemorfo a $R^n$
  \end{proof}
\end{corollary}


\begin{theorem}
Sea $E$ un espacio métrico de dimension finita entonces $E$ es completo (de Banach)

\begin{proof}
  Sea $(x_{n})_{n \in \N} \in \E$ sucesión de Cauchy, luego como sabemos que tenemos un homemorfismo $T$ con con $R^n$ y que homemorfismo uniforme preserva sucesion Cauchy y sucesiones convergentes

$(T(x_{n}))_{n \in \N} \in \R^n$ es de Cauchy. Luego como $\R^n$ es completo sucesiones de Cauchy convergen $\Ra T(x_{n})$ converge

Finalmente $x_{n} = T^{-1}(T(x_{n}))$ converge.

$\Ra$ Toda sucesion de Cauchy de $\E$ converge. Entonces $\E$ es completo 
\end{proof}
\end{theorem}

\begin{remark}
	La bola unidad abierta de $E$ es $$ B_{E} = \{X \in \E: \norm{x} < 1\}=B(0,1)$$

	La bola unidad cerrada de $E$ es $$ \ol B_{E} = \{X \in \E: \norm{x} \leq 1\}=\ol B(0,1)$$

\end{remark}
\begin{corollary}
	Si $E$ es un espacio normado de dimensión finita entonces $\ol B_{E}$ es un compacto.  Mas aún, todo cerrado y acotado es compacto
	\begin{proof}
	  Sea $E$ normado de dimensión finita sabemos que existe $f:(R^{n},d_{1}) \ra (\E,d_{2})$ homemorfismo uniforme y ademas sabemos que $d_{1}$ y $d_{2}$ son fuertemente equivalentes


		Sea $X \subseteq \E$ cerrado y acotado, considerando que en continuas preimagen de cerrado es cerrado $f^{-1}(X) \in R^n$ es un cerrado. 

		Luego tenemos que cerrado en $R^n$ es completo, pero además homemorfismo manda completos en completos luego $X = f(f^{-1}(X))$ es completo

		Sabiendo que homemorfismo preservan totalmente acotados y que ademas por tener equivalencia fuerte (uniforme) $f$ preserva acotados, usamos la misma idea y vemos que $X$ es totalmente acotado 

		$X$ es completo y totalmente acotado por tanto es compacto

	\end{proof}

\end{corollary}

\begin{remark} 
	Si $T:R^n \ra R^m$ es una transformación lineal.
	\begin{itemize}
		\item T es función continua si consideramos en ambos la norma euclídea
		\item Luego T es continua para cualquier par normas que pongamos en $R^n$ $R^m$ por que todas son equivalentes
		\item Y lo mismo pasa con una transformación lineal entre dos espacios normados de dimensión finita

		Esto requiere una demostración por lo menos informal
	\begin{proof}
			Sea $T:E \ra F$ tal que $dim(T)= n$ y $dim(F) =m$ luego tenemos $R,S$ isomorfismos lineales dados por tomar coordenadas en $R^n$ y $R^m$ respectivamente 

		Entonces tenemos $\ol T : R^{n} \ra R^{m}$ dada por $\ol T =S \circ T \circ R^{-1}$ que es lineal por ser composicion de lineales 

		Luego por el insciso anteriór $\ol T$ es continua entonces $T:E \ra F$ dada por $T = S^{-1} \circ \ol T \circ R$ es continua

	\end{proof}
	\item Mas aún si $E$ y $F$ son normados y $E$ es de dimensión finita, entonces la transformación lineal $T:E \ra F$ es continua 
	
	\begin{proof}

		Por lineal sabemos que si $dim (E)$ es finita luego $T:E \ra F$ 

		Tenemos $dim (Im(T)) \leq \infty$. Entonces $T^{0}:E \ra Im(T)$ sabemos que es continua por el insciso anteriór. Tambien tenemos $T^1$ la operacion contención (que es continua). 

		Luego tenemos $T:E \ra F$ dada por $T = T^0 \circ T^1$ que es continua por ser composición de continuas 
		

	\end{proof}
	\end{itemize}
\end{remark}
\begin{remark}
  Transformación lineal , operador lineal , funcion lineal o aplicacion lineal son lo mismo.
\end{remark}

\begin{definition}
	Sean $E,F$ dos espacios vectoriales normados (sobre el mismo cuerpo de escalares). Un mapa $T:E \ra F$ es un operador lineal continuo si:
	\begin{itemize}
		\item Es una transformación lineal
			\begin{itemize}
				\item $T(x+y) = T(x) + T(y) \quad \forall x,y \in E$
				\item $T(\lambda x) = \lambda T(x) \quad \forall \lambda \in \K \quad \forall x \in E$
			\end{itemize}

		\item Es una funcion continua, con las métricas inducidas por las normas

	\end{itemize}
\end{definition}

\begin{remark}
  Si $T:E \ra W$ es lineal tenemos $T(0_{E}) = 0_{W}$. Si $T$ es continua en 0 luego $$\forall \epsilon > 0 \quad \exists \delta > 0 \quad  T(B(0_{E},\delta))  \subseteq B(0_{W},\epsilon) $$
\end{remark}


\begin{lemm}
  Sea $T$ transformación lineal. $T$ continua en 0 $\iff T$ es unforme continua 	
	\begin{proof}
  		$\Ra ) \quad T$ continua en 0 luego dado $\epsilon > 0 \quad \exists \delta > 0 / \quad T(B(0_{E},\delta)) \subset B(0_{W},\epsilon)$\\

    Luego sea $$ x_{0} ,x_{1} \in \E / \quad \norm{x_{1}-x_{0}}_{E} < \delta \Ra  x_{1}-x_{0} \in B(0_{E},\delta)$$

    $$T(x_{1}-x_{0}) \in B(0_{W},\epsilon) \Ra \norm {T(x_{1}-x_{0})}_{W} < \epsilon $$

    Entonces $$\norm{T(x_{1}) - T(x_{0})} < \epsilon$$ 

    Además este $\delta$ sirve $\forall x_{0},x_{1} \in E \Ra T$ es uniformemente continua \\
	
$\quad \Leftarrow)$ Sea $z_{0} \in E / \quad T$ es uniforme continua en $z_{0}$. Entonces es continua en este punto 

	Luego dado $\epsilon > 0$ 

	$$ \exists \delta > 0 / \quad T(B(z_{0},\delta)) \subset B(T(z_{0}),\epsilon )$$

	Sea $x \in E $ tal que $\norm{x} < \delta$ 

	$$ x \in B(0,\delta) \Ra  z_{0} + x \in B(z_{0},\delta ) $$

      Por lo que $T(z_{0} + x) \in B(T(z_{0}),\epsilon)$

      Luego $$\norm{T(z_{0} + x) - T(z_{0})} = \norm{T(x)} < \epsilon  $$

      Recapitulando dado $\epsilon > 0 \quad \exists \delta > 0 / \quad \norm{x} < \delta \Ra \norm{T(x)} < \epsilon $

      $\Ra T$ es continua en 0

  \end{proof}
  \end{lemm}
  \begin{theorem}
    Sean $E,W$ espacios normados y $T:E \ra F$ transformación lineal
    \begin{enumerate}
      	 \item $T$ es continua en el origen.
	 \item $T$ es continua en algún punto
	 \item $T$ es continua
	 \item $T$ es uniformemente continua 
    \end{enumerate}

    \begin{proof}
      Ya vimos que $(2) \Ra (1)$ y $(1) \Ra (4)$
      Es evidente que $(4) \Ra (3)$ y $(3) \Ra (2)$
    \end{proof}
\end{theorem}

\begin{definition}
Decimos que una transformación lineal $T:E \ra F$ es acotada si $\exists c > 0 $ tal que 

    $$ \norm{T(x)}_{F} \leq c \norm{x}_{E} \quad \forall x \in E$$ \\

 \noindent   Equivalentemente $T$ es acotado si:

    $$ \sup_{x \in B_{E}}{\norm{T(x)}_{F}} < \infty$$
    
  \end{definition}


\begin{proof}
	$1) \Ra 2)$ Sea $x \in B_{E} \ra \norm{x} < 1$ 

	luego $$ \norm{T(x)} \leq c \norm{x} < c \quad \forall x \in B_{E}$$ 

	entonces $$ \sup_{x \in B_{E} }\norm{T(x)} < c $$
	
  $ 2) \Ra 1)$ Sea $$M = \sup_{x \in B_{E}}{\norm {T(x)}_{F}}$$

Ahora tomamos $ x \in E \quad x \neq 0 $

Luego $$ \frac{x}{(1+ \epsilon)\norm{x}} \in B_{E} \Ra \norm{T \biggl( \frac{x}{(1+ \epsilon)\norm{x}}\biggr) } \leq M$$
      
Como $T$ es lineal $$ \norm{\frac{1}{(1+\epsilon)\norm{x}}T(x)} \leq M \Ra \norm{T(x)} \leq M(1+ \epsilon) \norm{x}$$ 

Como $\epsilon$ es arbitario 

$$ \norm{T(x)}_{F} \leq M \norm{x}_{E} \quad \forall x \neq0$$ 

Y para $x=0$ sabemos que vale tambien 
      \end{proof}

      \begin{remark}
	$T$ transformación lineal continua y acotada

	$$M = \sup_{x \in B_{E}}{\norm {T(x)}_{F}}$$

	Es el menor $M$ tal que $$ \norm{T(x)}_{F} \leq M \norm{X}_{E}$$
      	
	
      \end{remark}

      \begin{theorem}
      $T:E \ra F$ transformación lineal entonces
     \begin{center}
       $T$ Lipschitz $\iff$ $T$ es acotada
     \end{center}

      \begin{proof}
	$\Ra)$ Como $T$ es Lipschitz luego es continua en todos lados en particular en el 0

       Entonces sea $\epsilon = 1 \quad \exists \delta > 0$ $$T(B(0,\delta)_{F}) \subset B(0_{E}, \epsilon) = B(0_{E},1)$$
	
       Sea $x \in E$ sabemos que $$\frac{\delta}{2}\frac{x}{\norm{x}} \in B(0,\delta)$$

       Luego $$ T\biggl( \frac{\delta}{2}\frac{x}{\norm{x}} \biggr) \in B(0_{E},1)$$

       Entonces $$ \norm{T\biggl( \frac{\delta}{2}\frac{x}{\norm{x}} \biggr)} \leq 1 \Ra \norm{T(x)} \leq \frac{2}{\delta}\norm{x}$$ 

       Luego T es acotada

     $\Leftarrow)$ Sea acotada luego $\norm {T(x)}_{F} \leq c \norm{x}_{E}$

     $$ d(T(x),T(y)) = \norm {T(x) - T(y)} = \norm{T(x-y)}\leq c\norm{x-y} = c.d(x,y)$$

     Luego $T$ es Lipschitz
      \end{proof}
      \end{theorem}

      \begin{corollary}
	Sea $T$ una transformación lineal. 
	\begin{center}Luego $T$ continua $\iff T$ es acotada $\iff T$ es Lipschitz

	\end{center}
      \end{corollary}

      \begin{definition}
	Sean $E,F$ espacios normados y $T:E \ra F$ un operador lineal acotado. Definitmos la norma de $T$ como
	$$\norm{T} = \sup_{x \in B_{E}} \norm{T(x)}_{F} = \inf{\{c: \norm{T(x)} \leq c\norm{x} \quad \forall x \in E \}}$$
	La demostracion es simple y queda como ejercicio para el lector

      \end{definition}

      \begin{definition}
	Sean $E, F$ espacios normados.
	\begin{center}
	  $L(E,F) =$ $\{ T:E \ra F$ con $T$ lineal y acotado $\}$
	\end{center}
	\begin{itemize}

	  \item $T_{1} T_{2} \in L(E,F) \Ra (T_{1}+T_{2})(x) + T_{1}(x) + T_{2}(x)$
	  \item $(\lambda T_{1}) (x) = \lambda T_{1}(x)$  

	\end{itemize}
      \end{definition}
      \begin{theorem}
	Sean $E,F$ espacios normados. Si $F$ es banach entonces $L(E,F)$ es Banach. En particular $L(E, \R)$ es banach para todo espacio normado $E$
      
	\begin{proof}
	  $(T_{n})_{n}$ sucesión de Cauchy en $L(E,F)$. En particular $T_{n}$ es acotada: 
	  $$\exists k > 0 / \quad \norm{T_{n}} \leq K$$
	Para $x \in E \quad x \neq 0 \quad (T_{n}(x))_{n} \subset F$
	$$\norm{T_{n}(x) - T_{m}(x)}_{F} = \norm{(T_{n} - T_{m})(x)} \leq \norm{T_{n} - T_{m}}_{L(E,F)} \norm{x}$$	
	Luego como $T_{n}$ es de Cauchy luego dado $\epsilon > 0$
	$$ \exists  n_{0} / \quad \norm{T_{n} - T_{m}} \leq \frac{\epsilon}{\norm{x}} \quad \forall n \geq n_{0} $$
Juntando todo
$$ \norm{T_{n}(x) - T_{m}(x)} < \frac{\epsilon}{\norm{x}} \norm{x} = \epsilon \quad \forall n \geq n_{0}$$ 
Luego $T_{n}(x)$ es de Cauchy en $F$ y como $F$ es completo $$ \exists \lim_{x \ra \infty}{T_{n}(x)}$$
(Para $x = 0 $ es evidente que el límite existe)

\begin{itemize}
  \item Definimos $\lim{T_{n}(x)} = T(x)$ es facil ver que $T$ es lineal
  \item $\norm{T_{n}(x)}_{F} \leq \norm{T_{n}}_{L(E,F)} \norm{x}_{E} \leq K \norm{x}_{E}$
    $$ \norm{T(x)} =\lim_{n \ra \infty}{\norm{T_{n} (x)}} \leq \lim_{n \ra \infty}{K \norm{x}} = K \norm{x}_{E} \Ra \norm{T(x)} \leq K \norm{x}$$
  \end{itemize}
    Luego $T \in L(E,F)$

    Por último como $T_{n}$ es de Cauchy dado $\epsilon > 0 \quad \exists n_{0} \norm{T_{n} - T_{m}} \leq \frac{\epsilon}{2} \quad \forall n,m \geq n_{0}$
    
    Si $x \in B_{E}$ $$ \norm{T_{n}(x) - T_{m}(x)} = \norm{(T_{n} -T_{m})(x)} \leq \norm{T_{n} - T_{m}} \leq \frac{\epsilon}{2} \quad \forall n,m \geq n_{0}$$ 
   
    (Se ve mirando la definición de norma de $T$)

  Como $T_{m}(x) \ra T(x)$ 
  $$ \norm {(T_{n} - T)(x)} = \norm{T_{n}(x) - T(x)} \leq \frac{\epsilon}{2} \quad \forall n \geq n_{0} \quad \forall x\ \in B_{E}$$
  
  Pero luego $$\sup_{x \in B_{E}}{\norm{(T_{n} - T)(x)}} \leq \frac{\epsilon}{2} $$

  Finalmente por definición de norma $$ \sup_{x \in B_{E}}{\norm{(T_{n} - T)(x)}} = \norm{T_{n} - T} \leq \frac{\epsilon}{2}$$ 

  $$ T_{n} \ra T$$

  Entonces partimos con una sucesión de Cauchy $T_{n} \in L(E,F)$ cualquiera y vimos que converge

  $\Ra L(E,F)$ es completo (Banach) 
      \end{proof}
	\end{theorem}
	\begin{lemma}
	  Sea $E$ un espacio normado y $S \subset E$ un subespacio cerrado propio. Dado $0 < \alpha < 1$ existe $x_{\alpha} \in E$ talque $\norm{x_{\alpha}} = 1$ y $\norm{x_{\alpha} - s} > \alpha $ para todo $s \in S$ 
	
	  \begin{proof}
	    $x \in E \setminus S \quad r=d(x,S) > 0.$ Luego tenemos  $0 < \alpha < 1 \Ra \frac{r}{\alpha} > r$

	    Entonces $$\exists b \in S / \quad \norm{x - b} = d(x,b) < \frac{r}{\alpha} \Ra \alpha < \frac{r}{\norm{x - b}} $$

	    Tomemos $x_{\alpha} = \frac{x - b}{\norm{x - b}}$ este cumple lo pedido, mostremosló

	  Sea $s \in S$ $$ \norm{x_{\alpha} - s} = \norm{\frac{x-b}{\norm{x-b}} - s} = \norm{\frac{1}{\norm{x - b}}(x -b -s \norm{x -b})} = \frac{1}{\norm{x - b}} \norm{x - (b + s \norm{x - b})}$$ 

	  Lo que está entre paréntesis pertenece a $S$ entonces 
	  $$\frac{1}{\norm{x - b}} \norm{x - (b + s \norm{x - b})} \geq \frac{1}{\norm{x - b}}r > \alpha  $$

	$$ \norm{x_{\alpha} - s} \geq \alpha \quad \forall s \in S $$
	  \end{proof}


	\end{lemma}

	\begin{corollary}
	  Sea $E$ un espacio normado. Entonces $E$ es de dimensión finita si y solo si $\ol B_{E}$ es un compacto
	
	  \begin{proof}
	  $\Ra)$ ya lo probamos antes viendo que $E$ dimensión finita es homemorfo con $R^n$ y etc

      $\quad \Leftarrow)$ Supongamos que $E$ es de dimensión infinita. Sea $x_{1} \in E$ con $\norm{x_{1}} = 1$. Luego sea $S = <x_{1}> $ sabemos que no es todo $E$ por que $E$ es infinito y ademas sabemos que es cerrado 
      
      \begin{itemize}
		\item Luego por lema de Riesz $\exists x_{2} \in E $ tal que $\norm{x_{2}} = 1 \quad \norm{x_{2} - s} > \frac{1}{2} \quad \forall s \in S$ 

		\item Luego tomando $S=<x_{1},x_{2}>$ hacemos lo mismo y conseguimos un $x_{3}$
	\end{itemize}

      Y así construimos una sucesión que esta en $\ol B_{E}$ dado que la norma de sus elementos es siempre 1. Sin embargo esta sucesión no converge dado que la distancia es siempre $\frac{1}{2}$ pero entonces $\ol B_{E}$ tiene suecesión sin subsucesión convergente luego no puede ser compacta lo que es absurdo
	  \end{proof}

	\end{corollary}

	\begin{theorem}
	  Si $E$ es un espacio de Banach de dimension infinita, entonces no puede tener una base (algebráica) numerable
		\begin{proof}
  			$\{v_{i}\}_{i \in I}$ base de $E$ 
			
			Luego $\forall x \in E \quad \exists (\alpha_{i})_{i \in I}$ con 
			\begin{center}$x = \sum_{i \in I}{\alpha_{i} v_{i}} \quad $ con $ \alpha_{i} \neq 0$ solo para finitos valores de $i$\end{center}
			Como $E$ es Banach y $dim(E) = \infty$ por teorema de Baire (suponiendo que es numerable) 

			$\Ra I$ es NO numerable
		\end{proof}
\end{theorem}

\begin{remark}
  Un hiperplaon es cerrado o denso
\end{remark}

\begin{theorem}
  Sea $\alpha: E \ra \R$ un funcional lineal no nula. Entonces $\alpha$ es  continua si y solo si $\ker{\alpha} $ es un subespacio cerrado
\begin{proof}
  $\alpha \neq 0 \Ra \ker{\alpha}$ es un hiperplano. 

  Luego \begin{center} 
    $\ker{\alpha}$ cerrado $\iff \alpha^{-1}(0)$ cerrado. 
  \end{center}
    Como la imagen es $R$ alcanza con mirar preimagen de 0
  
\end{proof}

\end{theorem}

\end{document}
