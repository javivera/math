\documentclass[12pt]{report}

\usepackage[margin=1in]{geometry}
\usepackage{enumerate}
\usepackage{amsmath}
\usepackage{amssymb}
\usepackage{mathtools}
\usepackage{amsfonts}
\usepackage{amsthm}
\usepackage{graphicx}
\usepackage{fancyhdr}
\pagestyle{fancy}

\newcommand{\n}{\aleph_{0}}
\newcommand{\F}{\mathhbb{F}}
\newcommand{\Q}{\mathbb{Q}}
\newcommand{\C}{\mathbb{C}}
\newcommand{\R}{\mathbb{R}}
\newcommand{\K}{\mathbb{K}}
\newcommand{\E}{\mathbb{E}}
\newcommand{\I}{\mathbb{I}}
\newcommand{\Z}{\mathbb{Z}}
\newcommand{\N}{\mathbb{N}}
\newcommand{\Ra}{\Rightarrow}
\newcommand{\ra}{\rightarrow}
\newcommand{\ol}{\overline}
\newcommand{\norm}[1]{\left\lVert#1\right\rVert}
\newcommand{\open}{\mathrm{o}}


\theoremstyle{definition}
\newtheorem{definition}{Definición}[section]
\newtheorem*{remark}{Observación}
\newtheorem{theorem}{Teorema}
\newtheorem{lemm}{Lema}
\newtheorem{corollary}{Corolario}[theorem]
\newtheorem{lemma}[theorem]{Lema}
\newtheorem{prop}{Proposición}
\newtheorem{ej}{Ejercicio}


\fancyhead[R]{Espacios Métricos}
\fancyhead[L]{Alumno Javier Vera}
\fancyhead[C]{Cálculo Avanzado}

\begin{document}

\begin{ej} Ver si las siguientes funciones son distancias.
  \begin{itemize}
    \item $d_{1}(x,y) = (x - y)^2$ No es distancia.
      \begin{proof}
	$d_{1}(-1,1) = 4 > 1 + 1 = d_{1}(-1,0) + d_{1}(0,1)$
      \end{proof}
    \item $d_{2}(x,y) = \sqrt{| x - y|}$ es una distancia
      \begin{enumerate}
	\item $d_{2}(x,y) = 0 \iff \sqrt{| x - y |} = 0 \iff |x-y| = 0 \iff x=y$
	\item $d_{2}(x,y) = d_{2}(y,x)$ es trivial
	\item $d_{2}(x,y)^2 = |x-y| < |x-z| + |z - y| \leq |x - z| + |z - y | + 2\sqrt{|x-z||z-y|} = (\sqrt{|x-z|}+ \sqrt{|z - y|})^2 = (d_{2}(x,z) + d_{2}(z,y))^2$ 
      
	  Luego $d_{2}(x,y)^2 \leq (d_{2}(x,z) + d_{2}(z,y))^2$

	  Y es tirivial ver que entonces $d_{2}(x,y) \leq d_{2}(x,z) + d_{2}(z,y)$
      \end{enumerate}
    \item $d_{3}(x,y) = |x^2 - y^2|$ Es facil ver que no es distancia $d_{3}(-2,2) = 0$
    \item $d_{4}(x,y) = |x -2y|$ Es trivial devuelta $d_{4}(2,1) = 0$
    \item $d_{5}(x,y) = \frac{|x-y|}{1 + |x-y|}$. Tomemos la función $f : \R \ra \R$ dada por $f(t) = \frac{t}{1+xt}   $. Viendo que su derivada es siempre mayor que 0 podemos notar que esta función es estrictamente creciente

	    Además $f(a) + f(b) - f(a + b) = \frac{a}{1+a} + \frac{b}{1+b} - \frac{a+b}{1 + a+b} = \frac{a((1+b)(1+a+b)) + b((1+a)(1+a+b)) - (a+b)((1+a)(1+b))}{(1+a)(1+b)(1+a+b)}$

	    $= \frac{(a+ab)(1+a + b) + (b+ab)(1 + b+ a) - (a+b)(1+a + b + ab)}{(1+a)(1+b)(1+a+b)} = \frac{a + 2ab + a^2+a^2b +ab^2 + b +2ab +b^2 + ab^2 + ba^2 - (a + b + a^2 +ba + ab +b^2 + a^2b + ab^2)}{  (1+a)(1+b)(1+a+b)} = \frac{ a + 4ab + a^2+2a^2b +2ab^2 + b  +b^2  - (a + b + a^2 +ba + ab +b^2 + a^2b + ab^2)  }{(1+a)(1+b)(1+a+b)}  = \frac{ 2ab  +ab^2  + a^2b }{(1+a)(1+b)(1+a+b)} \geq 0$

	    Por lo tanto $f(a + b) \leq f(a) + f(b)$

Entonces $d(x,y) = f(|x-y|) \leq f(|x-z| + |z-y|) \leq f(|x-z|) + f(|z-y|) = d(x,z) + d(z,y)$

La primera desigualdad vale por que $f$ es creciente y usando la desigualdad de módulos de siempre, la segunda vale por lo probado arriba


\end{itemize}
\end{ej}

\begin{ej}
	Es una clásica demostración de taller de cálculo. 
\end{ej}


\begin{ej}
  Sean $X$ un conjunto y $\delta : X \times X \ra \R$ definida por
$$
\delta(x,y) = \left\{
        \begin{array}{ll}
	 1 & \quad \text{si } x \neq y  \\
	  0 & \quad \text{si } x = y
        \end{array}
    \right.
$$

\noindent Verificar que $\delta$ es una métrica y hallar los abiertos de $(X, \delta)$

\noindent Nota: $\delta$ se llama metrica discreta y $(X,\delta)$ espacio métrico discreto

\begin{proof}
  \begin{enumerate}
    \item $\delta (x,y) = 0 \iff x = y$
    \item Supongamos $x\neq y \Ra \delta (x,y) = 1 = \delta(y,x)$
    \item Supongamos devuelta $x \neq y$ si no es obvio que vale , $\delta (x,y) = 1 \leq \delta (x,z) + \delta (z,y)$ esto vale seguro , por que no puede suceder $\delta (x,z) = \delta (z,y) = 0$ por que esto implicaría $x = z = y$ absurdo 
  \end{enumerate}
\end{proof}
\end{ej}

\begin{ej}
  Sea $N : \Z \ra \R$ la funcion definida por 

$$
N(x) = \left\{
        \begin{array}{ll}
	  2^{-n} & \quad \text{si } a \neq 0, \quad p^n | a \quad  \text{ y } \quad p^{n+1} \nmid a  \\
	  0 & \quad \text{si } a = 0
        \end{array}
    \right.
$$

\noindent donde $p$ es un primo fijo, y sea $d:\Z \times \Z \ra \R$ dada por $d(a,b) = N(a-b)$. Probar que $(\Z ,d)$ es un espacio métrico

\begin{proof}
  Primero definamos para cada entero no nulo $\phi_{p}(a)$ que es el mayor $n \in \N$ tal que $p^n |a $

  Es simple ver $\phi_{p}(a) = \phi_{p}(-a)$ tambien $\phi_{p}(a + b) \geq \min{\{\phi_{p}(a),\phi_{p}(b)}\}$

  Ahora podemos reescribir
$$
d(a,b) = \left\{
        \begin{array}{ll}
	  2^{-\phi_{p}(a-b)} & \quad \text{si } a \neq b  \\
	  0 & \quad \text{si } a = b
        \end{array}
    \right.
$$

\begin{enumerate}
  \item Sea $d(a,b) = 0$ entonces $a =b $ por definición , por que $2^n \neq 0 \quad \forall n \in \Z$
  \item Asumiendo $a \neq b$ tenemos $d(a,b) = 2^{-\phi_{p}(a-b)} = 2^{-\phi_{p}(-(a-b))} = 2^{-\phi_{p}(-a+b)}= d(b,a)$
  \item Ahora consideremos que $\phi_{p}(a - b) = \phi_{p}((a-c) + (c - b)) \geq \min{\{\phi_{p}(a-c) ,\phi_{p}(c-b)\}}$

    Tambien supongo por comodidad $a \neq b \neq c$ por comodidad, si alguno fuera igual la demostración es trivial

    \noindent $d(a,b) = 2^{-\phi_{p}(a-b)} \leq 2^{-\min{   \{\phi_{p}(a-c) ,\phi_{p}(c-b) \}}} = \min{ \{     2^{\phi_{p}(a-c)} ,2^{\phi_{p}(c-b)} \} } \leq 2^{\phi_{p}(a-c)}  + 2^{\phi_{p}(c-b)}$ 

    Finalmente $d(a,b) \leq d(a,c) + d(c,b)$

    Entonces $d$ es una métrica y por lo tanto $(\Z, d)$ es un pár conjunto, métrica o lo que es lo mismo , un espacio métrico
\end{enumerate}
\end{proof}
\end{ej}


\begin{ej}
  Sea $\ell_{\infty} = \{(a_{n})_{n \in \N} \subseteq \R : (a_{n})_{n \in \N} \text{ es acotada} \}$. Se considera $d: \ell_{\infty} \times \ell_{\infty} \ra \R$ definida por $d(a_{n},b_{n}) = \sup_{n \in \N}{|a_{n} - b_{n}|}$. Probar que $(\ell_{\infty},d)$ es un espacio métrico. 

  \begin{proof}
    \begin{enumerate}
      \item $d(a_{n},b_{n}) = 0 \iff \sup_{n \in \N}{|a_{n} - b_{n}| = 0} \iff 0 \leq |a_{n} - b_{n}| \leq 0 \quad \forall n \in \N $

$	\iff |a_{n} - b_{n}| = 0 \quad \forall n \in \N \iff a_{n} = b_{n} \quad \forall n \in \N$ 
      \item $d(a_{n},b_{n}) = \sup_{n \in \N}|a_{n} - b_{n}| = \sup_{n \in \N}|b_{n} - a_{n}| = d(b_{n},a_{n})$
      \item  Sabemos que $|a_{n} - b_{n}| \leq |a_{n} + c_{n}| + |c_{n} - b_{n}| $

	$   \sup_{n \in \N}|a_{n} - b_{n}| \leq  \sup_{n \in \N}(|a_{n} + c_{n}| + |c_{n} - b_{n}|) =   \sup_{n \in \N}|a_{n} + c_{n}| +  \sup_{n \in \N}|c_{n} - b_{n}| $

      $  d(a_{n},b_{n}) \leq d(a_{n},c_{n}) + d(c_{n},b_{n})  $
    \end{enumerate}
  \end{proof}
\end{ej}


\begin{ej}
  Dados $a,b \in \R, a < b$, se define $\mathcal{C}[a,b] = \{f:[a,b ] \ra \R : f \text{ es continua}\}$. Probar que son espacios métricos.
  \begin{enumerate}[i.]
    \item $(\mathcal{C}[a,b],d_{1})$ con $d_{1}(f,g) = \int_{a}^{b} | f(x) - g(x)|dx$
      
    \item $(\mathcal{C}[a,b],d_{\infty})$, con $d_{\infty}(f,g) = \sup_{x \in [a,b]}|f(x) - g(x)|$
      \begin{proof}
Son demostraciones de taller. De todas maneras las desigualdades salen usando $|f(x) - f(y)| \leq |f(x) - h(x)| + |h(x) - f(y)|$
	
      \end{proof}
  \end{enumerate}
\end{ej}

\begin{ej}
	Sea $(X,d)$ un espacio métrico. Se define $d'(x,y) = \frac{d(x,y)}{1+d(x,y)}
	$
	\begin{enumerate}[i)]
		\item $d'$ es una métrica en $X$, que satisface $0 \leq d'(x,y) \leq 1 \quad \forall x,y \in X$
			\begin{proof} 
			Veamos que es métrica , las dos primeras propiedades son triviales, veamos la desigualdad, usando la misma funcione que habíamos usado antes $f(t) = \frac{t}{1 + t}$ tenemos que $d'(x,y) = f(d(x,y)) \leq f(d(x,z) + d(z,y)) \leq f(d(x,z)) + f(d(z,y)) = d'(x,z) + d'(z,y)$

			Esto vale por que $f$ es creciente y por que $d$ es métrica

			Por otro lado es trivial que $0 \leq d'(x,y)$

			Supongamos que $d'(x,y) > 1 \Ra d(x,y) > 1 + d(x,y)$ lo que es absurdo , entonces $d'(x,y) \leq 1$
			\end{proof}
		\item $A \subseteq X$ es abierto para la métrica $d$ si y sólo si lo es para la métrica $d'$ 
			\begin{proof}
				Tomemos cualquier bola abierta en $d$ , $B_d(x,r)$

				Ahora dado $y \in B_{d}(x,r)$ sabemos que $d(x,y) < r \Ra d'(x,y) =  \frac{d(x,y)}{1 + d(x,y)} < d(x,y) <  r$ 

				Por lo tanto $y \in B_{d'}(x,r)$ entonces $B_d(x,r) \subseteq B_{d'}(x,r)$

				$\Leftarrow$) Entonces si tenemos un abierto con $d'$ en $A \subseteq X$ tenemos que dado $x \in X$ existe $B_{d'}(x,r) \subseteq A$ luego $B_d(x,r) \subseteq B_{d'}(x,r) \subseteq A$ por lo tanto tambien es abierto en $d$ 

				$\Ra)$ Ahora tomemos nuevamente un abierto en $A$ con respecto a $d$. Dado $x \in A$ tenemos $B_d(x,r) \subseteq A$. 

				Ahora si consideramos $r'= \frac{r}{r+1} $ podemos afirmar que $B_{d'}(x,r') \subseteq B_d(x,r) \subseteq A$

				Probémoslo, sea $y \in B_{d'}(x,r')$ entonces $d'(x,y) < r' =  \frac{r}{r+1}$, luego $\frac{d(x,y)}{1+d(x,y)} < \frac{r}{r+1} $ 

				Entonces $d(x,y) < \frac{r}{r+1}(1+d(x,y)) \leq \frac{r}{r+1}(1+r) = r$

				Finalmente $d(x,y) < r$ entonces $y \in B_d(x,r)$

				Entonces $A$ es abierto con respecto a $d'$
			\end{proof}

		\item Deducir que $(x_n)_n$ converge a $x$ con respecto en la métrica $d$ si y sólo si converge a $x$ con respecto a la métrica $d'$

			\begin{proof}
			$\Ra $) Converge en $d$ entonces $\forall \epsilon > 0$  $\exists n_0 \in \N$ tal que $d(x,x_n) \leq \epsilon \quad \forall n \geq n_0$ 

			Como $d'(x,x_n) < d(x,x_n) \quad \forall n \in \N$. Tenemos que:  

			$\forall \epsilon > 0$ $\exists n_0$ $d'(x,x_n) < d(x,x_n) \leq \epsilon \quad \forall n \geq n_0$

			Por lo tanto $x_n \ra x$ con la métrica $d'$

			$\Leftarrow )$ Supongamos que $x_n$ converge a $x$ con d'. Ahora dado un $\epsilon > 0$ sabemos que $\exists r>0 $ tal que $B_{d'}(x,r) \subseteq B_d(x,\epsilon)$ y por convergencia de $x_n$ tenemos que $\exists n_0$ tal que $x_n \in B_{d '}(x,r) \subseteq B_d(x,\epsilon)\quad \forall n \geq n_0 $ entonces dado un $\epsilon$ conseguimos un $n_0 \in \N$ tal que $x_n \in B_{d}(x,\epsilon) \quad \forall n \geq n_0$ y esto lo podemos hacer para cualquier $\epsilon >0$

			Entonces $x_n$ converge a $x$ con la métrica $d$
		\end{proof}
		
		
			
			
	\end{enumerate}
\end{ej}


\begin{ej}
Sean $(X_{1},d_{1})$ y $(X_{2}.d_{2})$ espacios métricos. Consideremos el conjunto $X_{1} \times X_{2}$ y la aplicación $d: (X_{1} \times X_{2}) \times (X_{1} \times X_{2}) \ra \R$ dada por $$d((x_{1},x_{2}),(y_{1},y_{2}) ) = d_{1}(x_{1},y_{1}) + d_{2}(x_{2},y_{2})$$ 

\begin{enumerate}[(a)]
    \item Probar que $d$ define una métrica en $X_{1} \times X_{2}$
      \begin{proof}
	\begin{enumerate}[i.]
	  \item Por comodida tomemos $x = (x_{1},x_{2})$ e $y = (y_{1},y_{2})$  $d(x,y) = d_{1}(x_{1},y_{1}) + d_{2}(x_{2},y_{2})$ como ambas $d_{1},d_{2}$ son distancias entonces son mayores a $0$ entonces   $d((x_{1},x_{2}),(y_{1},y_{2})) \geq 0$
	  \item $d(x,y) =  d_{1}(x_{1},y_{1}) + d_{2}(x_{2},y_{2}) =  d_{1}(y_{1},x_{1}) + d_{2}(y_{2},x_{2}) = d(y,x)   $
	  \item $d(x,y) =  d_{1}(x_{1},y_{1}) + d_{2}(x_{2},y_{2}) \leq d_{1}(x_{1},z_{1}) + d_{1}(z_{1},y_{1}) + d_{2}(x_{2},z_{2}) + d_{2}(z_{2},y_{2}) = d(x,z) + d(z,y) $
	\end{enumerate}
	  \end{proof}
  \item $\Ra )$ Sea $(a_n,b_n)_n$ convergente a $(a,b)$ entonces $d_1(a_n,a) + d_2(b_n,b) = d((a_n,b_n),(a,b)) \ra 0$ entonces $d_1(a_n,a) + d_2(b_n,b) \ra 0 $ dado que son distancias son ambas mayores o iguales que 0 , por lo tanto ambas convergen a 0 , si no el sumando no convergería. 

	  $\Leftarrow ) $ $a_b \ra a$ y $b_n \ra b$ entonces $d_2((a_n,b_n),(a,b)) = d_1(a_n,a) + d_2(b_n,b) \ra 0$
  \end{enumerate}
\end{ej}

\begin{ej}
	Sea $(X_n,d_n)_{n \in \N}$ una sucesión de espacios métricos, y consideramos el producto cartesiano $X = \prod_{n =1}^{\infty} X_n$. El objetivo del ejercicios es construír una métrica para $X$ en la cual la convergencia de una sucesión equivalga a la convergencia en cada coordenada, como en el ejercicios anteriór.
	\begin{enumerate}
		\item Supongamos primero que todos los $X_n$ tienen diámetro menor o igual que 1, es decir $d_n(x,y) \leq 1 \quad \forall n\in \N \quad \forall x,y \in X_n$. Dados dos elementos $x = (x_n)_n$ e $y= (y_n)_n$ en $X$, definimos 

			$$ d(x,y) = \sum_{n = 1}^{\infty} \frac{d_n(x_n,y_n)}{2^n}$$

			Probar que $d$ es una métrica
			\begin{proof}
				Las primeras dos propiedades son triviales considerando que cada $d_n$ es métrica

				Ahora veamos la desigualdad

				$$ d(x,y) = \sum_{i = 1}^\infty \frac{d_n(x_n,y_n)}{2^n} = d_1(x_1,y_1) + d_2(x_2,y_2) + \cdots  \leq d_1(x_1,z_1) + d_1(z_1,y_1) + \cdots $$ 

				$$= \sum d_n(x_n,z_n) + d_n(z_n,y_n) = d(x,z) + d(z,y) $$

				Entonces es una métrica
			\end{proof}
		\item Sea $x^1,x^2,x^3,..$ una sucesión de puntos de $X$, es decir, cada $x^k$ es una sucesión $(x_1^k,x_2^k,\cdots)$, en la cual $x_n^k \in X_n$ para cada $n \in \N$. Sea $x = (x_n)_n$ un elemento de $X$. Probar que, con la métrica $d$ definida en el ítem anteriór, $x^k \ra x $ en $X$ si y sólo para todo $n \in \N$ se cumple que $x_n^k \ra x_n $ en $X_n$	
			
			\begin{proof}
				Tenemos que $d(x^k, x) \ra 0$ o lo que es equivalente dado  $\epsilon > 0 $ existe $k \in \N$ tal que  $d(x^k,x) \leq \epsilon \quad \forall k\geq k_0$. 

			Entonces $\sum_{n = 0}^{\infty} \frac{d_n(x_n^k , x_n)}{2^n} \leq \epsilon $

			Luego  $\sum_{n = 0}^{\infty} \frac{d_n(x_n^k , x_n)}{2^n} \leq 2^n\epsilon $ 

			Ahora dado cualquier $n \in \N$ tenemos que $d_n(x_n^k , x_n) \leq C \epsilon \quad \forall k \geq k_0$ (Donde $C = 2^n$ es una constante)

			O lo que es lo mismo $x_n^k \ra x_n$. Y esto vale para cualquier $n \in \N$ que tomemos
			\end{proof}
			
			
	\end{enumerate}
\end{ej}



\begin{ej}
  Sea $(X,d)$ un espacio métrico y sean $A,B \subseteq X$
  \begin{enumerate}
    \item Probar las siguientes propiedades del interiór de un conjunto:
      \begin{enumerate}
	\item $$A^{\open} = \bigcup_{G \text{ abierto, } G \subseteq A} G$$
	  \begin{proof}
		  $\subseteq )$ Sea $x \in A^{\open}$ entonces existe $r>0$ tal que $B(x,r) \subseteq A$ y este es un abierto contenido en $A$ entonces $x \in B(x,r) \subseteq \bigcup G$
	  
	$\supseteq ) $ Sea $x \in \bigcup G$ entonces $x \in G$ para algún $G$ de la unión

	Como $G$ es abierto existe $B(x,r) \subseteq G$ y por otro lado $G \subseteq A$

	Entonces existe $B(x,r) \subseteq A$ entonces $x \in A^{\open}$
	  \end{proof}
	\item $\emptyset^{\open} = \emptyset$
	  \begin{proof}
	    Supongamos $\emptyset^{\open} \neq \emptyset$ entonces $\exists x \in X$ tal que $x \in \emptyset^{\open}$

	    Luego tiene que existir $B(x,r) \subseteq \emptyset$ que es absurdo
	  \end{proof}
	\item $X^{\open} = X$
	  \begin{proof}
	  $\subseteq )$ Vale siempre

	$\supseteq )$ Sea $x \in X$ supongamos que $x \notin X^{\open}$ entonces $\forall r>0 \quad B(x,r) \not\subseteq X$ entonces $\exists y \in B(x,r)$ tal que $y \notin X$

	Absurdo por que $X$ es todo no pueden existir cosas que no esten en $X$
	  \end{proof}
	\item $A \subseteq B \Ra A^{\open} \subseteq B^{\open}$
	  \begin{proof}
	    Sea $x \in A^{\open}$ entonces existe $B(x,r) \subseteq A \subseteq B$ luego $x \in B^{\open}$
	  \end{proof}
	\item $(A \cap B)^{\open} = A^{\open} \cap B^{\open}.$ ¿Se puede generalizar a una intersección infinita?
	  \begin{proof}
	  $\subseteq )$ Sea $x \in (A \cap B)^{\open}$ entonces existe $B(x,r) \subseteq A \cap B$

	  Pero entonces $B(x,r) \subseteq A$ por lo que $x \in A^{\open}$

	  Tambien $B(x,r) \subseteq B$ por lo que $x \in B^{\open}$

	  Luego $x \in A^{\open} \cap B^{\open}$

	  Esta se puede generalizar a infinito

	$\supseteq )$ Sea $x \in A^{\open} \cap B^{\open}$ entonces $x \in A^{\open}$ y $x \in B^{\open}$ 

	Entonces existe $B(x,r_{1}) \subseteq A$ y también $B(x,r_{2}) \subseteq B$

	Si tomamos $r = \min{ \{r_{1},r_{2}\}}$ tenemos que $B(x,r) \subseteq A$ y tambien $B(x,r) \subseteq B$

      Entonces $B(x,r) \subseteq A \cap B$ finalmente $ x \in (A \cap B)^{\open}$

      Esta no se puede generalizar , por que ahora no necesariamente tenemos mínimo , y tenemos un conjunto de radios que si bien está acotado inferiormente por 0, nada nos asegura que el infimo no sea el mismo 0 que no nos serviría como radio.

      Ejemplo $\bigcap B(x,\frac{1}{n}) = \bigcap (B(x,\frac{1}{n}))^{\circ}$ esto es porque las bolas son abiertas por ende iguales a su interiór

      $x \in \bigcap (B(x,\frac{1}{n}))^{\circ}$ sin embargo $x \notin (\bigcap B(x,\frac{1}{n}))^{\circ} = \{x\}^{\circ} = \emptyset$
	  \end{proof}
	\item $(A \cup B)^{\open} \supseteq A^{\open} \cup B^{\open}$ ¿Vale la igualdad?
	  \begin{proof}
	    $x \in A^{\open} \cup B^{\open}$ entonces $x$ esta en alguno de los dos o los dos interiores

	    Supongamos $x \in A^{\open}$ entonces existe $B(x,r) \subseteq A \subseteq A \cup B$

	    Entonces $x \in (A \cup B)^{\open}$

	    Si esta en ambos , en particular esta en una , asi que usamos lo de arriba nuevamente

	    No vale la igualdad por ejemplo $A = [1,2]$ y $ B =  [2,3]$

	    $A^{\open} \cup B^{\open} = (1,2) \cup (2,3) \neq (1,3) = ([1,3])^{\open} = (A \cup B)^{\open}$
	  \end{proof}
      \end{enumerate}
    \item Probar las siguiente propiedades de la clausura de un conjunto
      \begin{enumerate}
	\item $$\ol A = \bigcap_{F \text{ cerrado, } A \subseteq F} F$$
	  \begin{proof}
		  $\subseteq$) Sea $x \in \ol A$ entonces $\forall r>0 , B(x,r) \cap A \neq \emptyset$ ahora supongamos $x \notin F$ para algún $F$

	    Como $F = \ol F$ por ser cerrado, entonces $x \notin \ol F$ para algún $F$ en la intersección

	    Entonces $\exists r'>0$ tal que $B(x,r') \cap F = \emptyset$

	    Pero esto es absurdo dado que $A \subseteq F$ tenemos $\emptyset \neq B(x,r') \cap A \subseteq B(x,r') \cap F = \emptyset$

	    Provino de suponer que $x \notin F$ por lo tanto $x \in F $

	    Y esto vale para cualquier $F$ cerrado tal que $A \subseteq F$

	    Entonces $x$ esta en todos estos $F$ y por ende en la intersección

	$\supseteq )$ Supongamos que $x \in \bigcap F$ pero $x \notin \ol A$ entonces tiene que existir un $r>0$ tal que $B(x,r) \cap A = \emptyset$ luego tenemos que $A \subseteq X \setminus B(x,r)$ que ademas es cerrado por que es el complemento de $B(x,r)$ que es abierto

	Pero entonces $X \setminus B(x,r)$ es un cerrado que contiene a $A$ por ende es uno de los $F$ en la intersección

	Entonces $x \in X \setminus B(x,r)$ lo cual es absurdo

	Provino de suponer que existia un $r>0$ tal que $B(x,r) \cap A = \emptyset$

	Entonces $\forall r>0 \quad B(x,r) \cap A \neq \emptyset$ por lo tanto $x \in \ol A$
	  \end{proof}
	\item $\ol \emptyset = \emptyset$
	  \begin{proof}
	    Supongamos que son diferentes entonces $\exists x \in X$ tal que $x \in \ol \emptyset$

	    entonces $\forall r>0 \quad B(x,r) \cap \emptyset \neq \emptyset$ lo cual es absurdo
	  \end{proof}
	\item $\ol X = X$
	  \begin{proof}
		  $\supseteq )$ Sea $x \in X$ entonces $\forall r >0$ tenemos que $ B(x,r) \cap X \neq \emptyset$ por que $x \in B(x,r)$ y $x \in X \quad \forall r >0$

		  Entonces $x \in \ol X$

	$\subseteq ) $ Sea $x \in \ol X$ entonces $B(x,r) \cap X \neq \emptyset \quad \forall r>0$ 

	Tomemos radios $\frac{1}{n}$, entonces $B(x,\frac{1}{n}) \cap X \neq \emptyset \quad \forall n \in \N$

	Ahora $\emptyset \neq \bigcap_{n \in \N} B(x,\frac{1}{n}) \cap X = (\bigcap_{n \in \N} B(x,\frac{1}{n})) \cap X = \{x\} \cap X$

	Entonces $\{x\} \cap X \neq \emptyset$ por lo tanto $x \in X$

	  \end{proof}
	\item $A \subseteq B \Ra \ol A \subseteq \ol B$
	  \begin{proof}
	    Sea $x \in \ol A$ entonces $\forall r>0 \quad B(x,r) \cap A \neq \emptyset$

	    Tambien sabemos que $A \subseteq B$ entonces $B(x,r) \cap A \subseteq B(x,r) \cap B$

	    Entonces $\forall r>0 \quad B(x,r) \cap B \neq \emptyset$ luego $x \in \ol B$
	  \end{proof}
	\item $\ol{A \cup B} = \ol A \cup \ol B$ ¿ Se puede generalizar a unión infinita?
	  \begin{proof}
	$\subseteq ) $ Sea $x \in \ol{A \cup B}$ luego $\forall r>0 \quad B(x,r) \cap (A \cup B) \neq \emptyset$

	Supongamos $x \notin \ol A \cup \ol B$ entonces $x \notin \ol A$ y $x \notin \ol B$

	Entonces $B(x,r_{1}) \cap A = \emptyset$ y por otro lado $B(x,r_{2}) \cap B = \emptyset$

	Luego sea $r = \min{\{r_{1},r_{2}\}}$ tenemos que 
      $$B(x,r) \cap (A \cup B) =(B(x,r) \cap A) \cup (B(x,r) \cap B) \subseteq (B(x,r_{1}) \cap A ) \cup (B(x,r_{2}) \cap B) = \emptyset $$

      Absurdo entonces no puede ser que $x \notin \ol A$ y $x \notin \ol B$

      Por lo tanto $x \in \ol A$ o $x \in \ol B$ luego $x \in \ol A \cup \ol B$

      Creo que no vale la generalización 

      Por ejemplo consideremos los conjuntos $A_n = (\frac{1}{n}, 2]$. Luego $0 \in \ol{\bigcup_{n} A_n}$

      Vale por que podemos construír una sucesión $x_j \subseteq \bigcup_{n \in \N} A_n$ tal que $x_j \ra 0$

      La podemos armar por ejemplo dando $x_1 = 2 $ y despues $x_j \in A_j \setminus A_{j-1} \neq \emptyset$

      Pero por otro lado $\forall n \in \N \quad 0 \notin \ol A_n$ por lo tanto $0 \notin \bigcup_{n\in \N} \ol A_n$

      $\supseteq ) $ Sea  $x \in \ol A \cup \ol B$ supongamos $x \in \ol A$ luego $\forall r>0 \quad B(x,r) \cap A \neq \emptyset$

    Entonces dado que $B(x,r) \cap A \subseteq B(x,r) \cap (A \cup B)$ 

    Tenemos $B(x,r) \cap (A \cup B) \neq \emptyset$ por lo que $x \in \ol{A \cup B}$

    Esta se puede generalizar facilmente a infinitos
	  \end{proof}
	\item $\ol{A \cap B} \subseteq \ol A \cap \ol B$
	  \begin{proof}
	    Sea $x \in \ol{A \cap B}$ entonces $\forall r>0 \quad B(x,r) \cap (A \cap B) \neq \emptyset$

	    Entonces por asociatividad $(B(x,r) \cap A ) \cap B \neq \emptyset$

	    Entonces tenemos $\forall r>0 \quad B(x,r) \cap A \neq \emptyset$ por lo que $x \in \ol A$

	    Lo mismo podemos hacer con $B$. Entonces $x \in \ol A \cap \ol B$

	No vale la vuelta:  

	Sea $A = \Q$ y $B = \R \setminus \Q \quad \ol{A \cap B} = \ol{\emptyset} = \emptyset \neq \R = \R \cap \R = \ol \Q \cap \ol{ \R \setminus \Q} = \ol A \cap \ol B$ 
	  \end{proof}
	\item $x \in \ol A \iff$ existe una sucesión $(x_{n})_{n \in \N} \subseteq A$ tal que $x_{n} \longrightarrow x$
	  \begin{proof}
	  $\Ra )$ Sea $x \in \ol A$ entonces $\forall n \in \N \quad  B(x,\frac{1}{n}) \cap A \neq \emptyset$
	
	  Entonces $\forall n \in \N$ existe $a_{n} \in A \cap B(x,\frac{1}{n})$ entonces $a_{n} \in A$ y $a_{n} \in B(x,\frac{1}{n})$

	  Que es lo mismo que decir $\forall n \in \N \quad \exists a_{n} \in A$ tal que $ d(x,a_{n})\leq \frac{1}{n}$

	  Ademas podemos ver que $d(x,a_n) \leq d(x, a_{n_0})$ si $n \geq n_0$. 

	  Esto vale por que $B(x,\frac{1}{n}) \subseteq B(x,\frac{1}{n_0})$

	  Ahora dado cualquier $\epsilon$ sabemos que $\exists n_0 \in \N$ tal que $\frac{1}{n_0} \leq \epsilon$

	  Por lo tanto $\forall \epsilon > 0 \quad \exists n_{0}$ tal que $\forall n \geq n_{0} \quad d(x,a_{n}) \leq d(x,a_{n_0}) \leq \frac{1}{n_0}\leq \epsilon$

	  Juntando todo $\forall \epsilon > 0 \quad \exists n_0 \in \N$ tal que $d(x,a_n) \leq \epsilon \quad \forall n \geq n_0$

	  Entonces $a_n \ra x$

	$\Leftarrow ) $ Sea $a_{n} \in A \quad \forall n \in \N$ tal que $a_{n} \ra x $

	Entonces $\forall \epsilon > 0 \quad \exists a_{n} \in A $ tal que $d(x,a_{n}) \leq \epsilon$

	Luego $\forall \epsilon > 0$ tenemos $a_{n} \in B(x, \epsilon)$ con $a_{n} \in A$

      Por lo que $B(x,\epsilon) \cap A \neq \emptyset$

      Entonces $x \in \ol A$
	\end{proof}
      \end{enumerate}

    \item Probar las siguientes propiedades que relacionan interiór y clausura:
      \begin{enumerate}
	\item $(X \setminus A)^{\open} = X \setminus \ol A$
	  \begin{proof}
	  $\subseteq ) $ Sea $x \in (X\setminus A)^{\open}$ entonces existe $r>0$ tal que $B(x,r) \subseteq (X \setminus A)$

	    Entonces $B(x,r) \cap A = \emptyset$ luego $x \notin \ol A$ y sabemos que $x \in X$

	    Entonces $x \in X \setminus \ol A$

	  $\supseteq )$ Sea $x \in X \setminus \ol A$ entonces $x \notin \ol A$

	  Entonces $\exists r>0 \quad B(x,r) \cap A = \emptyset$

	Por lo tanto $B(x,r) \subseteq X \setminus A$ luego $x \in (X \setminus A)^{\open}$
	  \end{proof}
	\item $\ol{X \setminus A} = X \setminus A^{\open}$
	  \begin{proof}
	  Sea $x \in X\setminus A^{\open} \iff x \notin A^{\open} \iff \forall r > 0 \quad B(x,r) \not\subseteq A$ 

	  $\iff \forall r>0 \quad B(x,r) \cap (X\setminus A) \neq \emptyset \iff x \in \ol{X \setminus A}$
	  \end{proof}
	\item ¿Es cierto que vale $\ol A = \ol{A^{\open}}$?
	  \begin{proof}
	    Si $A \subseteq \R$ con $A = \{1\}$ entonces $\ol A = \{1\} \neq \emptyset = \ol{\emptyset} = \ol{A^{\open}}$
	  \end{proof}
	\item ¿Es cierto que vale $A^{\open} = (\ol A)^{\open}$? 
	  
	  $\Q^{\open} = \emptyset \neq \R = \R^{\open} = (\ol{\Q})^{\open}$
      \end{enumerate}

    \item Probar las siguientes propiedades de la frontera de un conjunto
      \begin{enumerate}
	\item $\partial A = \ol A \cap \ol{X \setminus A}$
	  \begin{proof}
	  $\subseteq )$ $x \in \partial A \iff \forall r>0 \quad B(x,r) \cap A \neq \emptyset$ y $B(x,r)\cap A^c \neq \emptyset $ 

	  $\iff x \in \ol A$ y $x \in \ol{A^c} = \ol{X \setminus A} \iff x \in \ol A \cap \ol{X \setminus A}$
	  \end{proof}
	\item $\partial A$ es cerrado
	  \begin{proof}
	    Esto es equivalente a ver que $\partial A = \ol{\partial A}$ una de las inclusiones es trivial

	    Veamos que $\ol{\partial A} \subseteq \partial A$. Sea $x \in \ol{\partial A}$ entonces $\forall r>0 \quad B(x,r) \cap \partial A \neq \emptyset$

	    Luego  $\forall r>0 \quad B(x,r) $ tenemos un $y \in \partial A$ tal que $y \in B(x,r)$

	    Como $y \in B(x,r)$ que es abierto $\exists r'$ tal que $B(y,r') \subseteq B(x,r)$

	  Como $y \in \partial A$ entonces $\forall r$ tenemos $B(y,r) \cap A \neq \emptyset$ y $B(y,r) \cap A^c \neq \emptyset$

	  En particular vale para $r'$, entonces $B(y,r') \cap A \neq \emptyset$ y $B(y,r') \cap A^c \neq \emptyset$

	  Entonces $\emptyset  \neq B(y,r') \cap A \subseteq B(x,r) \cap A$ y tambien sucede con $A^c$

	  Entones $x \in \partial A$

	  Otra opción es usar el ejercicio de arriba, como $\partial A = \ol A \cap \ol{X\setminus A}$ que es una intersección de dos cerrados entonces es cerrado 
	  \end{proof}
	\item $\partial A = \partial{(X \setminus A)}$
	  \begin{proof}
	  Esto sale por definición usando que $A^c = X \setminus A$ y que $A = (X\setminus A)^c$
	  \end{proof}
      \end{enumerate}
  \end{enumerate}
\end{ej}
\begin{ej}
	Sea $(X,d)$ un espacio métrico y sea $A \subseteq X$ un conjunto numerable. Probar que $\# \ol A \leq \mathfrak{c}$
	\begin{proof}
		Sea $B = \{(a_n)_n \subseteq A : a_n \text{ converge} \}$. Tenemos una sobreyección $f : B \ra \ol A$ donde $f(a_n) = a$ (con $a_n \ra a$) 

		Es evidentemente sobreyectiva, por que para cualquier $x \in \ol A$ tenemos un sucesión contenida en $A$ que converge a $x$

		Esta sucesión está en $B$ por ser una sucesión convergente contenida en $A$ 

		Luego $x$ tiene preimagen

		Luego $\# \ol A \leq \# B \leq \# A^{\N} \leq \# \N^{\N} = \mathfrak{c} $
	\end{proof}
	
	
\end{ej}


\begin{ej}
  Sea $(X,d)$ un espacio métrico y sean $G \subseteq X$ abierto y $F \subseteq X$ cerrado. Probar que $F \setminus G$ es cerrado y $G \setminus F$ es abierto
  \begin{proof}

	  Primero voy a probar que complemento de un abierto es cerrado. Sea $A$ abierto , supongamos que $A^c$ es abierto. 

	  Supongo $A^c$ no es cerrado, entonces existe algún punto de acumulación $x$ de $A^c$ tal que $x \notin A^c$ luego $x \in A$

	  Pero como $x \in A$ que es abierto entonces existe $r>0$ tal que $B(x,r) \subseteq A$, contradiciendo que $x$ era punto de acumulación de $A^c$

	  Por lo tanto, $A^c$ contiene todos sus puntos de acumulación entonces $A^c = \ol A$ por lo que es cerrado.

	  Que complemento de un cerrado $A$ es abierto, sale de forma similar , suponiendo que el $A^c$ no es abierto , entonces tenemos algún punto $x \in A^c$ que no es interiór

	  Por lo tanto para todo radio la bola de centro $x$ no está contenida enteramente en el complemento, luego podemos armar una sucesion de A que converga a $x$ y como $A$ es cerrado el límite de la sucesión tiene que estar en $A$ lo que es absurdo

Sabemos que $G \setminus F = G \cap F^c$ como $F$ cerrado entonces $F^c$ es abierto , por lo tanto tenemos una intersección de dos abiertos que sabemos que es abierta

Sabemos que $F \setminus A = F \cap G^c$ como $G$ es abierto entonces $G^c$ es cerrado, por lo tanto tenemos una intersección de dos cerrados que sabemos que es cerrada

  \end{proof}
\end{ej}

\begin{ej}
Sea $(X,d)$ un espacio métrico. Dados $a \in X$ y $r \in \R_{>0}$, llamamos bola cerrada de centro $a$ y radio $r$ al conjunto $\ol{B}(a,r) = \{x \in X : d(x,a) \leq r\}$
\begin{enumerate}
  \item Probar que $\ol{B}(a,r)$ es un conjunto cerrado y que $\ol{B(a,r)} \subseteq \ol{B}(a,r)$
    \begin{proof}
      Sea $y \in X \setminus \ol{B}(x,r)$, entonces $d(x,y) > r$ por lo tanto $\epsilon = d(x,y) -r > 0$

      Ahora sea $z \in B(y,\epsilon)$ entonces $d(z,x) + d(z,y) \geq d(x,y)$

      Luego $d(z,x) \geq d(x,y) - d(z,y) > d(x,y) - \epsilon = r $

      Entonces $z \in X \setminus \ol{B}(x,r) \quad \forall z \in B(y,\epsilon)$

      Por lo que $\forall y \in X \setminus \ol{B}(x,r) \quad \exists B(y,\epsilon) $ tal que $ B(y,\epsilon) \subseteq X \setminus \ol{B}(x,r)$

      Finalmente $X \setminus \ol{B}(x,r)$ es abierto entonces $\ol{B}(x,r)$ es cerrado

      Como sabemos que $B(x,r)\subseteq \ol{B}(x,r)$ y ahora sabiendo que $\ol{B}(x,r)$ cerrado

      Entonces $\ol{B(x,r)} \subseteq \ol{B}(x,r)$
    \end{proof}
  \item Dar un ejemplo de un espacio métrico y una bola abierta $B(a,r)$ cuya clausura no sea $\ol{B}(a,r)$

    Esto es dar un ejemeplo donde $\ol{B(x,r)} \not\supseteq \ol{B}(x,r)$

    Consideremos el espacio métrico $(\Z,\delta)$ donde $\delta$ es la distancia discreta 

    Para cualquier $x \in \Z$ tenemos $\ol{B(x,1)} = \ol{\{x\}} = \{x\}\not\supseteq \Z = \ol{B}(x,1)$
\end{enumerate}
\end{ej}

\begin{ej}
	Sean $(X,d)$ un espacio métrico, $p$ un punto de $X$ y $a,b$ números reales tales que $0 <a<b$. Probar que:
	\begin{enumerate}[i.]
		\item $ \{ x \in X \text{/} a < d(x,p)<b\}$ es abierto
			\begin{proof}
				Primero sea $A_1 = \{x \in X : d(x,p) < b\}$ entonces $A_1 = B(x,b)$ que ya demostramos que es abierta

				Y luego consideremos $A_2 = \{x \in X : a < d(x,p)\}$

				Ahora si miramos $A_2^c = \{x \in X : d(x,p) \leq a)\} = \ol B (x,a)$ la bola cerrada que ya demostramos que es cerrada

				Entonces $A_2$ tiene que ser abierto

				Pero $A = A_1 \cap A_2$ entonces por ser intersección de abiertos es abierto
			\end{proof}
			
		\item $ \{ x \in X \text{/} a \leq d(x,p) \leq b\}$ es cerrado. Sale igual que el $i.$
	\end{enumerate}
\end{ej}



\begin{ej}
  Sean $(X,d_{1})$ e $(Y,d_{2})$ espacios métricos. Se considera el espacio métrico $(X \times Y , d)$, donde la $d$ es la métrica definida en el Ejercicio 12. Probar que para $A \subseteq X$ y $B \subseteq Y$ valen: 
  \begin{enumerate}
    \item $(A\times B)^{\open} = A^{\open} \times B^{\open}$
      \begin{proof}
	      \begin{lemma}
	Sean $U$ y $V$ abiertos de $X$ e $Y$ respectivamente entonces $U \times V$ es abierto de $X\times Y$.

\begin{proof}
	



	
	Sea $(x,y) \in U \times V$. como $x \in U$ que es abierto existe $B(x,r_{1}) \subseteq U$ 

	Y lo mismo con $y$ existe $B(y,r_{2}) \subseteq V$

	Ahora si tomamos $r = \min{\{r_{1},r_{2}\}}$. 

	Tenemos $B((x,y),r)$, veamos que está contenida en $U \times V$ por que entonces habiendo tomado cualquier $(x,y) \in U\times V$ estaríamos encontrando $B((x,y),r)\subseteq U\times V$. Probando que $U \times V$ es abierto

	Sea $(x',y') \in B((x,y),r)$ luego $ r > d((x,y),(x',y')) = d_{1}(x,x') + d_{2}(y,y') $. Ambos sumandos son positivos por ser distancias. Luego ambos sumandos tienen que ser menores que $r$

	Entonces $d_{1}(x,x') < r \leq r_{1} $ entonces $x' \in B(x,r_{1}) \subseteq U$

	Y también $d_{2}(y,y') < r \leq r_{2}$ entonces $y' \in B(y,r_{2}) \subseteq V$

	Entonces $(x',y') \in U \times V$ luego $B_{r}(x,y) \subseteq U \times V$

	Entonces para cualquier $(x,y) \in U \times V$ encontramos $B_{r}(x,y) \subseteq U \times V$

	Luego $U \times V$ es abierto.
\end{proof}
\end{lemma}

Continua el ejercicio:

	Luego como $A^{\open}$ y $B^{\open}$ abierto entonces $A^{\open} \times B^{\open}$ abierto por lema

	Luego dado que $A^{\open} \times B^{\open} \subseteq A \times B$ y $A^{\open} \times B^{\open}$ es abierto. Entonces $A^{\open} \times B^{\open} \subseteq (A \times B)^{\open}$

	Veamos $A^{\open} \times B^{\open} \supseteq (A \times B)^{\open}$

	Sea $(x,y) \in (A \times B)^{\open}$ entonces existe $r>0 \quad B_{r}(x,y) \subseteq (A \times B)$

	Entonces si $x' \in B(x,\frac{r}{2})$ e $y' \in B(y,\frac{r}{2})$

	Luego $d((x',y')(x,y)) = d_{1}(x',x) + d_{2}(y',y) < \frac{r}{2} + \frac{r}{2} = r$

	entonces $(x',y') \in B_{r}(x,y) \subseteq A \times B$

	Luego $x' \in A$ y tambien $y' \in B$

	Por lo tanto $B(x,\frac{r}{2}) \subseteq A$ y por otro lado $B(y,\frac{r}{2}) \subseteq B$

	Entonces $x \in A^{\open}$ e $y \in B^{\open}$ luego $(x,y) \in A^{\open} \times B^{\open}$  
      \end{proof}
    \item $\ol{A\times B} = \ol{A} \times \ol{B}$
      \begin{proof}
	Siguiendo las ideas anteriores probemos que $F$ y $G$ cerrados entonces $F \times G$ es cerrado

	Sea $F$ e $ G$ cerrados entoncse $X \setminus F$ y $Y \setminus G$ son abiertos 

	Ahora como sabemos que $X$ e $Y$ son abiertos (y cerrados, pero no nos importa)

	Luego $X\setminus F \times Y  $ es abierto por ser producto de dos abiertos $X\setminus F$ e $Y$.

	Pero $ X\setminus F \times Y = X \times Y \setminus (F \times Y) $ y esto es el complemento de $F \times Y$

	Entonces el complemento de $F \times Y$ es abierto , por lo que $F \times Y$ es cerrado

	De la misma manera $X \times Y \setminus G = X \times Y \setminus X \times G$ abierto entonces $X \times G$ es cerrado

	Luego $(X \times G )\cap (F \times Y) = F\times G$ es intersección de cerrado

	Entonces $F \times G$ es cerrado

	Luego usando esto tenemos que $\ol A$ y $\ol B$ son cerrados por lo que $\ol A \times \ol B$ es cerrado

	Luego $A \times B \subseteq \ol A \times \ol B$ entonces $\ol{A \times B} \subseteq \ol{\ol A \times \ol B} = \ol A \times \ol B$

	Veamos la otra inclusión. Tenemos $(x,y) \in \ol A \times \ol B$ queremos ver $(x,y) \in \ol{A \times B}$

	Dado cualquier $r>0$ queremos que $B(r,(x,y)) \cap A \times B \neq \emptyset$

	Pero sabemos que $\forall r' > 0$ en particular para $\frac{r}{2}$ se da $B(x,\frac{r}{2}) \cap A \neq \emptyset$ y también $B(y,\frac{r}{2}) \cap B \neq \emptyset$

	Recordemos $((A\times B) \cap (C \times D) = A \cap C \times B \cap D)$

	Entonces $\emptyset \neq B(x,\frac{r}{2}) \cap A \times B(y,\frac{r}{2} \cap B) = (B(x,\frac{r}{2}) \times B(y,\frac{r}{2})) \cap (A \times B)$

	Veamos que $B(x,\frac{r}{2}) \times B(y,\frac{r}{2}) \subseteq B((x,y),r)$

	Sea $(x',y') \in  B(x,\frac{r}{2}) \times B(y,\frac{r}{2})$ entonces $x' \in B(x,\frac{r}{2})$  e $y'\in B(y,\frac{r}{2})$

	$d(x',x) < \frac{r}{2}$ y también $d(y,y') < r$

	Entonces $d((x',y')(x,y)) = d(x',x) + d(y',y) < \frac{r}{2} + \frac{r}{2} = r$ 

	Por lo tanto $(x',y ') \in B((x,r),r)$ luego $B(x,\frac{r}{2}) \times B(y,\frac{r}{2})\subseteq B((x,y),r)$

	Finalmente juntando todo lo que teníamos llegamos a 

	$$ \emptyset \neq B(x,\frac{r}{2}) \times B(y,\frac{r}{2}) \cap (A \times B) \subseteq B((x,y),r) \cap A \times B$$

	Esto lo podemos hace para cualquie radio, por lo tanto $(x,y) \in \ol{A\times B}$

      \end{proof}
  \end{enumerate}
\end{ej}

\begin{ej}
  Sea $(X,d)$ un espacio métrico y sean $A,B$ subconjunto de $X$.
  \begin{enumerate}
    \item Probar las siguientes propiedades del derivado de un conjunt:
      \begin{enumerate}
	\item $A'$ es cerrado.
	  \begin{proof}
	    Sea $(a_{n})_{n \in \N} \subseteq A'$ convergente tal que $a_{n} \ra a$, queremos ver que $a \in A'$ esto nos diría que $A' = \ol{A'}$

	    Como $a_{n} \ra a$ dado un $\epsilon > 0$ existe $n_{0}$ tal que $\forall n > n_{0} \quad d(a,a_{n}) \leq \epsilon$

	    Equivalentemente para cualquier $\epsilon > 0$ existe $n_{0}$ tal que $\forall n \geq n_{0}$ $a_{n} \in B(a,\epsilon)$. Pero tomemos solo un $a_{n}$ llamemoslo $a_{j}$ tal que $a_{j} \in B(a,\epsilon)$

	    Como $a_{j} \in B(a,\epsilon)$ es abierto entonces existe $r'$ tal que $B(a_{j},r') \subseteq B(a,r)$

	    Tambien sabemos que $a_{j} \in A'$ entonces existe $(x_{n})_{n \in \N} \subseteq A$ tal que $x_{n} \ra a_{j}$

	    Sea $\epsilon = r'$ tenemos que exsite  $n_{1}$ tal que $\forall n \geq n_{1}$ $d(x_{n}, a_{j}) \leq r'$

	    Entonces $\forall n \geq n_{1}$ $x_{n} \in B(a_{j},r')$

	    Por lo tanto hay numerables $x_{n} \in A$ tal que $x_{n}  \in B(a_{j},r') \subseteq B(a,r)$

	    Entonces hay numerables $x_{n} \in A$ tal que $x_{n} \in B(a,r)$

	    Por lo tanto $B(a,r) \cap A$ es numerable.

	    Entonces $a$ es un punto de acumulación, $a \in A'$

	    Luego $A' = \ol{A'}$ entonces $A'$ es cerrado

	    Otra forma:

	    Sabemos que $A'\subseteq \ol{A'}$ veamos que $\ol{A'} \subseteq A'$

	    Tomemos $p\in\ol{A'}$ queremos ver que es punto de acumulación de $A$ entonces $p \in A'$

	    Como $p \in \ol{A'}$ entonces $B(p,r) \cap A' \neq \emptyset \quad \forall r>0$

	    Una opción es que $p \in A'$ y ya estaríamos. La otra opción es que $p \notin A'$

	    Entonces existe un $p_1 \in A'$ tal que $d(p_1,p) < \frac{r}{2}$ con $p_1 \neq p$ 

	    Además $p_1 \in A'$ luego $(B(p_1,\frac{r}{2})\setminus\{p_1\})\cap A \neq \emptyset$ 

	    Entonces $\exists a \in A$ tal que $d(a,p_1) < \frac{r}{2}$ 

	    Ahora $d(a,p) \leq d(a,p_1) + d(p_1,p) = \frac{r}{2} + \frac{r}{2} = r$ 

	    Entonces $a \in B(p,r)$ además $a \in A$ y $a\neq p$ por lo tanto $(B(p,r)\setminus\{p\})\cap A \neq \emptyset$

	    Y esto lo podemos hacer para cualquier $r$. 

	    Finalmente $p$ es punto de acumulación de $A$ o lo que es lo mismo $p \in A'$
	  \end{proof}
	\item $A \subseteq B \Longrightarrow A' \subseteq B'$
	  \begin{proof}
	    Sea $x \in A'$ entonces existe $(x_{n} )_{n \in \N} \subseteq A$ tal que $x_{n} \ra x$

	    Como $A \subseteq B$ la misma sucesión $(x_{n})_{n} \subseteq B$ entonces $x \in B'$

	    Esto se aprovecha de algo que se prueba en este mismo ejercicio asi que voy a dar otra solución sin usarlo.

	    Sea $a \in A'$ entonces $\emptyset \neq (B(a,r)\setminus \{a\}) \cap A  \subseteq (B(a,r)\setminus \{a\}) \cap B\quad \forall r>0 $

	    Entonces $a \in B' $
	  \end{proof}
	\item $(A \cup B)' = A' \cup B'$
	  \begin{proof}
	  $\subseteq )$ Sea $x \in (A \cup B)'$ entonces existe $(x_{n})_{n} \subseteq A \cup B$ tal que $x_{n} \ra x$

	    Entonces $x_{n} \in A$ o $x_{n} \in B$ para infinitos términos , si no tendría infinitos términos fuera de $A$ y fuera de $B$ lo que es absurdo. Quizas para los dos , pero no importa.

	    Spd $x_{n} \in A$ para infinitos términos entonces me quedo con todos los términos de $x_{n}$ tal que $x_{n} \in A$ esto es una subsucesión de $x_{n}$ entonces converge a $x$ por lo tanto tengo una sucesión contenida en $A$ que converge a $x$ luego $x \in A'$

	    Entonces $x \in A' \cup B'$

	  $\supseteq )$ Sea $x \in A' \cup B'$ spd $x \in A'$ luego existe $(a_{n})_{n \in \N} \subseteq A$ tal que $a_{n} \ra a$
	    Sin usar sucesiones:

	    $\subseteq$) Sea $x \in (A \cup B) '$. Entonces: 

	    $$\emptyset \neq (B(x,r)\setminus\{x\}) \cap (A \cup B) = ((B(x,r)\setminus \{x\}) \cap A) \cup ((B(x,r)\setminus\{x\}) \cap B)$$

	    Entonces $B(x,r)\setminus\{x\} \cap A \neq \emptyset$ o $B(x,r)\setminus\{x\} \cap B \neq \emptyset \quad \forall r>0$
 
            Supongamos que $x \notin A'\cup B'$ entonces $x \notin A'$ y $x \notin B'$ 

	    Luego $\exists r_1>0$ tal que $(B(x,r_1)\setminus\{x\}) \cap A = \emptyset$ 

	    También $\exists r_2>0$ tal que $(B(x,r_2) \setminus\{x\} )\cap B = \emptyset$

	    Tomamos $r = min \{r_1,r_2\}$ 

	    Entonces $(B(x,r)\setminus \{x\} \cap A = \emptyset)$ y también $ (B(x,r)\setminus \{x\} \cap B = \emptyset)   $

	    Lo que es absurdo, por lo tanto $x \in A'\cup B'$

	  $\supseteq )$ Sea $x \in A'\cup B'$ entonces spd $x \in A'$ por lo tanto $B(x,r)\setminus\{x\} \cap A \neq \emptyset \quad \forall r>0$

	  Pero entonces $B(x,r)\setminus\{x\}\cap (A \cup B) \neq \emptyset\quad \forall r>0$ por lo tanto $x \in (A\cup B)'$
	  \end{proof}
	\item $\ol A = A \cup A'$
	  \begin{proof}
	  Primero notemos que si $x \in A'$ entonces $\forall r > 0 \quad  B(x,r) \cap A$ es infinita 

	 Por lo tanto diferente del vacio entonces $x \in \ol A$ entonces $A' \subseteq \ol A$

	 Otra forma de verlo es $\emptyset \neq B(x,r)\setminus\{x\} \cap A \subseteq B(x,r) \cap A \quad \forall r>0$

	 Luego $x \in \ol A$ entonces $A'\subseteq \ol A$

	  $\supseteq )$ $ A \cup A' \subseteq \ol A \cup A' \subseteq \ol A \cup \ol A= \ol A$

	$\subseteq )$ Sea $x \in \ol A$ entonces $\forall r>0 \quad  B(x,r) \cap A \neq \emptyset$

	Si $x \in B(x,r) \cap A \quad \forall r >0$ entonces $x \in A$ luego $x \in A\cup A'$

	Si $x \notin B(x,r) \cap A\quad \forall r>0$ entonces $\emptyset \neq B(x,r) \cap A = (B(x,r)\setminus \{x\}) \cap A$

	Por lo tanto $x \in A'$ luego $x \in A \cup A'$
	  \end{proof}
	\item $(\ol A)' = A'$
	  \begin{proof}
	$\supseteq )$ Usando el b) tenemos que como $A \subseteq \ol A \Ra A' \subseteq (\ol A)'$

      $\subseteq )$ Sea $x \in (\ol A)'$ entonces existe $(x_n)_{n\in \N} \subseteq \ol A \setminus \{x\}= (A \cup A') \setminus \{x\}$ tal que $x_n \ra x$

    Luego $x_{n}$ tiene infinitos términos en $A$ o en $A'$ o en las dos

  Si tiene infinitos en $A$ podemos armar una subsucesión $(x_{n_{j}})_{n\in \N} \subseteq A$ como es subsucesión $x_{n_j} \ra x$ entonces $x \in A'$

Si tiene infinitos en $A'$ similarmente llegamos a que $x \in (A')' \subseteq A'$

Si tiene infinitos en las dos , podemos usar cualquiera de los dos argumentos

\begin{remark}
 $(A')'\subseteq A'$ 
 \begin{proof}
 $A'$ es cerrado por lo tanto para cualquier $(x_{n})_n \subseteq A'$ tal que $x_n \ra x$ sucede que $x \in A'$. Si no , no sería cerrado 

Luego $A'$ contiene a todos sus puntos de acumulación por lo tanto $(A')' \subseteq A'$
 \end{proof} 
\end{remark}

	Sin usar sucesiones:

	Sea $x \in ( \ol A)'$ entonces $\emptyset \neq (B(x,r)\setminus \{x\})\cap \ol A = (B(x,r)\setminus \{x\})\cap (A \cup A')$

	Entonces $((B(x,r)\setminus\{x\} )\cap A) \cup ((B(x,r)\setminus\{x\}) \cap A') \neq \emptyset$

        Por lo tanto $(B(x,r)\setminus\{x\}) \cap A \neq \emptyset$ o $(B(x,r)\setminus\{x\}) \cap A' \neq \emptyset$

	Si pasa la segunda entonces $x \in (A')' \subseteq A'$ si no $x \in A '$ por lo tanto $x \in A'$

	\begin{remark}
		$(A') '\subseteq A'$. Sea $x \in (A') '$ 

		Entonces $\emptyset \neq (B(x,r)\setminus\{x\})\cap A'\subseteq B(x,r) \cap A'\quad \forall r>0$ 

		Por lo tanto $x \in \ol{A'}$ pero como sabemos que $A'$ es cerrado entonces $\ol{A'} = A'$

		Luego $x \in A'$
	\end{remark}

	  \end{proof}
      \end{enumerate}
      \newpage
    \item Probar que $x \in X$ es un punto de acumulación de $A \subseteq X$ si y solo si existe una sucesión $(x_{n})_{n \in \N} \subseteq A$ tal que $x_{n} \ra x$ y $(x_{n})_{n \in \N}$ no es casi constante. 
      \begin{proof}
    $\Ra )$ Sabemos que si $x \in A'$ entonces $\forall r>0 \quad B(x,r) \cap A$ es infinito entonces $(B(x,r)\setminus \{x\}) \cap A$ es también infinta. 

      Luego definamos $x_{n}$ tal que $x_{n} \in B(x,\frac{1}{n}) \setminus \{x\} \cap A$ para cada $n \in \N$

Ahora afirmo $x_{n} \ra x$ veamosló

Sea $\epsilon > 0$ sabemos por arquimedianidad que exsite $n_0$ tal que $\frac{1}{n_0} \leq \epsilon$

Luego por como construí $x_n$ tengo que $x_n \in B(x,\frac{1}{n_0}) \quad \forall n \geq n_0$

Entonces dado cualquier $\epsilon > 0$ tenemos que existe $n_0 \in \N$ tal que $d(x_n ,x ) \leq \epsilon \quad \forall n \geq n_0$

Por lo tanto $ \forall \epsilon > 0$ tenemos que existe $n_0 \in \N$ tal que $ d(x,x_n) \leq \epsilon \quad \forall n \geq n_0 $ 

Entonces $x_n \ra x$. Además $x_{n}$ no puede ser casi constante, si lo fuera existiría un $n_{0}$ tal que  $x_{n} = x \quad \forall n \geq n_{0}$ pero esto es absurdo por que sabemos que $x_n \neq x \quad \forall n \in \N$  
     
Si en cambio existiera un $n_0$ tal que $x_n = a \neq x \quad \forall n \geq n_0$ luego $a_n$ no convergería a $x$

    \end{proof}
  \end{enumerate}
\end{ej}

\begin{ej}
  Hallar interiór, clausura, conjunto derivado y frontera de cada uno de los siguientes subconjuntos de $\R$. Determinar cuales son abiertos o cerrado

  $$ [0,1] \quad ; \quad (0,1) \quad ; \quad \Q \quad ; \quad \Q \cap [0,1] \quad ; \quad \Z \quad ; \quad [0,1) \cup \{2\}$$

  \begin{proof}
    \begin{enumerate}
      \item $[0,1]$ Es facil ver que el interiór es $(0,1)$ viendo que cada punto es interión tomando un punto y usando como radio el minimo de las distancias hacia 0 y hacia 1
	
	La clausura es también simple por que todo punto en $[0,1]$ cumple trivialmente que la intersección con $[0,1]$ es diferente de vacía

	Todos los puntos en $[0,1]$ son de acumlación usando la sucesión constante

	La frontera es el conunto $\{0,1\}$ es facíl ver que son de la frontera y es facil ver que cualquier otro no cumple ser de la frontera

	Usando esto es facil ver que $[0,1]$ es cerrado

	Para $(0,1)$ el análisis es similar

	$\Q$ por densidad de $\I$ es facil ver que dado un $x \in \Q \quad \forall r>0 \quad B(x,r) \cap \I \neq \emptyset$

	Por ende ninguna bola puede estar contenida en $\Q$ y entonces su interiór es vacío

	Esta claro que todos $x \in \Q$ es de acumulación , usando la sucesión constante, pero además todo $x \in \I$ es de acumulación de $\Q$ por densidad de racionales es facil de probar

Sabiendo que $\ol \Q = \Q \cup \Q'$  tenemos que $\ol \Q = \Q \cup \I =  \R$

$\partial \Q = \ol{\Q} \setminus \Q^{\open} = \R \setminus \emptyset = \R$

Un análisis muy similar podemos hacer con $\Q \cap [0,1]$

$\Z$ devuelta su interiór es vacío, es facíl ver que todos sus puntos son aislados, entonces no pueden ser de acumulación

Luego $\Z ' = \emptyset$ entonces tambien tenemos que $\ol{\Z} = \Z \cup \Z ' = \Z$

$\partial{\Z} = \ol{\Z} \setminus \Z^{\open} = \ol{\Z} = \Z$

$A = [0,1) \cup \{2\}$ tenemos que $2$ no puede ser interiór usando $\forall r > 0 \quad B(2,r) \not\subseteq A$

Lo mismo con $0$ para cualquier $B(x,r)$ sabemos que existe un $x < 0$ tal que $x \in B(0,r)$ por ende $B(0,r) \not\subseteq A$ el $1 \notin A$ por lo tanto $1 \notin A^{\open}$

Para el resto de los puntos $y$ es facil encontrar un radio usando $d(y,1)$ o $d(y,0)$  

Finalmente tenemos $A^{\open} = (0,1)$

Es fácil ver que $0$ son puntos de acumulación usando una sucesión por derecha

Luego usando una sucesión de numeros menores que 1 vemos que 1 es de acumulación 

Entonces $A' = [0,1] $

Luego $\ol A = A \cup A' = [0,1] \cup \{2\}$

$ \partial A = \ol{A} \setminus A^{\open} = \{0,1,2\} $
    \end{enumerate} 
  \end{proof}
\end{ej}

\begin{ej}
  Caracterizar los abiertos y los cerrados de $\Z$ considerado como espacio métrico con la métrica inducida por la usual de $\R$. Generalizar a un subespacio discreto de un espacio métrico $X$.
  \begin{proof}
      
  \end{proof}
\end{ej}

\begin{ej}
  Sea $(X,d)$ un espacio métrico y sean $(x_n)_{n \in \N}, (y_n)_{n \in \N}$ sucesiones en $X$.
  \begin{enumerate}
    \item Si $\lim x_n = x$ y $\lim y_n = y$, probar que $\lim_{n \ra \infty} d(x_n,y_n) = d(x,y)$ 
      \begin{proof}
	Sabemos que $d(x_n , y_n) \leq d(x_n,x) + d(x,y_n) \leq d(x_n,x) + d(x,y) + d(y,y_n)$

	Entonces tenemos que $\lim_{n \ra \infty} d(x_n,y_n) \leq \lim_{n \ra \infty}{(d(x_n,x) + d(x,y) + d(y,y_n))}$

	Como todos los límites del lado derecho exiten los puedo separar $\lim d(x_n,y_n) \leq d(x,y)$

	Con la misma idea $d(x,y) \leq d(x,x_n) + d(x_n,y) \leq d(x,x_n) + d(x_n,y_n) + d(y_n ,y)$

	entonces $- d(x_n,y_n) \leq d(x,x_n) - d(x,y) + d(y_n,y)$

	$\lim - d(x_n,y_n) \leq \lim (d(x,x_n) - d(x,y) + d(y_n,y)) $ 

	Todos los límites existen entonces separando $-\lim d(x_n,y_n) \leq -d(x,y)$

	Finalmente $\lim d(x_n,y_n) \geq d(x,y)$

	Entonces $d(x_n,y_n) = d(x,y)$
      \end{proof}
      \newpage
    \item Si $(x_n)_{n \in \N}, (y_n)_{n \in \N}$ son de sucesiones de Cauchy de $X$, probar que la sucesión real $(d(x_n,y_n))_{n \in \N}$ es convergente
      \begin{proof}
	Sabemos que ambas sucesiones son de cauchy entonces 

	Dado un $\epsilon > 0$ tenemos que existe $n_0 \in \N$ tal que $d(x_n,x_m) \leq \frac{\epsilon}{2} \quad \forall n,m \geq n_0$

	Y con ese mismo dado $\epsilon > 0$ existe $n_1\in \N$ tal que $d(y_k,y_j) \leq \frac{\epsilon}{2} \quad \forall k,j\geq n_1 $

	Ahora si tomamos $n_2 = \max{\{n_1 ,n_0\}}$

	Tenemos ambas $d(x_n,x_m) \leq \frac{\epsilon}{2}$ y $d(y_k,y_j) \leq \frac{\epsilon}{2} \quad \forall n,m,j,k \geq n_2$


	Teniendo esto $d(x_n,y_n) \leq d(x_n ,x_{s}) + d(x_{s},y_n) \leq d(x_n,x_{s}) + d(x_{s},y_{s}) + d(y_{s},y_n)$

	Entonces dado $\epsilon > 0$ usando el $n_2$ tenemos $d(x_n,y_n) \leq \frac{\epsilon}{2} + d(x_s,y_s) + \frac{\epsilon}{2} \quad \forall n,s \geq n_2$

	Entonces dado el $\epsilon >0$ tenemos $n_2 \in \N$ tal que $d(x_n,y_n) \leq d(x_s,y_s) + \epsilon \quad \forall n,s \geq n_2$

	Hacieno el mismo proceso con $d(x_s,y_s)$ llegamos a que $d(x_s,y_s) - \epsilon \leq d(x_n,y_n)$
      
     Luego juntando estas dos ideas podemos notar que dado un $\epsilon > 0$ tenemos
     $$\exists n_2 \in \N \text{ tal que } |d(x_n,y_n) - d(x_s,y_s)| \leq \epsilon \quad \forall n,s > n_2$$ 

   Sabemos que para todo $\epsilon > 0$ podemos hacer el mismo proceso y encontrar un $n_2$ 
 $$\forall \epsilon > 0 \quad \exists n_2 \in \N \text{ tal que } |d(x_n,y_n) - d(x_s,y_s)| \leq \epsilon \quad \forall n,s \geq n_2$$ 


     Pero esto nos dice que $d(x_n,y_n)$ es de Cauchy y como $d(x_n.y_n) \in \R \quad \forall n \in \N$ y $\R$ es completo entonces $d(x_n,y_n)$ converge

      \end{proof}
  \end{enumerate}
\end{ej}
\begin{ej}
Un subconjunto de $A$ de un espacio métrico de $X$ se dice $G_{\delta}$ (respectivamente $F_\sigma$) si es intersección de una sucesión de abiertos (respectivamente unión de una sucesión de cerrados) de $X$
\begin{enumerate}
  \item Probar que el complemento de un $ G_{\delta}$ es un $F_{\sigma}$
    \begin{proof}
    Sea $G_{\delta} = \bigcap_{i \in I} G_i$ intersección de abiertos

    Luego $x \in (\bigcap_{i \in I} G_i)^c = G_{\delta}^c \iff x \notin \bigcap_{i \in I}G_i \iff$

    existe algún $G_i$ tal que $x \notin G_i \iff$ existe algún $G_i$ tal que $ x \in G_{i}^c$

    $\iff x \in \bigcup_{i \in I} G_i^c \iff x \in F_{\sigma}$
 
    Este último sí y solo sí vale por que $G_i$ es abierto, por lo tanto $G_{i}^c$ es cerrado, luego $\bigcup G_i^c$ es unión de cerrados por lo tanto un $F_{\sigma}$
    \end{proof}
    \newpage
  \item Probar que el complemento de un $F_{\sigma}$ es un $G_{\delta}$
    \begin{proof}
      Sea $F_{\sigma} = \bigcup_{i \in I}F_i$ unión de cerrados 
      
      Luego $x \in F_{\sigma}^c = (\bigcup_{i \in I} F_i)^c \iff x \notin \bigcup_{i \in I} F_i$ 

      $\iff \forall i \in I$ $x \notin F_i \iff x \in F_i^c \quad \forall i \in I \iff x \in \bigcap_{i \in I} F_i^c \iff x \in G_{\delta}$  

      El último si y solo si vale por que  $F_i$ es cerrado luego $F_i^c$ es abierto por lo tanto $\bigcap F_{i}^c$ es intersección de abiertos entonces es un $G_{\delta}$

    \end{proof}
  \item Probar que todo cerrado es un $G_{\delta}$. Deducir que todo abierto es un $F_{\delta}$
\begin{proof}
  Sea $F$ cerrado , definamos $U_n$ $$U_n = \bigcup_{x \in F} B(x,\frac{1}{n})$$

  $U_n$ es unión de abiertos por lo tanto abierto

  Ahora firmo que $F = \bigcap U_n$ osea intersección de abiertos. entonces $F$ es $G_{\delta}$

  Veamosló. $x \in F$ entonces $x \in B(x,\frac{1}{n})\quad \forall n \in \N$ entonces $x \in U_n \quad \forall n \in \N$

Entonces $y \in \bigcap U_n$

Sea $y \in \bigcap U_n$ entonces $y \in U_n \quad \forall n \in \N$ entonces para cada $n \in \N$ sabemos que $y $ pertenece a alguna de esas bolas, otra forma de decirlo $y \in B(x_n,\frac{1}{n})$ para algún $x_n \in F$ pero entonces dado un $\epsilon > 0$ sabemos que existe un $n_0 \in \N$ tal que $\frac{1}{n_0} \leq \epsilon$ pero ademas sabemos que para todo $n > n_0 $ sucede $\frac{1}{n} \leq \frac{1}{n_0} \leq \epsilon $ por ende $d(x_n , y) \leq \epsilon \quad \forall n \geq n_0$

Pero entonces $x_n$ converge a $y$ y además $x_n \in F \quad \forall n \in \N$ y como $F$ es cerrado tenemos que $y \in F$

Forma B. Sea $G$ abierto
$$U_n = \bigcup_{x \in X \setminus G}B(x,\frac{1}{n}) $$

Luego tenemos $F_n = X \setminus U_n$ que es complemento de abierto por lo tanto cerrado.

Ahora afirmo que $G = \bigcup F_n$ que es unión de cerrados por lo tanto $F_{\sigma}$

Veamosló, sea $y \in G$ supongamos $y \notin \bigcup F_n $ entonces $y \notin F_n \quad \forall n \in \N$

Entonces $y \in U_n \quad \forall n \in \N$ por lo tanto $y \in \bigcap U_n$ por el mismo argumento que antes esto implica que $y \in X \setminus G $ , lo que es absurdo. Luego $y \in \bigcup F_n$

Sea $y \in \bigcup F_n$ entonces $y \in F_n \quad \forall n \in \N$ entonces $y \notin U_n \quad \forall n \in \N$ 

Supongamos $y \notin G$ entonces $y \in X \setminus G$ pero entonces $y \in U_n$ para algún $n \in \N$ seguro

Lo que es absurdo , entonces $y \in G$
\newpage
\end{proof}
  \item 
    \begin{enumerate}
      \item Exhibir una sucesión de abiertos de $\R$ cuya intersección sea $[0,1)$. Idem con $[0,1]$
      \item Exhibir una sucesión de cerrados de $\R$ cuya unión sea $[0,1)$
      \item ¿Qué conclusión puede obtenerse de estos ejemplos? 
    \end{enumerate}
\end{enumerate}
\end{ej}


\begin{ej} a\\

  \begin{enumerate}
    \item Sea $(X,d)$ un espacio métrico. Se define $d'(x,y) = \frac{d(x,y)}{1 + d(x,y)}$. Probar que $d'$ es una métrica en $X$ topológicamente equivalente a $d$ (o sea, ambas dan a lugar a una misma noción de conjunto abierto). Observar que $0 \leq d'(x,y)\leq 1$ para todo $x,y \in X$
      \begin{proof}
	Consideremos $f = \frac{x}{1 + x}$ entonces podemos reescribir $d'(x,y) = f \circ d$

	También sabemos que $f$ es creciente dado que su derivada es mayor a $0 \quad \forall x \in \R$

	Y $f(0) = 0$. Estas dos cosas nos dicen que $f \circ d $ es distancia , por lo tanto $d'$ es distancia
      
  
Veamos que $d' = f \circ d$ y $d$ son topológicamente equivalentes

Sea $y \in B_{d'}(x,\epsilon)$  seguro existe un $r > 0$ tal que $\epsilon < \frac{r}{r+1}$ entonces $d'(y,x) \leq \epsilon \leq \frac{r}{r +1}$

$d'(x,y) = \frac{d(y,x)}{1 + d(y,x)} \leq \frac{r}{r+1} \iff \frac{r+1}{r} \leq \frac{1 + d(y,x)}{d(y,x)} \iff 1 + \frac{1}{r} \leq \frac{1}{d(y,x)} + 1 \iff d(y,x) \leq r$

Entonces $y \in B_d(x,r)$ por lo tanto $B_{d'}(x,\epsilon) \subseteq B_{d}(x,r)$

Ahora sea $y \in B_d(x,\epsilon)$ entonces $d(x,y) \leq \epsilon$ y seguro existe un $r \geq \epsilon$

Entonces $d(x,y) \leq \epsilon \leq r$ usando la misma idea llegamos a que entonces $d'(x,y) \leq \frac{r}{r+1}$

Luego $y \in B_{d'}(x,\frac{r}{r+1})$ luego $B_d(x,r) \subseteq B_{d'}(x,\frac{r}{r+1})$


      \end{proof}
    \item Sea $(X_n,d_n)_{n \in \N}$ una sucesión de espacios métricos tales que para cada $n \in \N$ vale $0 \leq d_n(x,y)\leq 1$ para todo par de elementos $x,y \in X_n$. 

      Para cada $x  = (X_n)_{n \in \N}, y = (y_n)_{n \in \N}$ definimos:
    $$ d(x,y) = \sum_{n = 1}^{\infty} \frac{d_n(x_n,y_n)}{2^n}$$
\item Sea $(X,d)$ un espacio métrico. Llamamos $X^{\N}$ al conjunto de las sucesiones de $X$. Mostrar que aplicando i) y ii) se le puede dar una métrica a $X^{\N}.$


  \end{enumerate}
\end{ej}
\newpage
\begin{ej}
  Sean $d_{\infty}$ y $d_2$ las métricas en $R^n$ definidas en el ejercicio 7. Mostrar que $d_{\infty}$ y $d_2$ son topológicamente equivalentes.
\begin{proof}
  Por un lado tenemos que $d_{\infty}(x,y) \leq d_1(x,y) \quad \forall x,y \in \R^n$

  Entonces si $x_k$ converge con $d_1$ entonces seguro converge con $d_{\infty}$

  Ahora por otro lado supongamos $x_k$ converge con $d_{\infty}$ 


  $x_k$ converge con $d_{\infty}$ entonces dado $\epsilon >0$ existe $n_0$ tal que $d_{\infty}(x_k,x) \leq \epsilon \quad \forall k \geq n_0$

  Entonces dado $\epsilon > 0$ existe $n_0$ tal que 
  $$d_1(x_k,x) = \sum_{j=1}^{n} |(x_k)_j - x_j| \leq n\sup_{1 \leq j \leq n} |(x_k)_j - x_j| = nd_{\infty}(x_k,x) \leq n\epsilon \quad \forall k \geq n_0$$ 

  \noindent Aclaración $j$ es el indice de componente, y $n$ es un número fijo, que sirve para cualquier $\epsilon$ y está dado por la dimensión de $\R^n$

  Luego $x_k$ converge en $d_1$. Entonces ambas distancias generan las mismas sucesiones convergentes, por lo tanto son equivalentes
\end{proof}
\end{ej}

\begin{ej}
  Sea $(X,d)$ un espacio métrico. Dados $A \subseteq X$ no vacío y $x \in X$, se define la $distancia$ de $x$ a $A$ como $d_A(x) = \inf{\{d(x,a):a\in A \}}$. Probar:
  \begin{enumerate}[i.]
    \item $|d_A(x) - d_A(y)| \leq d(x,y)$ para todo par de elementos $x,y \in X$
      \begin{proof}
	Tenemos que $d(x,a) \leq d(x,y) + d(y,a)$ 

	$d(x,A)  = \inf d(x,a) \leq \inf (d(x,y) + d(y,a)) = \inf d(x,y) + \inf d(y,a) = d(x,y) + d(y,A)$

	Entonces $d(x,A) - d(y,A) \leq d(x,y)$

	haciendo lo mismo pero arrancando de $d(y,a) \leq d(y,x) + d(x,a)$

	llegamos a $- d(x,y) \leq d(x,A) - d(y,A)$

	Juntando todo
	$$ |d_A(x) - d_A(y) | \leq d(x,y) $$
      \end{proof} 
    \item $x \in A \Ra d_A(x)=0$
      \begin{proof}
	Sea $D = \{d(x,a) : a \in A\}$ afirmo que $\inf D = 0$
	\begin{itemize}
	  \item $0 \leq d \quad \forall d \in D$ 

	    Si no fuera cierto existiria $d' \in D$ tal que $d' < 0$ entonces $d' = d(x,a) < 0$ para algún $a \in A$ lo que es absurdo
	  \item Sea $l \leq d \quad \forall d \in D $ entonces $l \leq 0$ Supongo que no es cierto, entonces existe $l \leq d \quad \forall d \in D$ con $l > 0$ , pero sabemos que $d(x,x) \in D$ y $d(x,x) = 0 < l$

	    Luego $0 = \inf D$ por lo tanto $d_{A}(x) = \inf D = 0$ 
	  \end{itemize}
	  \end{proof}
	  \newpage
	\item $d_A(x) = 0 \iff x \in \ol A$
	  \begin{proof}
	  $\Ra )$ Sea $D = \{d(x,a):A\in A\}$ luego $0 \inf D \iff$ 

	    Entonces existe un sucesión $d_n \in D \quad \forall n \in \N$ tal que $d_n \ra 0$

	    $\iff$ para cada $n \in \N$  existe $a \in A$ tal que $d_n = d(x,a)$ llamemosló $a_n$ 

	    Luego $d(x,a_n)= d_n \ra 0 \iff$ tenemos $a_n \in A \quad \forall n \in \N$ y además $a_n \ra x$

	     $\iff x \in \ol A$
	  \end{proof}
    \item $B_A(r) = \{x \in X : d_A(x) < r \}$ es abierto para todo $r>0$
      \begin{proof}
	Sea $x \in B_A(r)$ , primero una pequeña afirmación, 

	Como $x \in B_A(r)$ entonces $r > d_A(x) $ luego existe $\epsilon$ tal que $r - \epsilon > d_A(x)$

	Luego puedo tomar $r' = r  - \epsilon - d(x,A)$ y seguro $r > 0$

	Afirmo que $B(x,r') \subseteq B_A(r)$. Veamosló, sea $y \in B(x,r')$ entonces 

	$d(y,A) \leq d(y,x) + d(x,A) \leq r' + d(x,A) = r - \epsilon - d(x,A) + d(x,A) <r$

	Luego $y \in B_A(r)$ entonces $B(x,r') \subseteq B_A(r)$ 

	Finalmente $\forall x \in X \quad \exists r' >0 $ tal que $B(x,r') \subseteq B_A(r)$ 

	$B_A(r)$ es abierto
      \end{proof}
    \item $\ol B_{A}(r) = \{x \in X : d_A(x) \leq r\} $ es cerrado para todo $r>0$
      \begin{proof}
	Tomemos el complemento de la bola, $A = \{x \in X : d_A(x) > r\}$ veamos que es abierto

	Ahora sea $x \in A$ afirmo que $B(x,r') \subseteq A$ con $r' = d(x,A) - r > 0 $, veamosló

	Sea $y \in B(x,r')$ tenemos $d(x,A) - d(y,A) \leq |d(x,A) - d(y,A)| \leq d(x,y) < d(x,A) -r $

      Entonces $-d(y,A) < -r \Ra d(y,A) > r$ por lo tanto $y \in A$ luego $B(x,r') \subseteq A$

      Luego $\ol B_A(r)$ es complemento de un abierto , por lo tanto es cerrado
      \end{proof}
  \end{enumerate}
\end{ej}


\begin{ej}
  Sea $(X,d)$ un espacio métrico. Dados $A,B \subseteq X$ no vacíos se define la distancia entre $A$ y $B$ por $d(A,B) = \inf{\{d(a,b): a\in A \quad b \in B\}}$. Determinar si las siguientes afirmaciones son verdaderas o falsas:
  \begin{enumerate}
    \item $d$ es una distancia en $\mathcal{P}(X) \setminus \{\emptyset\}$ Es falso
      \begin{proof}
	Tomemos un $A \subseteq X$ con $A \neq \{\emptyset\}$  $B = A \cup \{x\} \quad x \in X$ 

	Entonces $d(A,B) = 0$ pero $A \neq B$ entonces no es una métrica
	\end{proof}
    \item $d(A,B) = d(A ,\ol  B)$ es verdadero

      Sea $L_1 = d(A,B) \quad L_2 = d( A ,\ol B)$ supongamos que son diferentes

      $L_1 < L_2$ entonces $\exists \epsilon > 0$ tal que $L_1 + \epsilon < L_2$

      Como $L_1$ es un ínfimo existe $a \in A, b \in B$ tal que $L_1 \leq d(a,b) \leq L_1 + \epsilon <  L_2 $

      Pero entonces existen $a \in A , b \in B$ tal que  $d(a,b) <  L_2  = \inf{\{d(a,b): a\in A \quad b \in \ol B\}}$

      Absurdo por que como $a \in A, b\in B$ entonces $d(a,b) \in \{d(a,b): a\in A \quad b \in \ol B\}$

      Ahora en cambio si $L_1 > L_2$ entonces usando el mismo argumento

    existe $a \in A \quad b' \in \ol B$ tal que $L_2 \leq d(a,b') \leq L_2 + \epsilon <  L_1$

    Entonces $d(a,b') < L_1$ entonces existe $\epsilon '$ tal que $d(a,b') + \epsilon ' < L_1$

  Ahora como $b' \in \ol B$ existe $(b_n )_n \subseteq B$ tal que $b_n \ra b$ Entonces $|d(a,b_n) - d(a,b')| \ra 0$ 

  Luego dado $\epsilon '$ existe $n_0$ tal que $d(a,b_n) \leq d(a,b') + \epsilon ' \quad \forall n \geq n_0$

  Por lo tanto $\forall n \geq n_0$ tenemos $d(a,b_n) < L_1$ pero con un $b_n \in B$ nos alcanza para decir que es absurdo dado que nuevamente $d(a,b_n) \in \{d(a,b) : a \in A \quad b \in B\}$ por ende $d(a,b_n)$ no puede ser menor que el infimo de un conjunto que lo contiene

Luego no sucede $L_1 < L_2 $ y tampoco $L_2 < L_1$ entonces $L_1 = L_2$
    \item $d(A,B) = 0 \iff A \cap B \neq \emptyset$ es falso
      \begin{proof}
	Sea $(\R^2,d)$.Con $d$ la distancia euclídea. 

	Sean $A = \{(x,0): x \in \N\} \quad B= \{(x,\frac{1}{x}): x \in \N \}$ 

	Sabemos que $A \cap B = \emptyset$ , sin embargo es facil ver que $d(A,B) = 0$. 

	Tomamos la sucesión $x_n = d((n,0), (n + \frac{1}{n})) = \sqrt{\frac{1}{n^2}}$. 
	
	$(x_n)_n \subseteq \{d(a,b) : a\in A \quad b \in B\}$ y además $x_n \ra 0$

	Por lo tanto $0$ es ínfimo

      \end{proof}
    \item $d(A,B) = 0 \iff \ol A \cap \ol B \neq \emptyset$
      \begin{proof}
	Sirve el mismo ejemplo que arriba, por que $A = \ol A$ y $B = \ol B$
      \end{proof}
    \item $d(A,B) \leq d(A,C) + d(C,B)$
      \begin{proof}
	$d(A,B)  \leq d(a,b) \leq d(a,c) + d(c,b) \quad \forall (a \in A \quad b \in B \quad c \in C)$

      $d(A,B) \leq d(a,c) + d(c,b) \quad \forall (a \in A \quad b \in B \quad c \in C)$ 

      Entonces $d(A,B) \leq \inf \{d(a,c) + d(c,b) : a \in A \quad b \in B\}$

      Que es igual a  $ \inf \{d(a,c) :  a \in A \quad c \in C\} + \inf \{d(c,b): c \in C \quad b \in B\}$

      o lo mismo $d(A,C) + d(C,B)$. Luego $d(A,B) \leq d(A,C) + d(C,B)$
      \end{proof}
  \end{enumerate}
\end{ej}



\end{document}
